\section{Quantitative Input Data Usage}
\labsec{quantitative-input-data-usage}
\newcommand*{\x}{\texttt{angle}}
\newcommand*{\y}{\texttt{speed}}
\newcommand*{\z}{\texttt{risk}}
\newcommand*{\lc}{\texttt{landing\_coeff}}

In this section we present some preliminaries on program computations, then we introduce our quantitative framework with the formal definitions of \rangename{} and \outcomesname.


Our goal is to quantify the impact of a specific input variable on the computation of the program.
To this end, we introduce the notion of impact, denoted by the function $\impactwrapper\in\setof\pairofstates\to\valuesposplus$, which maps program semantics to a non-negative domain of quantities, where $\definputvariable$ represents the input variable of interest in the program under analysis.

We implicitly assume the use of an \textit{output descriptor} $\outputdesc$ to determine the desired output of a program by observations on program states.

Specifically, $\reader\in \stateandbottom \to \valuesinf$ selects the output of interest from a given state and returns its corresponding value\sidenote{The option of returning $\pm\infty$ from the output descriptor is to deal with infinite traces, which do not have a final state ($\retrieveoutput\defseq = \statebottom$ for any $\defseq \in \infinitesequences$).}.
Additionally, $\filter\subseteq\valuesinf$ filters output states and selectively engages a subset of the potential outcomes.
Through this filtering mechanism, undesired outcomes are directly excluded, and a numerical value is ensured.

\begin{definition}[Output Descriptor]
  \labdef{outputdesc}
  Given $\reader\in\stateandbottom\to\valuesinf$ and $\filter\subseteq\valuesinf$, the tuple $\outputdesc$ is called an \textup{output descriptor}.
\end{definition}

The above output characterization $\outputdesc$ is generic enough to cover plenty of use cases.
We leverage this output descriptor to provide the end user of the framework the flexibility to choose the interpretation and meaning of program outputs, without establishing it beforehand.

\begin{marginlisting}
\begin{lstlisting}[language=customPython]
landing_coeff =
  abs(angle) + speed
if landing_coeff < 2:
  risk = 0
else if landing_coeff > 5:
  risk = 3
else:
  risk =
    floor(landing_coeff) - 2
\end{lstlisting}
\end{marginlisting}

\begin{example}
  Consider the~\refprog{landing-alarm-system} for the landing alarm system with program states $\state=\setdef{\langle a, b, c, d \rangle}{a\in\{-4,1\}\land b\in\{1,2,3\}\land c\in\N \land d\in\{0,1,2,3\}}$, where $a$ is the value of $\x$, $b$ of $\y$, $c$ of $\lc$, and $d$ of $\z$.
  Here, we abuse the notation and use $\state$ as set of tuples instead of a map between variables and values, the two views are equivalent.
  The output descriptor is instantiated with
  \[
  \reader(x) \DefeQ \begin{cases}
    d & \text{if } x = \langle a, b, c, d \rangle \\
    +\infty & \text{otherwise}
  \end{cases}
  \]
  and $\filter=\{0,1,2,3\}$ filters $+\infty$ from the possible outputs.
  In other words, we are interested in the value of $\z$ for terminating traces.

  However, the end-user of the analysis may be interested in only a subset of the possible outcomes of the program.
  For instance, only about the risk levels in $\{0, 1, 2\}$, forgetting about the value $3$.
  It is crucial that our impact definitions remain sound to the user assumption on post-conditions, even when it is under-approximating the exact one.
  Thus, the filter specifies this information by $\filter=\{0, 1, 2\}$, which is a subset of all the possible values of the output variable \z.
\end{example}

\begin{example}
  For other contexts, rather than considering a single output variable one may be interested in a custom operation.
  % For example, in the context of neural network classifiers, the output of the program is the index of the output variable with highest value.
  For example, the output of a neural network classifier is the index of the output neuron holding the highest value.
  Hence, for a network with $n+1$ output neurons, we could instantiate
  \[
    \reader(x_0,\dots,x_{w+n-1},x_{w+n})=\argmax_{0\le j\le n} x_{w+j}
  \] where the function $\argmax_j X_j$ returns the \textit{argument} $j$ of the value holding the \textit{maximum} among the indexed family $X_j$.
  The filter $\filter$ could be the set of all indices $\{0,\dots,n\}$, hence permitting all possible outcomes from the reader $\reader$.
\end{example}



We can now define our property of interest, the \textit{$\defbound$-bounded impact property} $\bounded$.
By extension, $\bounded$ is the set of trace semantics such that the impact of the set of traces allowed by the output descriptor $\outputdesc$, \wrt{} the input variable $\definputvariable$, is below the threshold $\defbound\in\valuesposplus$. Formally,
\begin{align}\labeq{eq:bounded}
  \BOUNDED \DefeQ \setdef
  {\tracesemanticsnoparam \in \setof\finiteinfinitesequences}
  {\impactwrapper(\setdef{\defseq\in\tracesemanticsnoparam}{\reader(\retrieveoutput{\defseq})\in\filter}) \impactsubseteq \defbound}
\end{align}
where $\impactwrapper$ is a parameter of the property $\bounded$ and returns the quantity computed on the given set of traces.
%
Following the definition of $\bounded$, our validation framework, \refeq{qualitative-soundness}, is instantiated as
%
\begin{align}
  \labeq{bounded-soundness}
  \defprogram \satisfies \BOUNDED \IfF \collectingsemantics \subseteq \BOUNDED
\end{align}
%
We are interested in properties $\bounded$ that are \textit{extensional}, namely, properties based on the observation of input-output relations of program states.
This means that, if $\tracesemantics\in\bounded$ and $\tracesemanticsnoparam\semanticsof{\defprogram'}$ consider the same set of input-output observations of $\tracesemantics$, then $\tracesemanticsnoparam\semanticsof{\defprogram'}\in\bounded$.
As a consequence, the impact is not affected by intermediate states,
$\impactwrapper(\tracesemantics)=\impactwrapper(\tracesemanticsnoparam\semanticsof{\defprogram'})$.
% Therefore, we derive the \textit{dependency semantics} $\dependencysemanticsnoparam$~\citep{Urban2020} as an abstraction of the collecting semantics $\collectingsemanticsnoparam$ removing intermediate computational states.
Furthermore, we require $\impactwrapper$ to be monotonic, \ie, for any $X, Y\in \setof\finiteinfinitesequences$, it holds that $X \subseteq Y$ if and only if $\impactwrapper(X) \le \impactwrapper(Y)$.
Intuitively, this ensures that an impact applied to an over-approximation of the program semantics can only produce a higher quantity, enabling the definition of a sound terminating static analysis.
Next, we formalize the already introduced impact metrics \rangename{} and \outcomesname.

\subsection{The \outcomesname{} impact definition}[The Outcomes Impact Definition]
\labsec{outcomes}
%
Formally $\outcomes\in\setof\pairofstates\to\Nplus$ counts the number of different output values, allowed by the output descriptor, reachable by varying the input variable $\definputvariable\in\inputvariables$.
First, we define step-by-step the quantity $\outcomes$,
followed by the instantiation of this quantity within the context of example of \refprog{landing-alarm-system}.
We present the formal definition at the end.


Intuitively,
for any possible input configuration $\definput\in\reducedstate$, we collect all the traces that are starting from an input configuration that is a variation of $\definput$ on the input variable $\definputvariable$, \ie, $\setdef{
  \defseq\in \defsetoftraces
}{
  \retrieveinput{\defseq} \stateeq{\inputvariableswithouti} \definput
}$, where $\defsetoftraces\in\setof\finiteinfinitesequences$.
Then, we collect the output values of this set of traces by means of the output descriptor $\outputdesc$, \ie, $\setdef{
  \reader(\retrieveoutput\defseq)
}{
  \defseq \in \defsetoftraces \land
    \retrieveinput{\defseq} \stateeq{\inputvariableswithouti} \definput
}$. Specifically, this set contains all the output readings performed by $\reader$.
%
Afterwards, we extract the number of elements via the cardinality operator $\cardinality{\cdot}$.
Finally, we iterate through each input configuration $\definput$ and return the maximum value to ensure the greatest impact is preserved.

\begin{example}[Landing Alarm System]
  \label{ex:range}
  \newcommand*{\inputa}{\tuple{-4}{1}} \newcommand*{\outputa}{\langle \outputvaluea\rangle} \newcommand*{\outputvaluea}{3}
  \newcommand*{\inputb}{\tuple{-4}{2}} \newcommand*{\outputb}{\langle \outputvalueb\rangle} \newcommand*{\outputvalueb}{3}
  \newcommand*{\inputc}{\tuple{-4}{3}} \newcommand*{\outputc}{\langle \outputvaluec\rangle} \newcommand*{\outputvaluec}{3}
  \newcommand*{\inputd}{\tuple{ 1}{1}} \newcommand*{\outputd}{\langle \outputvalued\rangle} \newcommand*{\outputvalued}{0}
  \newcommand*{\inpute}{\tuple{ 1}{2}} \newcommand*{\outpute}{\langle \outputvaluee\rangle} \newcommand*{\outputvaluee}{1}
  \newcommand*{\inputf}{\tuple{ 1}{3}} \newcommand*{\outputf}{\langle \outputvaluef\rangle} \newcommand*{\outputvaluef}{2}
  \newcommand*{\tracea}{\inputa\to\outputa}
  \newcommand*{\traceb}{\inputb\to\outputb}
  \newcommand*{\tracec}{\inputc\to\outputc}
  \newcommand*{\traced}{\inputd\to\outputd}
  \newcommand*{\tracee}{\inpute\to\outpute}
  \newcommand*{\tracef}{\inputf\to\outputf}
  Let us revisit the example of the landing alarm system, with program states $\state=\setdef{\langle a, b, c, d \rangle}{a\in\{-4,1\}\land b\in\{1,2,3\}\land c\in\N \land d\in\Nle{3}}$.
  The input variables are $\inputvariables = \{\x,\y\}$, consequently the input configurations are
  $\reducedstate=\{\inputa, \inputb, \inputc, \inputd, \inpute, \inputf\}$.
%
  We begin by considering $\definputvariable=\x$ and $\inputa$ as the first input configuration to be explored.
  Hence, we collect all traces that are
  starting from an input configuration that is a variation of $\inputa$, \ie, $\setdef{
    \defseq \in \tracesemantics
  }{
    \retrieveinput{\defseq} \stateeq{\inputvariables\setminus\{\x\}} \inputa
  }$, where $\inputvariables\setminus\{\x\} = \{\y\}$ and consequently $\retrieveinput{\defseq} \stateeq{\{\y\}} \inputa$ holds whenever the initial state of $\defseq$ has $\y=1$. A possible trace of this set is $\langle 1, 1, 0, 0\rangle \to \langle 1, 1, 2, 0\rangle\to\langle 1, 1, 2, 0\rangle$ where, at the beginning, we randomly assign $\lc=0$ and $\z=0$, respectively the third and fourth component of the initial state.
%
  We collect the output values of this set of traces, $\setdef{
    \reader(\retrieveoutput\defseq)
  }{
    \defseq \in \tracesemantics \land
      \retrieveinput{\defseq}(\y) = 1
  }$.
  As a result, we obtain the set of output values $\{0, 3\}$.
  For instance, the output value $0$ is the result of the trace we exhibited previously, where the last state is $\langle 1, 1, 2, 0\rangle$ and thus the $\z$ variable of this trace is the last component with value $0$.
%
  Finally, the cardinality operator returns the value $2$, $\cardinality{\{0, 3\}} = 2$.
  By doing so for all possible input configurations in $\reducedstate$, we obtain $\outcomesname_{\x}(\tracesemantics)=3$.
  \reftab{range-x} and~\reftab{range-y} in overview (\refsec{overview}) illustrate the steps for $\outcomesname_{\x}(\tracesemantics)$ and $\outcomesname_{\y}(\tracesemantics)$ respectively.
\end{example}

\begin{definition}[\outcomesname]\labdef{outcomes}
  Given an input variable $\definputvariable\in\inputvariables$, and an output descriptor $\outputdesc$,
  the quantity $\outcomes\in\setof\finiteinfinitesequences\to\Rposplus$ is defined as
  %
  \begin{align}
    \labeq{outcomes}
    \outcomes(\defsetoftraces) &\DefeQ \sup_{\definput\in\reducedstate}
      \cardinality{\setdef{
        \reader(\retrieveoutput{\defseq})
      }{
        \defseq \in \defsetoftraces \land \retrieveinput{\defseq} \stateeq{\inputvariableswithouti} \definput
      }}
  \end{align}
  where $\sup(X)$ is the supremum operator, \ie, the smallest $q$ such that $q\ge x$ for all $x\in X$.
\end{definition}

It is easy to note that $\outcomes(\defsetoftraces)$ uses only input-output states (\cf, $\retrieveinput{\defseq}$ and $\retrieveoutput{\defseq}$) of traces $\defseq\in\defsetoftraces$.
Therefore, assuming $\defsetoftraces, \defsetoftraces'\in\setof\finiteinfinitesequences$ share the same set of input-output observations, $\outcomes(\defsetoftraces) = \outcomes(\defsetoftraces')$ holds.
This implies that, $\bounded$ is an extensional property when $\outcomes$ is used to instantiate $\impactwrapper$ in $\bounded$.
Furthermore, this impact definition is monotone in the amount of traces. That is, the more traces in input, the higher the impact as only more dependencies can satisfy the condition of \refeq{outcomes}, \cf~$\retrieveinput\defstate \stateeq{\inputvariableswithouti} \definputvariable$, and hence increase the number of outcomes.


\subsection{The \rangename{} impact definition}[The Range Impact Definition]
\labsec{range}

%
The quantity $\range\in\setof\pairofstates\to\Rposplus$ determines the
length of the range of output values from all the possible variations in the input variable $\definputvariable\in\inputvariables$.
%
This definition employs the auxiliary function $\length\in\setof\valuesinf\to\valuesposplus$, defined as follows:
 $\length(X) \defeq \sup\;{X} - \inf\;{X}$ if $X\neq\emptyset$, where $\sup$ and $\inf$ are the supremum and infimum operators, while $\length(X)\defeq0$ otherwise.

\begin{example}[Landing Alarm System]
  \label{ex:range}
  We revisit again the example of the landing alarm system.
  Assuming $\definputvariable=\x$, $\tuple{-4}{1}$ is the first input configuration to be explored, we collect all traces that are
  starting from an input configuration that is a variation of $\tuple{-4}{1}$.
  As before, we obtain $\setdef{
    \retrieveoutput\defseq(\z)
  }{
    \defseq \in \defsetoftraces \land
      \retrieveinput{\defseq}(\y) = 1
  }=\{0,3\}$.
%
  Here, we apply the length function, hence $\length(\{0,3\})=3$.
  By doing so for all possible input configurations $\reducedstate$, we obtain $\rangename_{\x}(\tracesemantics)=2$.
  \reftab{range-count} illustrates the steps for both $\rangename_{\x}(\tracesemantics)$ and $\rangename_{\y}(\tracesemantics)$.
\end{example}

  Formally,
\begin{definition}[\rangename]\labdef{range}
  Given an input variable $\definputvariable\in\inputvariables$, and an output descriptor $\outputdesc$,
  the quantity $\range\in\dependencytype\to\Rposplus$ is defined as
  %
  \begin{align}
    \labeq{range}
    \range(\defsetoftraces) &\DefeQ \sup_{\definput\in\reducedstate}
      \length(\setdef{
        \reader(\retrieveoutput{\defseq})
      }{
        \defseq \in \defsetoftraces \land \retrieveinput{\defseq} \stateeq{\inputvariableswithouti} \definput
      })
  \end{align}
\end{definition}

This impact definition is monotone as well, in the amount of traces.
\begin{lemma}[\rangename{} is Monotonic]
  \lablemma{range-monotonic}
  For any $\defsetoftraces, \defsetoftraces'\in\dependencytype$, it holds that:
  \begin{align}
    \labeq{range-monotonic}
    \defsetoftraces \subseteq \defsetoftraces' \ImplieS \range(\defsetoftraces) \le \range(\defsetoftraces')
  \end{align}
\end{lemma}

