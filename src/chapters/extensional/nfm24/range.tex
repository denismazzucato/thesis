
\subsection{The \rangename{} impact definition}[The Range Impact Definition]
\labsec{range}

%
The quantity $\range\in\setof\pairofstates\to\Rposplus$ determines the
length of the range of output values from all the possible variations in the input variable $\definputvariable\in\inputvariables$.
%
This definition employs the auxiliary function $\length\in\setof\valuesinf\to\valuesposplus$, defined as follows:
 $\length(X) \defeq \sup\;{X} - \inf\;{X}$ if $X\neq\emptyset$, where $\sup$ and $\inf$ are the supremum and infimum operators, while $\length(X)\defeq0$ otherwise.

\begin{example}[Landing Alarm System]
  \label{ex:range}
  We revisit again the example of the landing alarm system.
  Assuming $\definputvariable=\x$, $\tuple{-4}{1}$ is the first input configuration to be explored, we collect all traces that are
  starting from an input configuration that is a variation of $\tuple{-4}{1}$.
  As before, we obtain $\setdef{
    \retrieveoutput\defseq(\z)
  }{
    \defseq \in \defsetoftraces \land
      \retrieveinput{\defseq}(\y) = 1
  }=\{0,3\}$.
%
  Here, we apply the length function, hence $\length(\{0,3\})=3$.
  By doing so for all possible input configurations $\reducedstate$, we obtain $\rangename_{\x}(\tracesemantics)=2$.
  \reftab{range-count} illustrates the steps for both $\rangename_{\x}(\tracesemantics)$ and $\rangename_{\y}(\tracesemantics)$.
\end{example}

  Formally,
\begin{definition}[\rangename]\labdef{range}
  Given an input variable $\definputvariable\in\inputvariables$, and an output descriptor $\outputdesc$,
  the quantity $\range\in\dependencytype\to\Rposplus$ is defined as
  %
  \begin{align}
    \labeq{range}
    \range(\defsetoftraces) &\DefeQ \sup_{\definput\in\reducedstate}
      \length(\setdef{
        \reader(\retrieveoutput{\defseq})
      }{
        \defseq \in \defsetoftraces \land \retrieveinput{\defseq} \stateeq{\inputvariableswithouti} \definput
      })
  \end{align}
\end{definition}

This impact definition is monotone as well, in the amount of traces.
\begin{lemma}[\rangename{} is Monotonic]
  \lablemma{range-monotonic}
  For any $\defsetoftraces, \defsetoftraces'\in\dependencytype$, it holds that:
  \begin{align}
    \labeq{range-monotonic}
    \defsetoftraces \subseteq \defsetoftraces' \ImplieS \range(\defsetoftraces) \le \range(\defsetoftraces')
  \end{align}
\end{lemma}
