%

\chapter{Quantitative Input Data Usage}
\labch{quantitative-input-data-usage}

In this chapter we present the quantitative framework used to measure the contributions of the input variables to the outcome of a program. First, we introduce a leading example of a landing-risk alarm system to illustrate the concepts. Then, we present the framework and its properties. Finally, since the framework depends on the notion of impact, we show a few possible definitions of impact.

\onlysectioncommands{\x,\y,\z,\exampleinput,\highlight,\inputa,\inputax,\inputay,\outputa,\inputb,\inputbx,\inputby,\outputb,\inputc,\inputcx,\inputcy,\outputc,\inputd,\inputdx,\inputdy,\outputd,\inpute,\inputex,\inputey,\outpute,\inputf,\inputfx,\inputfy,\outputf,\tracea,\traceax,\traceay,\traceb,\tracebx,\traceby,\tracec,\tracecx,\tracecy,\traced,\tracedx,\tracedy,\tracee,\traceex,\traceey,\tracef,\tracefx,\tracefy,\labelrotationangle}
\section{Landing Alarm System}
\labsec{landing-alarm-system}
\newcommand*{\x}{\texttt{angle}}
\newcommand*{\y}{\texttt{speed}}
\newcommand*{\z}{\texttt{risk}}


\begin{lstlisting}[language=customPython,escapechar=\%,label={lst:landing-alarm-system},caption={Program for the landing-risk alarm system.}]
landing_coeff = abs(angle) + speed %\labline{compute-risk}%
if landing_coeff < 2 then %\labline{low-risk-cond}%
  risk = 0 %\labline{low-risk}%
else if landing_coeff > 5 then %\labline{high-risk-cond}%
  risk = 3 %\labline{high-risk}%
else %\labline{medium-risk-branch}%
  risk = floor(landing_coeff) - 2 %\labline{medium-risk}%
\end{lstlisting}


The goal of \refprog{landing-alarm-system}, referred to as \landingprogram, is to inform the pilot about the level of risk associated with the landing approach.
It takes two input variables, denoted as \x and \y, for the aircraft-airstrip alignment angle and the aircraft speed, respectively.
A value of 1 represents a good alignment while -4 a non-aligned angle, whereas 1, 2, 3 denote low, medium, and high speed.
A safer approach is indicated by lower speed.
The landing risk coefficient combines the absolute landing angle and speed.
The output variable $\z{}$ is the danger level with possible values $\{0, 1, 2, 3\}$, where 0 represents low danger and 3 high danger.

\begin{marginfigure}
\centering
\begin{tikzpicture}
  % Grid
  \draw[help lines, color=gray!30, dashed] (0,0) grid (2.9,3.9);
  % x-axis
  \draw[->,ultra thick] (0,0)--(3,0) node[right]{\x};
  % y-axis
  \draw[->,ultra thick] (0,0)--(0,4) node[above]{\y};
  % x-axis ticks
  \draw (0+1,0.1) -- (0+1,-0.1) node[below] {-4};
  \draw (1+1,0.1) -- (1+1,-0.1) node[below] {1};
  % y-axis ticks
  \foreach \y in {1,2,3}
  \draw (0.1,\y) -- (-0.1,\y) node[left] {\y};
  % Nodes
  \fill[color=seabornRed]   (0+1,0+1) circle[radius=2pt];
  \node[above right] at (0+1,0+1) {$3$};
  \fill[color=seabornRed]   (0+1,1+1) circle[radius=2pt];
  \node[above right] at (0+1,1+1) {$3$};
  \fill[color=seabornRed]   (0+1,2+1) circle[radius=2pt];
  \node[above right] at (0+1,2+1) {$3$};
  \fill[color=seabornGreen] (1+1,0+1) circle[radius=2pt];
  \node[above right] at (1+1,0+1) {$0$};
  \fill[color=seabornYellow] (1+1,1+1) circle[radius=2pt];
  \node[above right] at (1+1,1+1) {$1$};
  \fill[color=seabornOrange]    (1+1,2+1) circle[radius=2pt];
  \node[above right] at (1+1,2+1) {$2$};
\end{tikzpicture}
\caption{Input space composition of \refprog{landing-alarm-system}.}
\labfig{input-space-composition}
\end{marginfigure}

\reffig{input-space-composition} shows the input space composition of this system, where the label near each input represents the degree of risk assigned to the corresponding input configuration.
It is easy to note that a nonaligned angle of approach corresponds to a considerably higher level of risk, whereas the risk with a correct angle depends mostly on the aircraft speed.
Our goal is to develop a static analysis capable of quantifying the contribution of each input variable to the computation of the output variable $\z{}$.


\newcommand{\exampleinput}[1][\defaultprogramexampleletter]{\textsc{Input}_{#1}}

We propose two impact definitions which, from value variations of the input variable under consideration, respectively focus on
\textit{the number of} resulting reachable outputs, and \textit{the distance of} extreme reachable outputs.
The column $\exampleinput[\landingprogram]$ in \reftab{outcome-count} shows all the possible input configurations $\tuple{angle}{speed}$ for the program $\landingprogram$.
For each input configuration, column \textsc{Relevant Traces} groups together the program traces resulting from value variation of the input variable of interest (in column $\textsc{variable}$), and column $\textsc{Outputs}$ collects the set of all reachable outputs.

\newcommand{\highlight}[1]{\textcolor{seabornBlue}{#1}}
\newcommand{\inputa}{\tuple{-4}{1}}
\newcommand{\inputax}{\tuple{\highlight{-4}}{1}}
\newcommand{\inputay}{\tuple{-4}{\highlight{1}}}
\newcommand{\outputa}{\langle \outputvaluea\rangle} \newcommand{\outputvaluea}{3}
\newcommand{\inputb}{\tuple{-4}{2}}
\newcommand{\inputbx}{\tuple{\highlight{-4}}{2}}
\newcommand{\inputby}{\tuple{-4}{\highlight{2}}}
\newcommand{\outputb}{\langle \outputvalueb\rangle} \newcommand{\outputvalueb}{3}
\newcommand{\inputc}{\tuple{-4}{3}}
\newcommand{\inputcx}{\tuple{\highlight{-4}}{3}}
\newcommand{\inputcy}{\tuple{-4}{\highlight{3}}}
\newcommand{\outputc}{\langle \outputvaluec\rangle} \newcommand{\outputvaluec}{3}
\newcommand{\inputd}{\tuple{ 1}{1}}
\newcommand{\inputdx}{\tuple{\highlight{ 1}}{1}}
\newcommand{\inputdy}{\tuple{ 1}{\highlight{1}}}
\newcommand{\outputd}{\langle \outputvalued\rangle} \newcommand{\outputvalued}{0}
\newcommand{\inpute}{\tuple{ 1}{2}}
\newcommand{\inputex}{\tuple{\highlight{ 1}}{2}}
\newcommand{\inputey}{\tuple{ 1}{\highlight{2}}}
\newcommand{\outpute}{\langle \outputvaluee\rangle} \newcommand{\outputvaluee}{1}
\newcommand{\inputf}{\tuple{ 1}{3}}
\newcommand{\inputfx}{\tuple{\highlight{ 1}}{3}}
\newcommand{\inputfy}{\tuple{ 1}{\highlight{3}}}
\newcommand{\outputf}{\langle \outputvaluef\rangle} \newcommand{\outputvaluef}{2}
\newcommand{\tracea}{\inputa\to\outputa}
\newcommand{\traceax}{\inputax\to\outputa}
\newcommand{\traceay}{\inputay\to\outputa}
\newcommand{\traceb}{\inputb\to\outputb}
\newcommand{\tracebx}{\inputbx\to\outputb}
\newcommand{\traceby}{\inputby\to\outputb}
\newcommand{\tracec}{\inputc\to\outputc}
\newcommand{\tracecx}{\inputcx\to\outputc}
\newcommand{\tracecy}{\inputcy\to\outputc}
\newcommand{\traced}{\inputd\to\outputd}
\newcommand{\tracedx}{\inputdx\to\outputd}
\newcommand{\tracedy}{\inputdy\to\outputd}
\newcommand{\tracee}{\inpute\to\outpute}
\newcommand{\traceex}{\inputex\to\outpute}
\newcommand{\traceey}{\inputey\to\outpute}
\newcommand{\tracef}{\inputf\to\outputf}
\newcommand{\tracefx}{\inputfx\to\outputf}
\newcommand{\tracefy}{\inputfy\to\outputf}

\subsection{First Impact Definition {\normalfont(\texorpdfstring{$\outcomesname$}{Outcomes})}}[Outcomes]
%
The first impact definition that we consider is
 $\outcomes(\defprogram)$, %(derived from $\outcomesentropy$),
where $\definputvariable$ is the input variable of interest and $\defprogram$ the program under analysis. Intuitively (the formal definition is given in \refsec{quantitative-input-feature-usage}), $\outcomes$ returns the maximum number of outputs that are reachable from value variations of the input variable $\definputvariable$.
For the program
$\landingprogram$, the result is shown in column $\outcomesname(\landingprogram)$~of \reftab{outcome-count}:
we obtain $\outcomesname_\x(\landingprogram)=2$ and $\outcomesname_\y(\landingprogram)=3$.

\newcommand{\labelrotationangle}{30}
\begin{table*}[t]
  \centering
  \caption{Impact of for $\outcomesname(\landingprogram)$  and $\rangename(\landingprogram)$ definitions for both $\x$ and $\y$ variables. Computational features are \highlight{highlighted in blue}.}
  \label{tab:outcome-count}
  \begin{tabular}{c|c|c|c|c|c}
  \rotatebox[origin=c]{\labelrotationangle}{\textsc{Variable}}~ & ~\rotatebox[origin=c]{\labelrotationangle}{$\exampleinput[\landingprogram]$}~ & ~\textsc{Relevant Traces}~ & ~\rotatebox[origin=c]{\labelrotationangle}{\textsc{Outputs}}~ & ~\rotatebox[origin=c]{\labelrotationangle}{$\outcomesname$}~ & ~\rotatebox[origin=c]{\labelrotationangle}{$\rangename$}~ \\
  \toprule
  \multirow{6}{*}{\x}
   & $\inputax$ & $\traceax, \tracedx$ & $\{\outputvaluea,\outputvalued\}$ & \multirow{6}{*}{$2$} & \multirow{6}{*}{$3$} \\
  \cline{2-4}
   & $\inputbx$ & $\tracebx, \traceex$ & $\{\outputvalueb,\outputvaluee\}$ & & \\
  \cline{2-4}
   & $\inputcx$ & $\tracecx, \tracefx$ & $\{\outputvaluec,\outputvaluef\}$ & & \\
   \cline{2-4}
   & $\inputdx$ & $\tracedx, \traceax$ & $\{\outputvalued,\outputvaluea\}$ & & \\
   \cline{2-4}
   & $\inputex$ & $\traceex, \tracebx$ & $\{\outputvaluee,\outputvalueb\}$ & & \\
   \cline{2-4}
   & $\inputfx$ & $\tracefx, \tracecx$ & $\{\outputvaluef,\outputvaluec\}$ & & \\
  \midrule
  \multirow{12}{*}{\y}
   & \multirow{2}{*}{$\inputay$} & $\traceay, \traceby,$ & \multirow{2}{*}{$\{\outputvaluea\}$} & \multirow{12}{*}{$3$} & \multirow{12}{*}{$2$} \\
   & & $\tracecy$ & & & \\
  \cline{2-4}
   & \multirow{2}{*}{$\inputby$} & $\traceay, \traceby,$ & \multirow{2}{*}{$\{\outputvaluea\}$} & & \\
   & & $\tracecy$ & & & \\
  \cline{2-4}
   & \multirow{2}{*}{$\inputcy$} & $\traceay, \traceby,$ & \multirow{2}{*}{$\{\outputvaluea\}$} & & \\
   & & $\tracecy$ & & & \\
   \cline{2-4}
   & \multirow{2}{*}{$\inputdy$} & $\tracedy, \traceey,$ & \multirow{2}{*}{$\{\outputvalued,\outputvaluee, \outputvaluef\}$} & & \\
   & & $\tracefy$ & & & \\
   \cline{2-4}
   & \multirow{2}{*}{$\inputey$} & $\tracedy, \traceey,$ & \multirow{2}{*}{$\{\outputvalued,\outputvaluee, \outputvaluef\}$} & & \\
   & & $\tracefy$ & & & \\
   \cline{2-4}
   & \multirow{2}{*}{$\inputfy$} & $\tracedy, \traceey,$ & \multirow{2}{*}{$\{\outputvalued,\outputvaluee, \outputvaluef\}$} & & \\
   & & $\tracefy$ & & \\
   \bottomrule
  \end{tabular}
\end{table*}
The conclusion is that $\y$ has a greater influence than $\x$ on the output of the program.

\subsection{Second Impact Definition {\normalfont(\texorpdfstring{$\rangename$}{Range})}}[Range]
%
The second impact definition is $\rangename_\definputvariable$, which yields the maximum difference between the maximum and the minimum outputs that are reachable from value variations of the input variable $\definputvariable$.
The result for program $\landingprogram$ is shown in column $\rangename(\landingprogram)$~of \reftab{outcome-count}:
%Following again \reftab{outcome-count}, we can see that
the range of reachable outputs from variations of $\x$ is, at most, the interval $[0, 3]$, with a length of 3. Instead, the range of reachable outputs from variations of $\y$ is, at most, the interval $[0, 2]$, with a length of 2. Therefore, we obtain $\rangename_\x(\landingprogram)=3$ and $\rangename_\y(\landingprogram)=2$.
In other words, varying the angle of approach might drastically alter the landing risk, whereas the speed has less influence.
%
This is in contrast to the conclusion of \outcomesname{} where $\y$ has a greater impact than $\x$.
Although it may seem counterintuitive at first, the difference between the two impact instances is due to the different program traits they explore.
$\rangename$ quantifies over the variance in the extreme values of the set of output values, while $\outcomesname$ quantifies over the variance in the number of unique output values.
Consequently, changes in $\x$ yield a bigger variation in the degree of risk compared to $\y$, while changes in $\y$ reach far more risk levels compared to $\x$.
%
Note that, the impact definitions presented above are not computationally practical as they rely on a complete enumeration of all possible input configurations.
% Note that, enumerating all possible input configurations is not computationally practical.
Specifically, when dealing with more complex input space compositions, this approach is highly inefficient or even infeasible (as in the case of continuous input spaces).
As a consequence, our approach is based on an abstraction of input-output relations, which allows us to automatically infer a sound upper bound on the program's impact.

\subsection{Abstract Analysis}

The analysis starts with a set of output abstractions called \textit{output buckets}.
A bucket is an abstract element representing a set of output states.
While this abstraction may limit the ability to precisely reason about the impact of output values within the same bucket, it permits automatic reasoning across different buckets.
Afterwards, an abstract interpretation-based static analyzer propagates each output bucket backward through the program under consideration.
The analyzer returns an abstract element for each output bucket, representing an over-approximation of the set of input configurations that lead to the output values inside the starting bucket.
This result contains also spurious input configurations that may not lead to a value inside the output bucket.
Based on the chosen impact definition \impactwrappername{} (e.g., \rangename{} or \outcomesname), we perform computations and comparisons on the abstract elements returned by the analysis to obtain an upper bound $\defbound'$. This upper bound is sound by construction of the theoretical framework, meaning that if $\defbound$ is the real (concrete) impact quantity obtained by \impactwrappername, then $\defbound\le\defbound'$.

The precision of our analysis is mostly affected by the choice of output buckets and the approximation induced by the backward analysis \denis{(as outlined by the use cases shown in~\\Section{experiments} and~\\Appendixfullexperimentaloverview)}

\section{Quantitative Input Data Usage}
\labsec{quantitative-input-data-usage}
\newcommand*{\x}{\texttt{angle}}
\newcommand*{\y}{\texttt{speed}}
\newcommand*{\z}{\texttt{risk}}
\newcommand*{\lc}{\texttt{landing\_coeff}}

In this section we present some preliminaries on program computations, then we introduce our quantitative framework with the formal definitions of \rangename{} and \outcomesname.


Our goal is to quantify the impact of a specific input variable on the computation of the program.
To this end, we introduce the notion of impact, denoted by the function $\impactwrapper\in\setof\pairofstates\to\valuesposplus$, which maps program semantics to a non-negative domain of quantities, where $\definputvariable$ represents the input variable of interest in the program under analysis.

We implicitly assume the use of an \textit{output descriptor} $\outputdesc$ to determine the desired output of a program by observations on program states.

Specifically, $\reader\in \stateandbottom \to \valuesinf$ selects the output of interest from a given state and returns its corresponding value\sidenote{The option of returning $\pm\infty$ from the output descriptor is to deal with infinite traces, which do not have a final state ($\retrieveoutput\defseq = \statebottom$ for any $\defseq \in \infinitesequences$).}.
Additionally, $\filter\subseteq\valuesinf$ filters output states and selectively engages a subset of the potential outcomes.
Through this filtering mechanism, undesired outcomes are directly excluded, and a numerical value is ensured.

\begin{definition}[Output Descriptor]
  \labdef{outputdesc}
  Given $\reader\in\stateandbottom\to\valuesinf$ and $\filter\subseteq\valuesinf$, the tuple $\outputdesc$ is called an \textup{output descriptor}.
\end{definition}

The above output characterization $\outputdesc$ is generic enough to cover plenty of use cases.
We leverage this output descriptor to provide the end user of the framework the flexibility to choose the interpretation and meaning of program outputs, without establishing it beforehand.

\begin{marginlisting}
\begin{lstlisting}[language=customPython]
landing_coeff =
  abs(angle) + speed
if landing_coeff < 2:
  risk = 0
else if landing_coeff > 5:
  risk = 3
else:
  risk =
    floor(landing_coeff) - 2
\end{lstlisting}
\end{marginlisting}

\begin{example}
  Consider the~\refprog{landing-alarm-system} for the landing alarm system with program states $\state=\setdef{\langle a, b, c, d \rangle}{a\in\{-4,1\}\land b\in\{1,2,3\}\land c\in\N \land d\in\{0,1,2,3\}}$, where $a$ is the value of $\x$, $b$ of $\y$, $c$ of $\lc$, and $d$ of $\z$.
  Here, we abuse the notation and use $\state$ as set of tuples instead of a map between variables and values, the two views are equivalent.
  The output descriptor is instantiated with
  \[
  \reader(x) \DefeQ \begin{cases}
    d & \text{if } x = \langle a, b, c, d \rangle \\
    +\infty & \text{otherwise}
  \end{cases}
  \]
  and $\filter=\{0,1,2,3\}$ filters $+\infty$ from the possible outputs.
  In other words, we are interested in the value of $\z$ for terminating traces.

  However, the end-user of the analysis may be interested in only a subset of the possible outcomes of the program.
  For instance, only about the risk levels in $\{0, 1, 2\}$, forgetting about the value $3$.
  It is crucial that our impact definitions remain sound to the user assumption on post-conditions, even when it is under-approximating the exact one.
  Thus, the filter specifies this information by $\filter=\{0, 1, 2\}$, which is a subset of all the possible values of the output variable \z.
\end{example}

\begin{example}
  For other contexts, rather than considering a single output variable one may be interested in a custom operation.
  % For example, in the context of neural network classifiers, the output of the program is the index of the output variable with highest value.
  For example, the output of a neural network classifier is the index of the output neuron holding the highest value.
  Hence, for a network with $n+1$ output neurons, we could instantiate
  \[
    \reader(x_0,\dots,x_{w+n-1},x_{w+n})=\argmax_{0\le j\le n} x_{w+j}
  \] where the function $\argmax_j X_j$ returns the \textit{argument} $j$ of the value holding the \textit{maximum} among the indexed family $X_j$.
  The filter $\filter$ could be the set of all indices $\{0,\dots,n\}$, hence permitting all possible outcomes from the reader $\reader$.
\end{example}



We can now define our property of interest, the \textit{$\defbound$-bounded impact property} $\bounded$.
By extension, $\bounded$ is the set of trace semantics such that the impact of the set of traces allowed by the output descriptor $\outputdesc$, \wrt{} the input variable $\definputvariable$, is below the threshold $\defbound\in\valuesposplus$. Formally,
\begin{align}\labeq{eq:bounded}
  \BOUNDED \DefeQ \setdef
  {\tracesemanticsnoparam \in \setof\finiteinfinitesequences}
  {\impactwrapper(\setdef{\defseq\in\tracesemanticsnoparam}{\reader(\retrieveoutput{\defseq})\in\filter}) \impactsubseteq \defbound}
\end{align}
where $\impactwrapper$ is a parameter of the property $\bounded$ and returns the quantity computed on the given set of traces.
%
Following the definition of $\bounded$, our validation framework, \refeq{qualitative-soundness}, is instantiated as
%
\begin{align}
  \labeq{bounded-soundness}
  \defprogram \satisfies \BOUNDED \IfF \collectingsemantics \subseteq \BOUNDED
\end{align}
%
We are interested in properties $\bounded$ that are \textit{extensional}, namely, properties based on the observation of input-output relations of program states.
This means that, if $\tracesemantics\in\bounded$ and $\tracesemanticsnoparam\semanticsof{\defprogram'}$ consider the same set of input-output observations of $\tracesemantics$, then $\tracesemanticsnoparam\semanticsof{\defprogram'}\in\bounded$.
As a consequence, the impact is not affected by intermediate states,
$\impactwrapper(\tracesemantics)=\impactwrapper(\tracesemanticsnoparam\semanticsof{\defprogram'})$.
% Therefore, we derive the \textit{dependency semantics} $\dependencysemanticsnoparam$~\citep{Urban2020} as an abstraction of the collecting semantics $\collectingsemanticsnoparam$ removing intermediate computational states.
Furthermore, we require $\impactwrapper$ to be monotonic, \ie, for any $X, Y\in \setof\finiteinfinitesequences$, it holds that $X \subseteq Y$ if and only if $\impactwrapper(X) \le \impactwrapper(Y)$.
Intuitively, this ensures that an impact applied to an over-approximation of the program semantics can only produce a higher quantity, enabling the definition of a sound terminating static analysis.
Next, we formalize the already introduced impact metrics \rangename{} and \outcomesname.

\subsection{The \outcomesname{} impact definition}
\labsec{outcomes}
%
Formally $\outcomes\in\setof\pairofstates\to\Nplus$ counts the number of different output values reachable by varying the input variable $\definputvariable\in\inputvariables$.
Intuitively,
for any possible input configuration $\definputvariable\in\reducedstate$, we gather the set $\defsetoftraces\in\setof\pairofstates$ of all input-output state dependencies with an input configuration that is a variation of $\definputvariable$ on the input variable $\definputvariable$, \ie, $\setdef{
  \inputoutputtuple{\defstate}\in \defsetoftraces
}{
  \retrieveinput{\defstate} \stateeq{\inputvariableswithouti} \definputvariable
}$.
Then, $\outcomes$ is the maximal cardinality of the output values $\setdef{
  \reader(\retrieveoutput\defstate)
}{
  \inputoutputtuple{\defstate} \in \defsetoftraces \land
    \retrieveinput{\defstate} \stateeq{\inputvariableswithouti} \definputvariable
}$.
%
Formally,
%
\begin{align}
  \label{eq:outcomes}
\outcomes(\defsetoftraces) &\DefeQ
  \sup_{\definputvariable\in\reducedstate}
  \cardinality{\seTDef{\reader(\retrieveoutput{\defstate})}{\inputoutputtuple\defstate\in\defsetoftraces\land\retrieveinput\defstate \stateeq{\inputvariableswithouti} \definputvariable
  }
  }
\end{align}
where $\cardinality{\cdot}$ is the cardinality operator,
and $\sup(X)$ is the supremum operator, \ie, the smallest $q$ such that $q\ge x$ for all $x\in X$.
From the definition above, it is easy to note that $\outcomes(\defsetoftraces)$ is monotone in the amount of dependencies $\defsetoftraces$. That is, the more dependencies in input, the higher the impact as only more dependencies can satisfy the condition of \refeq{outcomes}, \cf~$\retrieveinput\defstate \stateeq{\inputvariableswithouti} \definputvariable$, and hence increase the number of outcomes.

\subsection{The \rangename{} impact definition}
\labsec{range}
%
The quantity $\range\in\setof\pairofstates\to\Rposplus$ determines the
length of the range of output values from all the possible variations in the input variable $\definputvariable\in\inputvariables$.
%
This definition employs the auxiliary function $\length\in\setof\valuesinf\to\valuesposplus$, defined as follows:
 $\length(X) \defeq \sup\;{X} - \inf\;{X}$ if $X\neq\emptyset$, where $\sup$ and $\inf$ are the supremum and infimum operators, while $\length(X)\defeq0$ otherwise.
  Formally,
  \begin{align}
    \label{eq:range}
    \range(\defsetoftraces) &\DefeQ \sup_{\definput\in\reducedstate}
      \length(\seTDef{
        \reader(\retrieveoutput{\defstate})
      }{
        \inputoutputtuple\defstate \in \defsetoftraces \land \retrieveinput{\defstate} \stateeq{\inputvariableswithouti} \definput
      })
  \end{align}

Similarly to $\outcomesname$, $\rangename$ is monotone in the amount of dependencies $\defsetoftraces$.

\section{A Static Analysis for Quantitative Input Data Usage}
\labsec{static-analysis}

%
In this section, we introduce a sound computable static analysis to determine an upper bound on the impact of an input variable $\definputvariable$.
The soundness of the approach leverages two elements: $(1)$ an underlying abstract semantics $\backwardsemanticsnoparam$ to compute an over-approximation of the dependency semantics $\dependencysemanticsnoparam$; and $(2)$ a sound computable implementation of $\impactwrapper$, written $\impactinstance$, used in the property $\bounded$.

To quantify the usage of an input variable, we need to determine the input configurations leading to specific output values.
As our impact definitions $\outcomes$ and $\range$ measure over the different output values (i.e., $\reader(\retrieveoutput{\defstate})$) our underlying abstract semantics will be a \emph{backward} (co-)reachability semantics starting from \emph{disjoint} abstract post-conditions, over-approximating the (concrete) output values of the dependency semantics.
Specifically, we abstract the concrete output values with an indexed set $\buckets\in\vectorbuckets$ of $n$ disjoint \textit{output buckets}, where $\abstractdomainlattice$ is an abstract state domain with concretization function  $\abstractdomainconcretization\in\abstractdomain\to\setof\stateandbottom$. The choice of these output buckets is essential for obtaining a precise and meaningful analysis result.

For each output bucket $\bucket\in\abstractdomain$ where $j \le n$, our analysis computes an over-approximation of the dependency semantics restricted to the input configurations leading to $\abstractdomainconcretization(\bucket)$.
More formally, the reduction of the dependency semantics $\dependencysemanticsnoparam$ to the dependencies with final states in $X$ is defined as:
\[\reduce[\dependencysemanticsnoparam]{X} \DefeQ \setdef{\inputoutputtuple{\defstate}\in\dependencysemanticsnoparam}{\retrieveoutput{\defstate}\in X}\]
%
Our static analysis is parametrized by an underlying backward abstract family\sidenote{A family of semantics is a set of program semantics parametrized by an initialization.}
of semantics $\backwardsemanticsnoparam\in\backwardtype$ which computes the backward semantics $\backwardsemanticsnoparam(\bucket)$ from a given output bucket $\bucket\in\abstractdomain$.
The concretization function $\backwardconcretization\in(\backwardtype)\to\abstractdomain\to\setof\pairofstates$ employs %the abstract concretization
$\abstractdomainconcretization$ to restore all possible input-output dependencies, formally:
\[\backwardconcretization(\backwardsemanticsnoparam)\bucket \DefeQ \setdef{\inputoutputtuple{\defstate}}{\retrieveinput{\defstate}\in\abstractdomainconcretization(\backwardsemanticsnoparam\bucket)\land\retrieveoutput{\defstate}\in\abstractdomainconcretization(\bucket)}\]
We can thus define the soundness condition for the backward semantics with respect to the reduction of the dependency semantics.

\begin{definition}[Sound Over-Approximation for \texorpdfstring{$\backwardsemanticsnoparam$}{the backward semantics}]\labdef{sound-over-approximation}
  For all programs $\defprogram$, and output bucket $\bucket\in\abstractdomain$, the family of semantics $\backwardsemanticsnoparam$ is a \textup{sound over-approximation} of the dependency semantics $\dependencysemanticsnoparam$ reduced with  $\abstractdomainconcretization(\bucket)$, when it holds that:
  \[\reduceddependencysemantics \SubseteQ \backwardconcretization(\backwardsemantics)\bucket\]
\end{definition}

We define
$\multisemanticsnoparam\in\multitype$ as the backward semantics repeated on a set of output buckets $\buckets\in\vectorbuckets$, that is:
\[\multisemantics\buckets \DefeQ (\backwardsemantics\bucket)_{j\le n}\]
Again, the concretization function $\multiconcretization\in(\multitype)\to\vectorbuckets\to\setof\pairofstates$ employs the abstract concretization $\abstractdomainconcretization$ to restore all possible input-output dependencies over all the output buckets, formally:
\[\multiconcretization(\multisemantics)\buckets \DefeQ \bigsetjoin_{j\le n} \setdef{\inputoutputtuple{\defstate}}{\retrieveinput{\defstate}\in\abstractdomainconcretization((\multisemantics\buckets)_j)\land\retrieveoutput{\defstate}\in\abstractdomainconcretization(\bucket)}\]

\begin{lemma}[Sound Over-Approximation for \texorpdfstring{$\multisemanticsnoparam$}{the multi-bucket semantics}]\lablemma{sound-over-approximation-multi-bucket}
  For all programs $\defprogram$, output buckets $\buckets\in\vectorbuckets$, and a family of semantics $\backwardsemanticsnoparam$, the %multi-bucket family of
  semantics $\multisemanticsnoparam$ is a \textup{sound over-approximation} of the dependency semantics $\dependencysemanticsnoparam$ when reduced to $\bigsetjoin_{j\le n}\abstractdomainconcretization(\bucket)$:
  \[\reduce{\bigsetjoin_{j\le n}\abstractdomainconcretization(\bucket)} \SubseteQ \multiconcretization(\multisemantics)\buckets\]
\end{lemma}
\begin{proof}
\begin{align*}
  &\vphantom{=} \multiconcretization(\multisemanticsnoparam)\buckets
    && \text{by $\multiconcretization$} \\
  &= \bigsetjoin_{j \le n}\setdef{\inputoutputtuple{\defstate}}{\retrieveinput{\defstate}\in\abstractdomainconcretization((\multisemanticsnoparam(\buckets))_j)\land\retrieveoutput{\defstate}\in\abstractdomainconcretization(\bucket)}
    && \text{by $\multisemanticsnoparam$} \\
  &= \bigsetjoin_{j \le n} \setdef{\inputoutputtuple{\defstate}}{\retrieveinput{\defstate}\in\abstractdomainconcretization(((\backwardsemanticsnoparam(\bucket[t]))_{t\le n})_j)\land\retrieveoutput{\defstate}\in\abstractdomainconcretization(\bucket)}
  && \\
  &= \bigsetjoin_{j \le n} \setdef{\inputoutputtuple{\defstate}}{\retrieveinput{\defstate}\in\abstractdomainconcretization(\backwardsemanticsnoparam(\bucket))\land\retrieveoutput{\defstate}\in\abstractdomainconcretization(\bucket)}
  && \text{by $\backwardconcretization$} \\
  &= \bigsetjoin_{j \le n} \backwardconcretization(\backwardsemanticsnoparam)\bucket
\end{align*}
From \refdef{sound-over-approximation}, we obtain that $\foralldef{j \le n}{\reduceddependencysemanticsnoparam \subseteq \backwardconcretization(\backwardsemanticsnoparam(\bucket))}$.
Thus, by monotonicity of the union operator over set inclusion, it holds that $\bigsetjoin_{j\le n}\reduceddependencysemanticsnoparam \subseteq \bigsetjoin_{j\le n}\backwardconcretization(\backwardsemanticsnoparam(\bucket))$. We conclude by:
\begin{align*}
  \bigsetjoin_{j\le n}\reduceddependencysemanticsnoparam &= \bigsetjoin_{j\le n} \setdef{\inputoutputtuple{\defstate}\in\dependencysemanticsnoparam}{\retrieveoutput{\defstate}\in\abstractdomainconcretization(\bucket)}
    && \text{by $\reducenoparam{X}$} \\
  &= \setdef{\inputoutputtuple{\defstate}\in\dependencysemanticsnoparam}{\retrieveoutput{\defstate}\in\bigsetjoin_{j \le n}\abstractdomainconcretization(\bucket)}
    && \text{by set definition} \\
  &= \reducenoparam{\bigsetjoin_{j \le n}\abstractdomainconcretization(\bucket)}
    && \text{by $\reducenoparam{X}$}
\end{align*}
\end{proof}

Whenever the output buckets \textit{cover} the subset of potential outcomes $\filter$, $\multisemanticsnoparam$ is a sound over-approximation of $\dependencysemanticsnoparam$.
The concept of covering for output buckets ensures that no potential final state is missed from the analysis.

\begin{definition}[Covering]\label{def:covering}
  We say that the output buckets $\buckets\in\vectorbuckets$ \textit{cover} the subset of potential outcomes whenever it holds that:
  \[\filter\subseteq \setdef{\reader(\retrieveoutput\defstate)}{\retrieveoutput\defstate\in\bigsetjoin_{j\le n}\abstractdomainconcretization(\bucket)}\]
\end{definition}

% \newcommand{\resultofbucketj}{\hiX_{j}}
% \newcommand{\resultofbucketjk}{\hiX_{j,k}}
% \newcommand{\resultofbucket}{\hiX}

Next, we expect a sound implementation $\impactinstance\in\pair\vectorbuckets\vectorbuckets\to\valuesinf$ to return a bound on the impact which is always higher than the concrete counterpart $\impactwrapper$.

\begin{definition}[Sound Implementation]\labdef{sound-implementation}
  For all output buckets $\buckets$ and family of semantics $\backwardsemanticsnoparam$, $\impactinstance$ is a \textup{sound implementation} of $\impactwrappername$, whenever it holds that:
  \[
    \impactwrapper(
      \multiconcretization(\multisemantics)\buckets
    ) \LE \impactinstance(\multisemantics\buckets, \buckets)
  \]
\end{definition}

% \newcommand{\resultofproject}{\higher{Y}}
% \newcommand{\resultofprojectj}{\higher{Y}_{j}}
% \newcommand{\resultofprojectjk}{\higher{Y}_{j,k}}


The next result shows that our static analysis is sound when employed to verify the property of interest $\bounded$ for the program $\defprogram$.
That is, if %the computation of
$\impactinstance$ returns the bound $\defbound'$, and $\defbound'\le\defbound$, then the program $\defprogram$ satisfies the property $\bounded$, \cf{} $\defprogram \satisfies \bounded$.


\begin{theorem}[Soundness] \labthm{soundness}
  Let $\bounded$ be the property of interest we want to verify for the program $\defprogram$ and the input variable $\definputvariable\in\inputvariables$.
  Whenever,
  \begin{enumerate}[label=(\roman*)]
    \item \label{p:first} $\backwardsemanticsnoparam$ is sound with respect to $\dependencysemanticsnoparam$, \cf{} \refdef{sound-over-approximation},
    \item \label{p:second} $\buckets$ covers the subset of potential outcomes $\filter$, \cf{} \refdef{covering}, and
    \item \label{p:third} $\impactinstance$ is a sound implementation of $\impactwrapper$, \cf{} \refdef{sound-implementation},
  \end{enumerate}
  the following implication holds:
  \begin{align*}
    \impactinstance(\multisemantics\buckets, \buckets) = \defbound' \LanD \defbound' \le \defbound \ImplieS \defprogram \satisfies \bounded
  \end{align*}
\end{theorem}
\begin{proof}
  \begin{align*}
    k &\ge k' = \impactinstance(\multisemantics\buckets, \buckets)
      && \text{by hypothesis} \\
    &\ge \impactwrapper(\multiconcretization(\multisemantics\buckets))
      && \text{by \ref{p:third}}
  \end{align*}
  By \ref{p:first}, \ref{p:second}, and \reflemma{sound-over-approximation-multi-bucket} we obtain that $\dependencysemantics \subseteq \multiconcretization(\multisemantics\buckets)$. Thus, from monotonicity we obtain
  \[\impactwrapper(\dependencysemantics) \LE \impactwrapper(\multiconcretization(\multisemantics\buckets))\]

  It follows that $\impactwrapper(\dependencysemantics) \le k'$.
  Therefore, by definition of $\bounded$, \cf{} \refdef{bounded}, it holds that $\dependencysemantics \in \bounded$.
  From the definition of the collecting semantics $\collectingsemantics$, it follows that $\{\dependencysemantics\} \subseteq \bounded$.
  We conclude that $\defprogram \satisfies \bounded$ by \refdef{validation} applied to $\bounded$ as $\defproperty$.
\end{proof}

Finally,
we define $\abstractrange$ and $\abstractoutcomes$
as possible implementations for $\range$ and $\outcomes$, respectively.
%
We assume the underlying abstract state domain $\abstractdomain$ is equipped with an
operator $\abstractdomainproject\in\abstractdomain\to\abstractdomain$
to project away the input variable $\definputvariable$.
For example, in the context of the interval domain, where each input variable is related to a possibly unbounded lower and upper bound, $\abstractdomainproject(\langle\definputvariable \mapsto [1, 3], j \mapsto [2, 4]\rangle) = \langle \definputvariable \mapsto [-\infty, \infty], j \mapsto [2, 4] \rangle$
removes the constraints related to $\definputvariable$.
%
We assume a soundness condition on the project operator to ensure that $\abstractdomainproject(\defstate^\natural)$ represents all the concrete states result of perturbations on the variable $\definputvariable$ from a state represented by an abstract value $\defstate^\natural$.

\begin{definition}[Soundness of \texorpdfstring{$\abstractdomainproject$}{Project}]\labdef{soundness-project}
  Given an abstract value $\defstate^\natural\in\abstractdomain$, for all $\defstate \in \abstractdomainconcretization(\defstate^\natural)$, whenever it exists a state $\defstate'$ such that $\defstate \stateeq{\inputvariableswithouti} \defstate'$, then it holds that $\defstate' \in \abstractdomainconcretization(\abstractdomainproject(\defstate^\natural))$.
\end{definition}
The above condition ensures that no intersection is missed, potentially spurious ones are allowed by the abstraction.

The definition of $\abstractoutcomes$ first projects away the input variable $\definputvariable$ from all the given abstract values, then it collects all intersecting abstract values via the meet operator $\abstractdomainmeet$.
These intersections represent potential concrete input configurations where variations on the value of $\definputvariable$ lead to changes of program outcome, from a bucket to another.
We return the maximum number of abstract values that intersects after projections:
\begin{definition}[\texorpdfstring{$\abstractoutcomes$}{Abstract Outcomes}]\labdef{abstract-outcomes}
  We define $\abstractoutcomes\in\pair\vectorbuckets\vectorbuckets\to\valuesinf$ as:
  \begin{equation*}
  \abstractoutcomes(X^\natural, \buckets) \DefeQ \max~\setdef{\cardinalitynospaces{J}}{J \in \intersectallfunction((\abstractdomainproject(X^\natural_j))_{j\le\numberofbuckets})}
  \end{equation*}
\end{definition}
Note the use of $\max$ instead of $\sup$ as in the concrete counterpart (\refdef*{outcomes}) since the number of intersecting abstract values is bounded by $n$, the number of output buckets.
The function $\intersectallfunction$ takes as input an indexed set of abstract values and returns the set of indices of abstract values that intersect together, defined as follows:
\begin{gather*}
  \intersectallfunction(X^\natural\in\vectorbuckets) \DefeQ \\
  \setdef{J}{J \subseteq \N \land \forall j\le n, p\le n.~ j\in J \land p\in J \LanD X^\natural_j \abstractdomainmeet X^\natural_{p}}
\end{gather*}
Finding all the indices of intersecting abstract values is equivalent to find cliques in a graph, where each node represents an abstract value and an edge exists between two nodes if and only if the corresponding abstract values intersect.
Therefore, $\intersectallfunction$ can be efficiently implemented based on the graph algorithm by~\sidetextcite{Bron1973}.
%

In order to prove that the abstract impact $\abstractoutcomes$ is a sound over-approximation of the concrete impact $\outcomes$, we require the output buckets $\buckets$ to be \textit{compatible} with the output descriptor $\reader$.
Intuitively, compatibility ensures that the counting intersecting buckets in the abstract does not miss any concrete outcome.
%
\begin{definition}[Compatibility]\labdef{compatibility}
  Given the output buckets $\buckets\in\vectorbuckets$ and the output descriptor $\reader\in\stateandbottom\to\valuesinf$, we say that $\buckets$ is \textup{compatible} with $\reader$, whenever it holds:
  \[ \foralldef{\defstate_j \in \abstractdomainconcretization(\bucket), \defstate_p \in \abstractdomainconcretization(\bucket[p])}{\reader(\defstate_j) \neq \reader(\defstate_p)} \ImplieS \bucket \neq \bucket[p] \]
\end{definition}
%
Note that, $\outcomes$ is bounded by the number of buckets when the conditions of covering and compatibility hold for the output buckets.
\begin{lemma}[$\outcomes$ Upper Bound]\lablemma{outcomes-upper-bound}
  When the buckets $\buckets$ are compatible, \cf{} \refdef{compatibility}, and cover the subset of potential outcomes, \cf{} \refdef{covering}, it holds that
  $\outcomes(\dependencysemanticsnoparam) \le n$.
\end{lemma}
\begin{proof}
  We notice that $\outcomes(\dependencysemanticsnoparam) \le \cardinalitynospaces{\setdef{\reader(\retrieveoutput\defstate)}{\retrieveoutput\defstate\in\finalstatesdependency}}$ as the set of outputs for the dependency semantics is always bigger than any set of outputs.
  It is easy to note that the cardinality of $\setdef{\reader(\retrieveoutput\defstate)}{\retrieveoutput\defstate\in\finalstatesdependency}$ is upper bounded by $n$ since any two output states $\retrieveoutput\defstate, \retrieveoutput\defstate'$ that produce different output readings, \ie, $\reader(\retrieveoutput\defstate) \neq \reader(\retrieveoutput\defstate')$, belong to different buckets, \ie, $\retrieveoutput\defstate \in \abstractdomainconcretization(\bucket) \land \retrieveoutput\defstate' \in \abstractdomainconcretization(\bucket[p]) \land \bucket \neq \bucket[p]$ (by compatibility, \cf{} \refdef{compatibility}).
  Where the existence of the two buckets is guaranteed by covering (\cf{} \refdef{covering}).
  Therefore, there are at most $n$ different output readings.
\end{proof}

The next result shows that the abstract impact $\abstractoutcomes$ is a sound over-approximation of the concrete impact $\outcomes$.

\begin{lemma}[$\abstractoutcomes$ is a Sound Implementation of $\outcomes$]\lablemma{abstractoutcomes-is-sound}
  Let  $\definputvariable\in\variables$ the input variable of interest, $\abstractdomain$ the abstract domain, $\backwardsemanticsnoparam$ the family of semantics, and $\buckets\in\vectorbuckets$ the starting output buckets.
  Whenever the following conditions hold:
  \begin{enumerate}[label=(\roman*)]
    \item \label{proof:a} $\backwardsemanticsnoparam$ is sound with respect to $\dependencysemanticsnoparam$, \cf{} \refdef{sound-over-approximation},
    \item \label{proof:b2} $\buckets$ covers the subset of potential outcomes, \cf{} \refdef{covering},
    \item \label{proof:b1} $\buckets$ is compatible with $\reader$, \cf{} \refdef{compatibility}, and
    \item \label{proof:d} $\abstractdomainproject$ is sound, \cf{} \refdef{soundness-project};
  \end{enumerate}
  then, $\abstractoutcomes$ is a sound implementation of $\outcomes$.
\end{lemma}
\begin{proof}
  From \ref{proof:a}, \ref{proof:b2}, and the fact that $\outcomes$ is monotone, we obtain that $\outcomes(\dependencysemanticsnoparam) \le \outcomes(\multiconcretization(\multisemanticsnoparam)\buckets)$.
  By definitions of $\abstractoutcomes$, \cf{} \refdef{abstract-outcomes}, and $\outcomes$, \cf{} \refdef{outcomes}, we need to show that:
  \begin{gather*}
    \outcomes(\dependencysemanticsnoparam) = \sup_{\definputvariable\in\reducedstate}
    \cardinality{\setdef{\reader(\retrieveoutput{\defstate})}{\inputoutputtuple\defstate\in\dependencysemanticsnoparam\land\retrieveinput\defstate \stateeq{\inputvariableswithouti} \definputvariable}} \\
    \le \\
    \abstractoutcomes(X^\natural, \buckets) = \max~\setdef{\cardinalitynospaces{J}}{J \in \intersectallfunction((\abstractdomainproject(X^\natural_j))_{j\le\numberofbuckets})}
  \end{gather*}
  where $X^\natural = \multisemanticsnoparam(\buckets)$.
%
  First, from \ref{proof:b1} and \reflemma{outcomes-upper-bound} we know that $\outcomes$ is limited by the number of buckets $n$.
  Notably, $\outcomes$ cannot be unbounded, but it has to be a number, at most $n$. Thus, it exists an initial state $\overline\defstate\in\reducedstate$ such that $\outcomes(\dependencysemanticsnoparam) = \cardinalitynospaces{\defsetoftraces{\overline\defstate}}$, where $\defsetoftraces{\overline\defstate} = \setdef{\reader(\retrieveoutput{\defstate})}{\inputoutputtuple\defstate\in\dependencysemanticsnoparam\land\retrieveinput\defstate \stateeq{\inputvariableswithouti} \overline\defstate}$.
  We conclude in case this cardinality is $0$ as anything returned by $\abstractoutcomes$ would be greater. In the other case, by covering (\cf{} \ref{proof:b2}), for all dependencies in $\defsetoftraces{\overline\defstate}$
  it exists a bucket $\bucket$ such that $\retrieveoutput{\defstate'}\in\abstractdomainconcretization(\bucket)$.

  Furthermore, by compatibility (\cf{} \ref{proof:b1}), for any pair $\inputoutputtuple\defstate, \inputoutputtuple{\defstate'}\in\defsetoftraces{\overline\defstate}$ leading to two different outcomes $\reader(\retrieveoutput{\defstate}) \neq \reader(\retrieveoutput{\defstate'})$, we have two different buckets $\bucket, \bucket[p]$ such that $\retrieveoutput{\defstate}\in\abstractdomainconcretization(\bucket)$ and $\retrieveoutput{\defstate'}\in\abstractdomainconcretization(\bucket[p])$.
  Note that, by definition of $\defsetoftraces{\overline\defstate}$ it holds that $\retrieveinput\defstate \stateeq{\inputvariableswithouti} \overline{\defstate} \stateeq{\inputvariableswithouti} \retrieveinput{\defstate'}$, by transitivity $\retrieveinput\defstate \stateeq{\inputvariableswithouti} \retrieveinput{\defstate'}$.

  Let us call $\prefrombucket$ and $\prefrombucket[p]$ the corresponding abstract values from the backward analysis applied to the buckets $\bucket$ and $\bucket[p]$, respectively.
  From the fact that $\backwardsemanticsnoparam$ is sound with respect to $\dependencysemanticsnoparam$, \cf{} \ref{proof:a}, it holds that $\retrieveinput\defstate \in \abstractdomainconcretization(\prefrombucket)$ and $\retrieveinput{\defstate'} \in \abstractdomainconcretization(\prefrombucket[p])$.
  By the soundness condition of $\abstractdomainproject$, \cf{} \ref{proof:d}, we obtain that both states $\retrieveinput\defstate$ and $\retrieveinput\defstate'$ belong to each other projection, \ie, $\retrieveinput\defstate \in \abstractdomainconcretization(\abstractdomainproject(\prefrombucket))$ and $\retrieveinput{\defstate'} \in \abstractdomainconcretization(\abstractdomainproject(\prefrombucket[p]))$.

  Finally, the function \intersectallfunction{} applied to the projected preconditions $\prefrombucket$ and $\prefrombucket[p]$ finds an intersection between the indices $j$ and $p$ as $\abstractdomainproject(\prefrombucket) \abstractdomainmeet \abstractdomainproject(\prefrombucket[p])$ definitely holds since they share concrete states. Therefore, whenever it exists an intersection in the concrete, the two indices representing the respective precondition discovered by the backward analysis belong to the set $J$ in \refdef{abstract-outcomes}.
  As a consequence, the maximum cardinality of $J$ takes into account all the possible intersections in $\defsetoftraces{\overline\defstate}$, hence $\abstractoutcomes(X^\natural, \buckets) \ge \cardinalitynospaces{\defsetoftraces{\overline\defstate}}$.
\end{proof}

Similarly, we define $\abstractrange$ as the maximum length of the range of the extreme values of the buckets represented by intersecting abstract values after projections.
In such case, we assume $\abstractdomain$ is equipped with an additional abstract operator $\abstractdomainlength\in\abstractdomain\to\valuesposplus$, which returns the length of the given abstract element, otherwise $+\infty$ if the abstract element is unbounded or represents multiple variables.

\begin{definition}[\texorpdfstring{$\abstractrange$}{Abstract Range}]\labdef{abstract-range}
  We define $\abstractrange\in\pair\vectorbuckets\vectorbuckets\to\valuesposplus$ as:
  \begin{align*}
    \abstractrange(X^\natural, \buckets) \DefeQ& \max~\seTDef{\abstractdomainlength(K)}{K \in I} \\
    \text{where}~
    I ~=~& \seTDef{\abstractdomainjoin\seTDef{\bucket}{j\in J}}{J \in \intersectallfunction((\abstractdomainproject(X^\natural_j))_{j\le\numberofbuckets})}
  \end{align*}
\end{definition}

To prove that $\abstractrange$ is a sound implementation of $\range$, we require the following soundness condition on the abstract operator $\abstractdomainlength$ to ensure that the abstract length is always greater than the concrete one.

\begin{definition}[Soundness of \texorpdfstring{$\abstractdomainlength$}{Length}]\labdef{soundness-length}
  Given an abstract value $\defstate^\natural\in\abstractdomain$, it holds that:
  \[\abstractdomainlength(\defstate^\natural) \ge \length(\setdef{\reader(\defstate)}{\defstate\in\abstractdomainconcretization(\defstate^\natural)})\]
\end{definition}

As previously done, the next result shows that the abstract impact $\abstractrange$ is a sound over-approximation of the concrete impact $\range$, \cf{}\refdef*{range}.

\begin{lemma}[$\abstractrange$ is a Sound Implementation of $\range$]\lablemma{abstractrange-is-sound}
  Let  $\definputvariable\in\variables$ the input variable of interest, $\abstractdomain$ the abstract domain, $\backwardsemanticsnoparam$ the family of semantics, and $\buckets\in\vectorbuckets$ the starting output buckets.
  Whenever the following conditions hold:
  \begin{enumerate}[label=(\roman*)]
    \item $\backwardsemanticsnoparam$ is sound with respect to $\dependencysemanticsnoparam$, \cf{} \refdef{sound-over-approximation},
    \item $\buckets$ covers the subset of potential outcomes, \cf{} \refdef{covering},
    \item $\buckets$ is compatible with $\reader$, \cf{} \refdef{compatibility}, and
    \item $\abstractdomainproject$ is sound, \cf{} \refdef{soundness-project};
  \end{enumerate}
  then, $\abstractrange$ is a sound implementation of $\range$.
\end{lemma}

\section{Experimental Results}
\labsec{nfm24:experimental-results}
\newcommand*{\x}{\texttt{angle}}
\newcommand*{\y}{\texttt{speed}}
\newcommand*{\z}{\texttt{risk}}


The goal of this section is to highlight the potential of our static analysis for quantitative input data usage.
We implemented a proof-of-concept tool, called \impatto\sidenote{\rurl{github.com/denismazzucato/impatto}}, in Python 3 that employs the \interproc\sidenote{\rurl{github.com/jogiet/interproc}} abstract interpreter to perform the backward analysis.
Then, we exploited this tool to automatically derive a sound input data usage of six different use cases.
As each impact result must be interpreted with respect to what the program computes, we analyze each use case separately.

\subsection{Growth in a Time of Debt}
\labsec{rr}


Reinhart and Rogoff article ``Growth in a Time of Debt''~\sidecite{Reinhart2010} proposed a correlation between high levels of public debt and low economic growth,
and %. As a consequence, the article
was heavily cited to justify austerity measures around the world. %Notably,
One of the several errors discovered in the article is the incorrect usage of the input value relative to Norway's economic growth in 1964.
The data used in the article is publicly available but not the spreadsheet file. We reconstructed this simplified example based on
the technical critique by \sidetextcite{Herndon2014}, and an online discussion\sidenote{\rurl{economics.stackexchange.com/q/18553}}.
The~\refprog{rr}
computes the cross-country mean growth for the public debt-to-GDP $60-90\%$ category, key point to the article's conclusions.
The input data is the average growth rate for each country within this public dept-to-GDP category. The problem with this computation is that Norway has only one observation in such category, which alone could disrupt the mean computation among all the countries. Indeed, the year that Norway appears in the $60-90\%$ category achieved a growth rate of $10.2\%$, while the average growth rate for the other countries is $2.7\%$.
With such high rate, the mean growth rate raised to $3.4\%$, altering the article's conclusions.
We assume growth rate values between $-20\%$ and $20\%$ for all countries, consequentially, the output ranges are between these bounds as well. We instrumented the output buckets to cover the full output space in buckets of size $1$, \ie, $\setdef{t \le \texttt{avg} < t + 1}{-20 \le t \le 20}$.
%
\newcommand{\dg}{60}
\begin{table*}[t]
  \caption{Quantitative input usage for \refprog{rr} from the Reinhart and Rogoff's article.}
  \labtab{rr}
  \centering
  \begin{tabular}{c | ccccccccccc}
    \textsc{Impact} & \rotatebox{\dg}{\texttt{portugal1}} & \rotatebox{\dg}{\texttt{portugal2}} & \rotatebox{\dg}{\texttt{portugal3}} & \rotatebox{\dg}{\texttt{norway1}} & \rotatebox{\dg}{\texttt{uk1}} & \rotatebox{\dg}{\texttt{uk2}} & \rotatebox{\dg}{\texttt{uk3}} & \rotatebox{\dg}{\texttt{uk4}} & \rotatebox{\dg}{\texttt{us1}} & \rotatebox{\dg}{\texttt{us2}} & \rotatebox{\dg}{\texttt{us3}} \\
    \toprule
    \outcomesname{} & 5 & 5 & 5 & 10 & 2 & 2 & 2 & 2 & 3 & 3 & 3 \\
    \rangename{} & 5 & 5 & 5 & 10 & 2 & 2 & 2 & 2 & 3 & 3 & 3 \\
    \bottomrule
  \end{tabular}
\end{table*}
%
Results for both \outcomesname{} and \rangename{} are shown in \reftab{rr}.

\begin{lstlisting}[
  language=customPython,
  escapechar=\%,
  caption={Program computing the mean growth rate in the $60-90\%$ category.},
  label={lst:rr},
  % float,
  % floatplacement=H
]
 def mean_growth_rate_60_90(
     portugal1, portugal2, portugal3,
     norway1,
     uk1, uk2, uk3, uk4,
     us1, us2, us3):
   portugal_avg = (portugal1 + portugal2 + portugal3) / 3%\label{l:portugal-avg}%
   norway_avg = norway1%\label{l:norway-avg}%
   uk_avg = (uk1 + uk2 + uk3 + uk4) / 4%\label{l:uk-avg}%
   us_avg = (us1 + us2 + us3) / 3%\label{l:us-avg}%
   avg = (portugal_avg + norway_avg + uk_avg + us_avg) / 4%\label{l:final-avg}%
\end{lstlisting}
%\vspace{%-15pt}
%
%
The analysis discovers that the Norway's only observation for this category $\texttt{norway1}$ has the biggest impact on the output, as perturbations on its value are capable of reaching 10 different outcomes (\cf~column $\texttt{norway1}$), while the other countries only have 5, 2, and 3, respectively for Portugal, UK, and US.
The same applies to \rangename{} as the output buckets have size $1$ and all the input perturbations are only capable of reaching contiguous buckets. Hence, we obtain the same exact results.

Our analysis is able to discover the disproportionate impact of Norway's only observation in the mean computation, which would have prevented one of the several programming errors found in the article.
%Nevertheless,
From a review of~\refprog{rr}, it is clear that Norway's only observation has a greater contribution to the computation,
%of the average growth rate,
as it does not need to be averaged with other observations first.
However, such methodological error is less evident when dealing with a higher number of input observations ($1175$ observations in the original work) and the computation is hidden behind a spreadsheet.

% As noted in many reports, a possible solution would be to improve the weighting procedure or filter outliers.


\subsection{GPT-4 Turbo}
\labsec{gpt-4-turbo}

The second use case we present is drawn from Sam Altman's OpenAI keynote in September 2023\sidenote{\rurl{www.youtube.com/live/U9mJuUkhUzk?si=HOzuH3-gr_kTdhCt&t=2330}}, where he presented the GPT-4 Turbo.
This new version of the GPT-4 language model brings the ability to write and interpret code directly without the need of human interaction.
Hence, as showcased in the keynote, the user could prompt multiple information to the model, such as related to the organization of a holiday trip with friends in Paris, and the model automatically generates the code to compute the share of the total cost of the trip and run it in background.
In this environment, users are unable to directly view the code unless they access the backend console.
This limitation makes it challenging for them to evaluate whether the function has been implemented correctly or not, assuming users have the capability to do so.
%
From the keynote, we extracted the~\refprog{share-division} which computes the user's share of the total cost of a holiday trip to Paris, given the total cost of the Airbnb, the flight cost, and the number of friends going on the trip.
%
\begin{table}
  \caption{Quantitative input usage for \refprog{share-division} computing the share division among friends.}
  \labtab{gpt-4-turbo}
  \begin{tabular}{c | ccc}
    \textsc{Impact} & \rotatebox{45}{\texttt{airbnb\_total\_cost\_eur}} & \rotatebox{45}{\texttt{flight\_cost\_usd}} & \rotatebox{45}{\texttt{number\_of\_friends}} \\
    \toprule
    \outcomesname{} & 10 & 17 & 9 \\
    \rangename{} & 1099 & 1709 & 999 \\
    \bottomrule
  \end{tabular}
\end{table}
%

\begin{lstlisting}[
  language=customPython,
  escapechar=\%,
  label={lst:share-division},
  caption={Program computing share division for holiday planning among friends.},
  % float,
  % floatplacement=H
]
 def share_division(
     airbnb_total_cost_eur,
     flight_cost_usd,
     number_of_friends):
   share_airbnb = airbnb_total_cost_eur / number_of_friends
   usd_to_eur = 0.92
   flight_cost_eur = flight_cost_usd * usd_to_eur
   total_cost_eur = share_airbnb + flight_cost_eur
\end{lstlisting}
%
Regarding the input bounds, users are willing to spend between 500 and 2000 for the Airbnb, between 50 and 1000 for the flight, and travel with between 2 and 10 friends. As a result, they expect their share, variable $\texttt{total\_cost\_eur}$, to be between 90 and 1900.
To compute the impact of the input variables we choose the output buckets to cover the expected output space in buckets of size $100$, \ie, $\setdef{100t + 90 \le \texttt{total\_cost\_eur} < \min \{100(t + 1) + 90, 1900\}}{0 \le t \le 19}$.
The %analysis discovers similar
findings are similar for both the \outcomesname{} and \rangename{} analysis, see~\reftab{gpt-4-turbo}.
The input variable $\texttt{flight\_cost\_usd}$ has the biggest impact on the output, as perturbations on its value are capable of reaching 17 different output buckets (resp. a range of 1709 output values), while the other two, $\texttt{airbnb\_total\_cost\_eur}$ and $\texttt{number\_of\_friends}$, only reach 10 and 9 output buckets (resp. have ranges of size 1099 and 999), respectively.
%Since the output buckets reached by perturbations of input values are contiguous, the \rangename{} analysis shows similar findings: 1709, 1099, and 999, respectively for \texttt{flight\_cost\_usd}, \texttt{airbnb\_total\_cost\_eur}, and \texttt{number\_of\_friends},

These results confirm the user expectations about the proposed program from ChatGPT: the flight cost yields the biggest impact as it cannot be shared among friends.


\subsection{Termination Analysis (A)}
\labsec{termination-analysis-A}

%
\begin{marginlisting}
  \caption{Example program from termination analysis.}
  \labprog{timing-analysis}
  \vspace{0.5cm}
\begin{lstlisting}[
    language=customPython,
    escapechar=\%,
    ]
 def example(x, y):
   counter = 0
   while x >= 0:
     if y <= 50:
       x += 1
     else
       x -= 1
     y += 1
     counter += 1
\end{lstlisting}
\end{marginlisting}

\refprog{timing-analysis} is adapted from the termination category of the software verification competition \textsc{sv-comp}\sidenote{\rurl{sv-comp.sosy-lab.org/}}.
Assuming both input positives, $\texttt{x},\texttt{y} \ge 0$, this program terminates in $\texttt{x}+1$ iterations if $\texttt{y} >50$, otherwise it terminates in $\texttt{x} - 2\texttt{y} + 103$ iterations.
We define $\texttt{counter}$ as the output variable, with output buckets defined as $\setdef{10k \le \texttt{counter} < 10(k+1)}{0 \le k < 50}$ and $\{\texttt{counter}\ge 500\}$. These output buckets represent cumulative ranges of iterations required for termination.
The analysis results are illustrated in~\reftab{termination-analysis}, they show that the input variable $\texttt{x}$ has the biggest impact.
Modifying the value of $\texttt{x}$ can result in the program terminating within any of the other 50 iteration ranges.
On the other hand, perturbations on $\texttt{y}$ can only result in the program terminating within 10 different iteration ranges.
Such difference is motivated by the fact that $\texttt{y}$ is only used to determine the number of iterations in the case where $\texttt{y}$ is greater than 50, otherwise it is not used at all. Therefore, two values of $\texttt{y}$, \eg, $y_0$ and $y_1$, only result in two different ranges of iterations required to make the program terminate if either both of them are below $50$ or $y_0 < 50\land y_1 \ge 50$ or $y_0\ge50\land y_1 <50$, not in all the cases.


\begin{margintable}[-4cm]
  \caption{Quantitative input usage for \refprog{timing-analysis}.}
  \labtab{termination-analysis}
  \begin{tabular}{c | c@{\hskip 5pt}c}
    \textsc{Impact} & \rotatebox{0}{\texttt{x}} & \rotatebox{0}{\texttt{y}} \\
    \toprule
    \outcomesname{} & 50 & 10 \\
    \rangename{} & 499 & 99 \\
    \bottomrule
  \end{tabular}
\end{margintable}

The given results can be interpreted as follows: the speed of termination of this loop is highly dependent on the value of $\texttt{x}$, while $\texttt{y}$ has a much smaller impact.
% This information could be used to attack such program just by looking at its timing behavior.
% For instance, given the number of iterations \texttt{counter}, we infer that the value of \texttt{x} is either $\texttt{counter} - 1$ or $\texttt{counter} + 2\texttt{y} - 103$. On the other hand, since \texttt{y} has less impact as discovered by our tool, we cannot infer much about its value.



\subsection{Termination Analysis (B)}
\labsec{app:termination-analysis-B}

This use case comes from the software verification competition SV-Comp\sidenote{\rurl{sv-comp.sosy-lab.org/}}, where the goal is to verify the termination of a program. \refprog{termination-a} and~\refprog{termination-b} have originally been proposed by \sidetextcite{Chen2012}, respectively these are Example (2.16) and Example (2.21) of such work.

\begin{marginlisting}
  \caption{Program Ex2.16 from software verification competition SV-Comp.}
  \labprog{termination-a}
  \vspace{0.5cm}
\begin{lstlisting}[
  language=customPython,
  escapechar=\%,
]
def termination_a(x, y):
  while x > 0:
    x = y
    y = y - 1
  result = x + y
  return result
\end{lstlisting}
\end{marginlisting}


\refprog{termination-a} returns the value of \texttt{y} whenever $\texttt{x} = 0$, otherwise it returns $-1$.
We assume both input variables are positive up to $1000$, $0 \le \texttt{x} \le 1000$ and $0 \le \texttt{y} \le 1000$.
Regarding such a function, it is interesting to study its behaviors around $0$, thus the output bucket are $\{ \texttt{result < 0} \}, \{ \texttt{result} = 0 \}$, and $\{ \texttt{result > 0} \}$.
With the above parameters, the analysis \outcomesname{} returns 1 for both input variables.
Such result is not too interesting, but by looking at the internal stages of the analysis we notice that perturbations on the value of the variable \texttt{x} may be able to produce from an output negative value to zero or a positive one (and viceversa).
While perturbations on the value of the variable \texttt{y} are only able to produce from zero to positive (and viceversa).

As a second experiments, we consider the buckets from -1 to 19, $\setdef{ \texttt{result} = n}{-1 \le n \le 19}$, and we notice that the analysis \outcomesname{} returns 1 for the input variable \texttt{x} and 19 for \texttt{y}, meaning that the variable \texttt{y} is able to affect far more output values than \texttt{x}. However, combing the results of the previous experiment, only the variable \texttt{x} is able to affect the negative output values.

\begin{marginlisting}[-1.4cm]
  \caption{Program Ex2.21 from software verification competition SV-Comp.}
  \labprog{termination-b}
  \vspace{0.5cm}
\begin{lstlisting}[
  language=customPython,
  escapechar=\%,
]
def termination_b(x, y):
  while x > 0:
    x = x + y
    y = -y - 1
  result = x + y
  return result
\end{lstlisting}
\end{marginlisting}

From the same work, we also consider \refprog{termination-b} which returns the value of \texttt{y} whenever $\texttt{x} = 0$, otherwise it returns $-1$.
Unfortunately, the backward analysis does not capture a precise loop invariant, thus both the analyses \outcomesname{} and \rangename are inconclusive in such case.
The key takeaway is that our analysis is highly dependent on the precision of the underlying backward analysis.

As a conclusion, even though SV-Comp proposes challenging benchmarks for termination, reachability, and safety analyses, they are not amenable for information flow analysis.
Most of the time, their examples involve loops with complex invariant, but as input-output relations, the variables involved are just zeroed out after the loop.
Drawing examples from their dataset is less appealing to our work.

\subsection{Linear Loops}
\labsec{linear-loops}


\begin{marginlisting}
  \caption{Program computing the linear expression $(5x + 2y)$ via repeated additions.}
  \labprog{linear-expression}
  \vspace{0.9cm}
\begin{lstlisting}[
  language=customPython,
  escapechar=\%,
]
def linear_expression(x, y):
  result = 0
  i = 0
  while i < 5:
    result = result + x
    i += 1
  i = 0
  while i < 2:
    result = result + y
    i += 1
\end{lstlisting}
\end{marginlisting}

\refprog{linear-expression} computes the linear expression $(5x + 2y)$ via repeated additions.
Note that the invariant of the loop is indeed non-linear ($\texttt{result} = \texttt{i} * \texttt{x}$ and $\texttt{result} = \texttt{result}' + \texttt{i} * \texttt{y}$ respectively for the first and second loop, where $\texttt{result}'$ is the value of \texttt{result} before entering the second loop), but the loop is executed a fixed number of times, thus the analysis is able to compute the exact output buckets through loop unrolling.

For the analysis the input bounds are $0 \le \texttt{x} \le 1000$ and $0 \le \texttt{y} \le 1000$, while the output buckets are $\setdef{n * 100 \le \texttt{result} < (n + 1) * 100}{n \le 70}$.
Both analyses, \outcomesname{} and \rangename, show that \texttt{x} has an impact $\frac{5}{2}$ times bigger than \texttt{y} on the output.
Thus, the impact quantity provides insight about the termination speed.
Indeed, the loop for \texttt{x} is executed 5 times, while the one for \texttt{y} only 2.


\subsection{Landing Risk System}
\labsec{landing-risk}




\begin{figure}[t]
  \centering
\begin{tikzpicture}
  % Grid
  \draw[help lines, color=gray!30, dashed] (-0.1,-0.1) grid (9.9,3.9);
  % x-axis
  \draw[->,ultra thick] (0,0)--(10,0) node[rotate=90,below]{\x};
  % y-axis
  \draw[->,ultra thick] (0,0)--(0,4) node[above]{\y};
  % x-axis ticks
  \foreach \x in {-4,-3,-2,-1,0,1,2,3,4}
      \draw (\x+5,0.1) -- (\x+5,-0.1) node[below] {\x};
  % y-axis ticks
  \foreach \y in {1,2,3}
      \draw (0.1,\y) -- (-0.1,\y) node[left] {\y};
  % Polyhedra
  \fill[color=seabornGreen, opacity=0.5] (5,3) -- (7,1) -- (3,1) -- cycle;
  \draw[color=seabornGreen, ultra thick] (5,3) -- (7,1) -- (3,1) -- cycle;
  % Polyhedra
  \fill[color=seabornYellow, opacity=0.5] (4,3) -- (5,3) -- (3,1) -- (2,1) -- cycle;
  \draw[color=seabornYellow, ultra thick] (4,3) -- (5,3) -- (3,1) -- (2,1) -- cycle;
  % % Polyhedra
  \fill[color=seabornYellow, opacity=0.5] (5,3) -- (6,3) -- (8,1) -- (7,1) -- cycle;
  \draw[color=seabornYellow, ultra thick] (5,3) -- (6,3) -- (8,1) -- (7,1) -- cycle;
  % Polyhedra
  \fill[color=seabornOrange, opacity=0.5] (3,3) -- (4,3) -- (2,1) -- (1,1) -- cycle;
  \draw[color=seabornOrange, ultra thick] (3,3) -- (4,3) -- (2,1) -- (1,1) -- cycle;
  % % Polyhedra
  \fill[color=seabornOrange, opacity=0.5] (6,3) -- (7,3) -- (9,1) -- (8,1) -- cycle;
  \draw[color=seabornOrange, ultra thick] (6,3) -- (7,3) -- (9,1) -- (8,1) -- cycle;
  % Polyhedra
  \fill[color=seabornRed, opacity=0.5] (1,3) -- (3,3) -- (1,1) -- cycle;
  \draw[color=seabornRed, ultra thick] (1,3) -- (3,3) -- (1,1) -- cycle;
  % Polyhedra
  \fill[color=seabornRed, opacity=0.5] (7,3) -- (9,3) -- (9,1) -- cycle;
  \draw[color=seabornRed, ultra thick] (7,3) -- (9,3) -- (9,1) -- cycle;
  % Nodes
  \fill[color=seabornRed] (0+1,0+1) circle[radius=2pt];
  \node[above left] at (0+1,0+1) {$3$};
  \fill[color=seabornRed] (0+1,1+1) circle[radius=2pt];
  \node[above left] at (0+1,1+1) {$3$};
  \fill[color=seabornRed]    (0+1,2+1) circle[radius=2pt];
  \node[above left] at (0+1,2+1) {$3$};
  \fill[color=seabornOrange] (1+1,0+1) circle[radius=2pt];
  \node[above left] at (1+1,0+1) {$2$};
  \fill[color=seabornRed]    (1+1,1+1) circle[radius=2pt];
  \node[above left] at (1+1,1+1) {$3$};
  \fill[color=seabornRed] (1+1,2+1) circle[radius=2pt];
  \node[above left] at (1+1,2+1) {$3$};
  \fill[color=seabornYellow]    (2+1,0+1) circle[radius=2pt];
  \node[above left] at (2+1,0+1) {$1$};
  \fill[color=seabornOrange] (2+1,1+1) circle[radius=2pt];
  \node[above left] at (2+1,1+1) {$2$};
  \fill[color=seabornRed] (2+1,2+1) circle[radius=2pt];
  \node[above left] at (2+1,2+1) {$3$};
  \fill[color=seabornGreen] (3+1,0+1) circle[radius=2pt];
  \node[above left] at (3+1,0+1) {$0$};
  \fill[color=seabornYellow] (3+1,1+1) circle[radius=2pt];
  \node[above left] at (3+1,1+1) {$1$};
  \fill[color=seabornOrange]    (3+1,2+1) circle[radius=2pt];
  \node[above left] at (3+1,2+1) {$2$};
  \fill[color=seabornGreen] (4+1,0+1) circle[radius=2pt];
  \node[above left] at (4+1,0+1) {$0$};
  \fill[color=seabornGreen]    (4+1,1+1) circle[radius=2pt];
  \node[above left] at (4+1,1+1) {$0$};
  \fill[color=seabornYellow]   (4+1,2+1) circle[radius=2pt];
  \node[above left] at (4+1,2+1) {$1$};
  \fill[color=seabornGreen]    (5+1,0+1) circle[radius=2pt];
  \node[above right] at (5+1,0+1) {$0$};
  \fill[color=seabornYellow]   (5+1,1+1) circle[radius=2pt];
  \node[above right] at (5+1,1+1) {$1$};
  \fill[color=seabornOrange]   (5+1,2+1) circle[radius=2pt];
  \node[above right] at (5+1,2+1) {$2$};
  \fill[color=seabornYellow]   (6+1,0+1) circle[radius=2pt];
  \node[above right] at (6+1,0+1) {$1$};
  \fill[color=seabornOrange]   (6+1,1+1) circle[radius=2pt];
  \node[above right] at (6+1,1+1) {$2$};
  \fill[color=seabornRed]   (6+1,2+1) circle[radius=2pt];
  \node[above right] at (6+1,2+1) {$3$};
  \fill[color=seabornOrange]   (7+1,0+1) circle[radius=2pt];
  \node[above right] at (7+1,0+1) {$2$};
  \fill[color=seabornRed]   (7+1,1+1) circle[radius=2pt];
  \node[above right] at (7+1,1+1) {$3$};
  \fill[color=seabornRed]   (7+1,2+1) circle[radius=2pt];
  \node[above right] at (7+1,2+1) {$3$};
  \fill[color=seabornRed]   (8+1,0+1) circle[radius=2pt];
  \node[above right] at (8+1,0+1) {$3$};
  \fill[color=seabornRed]   (8+1,1+1) circle[radius=2pt];
  \node[above right] at (8+1,1+1) {$3$};
  \fill[color=seabornRed]   (8+1,2+1) circle[radius=2pt];
  \node[above right] at (8+1,2+1) {$3$};
\end{tikzpicture}
\caption{Input space composition with continuous input values.}
\labfig{extended}
\end{figure}


\begin{table}[t]
  \caption{Quantitative input usage for~\refprog{landing-alarm-system}.}
  \labtab{landing-risk}
  \centering
  \begin{tabular}{cc|cc|cc}
    \multicolumn{2}{c|}{\multirow{2}{*}{~\textbf{Input Bounds}}} & \multicolumn{2}{c|}{\outcomesname} & \multicolumn{2}{c}{\rangename} \\ \cline{3-6}
    & & \texttt{angle} & \hspace{-5.5pt}\texttt{speed} & \texttt{angle} & \hspace{-5.5pt}\texttt{speed} \\ \hline\hline
    % $\texttt{angle} = -1 \lor \texttt{angle} = 4$ & \multirow{4}{*}{$1 \le \texttt{speed} \le 3$} & &
    % 1  &  2  & 3  & 2  \\ \cline{1-1} \cline{4-7}
    $-4 \le \texttt{angle} \le 4$ & \multirow{3}{*}{$~~\land 1 \le \texttt{speed} \le 3$} &
    3  &  3  & 3  & 3  \\ \cline{1-1} \cline{3-6}
    $-4 \le \texttt{angle} \le 0$ & &
    3  &  2  & 3  & 2  \\ \cline{1-1} \cline{3-6}
    $0 \le \texttt{angle} \le 4$ & &
    3 &  2 & 3 & 2 \\
  \end{tabular}
\end{table}

Finally, we apply our quantitative analysis to~\refprog{landing-alarm-system}\marginprop{landing-alarm-system} (reported on the side) for the landing alarm system extended with the  continuous input space for the aircraft angle of approach, where $(-4 \le \texttt{angle} \le 4) \land (1 \le \texttt{speed} \le 3)$, see \reffig{extended}.
In this instance, the precision of the abstraction drastically drops as convex abstract domains are not able to capture the symmetric features of the input space around 0.
Indeed, the analysis result, first row of~\reffig{analysis}, is unable to reveal any difference in the input usage of input variables as all the abstract preconditions result of the backward analysis intersect together.
As a consequence, \outcomesname{} and \rangename{} are unable to provide any meaningful information, first row of \reftab{landing-risk}.

A possible approach to overcome the non-convexity of the input space is to split the input space into two subspaces (as a bounded set of disjunctive polyhedra), $-4 \le \texttt{angle} \le 0$ and $0 \le \texttt{angle} \le 4$, second and third row of \reftab{landing-risk}.
In the first subset $-4 \le \texttt{angle} \le 0$, we are able to perfectly captures the input regions that lead to each output bucket with our abstract analysis, second row of~\reffig{analysis}.
Therefore, we are able to recover the information that the input configurations from the bucket $\{\texttt{risk} =3\}$ do not intersect with the ones from the bucket $\{\texttt{risk} = 0\}$ after projecting away the axis \texttt{speed}.
As the end, our analysis notices that variations in the value of the input \texttt{angle} results in three possible output values, while variations in the \texttt{speed} input lead to two.
Similarly, regarding the range of values, variations in the \texttt{angle} input cover the entire spectrum of output values, whereas to the \texttt{speed} input only span a range of 2 since it exists no input value such that modifications in the \texttt{speed} value could obtain a range of output values bigger than 2.
The same reasoning applies to the other subspace with $0 \le \texttt{angle} \le 4$.


\begin{figure*}[t]
  \centering
  \begin{subfigure}{\textwidth}
  \begin{subfigure}[b]{0.24\textwidth}
    \begin{tikzpicture}[scale=0.8]
      % Grid
      \foreach \y in {0.5, 1.5, 2.5} {
        \draw[help lines, color=gray!30, dashed] (0,\y) -- (2.9,\y);
      }
      \foreach \x in {0.5, 1, 1.5, 2, 2.5} {
        \draw[help lines, color=gray!30, dashed] (\x, 0) -- (\x, 2.9);
      }
      % x-axis
      \draw[->,ultra thick] (0,0)--(3,0);
      % \draw[->,ultra thick] (0,0)--(3,0) node[rotate=90,below]{\x};
      % % y-axis
      \draw[->,ultra thick] (0,0)--(0,3) node[above]{\y};
      % % x-axis ticks
      \draw (0.5,0.1) -- (0.5,-0.1) node[below] {$-4$};
      % \draw (1,0.1) -- (1,-0.1) node[below] {$-2$};
      \draw (1,0.1) -- (1,-0.1);
      \draw (1.5,0.1) -- (1.5,-0.1) node[below] {$0$};
      % \draw (2,0.1) -- (2,-0.1) node[below] {$2$};
      \draw (2,0.1) -- (2,-0.1);
      \draw (2.5,0.1) -- (2.5,-0.1) node[below] {$4$};
      % % y-axis ticks
      \foreach \y in {1,2,3}
          \draw (0.1,\y-0.5) -- (-0.1,\y-0.5) node[left] {\y};
      % % Polyhedra
      \fill[color=seabornRed, opacity=0.5] (0.5,0.5) -- (2.5,0.5) -- (2.5,2.5) -- (0.5,2.5) -- cycle;
      \draw[color=seabornRed, ultra thick] (0.5,0.5) -- (2.5,0.5) -- (2.5,2.5) -- (0.5,2.5) -- cycle;
    \end{tikzpicture}
    % \caption{$\{\texttt{risk} = 3\}$}
  \end{subfigure}
  \hfill
  \begin{subfigure}[b]{0.23\textwidth}
    \begin{tikzpicture}[scale=0.8]
      % Grid
      \foreach \y in {0.5, 1.5, 2.5} {
        \draw[help lines, color=gray!30, dashed] (0,\y) -- (2.9,\y);
      }
      \foreach \x in {0.5, 1, 1.5, 2, 2.5} {
        \draw[help lines, color=gray!30, dashed] (\x, 0) -- (\x, 2.9);
      }
      % x-axis
      \draw[->,ultra thick] (0,0)--(3,0);
      % \draw[->,ultra thick] (0,0)--(3,0) node[rotate=90,below]{\x};
      % % y-axis
      % \draw[->,ultra thick] (0,0)--(0,3) node[above]{\y};
      % % x-axis ticks
      \draw (0.5,0.1) -- (0.5,-0.1) node[below] {$-4$};
      % \draw (1,0.1) -- (1,-0.1) node[below] {$-2$};
      \draw (1,0.1) -- (1,-0.1);
      \draw (1.5,0.1) -- (1.5,-0.1) node[below] {$0$};
      % \draw (2,0.1) -- (2,-0.1) node[below] {$2$};
      \draw (2,0.1) -- (2,-0.1);
      \draw (2.5,0.1) -- (2.5,-0.1) node[below] {$4$};
      % % y-axis ticks
      % \foreach \y in {1,2,3}
      %     \draw (0.1,\y-0.5) -- (-0.1,\y-0.5) node[left] {\y};
      % % Polyhedra
      \fill[color=seabornOrange, opacity=0.5] (0.5,0.5) -- (2.5,0.5) -- (2.25,2.5) -- (0.75,2.5) -- cycle;
      \draw[color=seabornOrange, ultra thick] (0.5,0.5) -- (2.5,0.5);
      \draw[color=seabornOrange, ultra thick] (0.75,2.5) -- (2.25,2.5);
      \draw[color=seabornOrange, ultra thick, dotted] (2.5,0.5) -- (2.25,2.5);
      \draw[color=seabornOrange, ultra thick, dotted] (0.75,2.5) -- (0.5,0.5);
    \end{tikzpicture}
    % \caption{$\{\texttt{risk} = 2\}$}
  \end{subfigure}
  % \hfill
  \begin{subfigure}[b]{0.23\textwidth}
    \begin{tikzpicture}[scale=0.8]
      % Grid
      \foreach \y in {0.5, 1.5, 2.5} {
        \draw[help lines, color=gray!30, dashed] (0,\y) -- (2.9,\y);
      }
      \foreach \x in {0.5, 1, 1.5, 2, 2.5} {
        \draw[help lines, color=gray!30, dashed] (\x, 0) -- (\x, 2.9);
      }
      % x-axis
      \draw[->,ultra thick] (0,0)--(3,0);
      % \draw[->,ultra thick] (0,0)--(3,0) node[rotate=90,below]{\x};
      % % y-axis
      % \draw[->,ultra thick] (0,0)--(0,3) node[above]{\y};
      % % x-axis ticks
      \draw (0.5,0.1) -- (0.5,-0.1) node[below] {$-4$};
      % \draw (1,0.1) -- (1,-0.1) node[below] {$-2$};
      \draw (1,0.1) -- (1,-0.1);
      \draw (1.5,0.1) -- (1.5,-0.1) node[below] {$0$};
      % \draw (2,0.1) -- (2,-0.1) node[below] {$2$};
      \draw (2,0.1) -- (2,-0.1);
      \draw (2.5,0.1) -- (2.5,-0.1) node[below] {$4$};
      % % y-axis ticks
      % \foreach \y in {1,2,3}
      %     \draw (0.1,\y-0.5) -- (-0.1,\y-0.5) node[left] {\y};
      % % Polyhedra
      \fill[color=seabornYellow, opacity=0.5] (1,0.5) -- (2,0.5) -- (1.75,2.5) -- (1.25,2.5) -- cycle;
      \draw[color=seabornYellow, ultra thick] (1,0.5) -- (2,0.5);
      \draw[color=seabornYellow, ultra thick] (1.75,2.5) -- (1.25,2.5);
      \draw[color=seabornYellow, ultra thick, dotted] (2,0.5) -- (1.75,2.5);
      \draw[color=seabornYellow, ultra thick, dotted] (1.25,2.5) -- (1,0.5);
    \end{tikzpicture}
    % \caption{$\{\texttt{risk} = 1\}$}
  \end{subfigure}
  % \hfill
  \begin{subfigure}[b]{0.24\textwidth}
    \begin{tikzpicture}[scale=0.8]
      % Grid
      \foreach \y in {0.5, 1.5, 2.5} {
        \draw[help lines, color=gray!30, dashed] (0,\y) -- (2.9,\y);
      }
      \foreach \x in {0.5, 1, 1.5, 2, 2.5} {
        \draw[help lines, color=gray!30, dashed] (\x, 0) -- (\x, 2.9);
      }
      % x-axis
      % \draw[->,ultra thick] (0,0)--(3,0);
      \draw[->,ultra thick] (0,0)--(3,0) node[rotate=90,below]{\x};
      % % y-axis
      % \draw[->,ultra thick] (0,0)--(0,3) node[above]{\y};
      % % x-axis ticks
      \draw (0.5,0.1) -- (0.5,-0.1) node[below] {$-4$};
      % \draw (1,0.1) -- (1,-0.1) node[below] {$-2$};
      \draw (1,0.1) -- (1,-0.1);
      \draw (1.5,0.1) -- (1.5,-0.1) node[below] {$0$};
      % \draw (2,0.1) -- (2,-0.1) node[below] {$2$};
      \draw (2,0.1) -- (2,-0.1);
      \draw (2.5,0.1) -- (2.5,-0.1) node[below] {$4$};
      % % y-axis ticks
      % \foreach \y in {1,2,3}
      %     \draw (0.1,\y-0.5) -- (-0.1,\y-0.5) node[left] {\y};
      % % Polyhedra
      \fill[color=seabornGreen, opacity=0.5] (1.25,0.5) -- (1.75,0.5) -- (1.5,2.5) -- cycle;
      \draw[color=seabornGreen, ultra thick] (1.25,0.5) -- (1.75,0.5);
      \draw[color=seabornGreen, ultra thick, dotted] (1.75,0.5) -- (1.5,2.5);
      \draw[color=seabornGreen, ultra thick, dotted] (1.5,2.5) -- (1.25,0.5);
    \end{tikzpicture}
    % \caption{$\{\texttt{risk} = 0\}$}
  \end{subfigure}
% \caption{Analysis result using the polyhedra domain.}
\label{fig:analysis-extended}
\end{subfigure}
\begin{subfigure}{\textwidth}
  \begin{subfigure}[b]{0.24\textwidth}
    \begin{tikzpicture}[scale=0.8]
      % Grid
      \foreach \y in {0.5, 1.5, 2.5} {
        \draw[help lines, color=gray!30, dashed] (0,\y) -- (2.9,\y);
      }
      \foreach \x in {0.5, 1, 1.5, 2, 2.5} {
        \draw[help lines, color=gray!30, dashed] (\x, 0) -- (\x, 2.9);
      }
      % x-axis
      \draw[->,ultra thick] (0,0)--(3,0);
      % \draw[->,ultra thick] (0,0)--(3,0) node[rotate=90,below]{\x};
      % % y-axis
      \draw[->,ultra thick] (0,0)--(0,3) node[above]{\y};
      \draw[dashed] (1.5,0)--(1.5,3);
      % % x-axis ticks
      \draw (0.5,0.1) -- (0.5,-0.1) node[below] {$-4$};
      % \draw (1,0.1) -- (1,-0.1) node[below] {$-2$};
      \draw (1,0.1) -- (1,-0.1);
      \draw (1.5,0.1) -- (1.5,-0.1) node[below] {$0$};
      % \draw (2,0.1) -- (2,-0.1) node[below] {$2$};
      \draw (2,0.1) -- (2,-0.1);
      \draw (2.5,0.1) -- (2.5,-0.1) node[below] {$4$};
      % % y-axis ticks
      \foreach \y in {1,2,3}
          \draw (0.1,\y-0.5) -- (-0.1,\y-0.5) node[left] {\y};
      % % Polyhedra
      \fill[color=seabornRed, opacity=0.5] (0.5,0.5) -- (1,2.5) -- (0.5,2.5) -- cycle;
      \fill[color=seabornRed, opacity=0.5] (2.5,0.5) -- (2.5,2.5) -- (2,2.5) -- cycle;
      \draw[color=seabornRed, ultra thick] (0.5,0.5) -- (1,2.5) -- (0.5,2.5) -- cycle;
      \draw[color=seabornRed, ultra thick] (2.5,0.5) -- (2.5,2.5) -- (2,2.5) -- cycle;
    \end{tikzpicture}
    % \caption{$\{\texttt{risk} = 3\}$}
  \end{subfigure}
  \hfill
  \begin{subfigure}[b]{0.23\textwidth}
    \begin{tikzpicture}[scale=0.8]
      % Grid
      \foreach \y in {0.5, 1.5, 2.5} {
        \draw[help lines, color=gray!30, dashed] (0,\y) -- (2.9,\y);
      }
      \foreach \x in {0.5, 1, 1.5, 2, 2.5} {
        \draw[help lines, color=gray!30, dashed] (\x, 0) -- (\x, 2.9);
      }
      % x-axis
      \draw[->,ultra thick] (0,0)--(3,0);
      \draw[dashed] (1.5,0)--(1.5,3);
      % \draw[->,ultra thick] (0,0)--(3,0) node[rotate=90,below]{\x};
      % % y-axis
      % \draw[->,ultra thick] (0,0)--(0,3) node[above]{\y};
      % % x-axis ticks
      \draw (0.5,0.1) -- (0.5,-0.1) node[below] {$-4$};
      % \draw (1,0.1) -- (1,-0.1) node[below] {$-2$};
      \draw (1,0.1) -- (1,-0.1);
      \draw (1.5,0.1) -- (1.5,-0.1) node[below] {$0$};
      % \draw (2,0.1) -- (2,-0.1) node[below] {$2$};
      \draw (2,0.1) -- (2,-0.1);
      \draw (2.5,0.1) -- (2.5,-0.1) node[below] {$4$};
      % % y-axis ticks
      % \foreach \y in {1,2,3}
      %     \draw (0.1,\y-0.5) -- (-0.1,\y-0.5) node[left] {\y};
      % % Polyhedra
      \fill[color=seabornOrange, opacity=0.5] (0.5,0.5) -- (0.75,0.5) -- (1,2.5) -- (0.75,2.5) -- cycle;
      \fill[color=seabornOrange, opacity=0.5] (2.25,0.5) -- (2.5,0.5) -- (2.25,2.5) -- (2,2.5) -- cycle;
      \draw[color=seabornOrange, ultra thick] (0.5,0.5) -- (0.75,0.5);
      \draw[color=seabornOrange, ultra thick] (2.25,0.5) -- (2.5,0.5);
      \draw[color=seabornOrange, ultra thick] (0.75,2.5) -- (1,2.5);
      \draw[color=seabornOrange, ultra thick] (2,2.5) -- (2.25,2.5);
      \draw[color=seabornOrange, ultra thick, dotted] (2.5,0.5) -- (2.25,2.5);
      \draw[color=seabornOrange, ultra thick] (2.25,0.5) -- (2,2.5);
      \draw[color=seabornOrange, ultra thick, dotted] (0.75,2.5) -- (0.5,0.5);
      \draw[color=seabornOrange, ultra thick] (1,2.5) -- (0.75,0.5);
    \end{tikzpicture}
    % \caption{$\{\texttt{risk} = 2\}$}
  \end{subfigure}
  % \hfill
  \begin{subfigure}[b]{0.23\textwidth}
    \begin{tikzpicture}[scale=0.8]
      % Grid
      \foreach \y in {0.5, 1.5, 2.5} {
        \draw[help lines, color=gray!30, dashed] (0,\y) -- (2.9,\y);
      }
      \foreach \x in {0.5, 1, 1.5, 2, 2.5} {
        \draw[help lines, color=gray!30, dashed] (\x, 0) -- (\x, 2.9);
      }
      % x-axis
      \draw[->,ultra thick] (0,0)--(3,0);
      \draw[dashed] (1.5, 0)--(1.5, 3);
      % \draw[->,ultra thick] (0,0)--(3,0) node[rotate=90,below]{\x};
      % % y-axis
      % \draw[->,ultra thick] (0,0)--(0,3) node[above]{\y};
      % % x-axis ticks
      \draw (0.5,0.1) -- (0.5,-0.1) node[below] {$-4$};
      % \draw (1,0.1) -- (1,-0.1) node[below] {$-2$};
      \draw (1,0.1) -- (1,-0.1);
      \draw (1.5,0.1) -- (1.5,-0.1) node[below] {$0$};
      % \draw (2,0.1) -- (2,-0.1) node[below] {$2$};
      \draw (2,0.1) -- (2,-0.1);
      \draw (2.5,0.1) -- (2.5,-0.1) node[below] {$4$};
      % % y-axis ticks
      % \foreach \y in {1,2,3}
      %     \draw (0.1,\y-0.5) -- (-0.1,\y-0.5) node[left] {\y};
      % % Polyhedra
      \fill[color=seabornYellow, opacity=0.5] (1,0.5) -- (1.25,0.5) -- (1.5,2.5) -- (1.75,0.5) -- (2,0.5) -- (1.75,2.5) -- (1.25,2.5) -- cycle;
      \draw[color=seabornYellow, ultra thick] (1,0.5) -- (1.25,0.5);
      \draw[color=seabornYellow, ultra thick] (1.75,0.5) -- (2,0.5);
      \draw[color=seabornYellow, ultra thick] (1.75,2.5) -- (1.25,2.5);
      \draw[color=seabornYellow, ultra thick, dotted] (2,0.5) -- (1.75,2.5);
      \draw[color=seabornYellow, ultra thick] (1.75,0.5) -- (1.5,2.5);
      \draw[color=seabornYellow, ultra thick, dotted] (1.25,2.5) -- (1,0.5);
      \draw[color=seabornYellow, ultra thick] (1.5,2.5) -- (1.25,0.5);
    \end{tikzpicture}
    % \caption{$\{\texttt{risk} = 1\}$}
  \end{subfigure}
  % \hfill
  \begin{subfigure}[b]{0.24\textwidth}
    \begin{tikzpicture}[scale=0.8]
      % Grid
      \foreach \y in {0.5, 1.5, 2.5} {
        \draw[help lines, color=gray!30, dashed] (0,\y) -- (2.9,\y);
      }
      \foreach \x in {0.5, 1, 1.5, 2, 2.5} {
        \draw[help lines, color=gray!30, dashed] (\x, 0) -- (\x, 2.9);
      }
      % x-axis
      % \draw[->,ultra thick] (0,0)--(3,0);
      \draw[->,ultra thick] (0,0)--(3,0) node[rotate=90,below]{\x};
      \draw[dashed] (1.5, 0)--(1.5, 3);
      % % y-axis
      % \draw[->,ultra thick] (0,0)--(0,3) node[above]{\y};
      % % x-axis ticks
      \draw (0.5,0.1) -- (0.5,-0.1) node[below] {$-4$};
      % \draw (1,0.1) -- (1,-0.1) node[below] {$-2$};
      \draw (1,0.1) -- (1,-0.1);
      \draw (1.5,0.1) -- (1.5,-0.1) node[below] {$0$};
      % \draw (2,0.1) -- (2,-0.1) node[below] {$2$};
      \draw (2,0.1) -- (2,-0.1);
      \draw (2.5,0.1) -- (2.5,-0.1) node[below] {$4$};
      % % y-axis ticks
      % \foreach \y in {1,2,3}
      %     \draw (0.1,\y-0.5) -- (-0.1,\y-0.5) node[left] {\y};
      % % Polyhedra
      \fill[color=seabornGreen, opacity=0.5] (1.25,0.5) -- (1.75,0.5) -- (1.5,2.5) -- cycle;
      \draw[color=seabornGreen, ultra thick] (1.25,0.5) -- (1.75,0.5);
      \draw[color=seabornGreen, ultra thick, dotted] (1.75,0.5) -- (1.5,2.5);
      \draw[color=seabornGreen, ultra thick, dotted] (1.5,2.5) -- (1.25,0.5);
      \draw[color=seabornGreen, ultra thick, dotted] (1.5,2.5) -- (1.5,0.5);
    \end{tikzpicture}
    % \caption{$\{\texttt{risk} = 0\}$}
  \end{subfigure}
% \caption{Analysis result after splitting the input space into two subspaces around $\texttt{angle}=0$.}
\end{subfigure}
\caption{Above, result of the analysis with convex polyhedra. Below, result after splitting the input space into two subspaces around $\texttt{angle}=0$.}
%\vspace{%-15pt}
\labfig{analysis}
\end{figure*}

