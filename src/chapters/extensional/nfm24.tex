%

\chapter{Quantitative Input Data Usage}
\labch{quantitative-input-data-usage}

In this chapter we present the quantitative framework used to measure the contributions of the input variables to the outcome of a program. First, we introduce a leading example of a landing-risk alarm system to illustrate the concepts. Then, we present the framework and its properties. Finally, since the framework depends on the notion of impact, we show a few possible definitions of impact.

\onlysectioncommands{\x,\y,\z,\lc,\exampleinput,\highlight,\inputa,\inputax,\inputay,\outputa,\inputb,\inputbx,\inputby,\outputb,\inputc,\inputcx,\inputcy,\outputc,\inputd,\inputdx,\inputdy,\outputd,\inpute,\inputex,\inputey,\outpute,\inputf,\inputfx,\inputfy,\outputf,\tracea,\traceax,\traceay,\traceb,\tracebx,\traceby,\tracec,\tracecx,\tracecy,\traced,\tracedx,\tracedy,\tracee,\traceex,\traceey,\tracef,\tracefx,\tracefy,\labelrotationangle,
\outputvaluea,\outputvalueb,\outputvaluec,\outputvalued,\outputvaluee,\outputvaluef}

%

\section{Landing Alarm System}
\labsec{landing-alarm-system}
\newcommand*{\x}{\texttt{angle}}
\newcommand*{\y}{\texttt{speed}}
\newcommand*{\z}{\texttt{risk}}
\newcommand*{\lc}{\texttt{landing\_coeff}}


\begin{marginlisting}
  \caption{Program for the landing-risk alarm system.}
  \labprog{landing-alarm-system}
  \vspace{0.5cm}
\begin{lstlisting}[language=customPython,escapechar=\%]
landing_coeff = abs(angle) + speed %\labline{compute-risk}%
if landing_coeff < 2 then %\labline{low-risk-cond}%
  risk = 0 %\labline{low-risk}%
else if landing_coeff > 5 then %\labline{high-risk-cond}%
  risk = 3 %\labline{high-risk}%
else %\labline{medium-risk-branch}%
  risk = floor(landing_coeff) - 2 %\labline{medium-risk}%
\end{lstlisting}
\end{marginlisting}


The goal of \refprog{landing-alarm-system}, referred to as \landingprogram, is to inform the pilot about the level of risk associated with the landing approach.
It takes two input variables, denoted as \x{} and \y, for the aircraft-airstrip alignment angle and the aircraft speed, respectively.
A value of 1 represents a good alignment while -4 a non-aligned angle, whereas 1, 2, 3 denote low, medium, and high speed.
A safer approach is indicated by lower speed.
The program \landingprogram{} computes first the landing risk coefficient, denoted as \lc, at \refline{compute-risk}.
This coefficient is obtained by taking the absolute value of the landing angle (accounting for both negative and positive approach angles) and adding it to the speed.
Afterwards, \refline{low-risk-cond} and~\refline{high-risk-cond} restrict \lc{} to range between values 2 and 5.
Values below 2 signify low danger, while values above 5 indicate high danger.
Respectively, we assign to the output variable \z{} the value of 0 for low danger and the value of 3 for high danger.
The medium degree of dangerousness is computed at \refline{medium-risk} and ranges between 1 and 2, which assigns to the output variable \z{} the largest integer less than, or equal, to $\lc-2$.
The output variable $\z{}$ is the danger level with possible values $\{0, 1, 2, 3\}$.

\begin{marginfigure}
\centering
\begin{tikzpicture}
  % Grid
  \draw[help lines, color=gray!30, dashed] (0,0) grid (2.9,3.9);
  % x-axis
  \draw[->,ultra thick] (0,0)--(3,0) node[right]{\x};
  % y-axis
  \draw[->,ultra thick] (0,0)--(0,4) node[above]{\y};
  % x-axis ticks
  \draw (0+1,0.1) -- (0+1,-0.1) node[below] {-4};
  \draw (1+1,0.1) -- (1+1,-0.1) node[below] {1};
  % y-axis ticks
  \foreach \y in {1,2,3}
  \draw (0.1,\y) -- (-0.1,\y) node[left] {\y};
  % Nodes
  \fill[color=seabornRed]   (0+1,0+1) circle[radius=2pt];
  \node[above right] at (0+1,0+1) {$3$};
  \fill[color=seabornRed]   (0+1,1+1) circle[radius=2pt];
  \node[above right] at (0+1,1+1) {$3$};
  \fill[color=seabornRed]   (0+1,2+1) circle[radius=2pt];
  \node[above right] at (0+1,2+1) {$3$};
  \fill[color=seabornGreen] (1+1,0+1) circle[radius=2pt];
  \node[above right] at (1+1,0+1) {$0$};
  \fill[color=seabornYellow] (1+1,1+1) circle[radius=2pt];
  \node[above right] at (1+1,1+1) {$1$};
  \fill[color=seabornOrange]    (1+1,2+1) circle[radius=2pt];
  \node[above right] at (1+1,2+1) {$2$};
\end{tikzpicture}
\caption{Input space composition of \refprog{landing-alarm-system}.}
\labfig{input-space-composition}
\end{marginfigure}

\reffig{input-space-composition} shows the input space composition of this system, where the label near each input represents the degree of risk assigned to the corresponding input configuration.
It is easy to note that a nonaligned angle of approach corresponds to a considerably higher level of risk, whereas the risk with a correct angle depends mostly on the aircraft speed.
Our goal is to develop a static analysis capable of quantifying the contribution of each input variable to the computation of the output variable $\z{}$.


\newcommand{\exampleinput}[1][\defaultprogramexampleletter]{\textsc{Input}_{#1}}

Understanding the effects of input variables on program executions allows us to reveal potential bugs or certify intended behavior.
Therefore, we present a framework to quantify the impact of input variables on the outcome of a program.
Intuitively, the definition of impact of a variable depends on various factors such as program structure, environment, developer expertise, and researcher intuition.
Different quantifiers of impact leverage different traits of the input-output relations in computation.
As a result, it is not possible to formalize a definition that fits all requirements.

In general, all possible definitions of impact establish some relationship between variations of the input and the output values.
In this section, we introduce two impact quantifiers.
The first impact definition focuses on \textit{the size of} extreme reachable values resulting from variations in the input variable.
The second exploits \textit{the number of} reachable outcomes.
Both definitions give us different insights on the program \landingprogram.
In particular, the first quantity tells us which variation in the values of input variables results in larger differences between output values, while the second indicates which variation reaches a greater number of output values.

\newcommand*{\highlight}[1]{\textcolor{seabornBlue}{#1}}
\newcommand*{\inputa}{\tuple{-4}{1}}
\newcommand*{\inputax}{\tuple{\highlight{-4}}{1}}
\newcommand*{\inputay}{\tuple{-4}{\highlight{1}}}
\newcommand*{\outputa}{\langle \outputvaluea\rangle} \newcommand*{\outputvaluea}{3}
\newcommand*{\inputb}{\tuple{-4}{2}}
\newcommand*{\inputbx}{\tuple{\highlight{-4}}{2}}
\newcommand*{\inputby}{\tuple{-4}{\highlight{2}}}
\newcommand*{\outputb}{\langle \outputvalueb\rangle} \newcommand*{\outputvalueb}{3}
\newcommand*{\inputc}{\tuple{-4}{3}}
\newcommand*{\inputcx}{\tuple{\highlight{-4}}{3}}
\newcommand*{\inputcy}{\tuple{-4}{\highlight{3}}}
\newcommand*{\outputc}{\langle \outputvaluec\rangle} \newcommand*{\outputvaluec}{3}
\newcommand*{\inputd}{\tuple{ 1}{1}}
\newcommand*{\inputdx}{\tuple{\highlight{ 1}}{1}}
\newcommand*{\inputdy}{\tuple{ 1}{\highlight{1}}}
\newcommand*{\outputd}{\langle \outputvalued\rangle} \newcommand*{\outputvalued}{0}
\newcommand*{\inpute}{\tuple{ 1}{2}}
\newcommand*{\inputex}{\tuple{\highlight{ 1}}{2}}
\newcommand*{\inputey}{\tuple{ 1}{\highlight{2}}}
\newcommand*{\outpute}{\langle \outputvaluee\rangle} \newcommand*{\outputvaluee}{1}
\newcommand*{\inputf}{\tuple{ 1}{3}}
\newcommand*{\inputfx}{\tuple{\highlight{ 1}}{3}}
\newcommand*{\inputfy}{\tuple{ 1}{\highlight{3}}}
\newcommand*{\outputf}{\langle \outputvaluef\rangle} \newcommand*{\outputvaluef}{2}
\newcommand*{\tracea}{\inputa\to\outputa}
\newcommand*{\traceax}{\inputax\to\outputa}
\newcommand*{\traceay}{\inputay\to\outputa}
\newcommand*{\traceb}{\inputb\to\outputb}
\newcommand*{\tracebx}{\inputbx\to\outputb}
\newcommand*{\traceby}{\inputby\to\outputb}
\newcommand*{\tracec}{\inputc\to\outputc}
\newcommand*{\tracecx}{\inputcx\to\outputc}
\newcommand*{\tracecy}{\inputcy\to\outputc}
\newcommand*{\traced}{\inputd\to\outputd}
\newcommand*{\tracedx}{\inputdx\to\outputd}
\newcommand*{\tracedy}{\inputdy\to\outputd}
\newcommand*{\tracee}{\inpute\to\outpute}
\newcommand*{\traceex}{\inputex\to\outpute}
\newcommand*{\traceey}{\inputey\to\outpute}
\newcommand*{\tracef}{\inputf\to\outputf}
\newcommand*{\tracefx}{\inputfx\to\outputf}
\newcommand*{\tracefy}{\inputfy\to\outputf}

\paragraph{First Impact Definition {\normalfont(\texorpdfstring{$\outcomesname$}{Outcomes})}.}
\labsec{overview:outcomes}
%
The first impact definition that we consider is
 $\outcomes(\defprogram)$, %(derived from $\outcomesentropy$),
where $\definputvariable$ is the input variable of interest and $\defprogram$ the program under analysis. Intuitively\sidenote{the formal definition is given in \refsec{quantitative-input-data-usage}}, $\outcomes$ returns the maximum number of outputs that are reachable from value variations of the input variable $\definputvariable$.
For the program
$\landingprogram$, the result is shown in column $\outcomesname(\landingprogram)$~of \reftab{overview}:
we obtain $\outcomesname_\x(\landingprogram)=2$ and $\outcomesname_\y(\landingprogram)=3$.

\newcommand*{\labelrotationangle}{30}
\begin{table*}[t]
  \centering
  \caption{Impact of for $\outcomesname(\landingprogram)$  and $\rangename(\landingprogram)$ definitions for both $\x$ and $\y$ variables. Computational features are \highlight{highlighted in blue}.}
  \labtab{overview}
  \begin{tabular}{c|c|c|c|c|c}
  \rotatebox[origin=c]{\labelrotationangle}{\textsc{Variable}}~ & ~\rotatebox[origin=c]{\labelrotationangle}{$\exampleinput[\landingprogram]$}~ & ~\textsc{Relevant Traces}~ & ~\rotatebox[origin=c]{\labelrotationangle}{\textsc{Outputs}}~ & ~\rotatebox[origin=c]{\labelrotationangle}{$\outcomesname$}~ & ~\rotatebox[origin=c]{\labelrotationangle}{$\rangename$}~ \\
  \toprule
  \multirow{6}{*}{\x}
   & $\inputax$ & $\traceax, \tracedx$ & $\{\outputvaluea,\outputvalued\}$ & \multirow{6}{*}{$2$} & \multirow{6}{*}{$3$} \\
  \cline{2-4}
   & $\inputbx$ & $\tracebx, \traceex$ & $\{\outputvalueb,\outputvaluee\}$ & & \\
  \cline{2-4}
   & $\inputcx$ & $\tracecx, \tracefx$ & $\{\outputvaluec,\outputvaluef\}$ & & \\
   \cline{2-4}
   & $\inputdx$ & $\tracedx, \traceax$ & $\{\outputvalued,\outputvaluea\}$ & & \\
   \cline{2-4}
   & $\inputex$ & $\traceex, \tracebx$ & $\{\outputvaluee,\outputvalueb\}$ & & \\
   \cline{2-4}
   & $\inputfx$ & $\tracefx, \tracecx$ & $\{\outputvaluef,\outputvaluec\}$ & & \\
  \midrule
  \multirow{12}{*}{\y}
   & \multirow{2}{*}{$\inputay$} & $\traceay, \traceby,$ & \multirow{2}{*}{$\{\outputvaluea\}$} & \multirow{12}{*}{$3$} & \multirow{12}{*}{$2$} \\
   & & $\tracecy$ & & & \\
  \cline{2-4}
   & \multirow{2}{*}{$\inputby$} & $\traceay, \traceby,$ & \multirow{2}{*}{$\{\outputvaluea\}$} & & \\
   & & $\tracecy$ & & & \\
  \cline{2-4}
   & \multirow{2}{*}{$\inputcy$} & $\traceay, \traceby,$ & \multirow{2}{*}{$\{\outputvaluea\}$} & & \\
   & & $\tracecy$ & & & \\
   \cline{2-4}
   & \multirow{2}{*}{$\inputdy$} & $\tracedy, \traceey,$ & \multirow{2}{*}{$\{\outputvalued,\outputvaluee, \outputvaluef\}$} & & \\
   & & $\tracefy$ & & & \\
   \cline{2-4}
   & \multirow{2}{*}{$\inputey$} & $\tracedy, \traceey,$ & \multirow{2}{*}{$\{\outputvalued,\outputvaluee, \outputvaluef\}$} & & \\
   & & $\tracefy$ & & & \\
   \cline{2-4}
   & \multirow{2}{*}{$\inputfy$} & $\tracedy, \traceey,$ & \multirow{2}{*}{$\{\outputvalued,\outputvaluee, \outputvaluef\}$} & & \\
   & & $\tracefy$ & & \\
   \bottomrule
  \end{tabular}
\end{table*}
The conclusion is that $\y$ has a greater influence than $\x$ on the output of the program.

\paragraph{Second Impact Definition {\normalfont(\texorpdfstring{$\rangename$}{Range})}.}
\labsec{overview:range}
%
The second impact definition we propose is $\rangename_\definputvariable$, which yields the maximum difference between the maximum and the minimum outputs that are reachable from value variations of the input variable $\definputvariable$.
The result for program $\landingprogram$ is shown in column $\rangename(\landingprogram)$~of \reftab{overview}:
the range of reachable outputs from variations of $\x$ is, at most, the interval $[0, 3]$, with a length of 3. Instead, the range of reachable outputs from variations of $\y$ is, at most, the interval $[0, 2]$, with a length of 2. Therefore, we obtain $\rangename_\x(\landingprogram)=3$ and $\rangename_\y(\landingprogram)=2$.
In other words, varying the angle of approach might drastically alter the landing risk, whereas the speed has less influence.
%
This is in contrast to the conclusion of \outcomesname{} where $\y$ has a greater impact than $\x$.
Although it may seem counterintuitive at first, the difference between the two impact instances is due to the different program traits they explore.
$\rangename$ quantifies over the variance in the extreme values of the set of output values, while $\outcomesname$ quantifies over the variance in the number of unique output values.
Consequently, changes in $\x$ yield a bigger variation in the degree of risk compared to $\y$, while changes in $\y$ reach far more risk levels compared to $\x$.
%
Note that, the impact definitions presented above are not computationally practical as they rely on a complete enumeration of all possible input configurations.
% Note that, enumerating all possible input configurations is not computationally practical.
Specifically, when dealing with more complex input space compositions, this approach is highly inefficient or even infeasible (as in the case of continuous input spaces).
As a consequence, our approach is based on an abstraction of input-output relations, which allows us to automatically infer a sound upper bound on the program's impact.

\paragraph{Static Analysis.}
\labsec{overview:static-analysis}

To quantify the impact of a program, one can rely solely on the input-output observations of the program.
Thus, our approach is based on an abstraction of input-output relations, which allows us to automatically infer a sound upper bound on the program's impact.

The analysis starts with a set of output abstractions called \textit{output buckets}.
A bucket is an abstract element representing a set of output states.
While this abstraction may limit the ability to precisely reason about the impact of output values within the same bucket, it permits automatic reasoning across different buckets.
Afterwards, an abstract interpretation-based static analyzer propagates each output bucket backward through the program under consideration.
The analyzer returns an abstract element for each output bucket, representing an over-approximation of the set of input configurations that lead to the output values inside the starting bucket.
This result contains also spurious input configurations that may not lead to a value inside the output bucket.
Based on the chosen impact definition \impactwrappername{} (e.g., \rangename{} or \outcomesname), we perform computations and comparisons on the abstract elements returned by the analysis to obtain an upper bound $\defbound'$. This upper bound is sound by construction of the theoretical framework, meaning that if $\defbound$ is the real (concrete) impact quantity obtained by \impactwrappername, then $\defbound\le\defbound'$.

Assuming \rangename{} as our impact of interest, given the set of abstract elements returned from the backward analysis (one for each bucket), we first project away the input variable $\definputvariable$ under consideration.
This means that our abstract elements now represent input configurations without the variable $\definputvariable$.
To be precise, we obtain \textit{partial input configurations} taking into account all possible variations of the input $\definputvariable$.
After the projection, we check for intersections among the abstract elements.
Any intersection may indicate two input configurations that only differ in the value of the variable $\definputvariable$, leading to two different output buckets.

Continuing our landing alarm system example, we start with three output buckets: $b_0=\{-1, 0\}$, $b_1=\{1, 2\}$, and $b_2=\{3, 4\}$ for instance.
Note that, only $\{0, 1, 2, 3\}$ are reachable \z{} values, but we may not have the exact post-condition of a program.
Therefore, for each of these three buckets, we run the backward analysis.
The backward analysis is parametric in the choice of the abstract domain, for instance, the disjunctive polyhedra domain based on the convex polyhedra domain~\sidecite{Cousot1978}.
Without showing the details of the analysis, starting with the bucket $b_0$, only the first branch of the if statement can be processed, \refline{low-risk}.
Therefore, \lc{} should be smaller than 2 at the end of \refline{compute-risk}.
Consequently, the sum of the absolute values of \x{} and \y{} should be smaller than 2.
Thus, our analysis discovers that only the input configuration $\tuple{1}{1}$ leads to the first bucket.
Similarly, we discover that from the bucket $b_1$, we reach the input configurations $\tuple{1}{2}$ and $\tuple{1}{3}$.
Lastly, from the bucket $b_2$, we reach the input configurations $\tuple{-4}{1}$, $\tuple{-4}{2}$, and $\tuple{-4}{3}$.

Regarding the case in which we quantify the impact of \x{}, we remove \x{} from each of the abstract elements returned by the analysis.
For example, from the bucket $b_0$, we have the abstract input configuration $\langle 1\rangle$\sidenote{The analysis discovers that only the input configuration $\tuple{1}{1}$ leads to the first bucket $b_0$. Thus, by removing \x{} from the input configuration, we obtain the tuple of a single element $\langle 1\rangle$, we use this notation for the rest of the section.}
; from the bucket $b_1$, we have $\langle 2\rangle$ and $\langle 3\rangle$; and from the bucket $b_2$, we have $\langle 1\rangle$, $\langle 2\rangle$, and $\langle 3\rangle$.
By checking the intersections, we find that bucket $b_0$ intersects with bucket $b_2$ since they have $\langle 1\rangle$ in common.
Bucket $b_2$ also intersects with bucket $b_1$ since they have both $\langle 2\rangle$ and $\langle 3\rangle$ in common.
Understanding the meaning of intersections is crucial.
For example, the first intersection between buckets $b_0$ and $b_2$ means that there exist two input configurations (namely $\tuple{1}{1}$ and $\tuple{-4}{1}$) that only differ in the value of \x{} (from $\x=1$ in the first configuration $\tuple{1}{1}$ to $\x=-4$ in the second configuration $\tuple{-4}{1}$), where the first configuration leads to an output in bucket $b_0$ and the second configuration leads to an output in the other bucket $b_2$.
Therefore, we are able to assign a range to this variation by taking the minimum of bucket $b_0$ and the maximum of bucket $b_2$, resulting in the range $[-1,4]$ with a size of 5.
Similarly, we obtain the range $[1,4]$ with a size of 3 from the other intersection between buckets $b_1$ and $b_2$.
Finally, we return the maximum among all possible sizes, which is 5.
This result is an over-approximation of the impact value for the function $\rangename_\x(\landingprogram)$.

Similarly, from the result of the backward analysis, we repeat the steps to quantify the impact regarding the other input variable \y{}.
Therefore, the analysis result is an impact of 3 since the only buckets intersecting after projecting away \y{} are buckets $b_0$ and $b_1$, with abstract input configurations of $\langle 1\rangle$ for both, while bucket $b_2$ contains only the value $\langle -4\rangle$.

Regarding the impact definition \outcomesname, whenever we discover intersections among different buckets, we count the total number of different outcome values they carry.
Consequently, we find that the two intersections for both variables \x{} and \y{} hold four different outcome values in total.
In such case, our analysis would conclude that the two variables hold the same impact.
Indeed, the choice of the starting output buckets is critical.
For instance, with $b_0=\{0\}$, $b_1=\{1\}$, $b_2=\{2\}$, and $b_3=\{3\}$ one would recover maximum precision and obtain a useful analysis quantification.

In conclusion, the sound upper bound discovered by our analysis is always higher than the concrete one by construction of the theoretical framework.
The precision of our analysis is mostly affected by the choice of output buckets and the approximation induced by the backward analysis \sidenote{as outlined by the use cases shown in~\refsec{nfm24:experimental-results}.}

\section{Quantitative Input Data Usage}
\labsec{quantitative-input-data-usage}
\newcommand*{\x}{\texttt{angle}}
\newcommand*{\y}{\texttt{speed}}
\newcommand*{\z}{\texttt{risk}}
\newcommand*{\lc}{\texttt{landing\_coeff}}

In this section we introduce our quantitative framework with the formal definitions of \rangename{} and \outcomesname.


Our goal is to quantify the impact of a specific input variable on the computation of the program.
To this end, all the impact definition described by this work are extensional, \ie, they are based on the observation of input-output relations of program states.
Indeed, we base our definitions on the dependency semantics $\dependencysemanticsnoparam$, \refdef{dependency-semantics}.
We introduce the notion of impact, denoted by the function $\impactwrapper\in\setof\pairofstates\to\valuesposplus$, which maps program semantics to a non-negative domain of quantities, where $\definputvariable$ represents the input variable of interest in the program under analysis.

We implicitly assume the use of an \textit{output descriptor} $\outputdesc$ to determine the desired output of a program by observations on program states.
%
Specifically, $\reader\in \stateandbottom \to \valuesinf$ selects the output of interest from a given state and returns its corresponding value\sidenote{The option of returning $\pm\infty$ from the output descriptor is to deal with infinite traces, which do not have a final state ($\retrieveoutput\defseq = \statebottom$ for any $\defseq \in \infinitesequences$).}.
Additionally, $\filter\subseteq\valuesinf$ filters output states and selectively engages a subset of the potential outcomes.
Through this filtering mechanism, undesired outcomes are directly excluded, and a numerical value is ensured.

\begin{definition}[Output Descriptor]
  \labdef{output-descriptor}
  Given $\reader\in\stateandbottom\to\valuesinf$ and $\filter\subseteq\valuesinf$, the tuple $\outputdesc$ is called an \textup{output descriptor}.
\end{definition}
% \marginprop{output-descriptor}

The above output characterization $\outputdesc$ is generic enough to cover plenty of use cases.
We leverage this output descriptor to provide the end user of the framework the flexibility to choose the interpretation and meaning of program outputs, without establishing it beforehand.

\begin{example}
\marginprop{landing-alarm-system}

  Consider the~\refprog{landing-alarm-system} for the landing alarm system (provided on the side of the page) with program states $\state=\setdef{\langle a, b, c, d \rangle}{a\in\{-4,1\}\land b\in\{1,2,3\}\land c\in\N \land d\in\{0,1,2,3\}}$, where $a$ is the value of $\x$, $b$ of $\y$, $c$ of $\lc$, and $d$ of $\z$.
  Here, we abuse the notation and use $\state$ as set of tuples instead of a map between variables and values, the two views are equivalent.
  The output descriptor is instantiated with
  \[
  \reader(x) \DefeQ \begin{cases}
    d & \text{if } x = \langle a, b, c, d \rangle \\
    +\infty & \text{otherwise}
  \end{cases}
  \]
  and $\filter=\{0,1,2,3\}$ filters $+\infty$ from the possible outputs.
  In other words, we are interested in the value of $\z$ for terminating traces.

  However, the end-user of the analysis may be interested in only a subset of the possible outcomes of the program.
  For instance, only about the risk levels in $\{0, 1, 2\}$, forgetting about the value $3$.
  It is crucial that our impact definitions remain sound to the user assumption on post-conditions, even when it is under-approximating the exact one.
  Thus, the filter specifies this information by $\filter=\{0, 1, 2\}$, which is a subset of all the possible values of the output variable \z.
\end{example}

\begin{example}
  For other contexts, rather than considering a single output variable one may be interested in a custom operation.
  % For example, in the context of neural network classifiers, the output of the program is the index of the output variable with highest value.
  For example, the output of a neural network classifier is the index of the output neuron holding the highest value.
  Hence, for a network with $n+1$ output neurons, we could instantiate
  \[
    \reader(x_0,\dots,x_{w+n-1},x_{w+n})=\argmax_{0\le j\le n} x_{w+j}
  \] where the function $\argmax_j X_j$ returns the \textit{argument} $j$ of the value holding the \textit{maximum} among the indexed family $X_j$.
  The filter $\filter$ could be the set of all indices $\{0,\dots,n\}$, hence permitting all possible outcomes from the reader $\reader$.
\end{example}



We can now define our property of interest, the \textit{$\defbound$-bounded impact property} $\bounded$.
By extension, $\bounded$ is the set of dependencies such that their impact, \wrt{} to the output descriptor $\outputdesc$ and the input variable $\definputvariable$, is below the threshold $\defbound\in\valuesposplus$.
%
\begin{definition}[$\defbound$-Bounded Impact Property]
  \labdef{bounded}
  Let $\dependencytype$ be the dependency semantics, $\outputdesc$ the output descriptor, $\definputvariable$ the input variable of interest, and $\bounded\in\valuesposplus$ the threshold.
  \begin{align*}
    \BOUNDED \DefeQ& \setdef
    {\dependencysemanticsnoparam \in \dependencytype\\ & \qquad}
    {\impactwrapper(\setdef{\inputoutputtuple\defstate\in\dependencysemanticsnoparam}{\reader(\retrieveoutput{\defstate})\in\filter}) \impactsubseteq \defbound}
  \end{align*}
  where $\impactwrapper$ is a parameter of the property $\bounded$ and returns the quantity computed on the given set of dependencies.
\end{definition}



%
% \marginprop{validation}
% \marginprop{collecting}
Following the definition of $\bounded$, our validation framework, \refdef*{validation}, is instantiated as
%
\begin{align}
  \labeq{bounded-soundness}
  \defprogram \satisfies \BOUNDED \IfF \collectingsemantics \subseteq \BOUNDED
\end{align}
%
We require $\impactwrapper$ to be monotonic, \ie, for any $X, Y\in \setof\finiteinfinitesequences$, it holds that $X \subseteq Y$ if and only if $\impactwrapper(X) \le \impactwrapper(Y)$.
Intuitively, this ensures that an impact applied to an over-approximation of the program semantics can only produce a higher quantity, enabling the definition of a sound terminating static analysis.
%
Next, we formalize the already introduced impact metrics \outcomesname{} and \rangename.

\subsection{The \outcomesname{} impact definition}[The Outcomes Impact Definition]
\labsec{outcomes}
%
Formally $\outcomes\in\setof\pairofstates\to\Nplus$ counts the number of different output values, allowed by the output descriptor, reachable by varying the input variable $\definputvariable\in\inputvariables$.
First, we define step-by-step the quantity $\outcomes$,
followed by the instantiation of this quantity within the context of example of \refprog{landing-alarm-system}.
We present the formal definition at the end.


Intuitively,
for any possible input configuration $\definput\in\reducedstate$, we collect all the traces that are starting from an input configuration that is a variation of $\definput$ on the input variable $\definputvariable$, \ie, $\setdef{
  \defseq\in \defsetoftraces
}{
  \retrieveinput{\defseq} \stateeq{\inputvariableswithouti} \definput
}$, where $\defsetoftraces\in\setof\finiteinfinitesequences$.
Then, we collect the output values of this set of traces by means of the output descriptor $\outputdesc$, \ie, $\setdef{
  \reader(\retrieveoutput\defseq)
}{
  \defseq \in \defsetoftraces \land
    \retrieveinput{\defseq} \stateeq{\inputvariableswithouti} \definput
}$. Specifically, this set contains all the output readings performed by $\reader$.
%
Afterwards, we extract the number of elements via the cardinality operator $\cardinality{\cdot}$.
Finally, we iterate through each input configuration $\definput$ and return the maximum value to ensure the greatest impact is preserved.

\begin{example}[Landing Alarm System]
  \label{ex:range}
  \newcommand*{\inputa}{\tuple{-4}{1}} \newcommand*{\outputa}{\langle \outputvaluea\rangle} \newcommand*{\outputvaluea}{3}
  \newcommand*{\inputb}{\tuple{-4}{2}} \newcommand*{\outputb}{\langle \outputvalueb\rangle} \newcommand*{\outputvalueb}{3}
  \newcommand*{\inputc}{\tuple{-4}{3}} \newcommand*{\outputc}{\langle \outputvaluec\rangle} \newcommand*{\outputvaluec}{3}
  \newcommand*{\inputd}{\tuple{ 1}{1}} \newcommand*{\outputd}{\langle \outputvalued\rangle} \newcommand*{\outputvalued}{0}
  \newcommand*{\inpute}{\tuple{ 1}{2}} \newcommand*{\outpute}{\langle \outputvaluee\rangle} \newcommand*{\outputvaluee}{1}
  \newcommand*{\inputf}{\tuple{ 1}{3}} \newcommand*{\outputf}{\langle \outputvaluef\rangle} \newcommand*{\outputvaluef}{2}
  \newcommand*{\tracea}{\inputa\to\outputa}
  \newcommand*{\traceb}{\inputb\to\outputb}
  \newcommand*{\tracec}{\inputc\to\outputc}
  \newcommand*{\traced}{\inputd\to\outputd}
  \newcommand*{\tracee}{\inpute\to\outpute}
  \newcommand*{\tracef}{\inputf\to\outputf}
  Let us revisit the example of the landing alarm system, with program states $\state=\setdef{\langle a, b, c, d \rangle}{a\in\{-4,1\}\land b\in\{1,2,3\}\land c\in\N \land d\in\Nle{3}}$.
  The input variables are $\inputvariables = \{\x,\y\}$, consequently the input configurations are
  $\reducedstate=\{\inputa, \inputb, \inputc, \inputd, \inpute, \inputf\}$.
%
  We begin by considering $\definputvariable=\x$ and $\inputa$ as the first input configuration to be explored.
  Hence, we collect all traces that are
  starting from an input configuration that is a variation of $\inputa$, \ie, $\setdef{
    \defseq \in \tracesemantics
  }{
    \retrieveinput{\defseq} \stateeq{\inputvariables\setminus\{\x\}} \inputa
  }$, where $\inputvariables\setminus\{\x\} = \{\y\}$ and consequently $\retrieveinput{\defseq} \stateeq{\{\y\}} \inputa$ holds whenever the initial state of $\defseq$ has $\y=1$. A possible trace of this set is $\langle 1, 1, 0, 0\rangle \to \langle 1, 1, 2, 0\rangle\to\langle 1, 1, 2, 0\rangle$ where, at the beginning, we randomly assign $\lc=0$ and $\z=0$, respectively the third and fourth component of the initial state.
%
  We collect the output values of this set of traces, $\setdef{
    \reader(\retrieveoutput\defseq)
  }{
    \defseq \in \tracesemantics \land
      \retrieveinput{\defseq}(\y) = 1
  }$.
  As a result, we obtain the set of output values $\{0, 3\}$.
  For instance, the output value $0$ is the result of the trace we exhibited previously, where the last state is $\langle 1, 1, 2, 0\rangle$ and thus the $\z$ variable of this trace is the last component with value $0$.
%
  Finally, the cardinality operator returns the value $2$, $\cardinality{\{0, 3\}} = 2$.
  By doing so for all possible input configurations in $\reducedstate$, we obtain $\outcomesname_{\x}(\tracesemantics)=3$.
  \reftab{range-x} and~\reftab{range-y} in overview (\refsec{overview}) illustrate the steps for $\outcomesname_{\x}(\tracesemantics)$ and $\outcomesname_{\y}(\tracesemantics)$ respectively.
\end{example}

\begin{definition}[\outcomesname]\labdef{outcomes}
  Given an input variable $\definputvariable\in\inputvariables$, and an output descriptor $\outputdesc$,
  the quantity $\outcomes\in\setof\finiteinfinitesequences\to\Rposplus$ is defined as
  %
  \begin{align}
    \labeq{outcomes}
    \outcomes(\defsetoftraces) &\DefeQ \sup_{\definput\in\reducedstate}
      \cardinality{\setdef{
        \reader(\retrieveoutput{\defseq})
      }{
        \defseq \in \defsetoftraces \land \retrieveinput{\defseq} \stateeq{\inputvariableswithouti} \definput
      }}
  \end{align}
  where $\sup(X)$ is the supremum operator, \ie, the smallest $q$ such that $q\ge x$ for all $x\in X$.
\end{definition}

It is easy to note that $\outcomes(\defsetoftraces)$ uses only input-output states (\cf, $\retrieveinput{\defseq}$ and $\retrieveoutput{\defseq}$) of traces $\defseq\in\defsetoftraces$.
Therefore, assuming $\defsetoftraces, \defsetoftraces'\in\setof\finiteinfinitesequences$ share the same set of input-output observations, $\outcomes(\defsetoftraces) = \outcomes(\defsetoftraces')$ holds.
This implies that, $\bounded$ is an extensional property when $\outcomes$ is used to instantiate $\impactwrapper$ in $\bounded$.
Furthermore, this impact definition is monotone in the amount of traces. That is, the more traces in input, the higher the impact as only more dependencies can satisfy the condition of \refeq{outcomes}, \cf~$\retrieveinput\defstate \stateeq{\inputvariableswithouti} \definputvariable$, and hence increase the number of outcomes.


\subsection{The \rangename{} impact definition}[The Range Impact Definition]
\labsec{range}

%
The quantity $\range\in\setof\pairofstates\to\Rposplus$ determines the
length of the range of output values from all the possible variations in the input variable $\definputvariable\in\inputvariables$.
%
This definition employs the auxiliary function $\length\in\setof\valuesinf\to\valuesposplus$, defined as follows:
 $\length(X) \defeq \sup\;{X} - \inf\;{X}$ if $X\neq\emptyset$, where $\sup$ and $\inf$ are the supremum and infimum operators, while $\length(X)\defeq0$ otherwise.

\begin{example}[Landing Alarm System]
  \label{ex:range}
  We revisit again the example of the landing alarm system.
  Assuming $\definputvariable=\x$, $\tuple{-4}{1}$ is the first input configuration to be explored, we collect all traces that are
  starting from an input configuration that is a variation of $\tuple{-4}{1}$.
  As before, we obtain $\setdef{
    \retrieveoutput\defseq(\z)
  }{
    \defseq \in \defsetoftraces \land
      \retrieveinput{\defseq}(\y) = 1
  }=\{0,3\}$.
%
  Here, we apply the length function, hence $\length(\{0,3\})=3$.
  By doing so for all possible input configurations $\reducedstate$, we obtain $\rangename_{\x}(\tracesemantics)=2$.
  \reftab{range-count} illustrates the steps for both $\rangename_{\x}(\tracesemantics)$ and $\rangename_{\y}(\tracesemantics)$.
\end{example}

  Formally,
\begin{definition}[\rangename]\labdef{range}
  Given an input variable $\definputvariable\in\inputvariables$, and an output descriptor $\outputdesc$,
  the quantity $\range\in\dependencytype\to\Rposplus$ is defined as
  %
  \begin{align}
    \labeq{range}
    \range(\defsetoftraces) &\DefeQ \sup_{\definput\in\reducedstate}
      \length(\setdef{
        \reader(\retrieveoutput{\defseq})
      }{
        \defseq \in \defsetoftraces \land \retrieveinput{\defseq} \stateeq{\inputvariableswithouti} \definput
      })
  \end{align}
\end{definition}

This impact definition is monotone as well, in the amount of traces.
\begin{lemma}[\rangename{} is Monotonic]
  \lablemma{range-monotonic}
  For any $\defsetoftraces, \defsetoftraces'\in\dependencytype$, it holds that:
  \begin{align}
    \labeq{range-monotonic}
    \defsetoftraces \subseteq \defsetoftraces' \ImplieS \range(\defsetoftraces) \le \range(\defsetoftraces')
  \end{align}
\end{lemma}


\subsection{Filter Semantics}
\labsec{filter-semantics}
\newcommand*{\X}{X}
\newcommand*{\Y}{Y}
\newcommand*{\hiX}{\higher{X}}
\newcommand*{\hiY}{\higher{Y}}

From the dependency abstractions we derive the \textit{filter semantics}.
We exploit the output descriptor $\outputdesc$ to remove dependencies (tuple of input-output values) that are not relevant for our property, that is, not in $\filter$.

Formally, the pair of right-left adjoints $\tuple{\filterabstraction}{\filterconcretization}$ with $\filterabstraction, \filterconcretization\in\dependencytype\to\filtertype$ is defined as:
\begin{align}
  \labeq{filter-abstraction}
  \filterabstraction(\hiX)&\DefeQ\setdef{\setdef{\inputoutputtuple{\defseq}\in X}{\reader(\retrieveoutput{\defseq})\in\filter}}{X\in\hiX}\\
  \labeq{filter-concretization}
%   \filterconcretization(\hiY) &\DefeQ \setdef{\Y\setjoin\setdef{\inputoutputtuple{\defseq}}{\reader(\retrieveoutput{\defseq})\in Z\land \retrieveinput{\defseq}\in I}}{
%   \Y\in\hiY \land Z\subseteq\readerdomain\setminus\filter \land I\subseteq\state
% }
  \filterconcretization(\hiY) &\DefeQ \setdef{\Y\setjoin\setdef{\inputoutputtuple{\defseq}}{\tuple{\retrieveinput{\defseq}}{z}\in W \land \retrieveoutput{\defseq}\in \state \land \reader(\retrieveoutput{\defseq})=z}}{
  \Y\in\hiY \land W \subseteq \pair{\state}{(\readerdomain\setminus\filter)}
  }
\end{align}
where $
\filterabstraction$ abstracts away pair of states that lead to outputs that are not allowed by the output descriptor $\outputdesc$ maintaining the set-structure of $\hiX$.
% Similarly, double dot notation (written $\hihiX$) is a set of sets of sets of elements.
% The concretization $\filterconcretization(\hiY) \defeq \setdef{\X\in\setof\pairofstates}{
%   \exists \Y\in\hiY, Z\subseteq\readerdomain\setminus\filter, I\subseteq\state.~
%     \X=
%       \setdef{\inputoutputtuple{\defseq}}{\defseq\in \Y}\setjoin\setdef{\inputoutputtuple{\defseq}}{\reader(\retrieveoutput{\defseq})\in Z\land \retrieveinput{\defseq}\in I}
% }$ considers all the semantics $Y\in\setof\pairofstates$ such that it exists a semantics $\Y\in\hiY$ that shares the same input-output observations plus arbitrary observations made of discarded output values in $\readerdomain\setminus\filter$.
The concretization $\filterconcretization$ considers all the semantics $\Y\in\hiY$ extended with arbitrary observations made of discarded output values in $\readerdomain\setminus\filter$.
The abstract subset operator $\hiX \filtersubseteq \hiY$ checks subset inclusion among sets of traces allowed by the output descriptor $\outputdesc$, formally defined as:
\begin{align}
  \labeq{filter-subseteq}
  \hiX \filtersubseteq \hiY \iff
  \setdef{\X\in\hiX}{\foralldef{\inputoutputtuple{\defstate}\in\X}{\reader(\retrieveoutput{\defstate})\in\filter}}
  \subseteq
  \setdef{\Y\in\hiY}{\foralldef{\inputoutputtuple{\defstate}\in\Y}{\reader(\retrieveoutput{\defstate})\in\filter}}
\end{align}
%
Therefore, we define the following Galois connection.
%
\begin{theorem}
  \label{th:dependency-filter-galois}
  $\galoisbetweensemantics{dependency}{filter}$ is a Galois connection.
  \begin{proof}
    Given two set of sets of input-output observations $\hiX, \higher{Y}\in\setofsetof\pairofstates$ such that $\filterabstraction(\hiX)\filtersubseteq \higher{Y}$, we obtain that $\hiX\dependencysubseteq\filterconcretization(\higher{Y})$ since the concretization $\filterconcretization$ builds all the possible set of input-output observations enhanced with auxiliary observations made of output values previously discarded by $\filterabstraction(\hiX)$.
    On the other hand, we have $\hiX\dependencysubseteq\filterconcretization(\higher{Y})$.
    By abstracting we obtain $\filterabstraction(\hiX)$, which filters all semantics by removing traces that do not agree with $\outputdesc$.
    It holds that $\filterabstraction(\hiX)\filtersubseteq \higher{Y}$ since the $\filtersubseteq$ operator removes, in turn, traces that do not agree with $\outputdesc$.
  \end{proof}
\end{theorem}
% Note that, applying
% \begin{corollary}
%   Given $\hiX\in\filtertype$ such that $\foralldef{\X\in\hiX,\inputoutputtuple{\defseq}\in\X}{\reader(\retrieveoutput{\defseq})\in\filter}$, it holds that $\filterabstraction(\filterconcretization(\hiX))=\hiX$.
% \end{corollary}
%
We derive the filter semantics $\filtersemanticsnoparam\in\setofsetof\pairofstates$ as follows:
%
\begin{align}
  \labeq{filter}
  \filtersemanticsnoparam &\DefeQ \filterabstraction(\dependencysemanticsnoparam) = \filterabstraction(\{\setdef{
    \inputoutputtuple{\defseq}
  }{
    \defseq\in \tracesemanticsnoparam
  }\}) \\
    &\;=~ \{{\setdef{\inputoutputtuple{\defseq}\in \setdef{
      \inputoutputtuple{\defseq}
    }{
      \defseq\in \tracesemanticsnoparam
    }}{\reader(\retrieveoutput{\defseq})\in\filter}}\} \\
    &\;=~ \{{\setdef{\inputoutputtuple{\defseq}}{\defseq\in \tracesemanticsnoparam\land\reader(\retrieveoutput{\defseq})\in\filter}}\}
\end{align}
%
The next result shows that $\filtersemanticsnoparam$ allows sound and complete verification for proving that the impact of a program is bounded by $\defbound$.
%
\begin{theorem}
  \label{th:filter-soundness}
  $\collectingsemantics \subseteq \BOUNDED \IfF \filtersemantics \filtersubseteq \filterabstraction(\dependencyabstraction(\BOUNDED))$
  % \begin{proof}
  %   Let $\dependencysemantics \subseteq \dependencyabstraction(\bounded)$. From the Galois connection described in \refthm{dependency-filter-galois}, we have that $\filterabstraction(\dependencysemantics) \subseteq \filterabstraction(\dependencyabstraction(\bounded))$.
  %   From the definition of $\filtersemantics$, \refeq{filter}, we can conclude $\filtersemantics \subseteq \filterabstraction(\dependencyabstraction(\bounded))$. The other direction is symmetric.
  % \end{proof}
  \begin{proof}
    The implication $(\implies)$ follows from \refthm{dependency-soundness}, the monotonicity of $\filterabstraction$, \refthm{dependency-filter-galois}, and the definition of $\filtersemanticsnoparam$, \refeq{filter}, \ie,
    $
    \collectingsemanticsnoparam \subseteq \bounded \implies
      \dependencysemanticsnoparam \subseteq \dependencyabstraction(\bounded) \implies \filterabstraction(\dependencysemanticsnoparam) \filtersubseteq \filterabstraction(\dependencyabstraction(\bounded))\implies \filtersemanticsnoparam \filtersubseteq \filterabstraction(\dependencyabstraction(\bounded))
    $.
    Regarding the other implication $(\Leftarrow)$, by the definition of $\filtersemanticsnoparam$, $\dependencysemanticsnoparam$, and the property of Galois connections, we obtain
    \begin{align}
      \filtersemanticsnoparam
        \filtersubseteq &\;\, \filterabstraction(\dependencyabstraction(\bounded)) \\
      &~\implies \filterabstraction(\dependencysemanticsnoparam) \filtersubseteq \filterabstraction(\dependencyabstraction(\bounded)) &&(\text{by \refeq{filter}}) \\
      &~\implies \dependencysemanticsnoparam
        \subseteq \filterconcretization(\filterabstraction(\dependencyabstraction(\bounded))) &&(\text{by \refthm{dependency-filter-galois}}) \\
      &~\implies \setdef{\inputoutputtuple{\defseq}}{\defseq\in\tracesemanticsnoparam} \in \filterconcretization(\filterabstraction(\dependencyabstraction(\bounded))) &&(\text{by \refeq{dependency}})
    \end{align}
    We split two cases depending whether $\setdef{\inputoutputtuple{\defseq}}{\defseq\in\tracesemanticsnoparam}$ contains only traces allowed by $\outputdesc$:
    \begin{enumerate}[label=(\roman*)]
      \item $\foralldef{\defseq\in\tracesemanticsnoparam}{\reader(\retrieveoutput{\defseq})\in\filter}$ holds. In this case it is easy to note that $\setdef{\inputoutputtuple{\defseq}}{\defseq\in\tracesemanticsnoparam} \in \dependencyabstraction(\bounded)$ since $\filterconcretization$ and $\filterabstraction$ do not affect semantics that already satisfy the output descriptor. \label{first-case}
      \item $\existsdef{\defseq\in\tracesemanticsnoparam}{\reader(\retrieveoutput{\defseq})\not\in\filter}$ holds. Given $\defsetoftraces\subseteq\setdef{\inputoutputtuple{\defseq}}{\defseq\in\tracesemanticsnoparam}$ such that any trace $\defseq\in\defsetoftraces$ is allowed by $\reader(\retrieveoutput{\defseq})\in\filter$, from the previous case it is evident that $\defsetoftraces\in\dependencyabstraction(\bounded)$. By the definition of $\bounded$, \refeq{bounded}, it holds that any superset $\defsetoftraces'\supseteq\defsetoftraces$ belongs to $\bounded$ when $\defsetoftraces$ contains only traces allowed by the output descriptor and $\defsetoftraces\in\bounded$. Hence, it holds that $\setdef{\inputoutputtuple{\defseq}}{\defseq\in\tracesemanticsnoparam} \in \dependencyabstraction(\bounded)$. \label{second-case}
    \end{enumerate}
    In both cases~\ref{first-case} and~\ref{second-case} we obtain $\setdef{\inputoutputtuple{\defseq}}{\defseq\in\tracesemanticsnoparam} \in \dependencyabstraction(\bounded)$.
    The conclusion $\collectingsemanticsnoparam \subseteq \bounded$ trivially follows from the definition of $(\subseteq)$ and \refthm{dependency-soundness}.
  \end{proof}
\end{theorem}

This semantics still contains enough information to soundly and completely prove our any property $\bounded$, therefore potentially undecidable.
% Next step is to define sound-only semantics that are approximating the property of interest therefore leading to an impact bound which is higher than the actual yet undecidable one.

\section{A Static Analysis for Quantitative Input Data Usage}
\labsec{static-analysis}

%
In this section, we introduce a sound computable static analysis to determine an upper bound on the impact of an input variable $\definputvariable$.
The soundness of the approach leverages two elements: $(1)$ an underlying abstract semantics $\backwardsemanticsnoparam$ to compute an over-approximation of the filter semantics $\filtersemanticsnoparam$; and $(2)$ a sound computable implementation of $\impactwrapper$, written $\impactinstance$, used in the property $\bounded$.

To quantify the usage of an input variable, we need to determine the input configurations leading to specific output values.
As our impact definitions $\outcomes$ and $\range$ measure over the different output values (i.e., $\reader(\retrieveoutput{\defstate})$) our underlying abstract semantics will be a \emph{backward} (co-)reachability semantics starting from \emph{disjoint} abstract post-conditions, over-approximating the (concrete) output values of the dependency semantics.
Specifically, we abstract the concrete output values with an indexed set $\buckets\in\vectorbuckets$ of $n$ disjoint \textit{output buckets}, where $\abstractdomainlattice$ is an abstract state domain with concretization function  $\abstractdomainconcretization\in\abstractdomain\to\setof\stateandbottom$. The choice of these output buckets is essential for obtaining a precise and meaningful analysis result.

For each output bucket $\bucket\in\abstractdomain$, our analysis computes an over-approximation of the dependency semantics restricted to the input configurations leading to $\abstractdomainconcretization(\bucket)$.
More formally, let $\reduce[\dependencysemanticsnoparam]{X} \defeq \setdef{\inputoutputtuple{\defstate}\in\dependencysemanticsnoparam}{\retrieveoutput{\defstate}\in X}$ be the reduction of the dependency semantics $\dependencysemanticsnoparam$ to the dependencies with final states in $X$.
%
Our static analysis is parametrized by an underlying backward abstract family\sidenote{A family of semantics is a set of program semantics parametrized by an initialization.}
of semantics $\backwardsemantics\in\backwardtype$ which computes the backward semantics $\backwardsemantics\bucket$ from a given output bucket $\bucket\in\abstractdomain$.
The concretization function $\backwardconcretization\in(\backwardtype)\to\abstractdomain\to\setof\pairofstates$ employs %the abstract concretization
$\abstractdomainconcretization$ to restore all possible input-output dependencies, \ie, $\backwardconcretization(\backwardsemantics)\bucket \defeq \setdef{\inputoutputtuple{\defstate}}{\retrieveinput{\defstate}\in\abstractdomainconcretization(\backwardsemantics\bucket)\land\retrieveoutput{\defstate}\in\abstractdomainconcretization(\bucket)}$.
We can thus define the soundness condition for the backward semantics with respect to the reduction of the dependency semantics.

\begin{definition}[Sound Over-Approximation for \texorpdfstring{$\backwardsemanticsnoparam$}{the backward semantics}]\label{def:sound-over-approximation}
  For all programs $\defprogram$, and output bucket $\bucket\in\abstractdomain$, the family of semantics $\backwardsemanticsnoparam$ is a \textup{sound over-approximation} of the dependency semantics $\dependencysemanticsnoparam$ reduced with  $\abstractdomainconcretization(\bucket)$, when it holds that:
  \[\reduceddependencysemantics \SubseteQ \backwardconcretization(\backwardsemantics)\bucket\]
\end{definition}

We define
$\multisemanticsnoparam\in\multitype$ as the backward semantics repeated on a set of output buckets $\buckets\in\vectorbuckets$, that is, $\multisemantics\buckets \defeq (\backwardsemantics\bucket)_{j\le n}$.
Again, the concretization function $\multiconcretization\in(\multitype)\to\vectorbuckets\to\setof\pairofstates$ employs the abstract concretization $\abstractdomainconcretization$ to restore all possible input-output dependencies over all the output buckets, \ie,
$\multiconcretization(\multisemantics)\buckets \defeq \bigsetjoin_{j\le n} \setdef{\inputoutputtuple{\defstate}}{\retrieveinput{\defstate}\in\abstractdomainconcretization((\multisemantics\buckets)_j)\land\retrieveoutput{\defstate}\in\abstractdomainconcretization(\bucket)}$.

\begin{lemma}[Sound Over-Approximation for \texorpdfstring{$\multisemanticsnoparam$}{the multi-bucket semantics}]\label{lm:sound-over-approximation-multi-bucket}
  For all programs $\defprogram$, output buckets $\buckets\in\vectorbuckets$, and a family of semantics $\backwardsemanticsnoparam$, the %multi-bucket family of
  semantics $\multisemanticsnoparam$ is a \textup{sound over-approximation} of the dependency semantics $\dependencysemanticsnoparam$ when reduced to $\bigsetjoin_{j\le n}\abstractdomainconcretization(\bucket)$:
  \[\reduce{\bigsetjoin_{j\le n}\abstractdomainconcretization(\bucket)} \SubseteQ \multiconcretization(\multisemantics)\buckets\]
\end{lemma}

Whenever the output buckets \textit{cover} the whole output space, $\multisemanticsnoparam$ is a sound over-approximation of $\dependencysemanticsnoparam$.
The concept of covering for output buckets ensures that no final states of the dependency semantics, \ie{} $\finalstatesdependency\defeq\setdef{\retrieveoutput{\defstate}}{\inputoutputtuple\defstate\in\dependencysemanticsnoparam}$, are missed from the analysis.

\begin{definition}[Covering]\label{def:covering}
  We say that the output buckets $\buckets\in\vectorbuckets$ \textit{cover} the whole output space whenever $\finalstatesdependency\subseteq\bigsetjoin_{j\le n}\abstractdomainconcretization(\bucket)$.
\end{definition}

% \newcommand{\resultofbucketj}{\hiX_{j}}
% \newcommand{\resultofbucketjk}{\hiX_{j,k}}
% \newcommand{\resultofbucket}{\hiX}

Next, we expect a sound implementation $\impactinstance\in\pair\vectorbuckets\vectorbuckets\to\valuesinf$ to return a bound on the impact which is always higher than the concrete counterpart $\impactwrapper$.

\begin{definition}[Sound Implementation]\label{def:sound-implementation}
  For all output buckets $\buckets$ and family of semantics $\backwardsemanticsnoparam$, $\impactinstance$ is a \textup{sound implementation} of $\impactwrappername$, whenever
  \[
    \impactwrapper(
      \multiconcretization(\multisemantics)\buckets
    ) \LE \impactinstance(\multisemantics\buckets, \buckets)
  \]
\end{definition}

% \newcommand{\resultofproject}{\higher{Y}}
% \newcommand{\resultofprojectj}{\higher{Y}_{j}}
% \newcommand{\resultofprojectjk}{\higher{Y}_{j,k}}


The next result shows that our static analysis is sound when employed to verify the property of interest $\bounded$ for the program $\defprogram$.
That is, if %the computation of
$\impactinstance$ returns the bound $\defbound'$, and $\defbound'\le\defbound$, then the program $\defprogram$ satisfies the property $\bounded$, \cf{} $\defprogram \satisfies \bounded$.


\begin{theorem}[Soundness] \label{th:soundness}
  Let $\bounded$ be the property of interest we want to verify for the program $\defprogram$ and the input variable $\definputvariable\in\inputvariables$.
  Whenever,
  \begin{enumerate}[label=(\roman*)]
    \item $\backwardsemanticsnoparam$ is sound with respect to $\dependencysemanticsnoparam$, \cf{} \refdef{sound-over-approximation}, and
    \item $\buckets$ covers the whole output space, \cf{} \refdef{covering}, and
    \item $\impactinstance$ is a sound implementation of $\impactwrapper$, \cf{} \refdef{sound-implementation},
  \end{enumerate}
  the following implication holds:
  \begin{align}
    \impactinstance(\multisemantics\buckets, \buckets) = \defbound' \LanD \defbound' \le \defbound \ImplieS \defprogram \satisfies \bounded
  \end{align}
\end{theorem}

Finally,
we define $\abstractrange$ and $\abstractoutcomes$
as possible implementations for $\range$ and $\outcomes$, respectively.
%
We assume the underlying abstract state domain $\abstractdomain$ is equipped with an
operator $\abstractdomainproject\in\abstractdomain\to\abstractdomain$
to project away the input variable $\definputvariable$.
For example, in the context of the interval domain, where each input variable is related to a possibly unbounded lower and upper bound, $\abstractdomainproject(\langle\definputvariable \mapsto [1, 3], j \mapsto [2, 4]\rangle) = \langle \definputvariable \mapsto [-\infty, \infty], j \mapsto [2, 4] \rangle$
removes the constraints related to $\definputvariable$.

The definition of $\abstractoutcomes$ first projects away the input variable $\definputvariable$ from all the given abstract values, then it collects all intersecting abstract values via the meet operator $\abstractdomainmeet$.
These intersections represent potential concrete input configurations where variations on the value of $\definputvariable$ lead to changes of program outcome, from a bucket to another.
We return the maximum number of abstract values that intersects after projections:
\begin{equation}\label{eq:abstract-outcomes}
  \abstractoutcomes(X^\natural, \buckets) \defeq \max~\setdef{\cardinalitynospaces{J}}{J \in \intersectallfunction((\abstractdomainproject(X^\natural_j))_{j\le\numberofbuckets})}
\end{equation}
Note the use of $\max$ instead of $\sup$ as in the concrete counterpart (\refeq{outcomes}) since the number of intersecting abstract values is bounded by $n$, i.e., the number of output buckets.
The function $\intersectallfunction$ takes as input an indexed set of abstract values and returns the set of indices of abstract values that intersect together, defined as follows:
\begin{equation*}
  \intersectallfunction(X^\natural\in\vectorbuckets) \defeq \setdef{J}{J \subseteq \N \land \forall j\le n, p\le n.~ j\in J \land p\in J \LanD X^\natural_j \abstractdomainmeet X^\natural_{p}}
\end{equation*}
Finding all the indices of intersecting abstract values is equivalent to find cliques in a graph, where each node represents an abstract value and an edge exists between two nodes if and only if the corresponding abstract values intersect.
Therefore, $\intersectallfunction$ can be efficiently implemented based on the graph algorithm by~\sidecite{Bron1973}.

Similarly, we define $\abstractrange$ as the maximum length of the range of the extreme values of the buckets represented by intersecting abstract values after projections.
In such case, we assume $\abstractdomain$ is equipped with an additional abstract operator $\abstractdomainlength\in\abstractdomain\to\valuesposplus$, which returns the length of the given abstract element, otherwise $+\infty$ if the abstract element is unbounded or represents multiple variables.
\begin{align}\label{eq:abstract-range}
  \abstractrange(X^\natural, \buckets) \DefeQ& \max~\seTDef{\abstractdomainlength(K)}{K \in I} \\
  \text{where}~
    I ~=~& \seTDef{\abstractdomainjoin\seTDef{\bucket}{j\in J}}{J \in \intersectallfunction((\abstractdomainproject(X^\natural_j))_{j\le\numberofbuckets})}
\end{align}

\section{Experimental Results}
\labsec{nfm24:experimental-results}
\newcommand*{\x}{\texttt{angle}}
\newcommand*{\y}{\texttt{speed}}
\newcommand*{\z}{\texttt{risk}}


The goal of this section is to highlight the potential of our static analysis for quantitative input data usage.
We implemented a proof-of-concept tool, called \impatto\sidenote{\rurl{github.com/denismazzucato/impatto}}, in Python 3 that employs the \interproc\sidenote{\rurl{github.com/jogiet/interproc}} abstract interpreter to perform the backward analysis.
Then, we exploited this tool to automatically derive a sound input data usage of six different use cases.
As each impact result must be interpreted with respect to what the program computes, we analyze each use case separately.

\subsection{Growth in a Time of Debt}
\labsec{rr}


Reinhart and Rogoff article ``Growth in a Time of Debt''~\sidecite{Reinhart2010} proposed a correlation between high levels of public debt and low economic growth,
and %. As a consequence, the article
was heavily cited to justify austerity measures around the world. %Notably,
One of the several errors discovered in the article is the incorrect usage of the input value relative to Norway's economic growth in 1964.
The data used in the article is publicly available but not the spreadsheet file. We reconstructed this simplified example based on
the technical critique by \sidetextcite{Herndon2014}, and an online discussion\sidenote{\rurl{economics.stackexchange.com/q/18553}}.
The~\refprog{rr}
computes the cross-country mean growth for the public debt-to-GDP $60-90\%$ category, key point to the article's conclusions.
The input data is the average growth rate for each country within this public dept-to-GDP category. The problem with this computation is that Norway has only one observation in such category, which alone could disrupt the mean computation among all the countries. Indeed, the year that Norway appears in the $60-90\%$ category achieved a growth rate of $10.2\%$, while the average growth rate for the other countries is $2.7\%$.
With such high rate, the mean growth rate raised to $3.4\%$, altering the article's conclusions.
We assume growth rate values between $-20\%$ and $20\%$ for all countries, consequentially, the output ranges are between these bounds as well. We instrumented the output buckets to cover the full output space in buckets of size $1$, \ie, $\setdef{t \le \texttt{avg} < t + 1}{-20 \le t \le 20}$.
%
\newcommand{\dg}{60}
\begin{table*}[t]
  \caption{Quantitative input usage for \refprog{rr} from the Reinhart and Rogoff's article.}
  \labtab{rr}
  \centering
  \begin{tabular}{c | ccccccccccc}
    \textsc{Impact} & \rotatebox{\dg}{\texttt{portugal1}} & \rotatebox{\dg}{\texttt{portugal2}} & \rotatebox{\dg}{\texttt{portugal3}} & \rotatebox{\dg}{\texttt{norway1}} & \rotatebox{\dg}{\texttt{uk1}} & \rotatebox{\dg}{\texttt{uk2}} & \rotatebox{\dg}{\texttt{uk3}} & \rotatebox{\dg}{\texttt{uk4}} & \rotatebox{\dg}{\texttt{us1}} & \rotatebox{\dg}{\texttt{us2}} & \rotatebox{\dg}{\texttt{us3}} \\
    \toprule
    \outcomesname{} & 5 & 5 & 5 & 10 & 2 & 2 & 2 & 2 & 3 & 3 & 3 \\
    \rangename{} & 5 & 5 & 5 & 10 & 2 & 2 & 2 & 2 & 3 & 3 & 3 \\
    \bottomrule
  \end{tabular}
\end{table*}
%
Results for both \outcomesname{} and \rangename{} are shown in \reftab{rr}.

\begin{lstlisting}[
  language=customPython,
  escapechar=\%,
  caption={Program computing the mean growth rate in the $60-90\%$ category.},
  label={lst:rr},
  % float,
  % floatplacement=H
]
 def mean_growth_rate_60_90(
     portugal1, portugal2, portugal3,
     norway1,
     uk1, uk2, uk3, uk4,
     us1, us2, us3):
   portugal_avg = (portugal1 + portugal2 + portugal3) / 3%\label{l:portugal-avg}%
   norway_avg = norway1%\label{l:norway-avg}%
   uk_avg = (uk1 + uk2 + uk3 + uk4) / 4%\label{l:uk-avg}%
   us_avg = (us1 + us2 + us3) / 3%\label{l:us-avg}%
   avg = (portugal_avg + norway_avg + uk_avg + us_avg) / 4%\label{l:final-avg}%
\end{lstlisting}
%\vspace{%-15pt}
%
%
The analysis discovers that the Norway's only observation for this category $\texttt{norway1}$ has the biggest impact on the output, as perturbations on its value are capable of reaching 10 different outcomes (\cf~column $\texttt{norway1}$), while the other countries only have 5, 2, and 3, respectively for Portugal, UK, and US.
The same applies to \rangename{} as the output buckets have size $1$ and all the input perturbations are only capable of reaching contiguous buckets. Hence, we obtain the same exact results.

Our analysis is able to discover the disproportionate impact of Norway's only observation in the mean computation, which would have prevented one of the several programming errors found in the article.
%Nevertheless,
From a review of~\refprog{rr}, it is clear that Norway's only observation has a greater contribution to the computation,
%of the average growth rate,
as it does not need to be averaged with other observations first.
However, such methodological error is less evident when dealing with a higher number of input observations ($1175$ observations in the original work) and the computation is hidden behind a spreadsheet.

% As noted in many reports, a possible solution would be to improve the weighting procedure or filter outliers.


\subsection{GPT-4 Turbo}
\labsec{gpt-4-turbo}

The second use case we present is drawn from Sam Altman's OpenAI keynote in September 2023\sidenote{\rurl{www.youtube.com/live/U9mJuUkhUzk?si=HOzuH3-gr_kTdhCt&t=2330}}, where he presented the GPT-4 Turbo.
This new version of the GPT-4 language model brings the ability to write and interpret code directly without the need of human interaction.
Hence, as showcased in the keynote, the user could prompt multiple information to the model, such as related to the organization of a holiday trip with friends in Paris, and the model automatically generates the code to compute the share of the total cost of the trip and run it in background.
In this environment, users are unable to directly view the code unless they access the backend console.
This limitation makes it challenging for them to evaluate whether the function has been implemented correctly or not, assuming users have the capability to do so.
%
From the keynote, we extracted the~\refprog{share-division} which computes the user's share of the total cost of a holiday trip to Paris, given the total cost of the Airbnb, the flight cost, and the number of friends going on the trip.
%
\begin{table}
  \caption{Quantitative input usage for \refprog{share-division} computing the share division among friends.}
  \labtab{gpt-4-turbo}
  \begin{tabular}{c | ccc}
    \textsc{Impact} & \rotatebox{45}{\texttt{airbnb\_total\_cost\_eur}} & \rotatebox{45}{\texttt{flight\_cost\_usd}} & \rotatebox{45}{\texttt{number\_of\_friends}} \\
    \toprule
    \outcomesname{} & 10 & 17 & 9 \\
    \rangename{} & 1099 & 1709 & 999 \\
    \bottomrule
  \end{tabular}
\end{table}
%

\begin{lstlisting}[
  language=customPython,
  escapechar=\%,
  label={lst:share-division},
  caption={Program computing share division for holiday planning among friends.},
  % float,
  % floatplacement=H
]
 def share_division(
     airbnb_total_cost_eur,
     flight_cost_usd,
     number_of_friends):
   share_airbnb = airbnb_total_cost_eur / number_of_friends
   usd_to_eur = 0.92
   flight_cost_eur = flight_cost_usd * usd_to_eur
   total_cost_eur = share_airbnb + flight_cost_eur
\end{lstlisting}
%
Regarding the input bounds, users are willing to spend between 500 and 2000 for the Airbnb, between 50 and 1000 for the flight, and travel with between 2 and 10 friends. As a result, they expect their share, variable $\texttt{total\_cost\_eur}$, to be between 90 and 1900.
To compute the impact of the input variables we choose the output buckets to cover the expected output space in buckets of size $100$, \ie, $\setdef{100t + 90 \le \texttt{total\_cost\_eur} < \min \{100(t + 1) + 90, 1900\}}{0 \le t \le 19}$.
The %analysis discovers similar
findings are similar for both the \outcomesname{} and \rangename{} analysis, see~\reftab{gpt-4-turbo}.
The input variable $\texttt{flight\_cost\_usd}$ has the biggest impact on the output, as perturbations on its value are capable of reaching 17 different output buckets (resp. a range of 1709 output values), while the other two, $\texttt{airbnb\_total\_cost\_eur}$ and $\texttt{number\_of\_friends}$, only reach 10 and 9 output buckets (resp. have ranges of size 1099 and 999), respectively.
%Since the output buckets reached by perturbations of input values are contiguous, the \rangename{} analysis shows similar findings: 1709, 1099, and 999, respectively for \texttt{flight\_cost\_usd}, \texttt{airbnb\_total\_cost\_eur}, and \texttt{number\_of\_friends},

These results confirm the user expectations about the proposed program from ChatGPT: the flight cost yields the biggest impact as it cannot be shared among friends.


\subsection{Termination Analysis (A)}
\labsec{termination-analysis-A}

%
\begin{marginlisting}
  \caption{Example program from termination analysis.}
  \labprog{timing-analysis}
  \vspace{0.5cm}
\begin{lstlisting}[
    language=customPython,
    escapechar=\%,
    ]
 def example(x, y):
   counter = 0
   while x >= 0:
     if y <= 50:
       x += 1
     else
       x -= 1
     y += 1
     counter += 1
\end{lstlisting}
\end{marginlisting}

\refprog{timing-analysis} is adapted from the termination category of the software verification competition \textsc{sv-comp}\sidenote{\rurl{sv-comp.sosy-lab.org/}}.
Assuming both input positives, $\texttt{x},\texttt{y} \ge 0$, this program terminates in $\texttt{x}+1$ iterations if $\texttt{y} >50$, otherwise it terminates in $\texttt{x} - 2\texttt{y} + 103$ iterations.
We define $\texttt{counter}$ as the output variable, with output buckets defined as $\setdef{10k \le \texttt{counter} < 10(k+1)}{0 \le k < 50}$ and $\{\texttt{counter}\ge 500\}$. These output buckets represent cumulative ranges of iterations required for termination.
The analysis results are illustrated in~\reftab{termination-analysis}, they show that the input variable $\texttt{x}$ has the biggest impact.
Modifying the value of $\texttt{x}$ can result in the program terminating within any of the other 50 iteration ranges.
On the other hand, perturbations on $\texttt{y}$ can only result in the program terminating within 10 different iteration ranges.
Such difference is motivated by the fact that $\texttt{y}$ is only used to determine the number of iterations in the case where $\texttt{y}$ is greater than 50, otherwise it is not used at all. Therefore, two values of $\texttt{y}$, \eg, $y_0$ and $y_1$, only result in two different ranges of iterations required to make the program terminate if either both of them are below $50$ or $y_0 < 50\land y_1 \ge 50$ or $y_0\ge50\land y_1 <50$, not in all the cases.


\begin{margintable}[-4cm]
  \caption{Quantitative input usage for \refprog{timing-analysis}.}
  \labtab{termination-analysis}
  \begin{tabular}{c | c@{\hskip 5pt}c}
    \textsc{Impact} & \rotatebox{0}{\texttt{x}} & \rotatebox{0}{\texttt{y}} \\
    \toprule
    \outcomesname{} & 50 & 10 \\
    \rangename{} & 499 & 99 \\
    \bottomrule
  \end{tabular}
\end{margintable}

The given results can be interpreted as follows: the speed of termination of this loop is highly dependent on the value of $\texttt{x}$, while $\texttt{y}$ has a much smaller impact.
% This information could be used to attack such program just by looking at its timing behavior.
% For instance, given the number of iterations \texttt{counter}, we infer that the value of \texttt{x} is either $\texttt{counter} - 1$ or $\texttt{counter} + 2\texttt{y} - 103$. On the other hand, since \texttt{y} has less impact as discovered by our tool, we cannot infer much about its value.



\subsection{Termination Analysis (B)}
\labsec{app:termination-analysis-B}

This use case comes from the software verification competition SV-Comp\sidenote{\rurl{sv-comp.sosy-lab.org/}}, where the goal is to verify the termination of a program. \refprog{termination-a} and~\refprog{termination-b} have originally been proposed by \sidetextcite{Chen2012}, respectively these are Example (2.16) and Example (2.21) of such work.

\begin{marginlisting}
  \caption{Program Ex2.16 from software verification competition SV-Comp.}
  \labprog{termination-a}
  \vspace{0.5cm}
\begin{lstlisting}[
  language=customPython,
  escapechar=\%,
]
def termination_a(x, y):
  while x > 0:
    x = y
    y = y - 1
  result = x + y
  return result
\end{lstlisting}
\end{marginlisting}


\refprog{termination-a} returns the value of \texttt{y} whenever $\texttt{x} = 0$, otherwise it returns $-1$.
We assume both input variables are positive up to $1000$, $0 \le \texttt{x} \le 1000$ and $0 \le \texttt{y} \le 1000$.
Regarding such a function, it is interesting to study its behaviors around $0$, thus the output bucket are $\{ \texttt{result < 0} \}, \{ \texttt{result} = 0 \}$, and $\{ \texttt{result > 0} \}$.
With the above parameters, the analysis \outcomesname{} returns 1 for both input variables.
Such result is not too interesting, but by looking at the internal stages of the analysis we notice that perturbations on the value of the variable \texttt{x} may be able to produce from an output negative value to zero or a positive one (and viceversa).
While perturbations on the value of the variable \texttt{y} are only able to produce from zero to positive (and viceversa).

As a second experiments, we consider the buckets from -1 to 19, $\setdef{ \texttt{result} = n}{-1 \le n \le 19}$, and we notice that the analysis \outcomesname{} returns 1 for the input variable \texttt{x} and 19 for \texttt{y}, meaning that the variable \texttt{y} is able to affect far more output values than \texttt{x}. However, combing the results of the previous experiment, only the variable \texttt{x} is able to affect the negative output values.

\begin{marginlisting}[-1.4cm]
  \caption{Program Ex2.21 from software verification competition SV-Comp.}
  \labprog{termination-b}
  \vspace{0.5cm}
\begin{lstlisting}[
  language=customPython,
  escapechar=\%,
]
def termination_b(x, y):
  while x > 0:
    x = x + y
    y = -y - 1
  result = x + y
  return result
\end{lstlisting}
\end{marginlisting}

From the same work, we also consider \refprog{termination-b} which returns the value of \texttt{y} whenever $\texttt{x} = 0$, otherwise it returns $-1$.
Unfortunately, the backward analysis does not capture a precise loop invariant, thus both the analyses \outcomesname{} and \rangename are inconclusive in such case.
The key takeaway is that our analysis is highly dependent on the precision of the underlying backward analysis.

As a conclusion, even though SV-Comp proposes challenging benchmarks for termination, reachability, and safety analyses, they are not amenable for information flow analysis.
Most of the time, their examples involve loops with complex invariant, but as input-output relations, the variables involved are just zeroed out after the loop.
Drawing examples from their dataset is less appealing to our work.

\subsection{Linear Loops}
\labsec{linear-loops}


\begin{marginlisting}
  \caption{Program computing the linear expression $(5x + 2y)$ via repeated additions.}
  \labprog{linear-expression}
  \vspace{0.9cm}
\begin{lstlisting}[
  language=customPython,
  escapechar=\%,
]
def linear_expression(x, y):
  result = 0
  i = 0
  while i < 5:
    result = result + x
    i += 1
  i = 0
  while i < 2:
    result = result + y
    i += 1
\end{lstlisting}
\end{marginlisting}

\refprog{linear-expression} computes the linear expression $(5x + 2y)$ via repeated additions.
Note that the invariant of the loop is indeed non-linear ($\texttt{result} = \texttt{i} * \texttt{x}$ and $\texttt{result} = \texttt{result}' + \texttt{i} * \texttt{y}$ respectively for the first and second loop, where $\texttt{result}'$ is the value of \texttt{result} before entering the second loop), but the loop is executed a fixed number of times, thus the analysis is able to compute the exact output buckets through loop unrolling.

For the analysis the input bounds are $0 \le \texttt{x} \le 1000$ and $0 \le \texttt{y} \le 1000$, while the output buckets are $\setdef{n * 100 \le \texttt{result} < (n + 1) * 100}{n \le 70}$.
Both analyses, \outcomesname{} and \rangename, show that \texttt{x} has an impact $\frac{5}{2}$ times bigger than \texttt{y} on the output.
Thus, the impact quantity provides insight about the termination speed.
Indeed, the loop for \texttt{x} is executed 5 times, while the one for \texttt{y} only 2.


\subsection{Landing Risk System}
\labsec{landing-risk}




\begin{figure}[t]
  \centering
\begin{tikzpicture}
  % Grid
  \draw[help lines, color=gray!30, dashed] (-0.1,-0.1) grid (9.9,3.9);
  % x-axis
  \draw[->,ultra thick] (0,0)--(10,0) node[rotate=90,below]{\x};
  % y-axis
  \draw[->,ultra thick] (0,0)--(0,4) node[above]{\y};
  % x-axis ticks
  \foreach \x in {-4,-3,-2,-1,0,1,2,3,4}
      \draw (\x+5,0.1) -- (\x+5,-0.1) node[below] {\x};
  % y-axis ticks
  \foreach \y in {1,2,3}
      \draw (0.1,\y) -- (-0.1,\y) node[left] {\y};
  % Polyhedra
  \fill[color=seabornGreen, opacity=0.5] (5,3) -- (7,1) -- (3,1) -- cycle;
  \draw[color=seabornGreen, ultra thick] (5,3) -- (7,1) -- (3,1) -- cycle;
  % Polyhedra
  \fill[color=seabornYellow, opacity=0.5] (4,3) -- (5,3) -- (3,1) -- (2,1) -- cycle;
  \draw[color=seabornYellow, ultra thick] (4,3) -- (5,3) -- (3,1) -- (2,1) -- cycle;
  % % Polyhedra
  \fill[color=seabornYellow, opacity=0.5] (5,3) -- (6,3) -- (8,1) -- (7,1) -- cycle;
  \draw[color=seabornYellow, ultra thick] (5,3) -- (6,3) -- (8,1) -- (7,1) -- cycle;
  % Polyhedra
  \fill[color=seabornOrange, opacity=0.5] (3,3) -- (4,3) -- (2,1) -- (1,1) -- cycle;
  \draw[color=seabornOrange, ultra thick] (3,3) -- (4,3) -- (2,1) -- (1,1) -- cycle;
  % % Polyhedra
  \fill[color=seabornOrange, opacity=0.5] (6,3) -- (7,3) -- (9,1) -- (8,1) -- cycle;
  \draw[color=seabornOrange, ultra thick] (6,3) -- (7,3) -- (9,1) -- (8,1) -- cycle;
  % Polyhedra
  \fill[color=seabornRed, opacity=0.5] (1,3) -- (3,3) -- (1,1) -- cycle;
  \draw[color=seabornRed, ultra thick] (1,3) -- (3,3) -- (1,1) -- cycle;
  % Polyhedra
  \fill[color=seabornRed, opacity=0.5] (7,3) -- (9,3) -- (9,1) -- cycle;
  \draw[color=seabornRed, ultra thick] (7,3) -- (9,3) -- (9,1) -- cycle;
  % Nodes
  \fill[color=seabornRed] (0+1,0+1) circle[radius=2pt];
  \node[above left] at (0+1,0+1) {$3$};
  \fill[color=seabornRed] (0+1,1+1) circle[radius=2pt];
  \node[above left] at (0+1,1+1) {$3$};
  \fill[color=seabornRed]    (0+1,2+1) circle[radius=2pt];
  \node[above left] at (0+1,2+1) {$3$};
  \fill[color=seabornOrange] (1+1,0+1) circle[radius=2pt];
  \node[above left] at (1+1,0+1) {$2$};
  \fill[color=seabornRed]    (1+1,1+1) circle[radius=2pt];
  \node[above left] at (1+1,1+1) {$3$};
  \fill[color=seabornRed] (1+1,2+1) circle[radius=2pt];
  \node[above left] at (1+1,2+1) {$3$};
  \fill[color=seabornYellow]    (2+1,0+1) circle[radius=2pt];
  \node[above left] at (2+1,0+1) {$1$};
  \fill[color=seabornOrange] (2+1,1+1) circle[radius=2pt];
  \node[above left] at (2+1,1+1) {$2$};
  \fill[color=seabornRed] (2+1,2+1) circle[radius=2pt];
  \node[above left] at (2+1,2+1) {$3$};
  \fill[color=seabornGreen] (3+1,0+1) circle[radius=2pt];
  \node[above left] at (3+1,0+1) {$0$};
  \fill[color=seabornYellow] (3+1,1+1) circle[radius=2pt];
  \node[above left] at (3+1,1+1) {$1$};
  \fill[color=seabornOrange]    (3+1,2+1) circle[radius=2pt];
  \node[above left] at (3+1,2+1) {$2$};
  \fill[color=seabornGreen] (4+1,0+1) circle[radius=2pt];
  \node[above left] at (4+1,0+1) {$0$};
  \fill[color=seabornGreen]    (4+1,1+1) circle[radius=2pt];
  \node[above left] at (4+1,1+1) {$0$};
  \fill[color=seabornYellow]   (4+1,2+1) circle[radius=2pt];
  \node[above left] at (4+1,2+1) {$1$};
  \fill[color=seabornGreen]    (5+1,0+1) circle[radius=2pt];
  \node[above right] at (5+1,0+1) {$0$};
  \fill[color=seabornYellow]   (5+1,1+1) circle[radius=2pt];
  \node[above right] at (5+1,1+1) {$1$};
  \fill[color=seabornOrange]   (5+1,2+1) circle[radius=2pt];
  \node[above right] at (5+1,2+1) {$2$};
  \fill[color=seabornYellow]   (6+1,0+1) circle[radius=2pt];
  \node[above right] at (6+1,0+1) {$1$};
  \fill[color=seabornOrange]   (6+1,1+1) circle[radius=2pt];
  \node[above right] at (6+1,1+1) {$2$};
  \fill[color=seabornRed]   (6+1,2+1) circle[radius=2pt];
  \node[above right] at (6+1,2+1) {$3$};
  \fill[color=seabornOrange]   (7+1,0+1) circle[radius=2pt];
  \node[above right] at (7+1,0+1) {$2$};
  \fill[color=seabornRed]   (7+1,1+1) circle[radius=2pt];
  \node[above right] at (7+1,1+1) {$3$};
  \fill[color=seabornRed]   (7+1,2+1) circle[radius=2pt];
  \node[above right] at (7+1,2+1) {$3$};
  \fill[color=seabornRed]   (8+1,0+1) circle[radius=2pt];
  \node[above right] at (8+1,0+1) {$3$};
  \fill[color=seabornRed]   (8+1,1+1) circle[radius=2pt];
  \node[above right] at (8+1,1+1) {$3$};
  \fill[color=seabornRed]   (8+1,2+1) circle[radius=2pt];
  \node[above right] at (8+1,2+1) {$3$};
\end{tikzpicture}
\caption{Input space composition with continuous input values.}
\labfig{extended}
\end{figure}


\begin{table}[t]
  \caption{Quantitative input usage for~\refprog{landing-alarm-system}.}
  \labtab{landing-risk}
  \centering
  \begin{tabular}{cc|cc|cc}
    \multicolumn{2}{c|}{\multirow{2}{*}{~\textbf{Input Bounds}}} & \multicolumn{2}{c|}{\outcomesname} & \multicolumn{2}{c}{\rangename} \\ \cline{3-6}
    & & \texttt{angle} & \hspace{-5.5pt}\texttt{speed} & \texttt{angle} & \hspace{-5.5pt}\texttt{speed} \\ \hline\hline
    % $\texttt{angle} = -1 \lor \texttt{angle} = 4$ & \multirow{4}{*}{$1 \le \texttt{speed} \le 3$} & &
    % 1  &  2  & 3  & 2  \\ \cline{1-1} \cline{4-7}
    $-4 \le \texttt{angle} \le 4$ & \multirow{3}{*}{$~~\land 1 \le \texttt{speed} \le 3$} &
    3  &  3  & 3  & 3  \\ \cline{1-1} \cline{3-6}
    $-4 \le \texttt{angle} \le 0$ & &
    3  &  2  & 3  & 2  \\ \cline{1-1} \cline{3-6}
    $0 \le \texttt{angle} \le 4$ & &
    3 &  2 & 3 & 2 \\
  \end{tabular}
\end{table}

Finally, we apply our quantitative analysis to~\refprog{landing-alarm-system}\marginprop{landing-alarm-system} (reported on the side) for the landing alarm system extended with the  continuous input space for the aircraft angle of approach, where $(-4 \le \texttt{angle} \le 4) \land (1 \le \texttt{speed} \le 3)$, see \reffig{extended}.
In this instance, the precision of the abstraction drastically drops as convex abstract domains are not able to capture the symmetric features of the input space around 0.
Indeed, the analysis result, first row of~\reffig{analysis}, is unable to reveal any difference in the input usage of input variables as all the abstract preconditions result of the backward analysis intersect together.
As a consequence, \outcomesname{} and \rangename{} are unable to provide any meaningful information, first row of \reftab{landing-risk}.

A possible approach to overcome the non-convexity of the input space is to split the input space into two subspaces (as a bounded set of disjunctive polyhedra), $-4 \le \texttt{angle} \le 0$ and $0 \le \texttt{angle} \le 4$, second and third row of \reftab{landing-risk}.
In the first subset $-4 \le \texttt{angle} \le 0$, we are able to perfectly captures the input regions that lead to each output bucket with our abstract analysis, second row of~\reffig{analysis}.
Therefore, we are able to recover the information that the input configurations from the bucket $\{\texttt{risk} =3\}$ do not intersect with the ones from the bucket $\{\texttt{risk} = 0\}$ after projecting away the axis \texttt{speed}.
As the end, our analysis notices that variations in the value of the input \texttt{angle} results in three possible output values, while variations in the \texttt{speed} input lead to two.
Similarly, regarding the range of values, variations in the \texttt{angle} input cover the entire spectrum of output values, whereas to the \texttt{speed} input only span a range of 2 since it exists no input value such that modifications in the \texttt{speed} value could obtain a range of output values bigger than 2.
The same reasoning applies to the other subspace with $0 \le \texttt{angle} \le 4$.


\begin{figure*}[t]
  \centering
  \begin{subfigure}{\textwidth}
  \begin{subfigure}[b]{0.24\textwidth}
    \begin{tikzpicture}[scale=0.8]
      % Grid
      \foreach \y in {0.5, 1.5, 2.5} {
        \draw[help lines, color=gray!30, dashed] (0,\y) -- (2.9,\y);
      }
      \foreach \x in {0.5, 1, 1.5, 2, 2.5} {
        \draw[help lines, color=gray!30, dashed] (\x, 0) -- (\x, 2.9);
      }
      % x-axis
      \draw[->,ultra thick] (0,0)--(3,0);
      % \draw[->,ultra thick] (0,0)--(3,0) node[rotate=90,below]{\x};
      % % y-axis
      \draw[->,ultra thick] (0,0)--(0,3) node[above]{\y};
      % % x-axis ticks
      \draw (0.5,0.1) -- (0.5,-0.1) node[below] {$-4$};
      % \draw (1,0.1) -- (1,-0.1) node[below] {$-2$};
      \draw (1,0.1) -- (1,-0.1);
      \draw (1.5,0.1) -- (1.5,-0.1) node[below] {$0$};
      % \draw (2,0.1) -- (2,-0.1) node[below] {$2$};
      \draw (2,0.1) -- (2,-0.1);
      \draw (2.5,0.1) -- (2.5,-0.1) node[below] {$4$};
      % % y-axis ticks
      \foreach \y in {1,2,3}
          \draw (0.1,\y-0.5) -- (-0.1,\y-0.5) node[left] {\y};
      % % Polyhedra
      \fill[color=seabornRed, opacity=0.5] (0.5,0.5) -- (2.5,0.5) -- (2.5,2.5) -- (0.5,2.5) -- cycle;
      \draw[color=seabornRed, ultra thick] (0.5,0.5) -- (2.5,0.5) -- (2.5,2.5) -- (0.5,2.5) -- cycle;
    \end{tikzpicture}
    % \caption{$\{\texttt{risk} = 3\}$}
  \end{subfigure}
  \hfill
  \begin{subfigure}[b]{0.23\textwidth}
    \begin{tikzpicture}[scale=0.8]
      % Grid
      \foreach \y in {0.5, 1.5, 2.5} {
        \draw[help lines, color=gray!30, dashed] (0,\y) -- (2.9,\y);
      }
      \foreach \x in {0.5, 1, 1.5, 2, 2.5} {
        \draw[help lines, color=gray!30, dashed] (\x, 0) -- (\x, 2.9);
      }
      % x-axis
      \draw[->,ultra thick] (0,0)--(3,0);
      % \draw[->,ultra thick] (0,0)--(3,0) node[rotate=90,below]{\x};
      % % y-axis
      % \draw[->,ultra thick] (0,0)--(0,3) node[above]{\y};
      % % x-axis ticks
      \draw (0.5,0.1) -- (0.5,-0.1) node[below] {$-4$};
      % \draw (1,0.1) -- (1,-0.1) node[below] {$-2$};
      \draw (1,0.1) -- (1,-0.1);
      \draw (1.5,0.1) -- (1.5,-0.1) node[below] {$0$};
      % \draw (2,0.1) -- (2,-0.1) node[below] {$2$};
      \draw (2,0.1) -- (2,-0.1);
      \draw (2.5,0.1) -- (2.5,-0.1) node[below] {$4$};
      % % y-axis ticks
      % \foreach \y in {1,2,3}
      %     \draw (0.1,\y-0.5) -- (-0.1,\y-0.5) node[left] {\y};
      % % Polyhedra
      \fill[color=seabornOrange, opacity=0.5] (0.5,0.5) -- (2.5,0.5) -- (2.25,2.5) -- (0.75,2.5) -- cycle;
      \draw[color=seabornOrange, ultra thick] (0.5,0.5) -- (2.5,0.5);
      \draw[color=seabornOrange, ultra thick] (0.75,2.5) -- (2.25,2.5);
      \draw[color=seabornOrange, ultra thick, dotted] (2.5,0.5) -- (2.25,2.5);
      \draw[color=seabornOrange, ultra thick, dotted] (0.75,2.5) -- (0.5,0.5);
    \end{tikzpicture}
    % \caption{$\{\texttt{risk} = 2\}$}
  \end{subfigure}
  % \hfill
  \begin{subfigure}[b]{0.23\textwidth}
    \begin{tikzpicture}[scale=0.8]
      % Grid
      \foreach \y in {0.5, 1.5, 2.5} {
        \draw[help lines, color=gray!30, dashed] (0,\y) -- (2.9,\y);
      }
      \foreach \x in {0.5, 1, 1.5, 2, 2.5} {
        \draw[help lines, color=gray!30, dashed] (\x, 0) -- (\x, 2.9);
      }
      % x-axis
      \draw[->,ultra thick] (0,0)--(3,0);
      % \draw[->,ultra thick] (0,0)--(3,0) node[rotate=90,below]{\x};
      % % y-axis
      % \draw[->,ultra thick] (0,0)--(0,3) node[above]{\y};
      % % x-axis ticks
      \draw (0.5,0.1) -- (0.5,-0.1) node[below] {$-4$};
      % \draw (1,0.1) -- (1,-0.1) node[below] {$-2$};
      \draw (1,0.1) -- (1,-0.1);
      \draw (1.5,0.1) -- (1.5,-0.1) node[below] {$0$};
      % \draw (2,0.1) -- (2,-0.1) node[below] {$2$};
      \draw (2,0.1) -- (2,-0.1);
      \draw (2.5,0.1) -- (2.5,-0.1) node[below] {$4$};
      % % y-axis ticks
      % \foreach \y in {1,2,3}
      %     \draw (0.1,\y-0.5) -- (-0.1,\y-0.5) node[left] {\y};
      % % Polyhedra
      \fill[color=seabornYellow, opacity=0.5] (1,0.5) -- (2,0.5) -- (1.75,2.5) -- (1.25,2.5) -- cycle;
      \draw[color=seabornYellow, ultra thick] (1,0.5) -- (2,0.5);
      \draw[color=seabornYellow, ultra thick] (1.75,2.5) -- (1.25,2.5);
      \draw[color=seabornYellow, ultra thick, dotted] (2,0.5) -- (1.75,2.5);
      \draw[color=seabornYellow, ultra thick, dotted] (1.25,2.5) -- (1,0.5);
    \end{tikzpicture}
    % \caption{$\{\texttt{risk} = 1\}$}
  \end{subfigure}
  % \hfill
  \begin{subfigure}[b]{0.24\textwidth}
    \begin{tikzpicture}[scale=0.8]
      % Grid
      \foreach \y in {0.5, 1.5, 2.5} {
        \draw[help lines, color=gray!30, dashed] (0,\y) -- (2.9,\y);
      }
      \foreach \x in {0.5, 1, 1.5, 2, 2.5} {
        \draw[help lines, color=gray!30, dashed] (\x, 0) -- (\x, 2.9);
      }
      % x-axis
      % \draw[->,ultra thick] (0,0)--(3,0);
      \draw[->,ultra thick] (0,0)--(3,0) node[rotate=90,below]{\x};
      % % y-axis
      % \draw[->,ultra thick] (0,0)--(0,3) node[above]{\y};
      % % x-axis ticks
      \draw (0.5,0.1) -- (0.5,-0.1) node[below] {$-4$};
      % \draw (1,0.1) -- (1,-0.1) node[below] {$-2$};
      \draw (1,0.1) -- (1,-0.1);
      \draw (1.5,0.1) -- (1.5,-0.1) node[below] {$0$};
      % \draw (2,0.1) -- (2,-0.1) node[below] {$2$};
      \draw (2,0.1) -- (2,-0.1);
      \draw (2.5,0.1) -- (2.5,-0.1) node[below] {$4$};
      % % y-axis ticks
      % \foreach \y in {1,2,3}
      %     \draw (0.1,\y-0.5) -- (-0.1,\y-0.5) node[left] {\y};
      % % Polyhedra
      \fill[color=seabornGreen, opacity=0.5] (1.25,0.5) -- (1.75,0.5) -- (1.5,2.5) -- cycle;
      \draw[color=seabornGreen, ultra thick] (1.25,0.5) -- (1.75,0.5);
      \draw[color=seabornGreen, ultra thick, dotted] (1.75,0.5) -- (1.5,2.5);
      \draw[color=seabornGreen, ultra thick, dotted] (1.5,2.5) -- (1.25,0.5);
    \end{tikzpicture}
    % \caption{$\{\texttt{risk} = 0\}$}
  \end{subfigure}
% \caption{Analysis result using the polyhedra domain.}
\label{fig:analysis-extended}
\end{subfigure}
\begin{subfigure}{\textwidth}
  \begin{subfigure}[b]{0.24\textwidth}
    \begin{tikzpicture}[scale=0.8]
      % Grid
      \foreach \y in {0.5, 1.5, 2.5} {
        \draw[help lines, color=gray!30, dashed] (0,\y) -- (2.9,\y);
      }
      \foreach \x in {0.5, 1, 1.5, 2, 2.5} {
        \draw[help lines, color=gray!30, dashed] (\x, 0) -- (\x, 2.9);
      }
      % x-axis
      \draw[->,ultra thick] (0,0)--(3,0);
      % \draw[->,ultra thick] (0,0)--(3,0) node[rotate=90,below]{\x};
      % % y-axis
      \draw[->,ultra thick] (0,0)--(0,3) node[above]{\y};
      \draw[dashed] (1.5,0)--(1.5,3);
      % % x-axis ticks
      \draw (0.5,0.1) -- (0.5,-0.1) node[below] {$-4$};
      % \draw (1,0.1) -- (1,-0.1) node[below] {$-2$};
      \draw (1,0.1) -- (1,-0.1);
      \draw (1.5,0.1) -- (1.5,-0.1) node[below] {$0$};
      % \draw (2,0.1) -- (2,-0.1) node[below] {$2$};
      \draw (2,0.1) -- (2,-0.1);
      \draw (2.5,0.1) -- (2.5,-0.1) node[below] {$4$};
      % % y-axis ticks
      \foreach \y in {1,2,3}
          \draw (0.1,\y-0.5) -- (-0.1,\y-0.5) node[left] {\y};
      % % Polyhedra
      \fill[color=seabornRed, opacity=0.5] (0.5,0.5) -- (1,2.5) -- (0.5,2.5) -- cycle;
      \fill[color=seabornRed, opacity=0.5] (2.5,0.5) -- (2.5,2.5) -- (2,2.5) -- cycle;
      \draw[color=seabornRed, ultra thick] (0.5,0.5) -- (1,2.5) -- (0.5,2.5) -- cycle;
      \draw[color=seabornRed, ultra thick] (2.5,0.5) -- (2.5,2.5) -- (2,2.5) -- cycle;
    \end{tikzpicture}
    % \caption{$\{\texttt{risk} = 3\}$}
  \end{subfigure}
  \hfill
  \begin{subfigure}[b]{0.23\textwidth}
    \begin{tikzpicture}[scale=0.8]
      % Grid
      \foreach \y in {0.5, 1.5, 2.5} {
        \draw[help lines, color=gray!30, dashed] (0,\y) -- (2.9,\y);
      }
      \foreach \x in {0.5, 1, 1.5, 2, 2.5} {
        \draw[help lines, color=gray!30, dashed] (\x, 0) -- (\x, 2.9);
      }
      % x-axis
      \draw[->,ultra thick] (0,0)--(3,0);
      \draw[dashed] (1.5,0)--(1.5,3);
      % \draw[->,ultra thick] (0,0)--(3,0) node[rotate=90,below]{\x};
      % % y-axis
      % \draw[->,ultra thick] (0,0)--(0,3) node[above]{\y};
      % % x-axis ticks
      \draw (0.5,0.1) -- (0.5,-0.1) node[below] {$-4$};
      % \draw (1,0.1) -- (1,-0.1) node[below] {$-2$};
      \draw (1,0.1) -- (1,-0.1);
      \draw (1.5,0.1) -- (1.5,-0.1) node[below] {$0$};
      % \draw (2,0.1) -- (2,-0.1) node[below] {$2$};
      \draw (2,0.1) -- (2,-0.1);
      \draw (2.5,0.1) -- (2.5,-0.1) node[below] {$4$};
      % % y-axis ticks
      % \foreach \y in {1,2,3}
      %     \draw (0.1,\y-0.5) -- (-0.1,\y-0.5) node[left] {\y};
      % % Polyhedra
      \fill[color=seabornOrange, opacity=0.5] (0.5,0.5) -- (0.75,0.5) -- (1,2.5) -- (0.75,2.5) -- cycle;
      \fill[color=seabornOrange, opacity=0.5] (2.25,0.5) -- (2.5,0.5) -- (2.25,2.5) -- (2,2.5) -- cycle;
      \draw[color=seabornOrange, ultra thick] (0.5,0.5) -- (0.75,0.5);
      \draw[color=seabornOrange, ultra thick] (2.25,0.5) -- (2.5,0.5);
      \draw[color=seabornOrange, ultra thick] (0.75,2.5) -- (1,2.5);
      \draw[color=seabornOrange, ultra thick] (2,2.5) -- (2.25,2.5);
      \draw[color=seabornOrange, ultra thick, dotted] (2.5,0.5) -- (2.25,2.5);
      \draw[color=seabornOrange, ultra thick] (2.25,0.5) -- (2,2.5);
      \draw[color=seabornOrange, ultra thick, dotted] (0.75,2.5) -- (0.5,0.5);
      \draw[color=seabornOrange, ultra thick] (1,2.5) -- (0.75,0.5);
    \end{tikzpicture}
    % \caption{$\{\texttt{risk} = 2\}$}
  \end{subfigure}
  % \hfill
  \begin{subfigure}[b]{0.23\textwidth}
    \begin{tikzpicture}[scale=0.8]
      % Grid
      \foreach \y in {0.5, 1.5, 2.5} {
        \draw[help lines, color=gray!30, dashed] (0,\y) -- (2.9,\y);
      }
      \foreach \x in {0.5, 1, 1.5, 2, 2.5} {
        \draw[help lines, color=gray!30, dashed] (\x, 0) -- (\x, 2.9);
      }
      % x-axis
      \draw[->,ultra thick] (0,0)--(3,0);
      \draw[dashed] (1.5, 0)--(1.5, 3);
      % \draw[->,ultra thick] (0,0)--(3,0) node[rotate=90,below]{\x};
      % % y-axis
      % \draw[->,ultra thick] (0,0)--(0,3) node[above]{\y};
      % % x-axis ticks
      \draw (0.5,0.1) -- (0.5,-0.1) node[below] {$-4$};
      % \draw (1,0.1) -- (1,-0.1) node[below] {$-2$};
      \draw (1,0.1) -- (1,-0.1);
      \draw (1.5,0.1) -- (1.5,-0.1) node[below] {$0$};
      % \draw (2,0.1) -- (2,-0.1) node[below] {$2$};
      \draw (2,0.1) -- (2,-0.1);
      \draw (2.5,0.1) -- (2.5,-0.1) node[below] {$4$};
      % % y-axis ticks
      % \foreach \y in {1,2,3}
      %     \draw (0.1,\y-0.5) -- (-0.1,\y-0.5) node[left] {\y};
      % % Polyhedra
      \fill[color=seabornYellow, opacity=0.5] (1,0.5) -- (1.25,0.5) -- (1.5,2.5) -- (1.75,0.5) -- (2,0.5) -- (1.75,2.5) -- (1.25,2.5) -- cycle;
      \draw[color=seabornYellow, ultra thick] (1,0.5) -- (1.25,0.5);
      \draw[color=seabornYellow, ultra thick] (1.75,0.5) -- (2,0.5);
      \draw[color=seabornYellow, ultra thick] (1.75,2.5) -- (1.25,2.5);
      \draw[color=seabornYellow, ultra thick, dotted] (2,0.5) -- (1.75,2.5);
      \draw[color=seabornYellow, ultra thick] (1.75,0.5) -- (1.5,2.5);
      \draw[color=seabornYellow, ultra thick, dotted] (1.25,2.5) -- (1,0.5);
      \draw[color=seabornYellow, ultra thick] (1.5,2.5) -- (1.25,0.5);
    \end{tikzpicture}
    % \caption{$\{\texttt{risk} = 1\}$}
  \end{subfigure}
  % \hfill
  \begin{subfigure}[b]{0.24\textwidth}
    \begin{tikzpicture}[scale=0.8]
      % Grid
      \foreach \y in {0.5, 1.5, 2.5} {
        \draw[help lines, color=gray!30, dashed] (0,\y) -- (2.9,\y);
      }
      \foreach \x in {0.5, 1, 1.5, 2, 2.5} {
        \draw[help lines, color=gray!30, dashed] (\x, 0) -- (\x, 2.9);
      }
      % x-axis
      % \draw[->,ultra thick] (0,0)--(3,0);
      \draw[->,ultra thick] (0,0)--(3,0) node[rotate=90,below]{\x};
      \draw[dashed] (1.5, 0)--(1.5, 3);
      % % y-axis
      % \draw[->,ultra thick] (0,0)--(0,3) node[above]{\y};
      % % x-axis ticks
      \draw (0.5,0.1) -- (0.5,-0.1) node[below] {$-4$};
      % \draw (1,0.1) -- (1,-0.1) node[below] {$-2$};
      \draw (1,0.1) -- (1,-0.1);
      \draw (1.5,0.1) -- (1.5,-0.1) node[below] {$0$};
      % \draw (2,0.1) -- (2,-0.1) node[below] {$2$};
      \draw (2,0.1) -- (2,-0.1);
      \draw (2.5,0.1) -- (2.5,-0.1) node[below] {$4$};
      % % y-axis ticks
      % \foreach \y in {1,2,3}
      %     \draw (0.1,\y-0.5) -- (-0.1,\y-0.5) node[left] {\y};
      % % Polyhedra
      \fill[color=seabornGreen, opacity=0.5] (1.25,0.5) -- (1.75,0.5) -- (1.5,2.5) -- cycle;
      \draw[color=seabornGreen, ultra thick] (1.25,0.5) -- (1.75,0.5);
      \draw[color=seabornGreen, ultra thick, dotted] (1.75,0.5) -- (1.5,2.5);
      \draw[color=seabornGreen, ultra thick, dotted] (1.5,2.5) -- (1.25,0.5);
      \draw[color=seabornGreen, ultra thick, dotted] (1.5,2.5) -- (1.5,0.5);
    \end{tikzpicture}
    % \caption{$\{\texttt{risk} = 0\}$}
  \end{subfigure}
% \caption{Analysis result after splitting the input space into two subspaces around $\texttt{angle}=0$.}
\end{subfigure}
\caption{Above, result of the analysis with convex polyhedra. Below, result after splitting the input space into two subspaces around $\texttt{angle}=0$.}
%\vspace{%-15pt}
\labfig{analysis}
\end{figure*}

