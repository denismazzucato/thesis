\def\customcitkey{}


\chapter*{Template Motivation}
\addcontentsline{toc}{chapter}{Template Motivation}

For this manuscript, we have selected the \textsc{Kaobook} class,\footnote{\rurl{github.com/fmarotta/kaobook}} a template featuring a wide margin specifically designed for writing books and graduate-level theses, based on Ken Arroyo Ohori's doctoral thesis\footnote{\rurl{3d.bk.tudelft.nl/ken/en/thesis}} and on the \textsc{Tufte}-\LaTeX{} class,\footnote{\rurl{ctan.org/pkg/tufte-latex}} offering a blend of style and function.

\emph{Why the narrow main text column?} It is all about the wide margin, an expansive playground for notes, code snippets, figures, tables, citations, and all those additional details that enhance the narrative.
This layout lets the main narrative shine in the spotlight while all the extra bits are placed in the margin, where they can be easily accessed without disrupting the flow of the text. It is like having a personal assistant handing you the information right when you need it.

With over 200 references in this manuscript, the use of side references becomes a lifesaver.
No need to dig back and forth through the bibliography to remember which citation matches \cite{\customcitkey}.
In this thesis, it is a matter you will not endure any more.\footnote{In fact, the 42nd citation is: \formatmargincitation{\customcitkey}}

The template also allows for brief restatements of definitions and theorems in the margins, linked to their original appearances.
This means you will not flip through the pages up and down this manuscript--once a concept is introduced, it conveniently pops up in the margin whenever relevant again. This keeps the discussion clear and accessible, which is especially valuable in a complex and detailed work like a PhD thesis. In short:
\begin{center}\itshape
this wide-margin template makes navigating through these pages a bit easier, \\
and hopefully, a bit more enjoyable too.
\end{center}
