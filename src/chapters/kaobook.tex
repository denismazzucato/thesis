\def\customcitkey{}


\chapter*{Template Motivation}
\addcontentsline{toc}{chapter}{Template Motivation}

For this manuscript, we have selected the \textsc{Kaobook} class,\footnote{\rurl{github.com/fmarotta/kaobook}} a template specifically designed for writing books and graduate-level theses, based on Ken Arroyo Ohori's doctoral thesis\footnote{\rurl{3d.bk.tudelft.nl/ken/en/thesis}} and on the \textsc{Tufte}-\LaTeX{} class.\footnote{\rurl{ctan.org/pkg/tufte-latex}}
The design features a relatively narrow main text column with an adjacent wide margin to house notes, code snippets, figures, tables, citations, and everything else that might be needed.
This layout effectively separates the primary narrative from supplementary material, allowing readers to easily access important information without interrupting the flow of the main text.

For instance, with over 200 references in this manuscript, the inclusion of side references is particularly beneficial, as it simplifies the process of locating and recalling specific citations.
It would be a suffering to remember which citation corresponds to \cite{\customcitkey} without a quick recap in the margin.\footnote{\formatmargincitation{\customcitkey}}
Moreover, the template's ability to briefly restate definitions and theorems in the margins—linked to their original locations—prevents readers from having to flip back through the document.
A notion is defined once, in the main text, and then reported in the margin whenever helpful.
This feature ensures that complex discussions remain coherent and accessible, making it especially valuable in a detailed work like a PhD thesis, where clarity and ease of navigation are crucial.
