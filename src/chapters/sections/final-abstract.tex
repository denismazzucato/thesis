Cette thèse vise à développer des méthodes efficaces et mathématiquement rigoureuses pour améliorer la fiabilité des logiciels en utilisant l'interprétation abstraite, un cadre formel pour approximer les comportements des programmes. Nous développons un cadre quantitatif pour mesurer l'influence des variables d'entrée sur le comportement des programmes, aidant ainsi à certifier les comportements corrects et à identifier les défauts. Le cadre proposé est flexible, permettant différentes mesures d'impact et garantissant des résultats rigoureux, ce qui signifie que l'impact calculé sera toujours une surestimation ou une sous-estimation.
%
Ce cadre est appliqué à la fois aux propriétés extensionnelles, qui mesurent le comportement entrée-sortie, et aux propriétés intentionnelles, qui évaluent des détails computationnels comme les itérations de boucles. Il a été implémenté dans trois outils : \textsc{Impatto}, \textsc{Libra} et \textsc{TimeSec}. Ces outils ciblent respectivement la fiabilité générale des logiciels, l'équité dans les réseaux neuronaux et les vulnérabilités de canaux auxiliaires. Les évaluations expérimentales ont validé ces outils, démontrant leur efficacité dans divers cas d'utilisation, tels que la détection de biais dans les réseaux neuronaux et les vulnérabilités dans les bibliothèques cryptographiques.
%
Dans l'ensemble, ce travail améliore la compréhension de l'utilisation des données d'entrée dans les logiciels, offrant à la fois des perspectives théoriques et des outils pratiques pour améliorer la fiabilité et la sécurité des systèmes.
