\setchapterpreamble[u]{\margintoc}

\chapter{Evaluation of the Timing Analysis}
\labch{sas24-eval}

\marginemptybox{2.4cm}

This chapter presents \timesec\sidenote{\label{ss69}\timesecurl}, written in about 3000 lines of \python{} code.
The global loop bound analysis is based on the numerical library \apron{} \sidecite{Jeannet2009}, we instrumented the analysis with widening and narrowing operators after 2 fixpoint iterations.
For the linear programming encoding, we used the \python{} library \scipy\sidenote{\label{ss420}\rurl{scipy.org}}.
In this chapter, we discuss some implementation features that make the analysis scalable, precise, and able to handle real-world programs.
We evaluate \timesec{} on the \bignum{} library\sidenote{\bignumurl}, a collection of arithmetic routines designed for cryptographic applications, and on the \svcomp{} benchmarks\sidenote{\label{ss69420}\svcompurl}.
An artifact of \timesec, including the source code, the benchmarks, and the evaluation results, is available on Zenodo\sidenote{\label{ss42069}\timeseczenodo}.
This chapter is based on the work presented at the 31st Static Analysis Symposium (SAS 2024)~\sidecite{Mazzucato2024c}.


\frenchdiv

\emph{Ce chapitre présente \timesec\sidenotemark[\ref{ss69}], écrit en environ 3000 lignes de code \python{}. L'analyse des bornes globales des boucles est basée sur la bibliothèque numérique \apron{} \cite{Jeannet2009}, et nous avons instrumenté l'analyse avec des opérateurs de renforcement et d'affaiblissement après 2 itérations du point fixe. Pour l'encodage en programmation linéaire, nous avons utilisé la bibliothèque \python{} \scipy\sidenotemark[\ref{ss420}]. Dans ce chapitre, nous discutons de certaines fonctionnalités d'implémentation qui rendent l'analyse évolutive, précise et capable de gérer des programmes réels. Nous évaluons \timesec{} sur la bibliothèque \bignum{} \sidenotemark[\ref{ss69420}], une collection de routines arithmétiques conçues pour des applications cryptographiques, ainsi que sur les benchmarks \svcomp{} \sidenotemark[\ref{ss420}]. Un artefact de \timesec, incluant le code source, les benchmarks et les résultats de l'évaluation, est disponible sur Zenodo\sidenotemark[\ref{ss42069}]. Ce chapitre est basé sur les travaux présentés lors du 31e Symposium sur l'analyse statique (SAS 2024)~\cite{Mazzucato2024c}.}





\section{Summary}

This chapter concludes the main body of the thesis, presenting the evaluation of our static analysis for quantitative program properties.
Next, we present the conclusion and future work.


\frenchdiv

\emph{Ce chapitre conclut le corps principal de la thèse en présentant l'évaluation de notre analyse statique pour les propriétés quantitatives des programmes. Ensuite, nous présenterons la conclusion et les perspectives de travaux futurs.}
