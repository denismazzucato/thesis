\chapter{Quantitative Fairness for Neural Networks}
\labch{quantitative-fairness}

% In this chapter, we introduce quantitative notions of impact in the context of neural networks.

% %

\section{Landing Alarm System}
\labsec{landing-alarm-system}
\newcommand*{\x}{\texttt{angle}}
\newcommand*{\y}{\texttt{speed}}
\newcommand*{\z}{\texttt{risk}}
\newcommand*{\lc}{\texttt{landing\_coeff}}


\begin{marginlisting}
  \caption{Program for the landing-risk alarm system.}
  \labprog{landing-alarm-system}
  \vspace{0.5cm}
\begin{lstlisting}[language=customPython,escapechar=\%]
landing_coeff = abs(angle) + speed %\labline{compute-risk}%
if landing_coeff < 2 then %\labline{low-risk-cond}%
  risk = 0 %\labline{low-risk}%
else if landing_coeff > 5 then %\labline{high-risk-cond}%
  risk = 3 %\labline{high-risk}%
else %\labline{medium-risk-branch}%
  risk = floor(landing_coeff) - 2 %\labline{medium-risk}%
\end{lstlisting}
\end{marginlisting}


The goal of \refprog{landing-alarm-system}, referred to as \landingprogram, is to inform the pilot about the level of risk associated with the landing approach.
It takes two input variables, denoted as \x{} and \y, for the aircraft-airstrip alignment angle and the aircraft speed, respectively.
A value of 1 represents a good alignment while -4 a non-aligned angle, whereas 1, 2, 3 denote low, medium, and high speed.
A safer approach is indicated by lower speed.
The program \landingprogram{} computes first the landing risk coefficient, denoted as \lc, at \refline{compute-risk}.
This coefficient is obtained by taking the absolute value of the landing angle (accounting for both negative and positive approach angles) and adding it to the speed.
Afterwards, \refline{low-risk-cond} and~\refline{high-risk-cond} restrict \lc{} to range between values 2 and 5.
Values below 2 signify low danger, while values above 5 indicate high danger.
Respectively, we assign to the output variable \z{} the value of 0 for low danger and the value of 3 for high danger.
The medium degree of dangerousness is computed at \refline{medium-risk} and ranges between 1 and 2, which assigns to the output variable \z{} the largest integer less than, or equal, to $\lc-2$.
The output variable $\z{}$ is the danger level with possible values $\{0, 1, 2, 3\}$.

\begin{marginfigure}
\centering
\begin{tikzpicture}
  % Grid
  \draw[help lines, color=gray!30, dashed] (0,0) grid (2.9,3.9);
  % x-axis
  \draw[->,ultra thick] (0,0)--(3,0) node[right]{\x};
  % y-axis
  \draw[->,ultra thick] (0,0)--(0,4) node[above]{\y};
  % x-axis ticks
  \draw (0+1,0.1) -- (0+1,-0.1) node[below] {-4};
  \draw (1+1,0.1) -- (1+1,-0.1) node[below] {1};
  % y-axis ticks
  \foreach \y in {1,2,3}
  \draw (0.1,\y) -- (-0.1,\y) node[left] {\y};
  % Nodes
  \fill[color=seabornRed]   (0+1,0+1) circle[radius=2pt];
  \node[above right] at (0+1,0+1) {$3$};
  \fill[color=seabornRed]   (0+1,1+1) circle[radius=2pt];
  \node[above right] at (0+1,1+1) {$3$};
  \fill[color=seabornRed]   (0+1,2+1) circle[radius=2pt];
  \node[above right] at (0+1,2+1) {$3$};
  \fill[color=seabornGreen] (1+1,0+1) circle[radius=2pt];
  \node[above right] at (1+1,0+1) {$0$};
  \fill[color=seabornYellow] (1+1,1+1) circle[radius=2pt];
  \node[above right] at (1+1,1+1) {$1$};
  \fill[color=seabornOrange]    (1+1,2+1) circle[radius=2pt];
  \node[above right] at (1+1,2+1) {$2$};
\end{tikzpicture}
\caption{Input space composition of \refprog{landing-alarm-system}.}
\labfig{input-space-composition}
\end{marginfigure}

\reffig{input-space-composition} shows the input space composition of this system, where the label near each input represents the degree of risk assigned to the corresponding input configuration.
It is easy to note that a nonaligned angle of approach corresponds to a considerably higher level of risk, whereas the risk with a correct angle depends mostly on the aircraft speed.
Our goal is to develop a static analysis capable of quantifying the contribution of each input variable to the computation of the output variable $\z{}$.


\newcommand{\exampleinput}[1][\defaultprogramexampleletter]{\textsc{Input}_{#1}}

Understanding the effects of input variables on program executions allows us to reveal potential bugs or certify intended behavior.
Therefore, we present a framework to quantify the impact of input variables on the outcome of a program.
Intuitively, the definition of impact of a variable depends on various factors such as program structure, environment, developer expertise, and researcher intuition.
Different notions of impact leverage different traits of the input-output relations in computation.
As a result, it is not possible to formalize a definition that fits all requirements.

In general, all possible definitions of impact establish some relationship between variations of the input and the output values.
In this section, we introduce two impact notions.
The first impact definition focuses on \textit{the size of} extreme reachable values resulting from variations in the input variable.
The second exploits \textit{the number of} reachable outcomes.
Both definitions give us different insights on the program \landingprogram.
In particular, the first quantity tells us which variation in the values of input variables results in larger differences between output values, while the second indicates which variation reaches a greater number of output values.

\newcommand*{\highlight}[1]{\textcolor{seabornBlue}{#1}}
\newcommand*{\inputa}{\tuple{-4}{1}}
\newcommand*{\inputax}{\tuple{\highlight{-4}}{1}}
\newcommand*{\inputay}{\tuple{-4}{\highlight{1}}}
\newcommand*{\outputa}{\langle \outputvaluea\rangle} \newcommand*{\outputvaluea}{3}
\newcommand*{\inputb}{\tuple{-4}{2}}
\newcommand*{\inputbx}{\tuple{\highlight{-4}}{2}}
\newcommand*{\inputby}{\tuple{-4}{\highlight{2}}}
\newcommand*{\outputb}{\langle \outputvalueb\rangle} \newcommand*{\outputvalueb}{3}
\newcommand*{\inputc}{\tuple{-4}{3}}
\newcommand*{\inputcx}{\tuple{\highlight{-4}}{3}}
\newcommand*{\inputcy}{\tuple{-4}{\highlight{3}}}
\newcommand*{\outputc}{\langle \outputvaluec\rangle} \newcommand*{\outputvaluec}{3}
\newcommand*{\inputd}{\tuple{ 1}{1}}
\newcommand*{\inputdx}{\tuple{\highlight{ 1}}{1}}
\newcommand*{\inputdy}{\tuple{ 1}{\highlight{1}}}
\newcommand*{\outputd}{\langle \outputvalued\rangle} \newcommand*{\outputvalued}{0}
\newcommand*{\inpute}{\tuple{ 1}{2}}
\newcommand*{\inputex}{\tuple{\highlight{ 1}}{2}}
\newcommand*{\inputey}{\tuple{ 1}{\highlight{2}}}
\newcommand*{\outpute}{\langle \outputvaluee\rangle} \newcommand*{\outputvaluee}{1}
\newcommand*{\inputf}{\tuple{ 1}{3}}
\newcommand*{\inputfx}{\tuple{\highlight{ 1}}{3}}
\newcommand*{\inputfy}{\tuple{ 1}{\highlight{3}}}
\newcommand*{\outputf}{\langle \outputvaluef\rangle} \newcommand*{\outputvaluef}{2}
\newcommand*{\tracea}{\inputa\to\outputa}
\newcommand*{\traceax}{\inputax\to\outputa}
\newcommand*{\traceay}{\inputay\to\outputa}
\newcommand*{\traceb}{\inputb\to\outputb}
\newcommand*{\tracebx}{\inputbx\to\outputb}
\newcommand*{\traceby}{\inputby\to\outputb}
\newcommand*{\tracec}{\inputc\to\outputc}
\newcommand*{\tracecx}{\inputcx\to\outputc}
\newcommand*{\tracecy}{\inputcy\to\outputc}
\newcommand*{\traced}{\inputd\to\outputd}
\newcommand*{\tracedx}{\inputdx\to\outputd}
\newcommand*{\tracedy}{\inputdy\to\outputd}
\newcommand*{\tracee}{\inpute\to\outpute}
\newcommand*{\traceex}{\inputex\to\outpute}
\newcommand*{\traceey}{\inputey\to\outpute}
\newcommand*{\tracef}{\inputf\to\outputf}
\newcommand*{\tracefx}{\inputfx\to\outputf}
\newcommand*{\tracefy}{\inputfy\to\outputf}

\paragraph{First Impact Definition {\normalfont(\texorpdfstring{$\outcomesname$}{Outcomes})}.}
\labsec{overview:outcomes}
%
The first impact definition that we consider is
 $\outcomes(\defprogram)$, %(derived from $\outcomesentropy$),
where $\definputvariable$ is the input variable of interest and $\defprogram$ the program under analysis. Intuitively\sidenote{the formal definition is given in \refsec{quantitative-input-data-usage}}, $\outcomes$ returns the maximum number of outputs that are reachable from value variations of the input variable $\definputvariable$.
For the program
$\landingprogram$, the result is shown in column $\outcomesname(\landingprogram)$~of \reftab{overview}:
we obtain $\outcomesname_\x(\landingprogram)=2$ and $\outcomesname_\y(\landingprogram)=3$.

\newcommand*{\labelrotationangle}{30}
\begin{table*}[t]
  \centering
  \caption{Impact of for $\outcomesname(\landingprogram)$  and $\rangename(\landingprogram)$ definitions for both $\x$ and $\y$ variables. Computational features are \highlight{highlighted in blue}.}
  \labtab{overview}
  \begin{tabular}{c|c|c|c|c|c}
  \rotatebox[origin=c]{\labelrotationangle}{\textsc{Variable}}~ & ~\rotatebox[origin=c]{\labelrotationangle}{$\exampleinput[\landingprogram]$}~ & ~\textsc{Relevant Traces}~ & ~\rotatebox[origin=c]{\labelrotationangle}{\textsc{Outputs}}~ & ~\rotatebox[origin=c]{\labelrotationangle}{$\outcomesname$}~ & ~\rotatebox[origin=c]{\labelrotationangle}{$\rangename$}~ \\
  \toprule
  \multirow{6}{*}{\x}
   & $\inputax$ & $\traceax, \tracedx$ & $\{\outputvaluea,\outputvalued\}$ & \multirow{6}{*}{$2$} & \multirow{6}{*}{$3$} \\
  \cline{2-4}
   & $\inputbx$ & $\tracebx, \traceex$ & $\{\outputvalueb,\outputvaluee\}$ & & \\
  \cline{2-4}
   & $\inputcx$ & $\tracecx, \tracefx$ & $\{\outputvaluec,\outputvaluef\}$ & & \\
   \cline{2-4}
   & $\inputdx$ & $\tracedx, \traceax$ & $\{\outputvalued,\outputvaluea\}$ & & \\
   \cline{2-4}
   & $\inputex$ & $\traceex, \tracebx$ & $\{\outputvaluee,\outputvalueb\}$ & & \\
   \cline{2-4}
   & $\inputfx$ & $\tracefx, \tracecx$ & $\{\outputvaluef,\outputvaluec\}$ & & \\
  \midrule
  \multirow{12}{*}{\y}
   & \multirow{2}{*}{$\inputay$} & $\traceay, \traceby,$ & \multirow{2}{*}{$\{\outputvaluea\}$} & \multirow{12}{*}{$3$} & \multirow{12}{*}{$2$} \\
   & & $\tracecy$ & & & \\
  \cline{2-4}
   & \multirow{2}{*}{$\inputby$} & $\traceay, \traceby,$ & \multirow{2}{*}{$\{\outputvaluea\}$} & & \\
   & & $\tracecy$ & & & \\
  \cline{2-4}
   & \multirow{2}{*}{$\inputcy$} & $\traceay, \traceby,$ & \multirow{2}{*}{$\{\outputvaluea\}$} & & \\
   & & $\tracecy$ & & & \\
   \cline{2-4}
   & \multirow{2}{*}{$\inputdy$} & $\tracedy, \traceey,$ & \multirow{2}{*}{$\{\outputvalued,\outputvaluee, \outputvaluef\}$} & & \\
   & & $\tracefy$ & & & \\
   \cline{2-4}
   & \multirow{2}{*}{$\inputey$} & $\tracedy, \traceey,$ & \multirow{2}{*}{$\{\outputvalued,\outputvaluee, \outputvaluef\}$} & & \\
   & & $\tracefy$ & & & \\
   \cline{2-4}
   & \multirow{2}{*}{$\inputfy$} & $\tracedy, \traceey,$ & \multirow{2}{*}{$\{\outputvalued,\outputvaluee, \outputvaluef\}$} & & \\
   & & $\tracefy$ & & \\
   \bottomrule
  \end{tabular}
\end{table*}
The conclusion is that $\y$ has a greater influence than $\x$ on the output of the program.

\paragraph{Second Impact Definition {\normalfont(\texorpdfstring{$\rangename$}{Range})}.}
\labsec{overview:range}
%
The second impact definition we propose is $\rangename_\definputvariable$, which yields the maximum difference between the maximum and the minimum outputs that are reachable from value variations of the input variable $\definputvariable$.
The result for program $\landingprogram$ is shown in column $\rangename(\landingprogram)$~of \reftab{overview}:
the range of reachable outputs from variations of $\x$ is, at most, the interval $[0, 3]$, with a length of 3. Instead, the range of reachable outputs from variations of $\y$ is, at most, the interval $[0, 2]$, with a length of 2. Therefore, we obtain $\rangename_\x(\landingprogram)=3$ and $\rangename_\y(\landingprogram)=2$.
In other words, varying the angle of approach might drastically alter the landing risk, whereas the speed has less influence.
%
This is in contrast to the conclusion of \outcomesname{} where $\y$ has a greater impact than $\x$.
Although it may seem counterintuitive at first, the difference between the two impact instances is due to the different program traits they explore.
$\rangename$ quantifies over the variance in the extreme values of the set of output values, while $\outcomesname$ quantifies over the variance in the number of unique output values.
Consequently, changes in $\x$ yield a bigger variation in the degree of risk compared to $\y$, while changes in $\y$ reach far more risk levels compared to $\x$.
%
Note that, the impact definitions presented above are not computationally practical as they rely on a complete enumeration of all possible input configurations.
% Note that, enumerating all possible input configurations is not computationally practical.
Specifically, when dealing with more complex input space compositions, this approach is highly inefficient or even infeasible (as in the case of continuous input spaces).
As a consequence, our approach is based on an abstraction of input-output relations, which allows us to automatically infer a sound upper bound on the program's impact.

\paragraph{Static Analysis.}
\labsec{overview:static-analysis}

To quantify the impact of a program, one can rely solely on the input-output observations of the program.
Thus, our approach is based on an abstraction of input-output relations, which allows us to automatically infer a sound upper bound on the program's impact.

The analysis starts with a set of output abstractions called \textit{output buckets}.
A bucket is an abstract element representing a set of output states.
While this abstraction may limit the ability to precisely reason about the impact of output values within the same bucket, it permits automatic reasoning across different buckets.
Afterwards, an abstract interpretation-based static analyzer propagates each output bucket backward through the program under consideration.
The analyzer returns an abstract element for each output bucket, representing an over-approximation of the set of input configurations that lead to the output values inside the starting bucket.
This result contains also spurious input configurations that may not lead to a value inside the output bucket.
Based on the chosen impact definition \impactwrappername{} (e.g., \rangename{} or \outcomesname), we perform computations and comparisons on the abstract elements returned by the analysis to obtain an upper bound $\defbound'$. This upper bound is sound by construction of the theoretical framework, meaning that if $\defbound$ is the real (concrete) impact quantity obtained by \impactwrappername, then $\defbound\le\defbound'$.

Assuming \rangename{} as our impact of interest, given the set of abstract elements returned from the backward analysis (one for each bucket), we first project away the input variable $\definputvariable$ under consideration.
This means that our abstract elements now represent input configurations without the variable $\definputvariable$.
To be precise, we obtain \textit{partial input configurations} taking into account all possible variations of the input $\definputvariable$.
After the projection, we check for intersections among the abstract elements.
Any intersection may indicate two input configurations that only differ in the value of the variable $\definputvariable$, leading to two different output buckets.

Continuing our landing alarm system example, we start with three output buckets: $b_0=\{-1, 0\}$, $b_1=\{1, 2\}$, and $b_2=\{3, 4\}$ for instance.
Note that, only $\{0, 1, 2, 3\}$ are reachable \z{} values, but we may not have the exact post-condition of a program.
Therefore, for each of these three buckets, we run the backward analysis.
The backward analysis is parametric in the choice of the abstract domain, for instance, the disjunctive polyhedra domain based on the convex polyhedra domain~\sidecite{Cousot1978}.
Without showing the details of the analysis, starting with the bucket $b_0$, only the first branch of the if statement can be processed, \refline{low-risk}.
Therefore, \lc{} should be smaller than 2 at the end of \refline{compute-risk}.
Consequently, the sum of the absolute values of \x{} and \y{} should be smaller than 2.
Thus, our analysis discovers that only the input configuration $\tuple{1}{1}$ leads to the first bucket.
Similarly, we discover that from the bucket $b_1$, we reach the input configurations $\tuple{1}{2}$ and $\tuple{1}{3}$.
Lastly, from the bucket $b_2$, we reach the input configurations $\tuple{-4}{1}$, $\tuple{-4}{2}$, and $\tuple{-4}{3}$.

Regarding the case in which we quantify the impact of \x{}, we remove \x{} from each of the abstract elements returned by the analysis.
For example, from the bucket $b_0$, we have the abstract input configuration $\langle 1\rangle$\sidenote{The analysis discovers that only the input configuration $\tuple{1}{1}$ leads to the first bucket $b_0$. Thus, by removing \x{} from the input configuration, we obtain the tuple of a single element $\langle 1\rangle$, we use this notation for the rest of the section.}
; from the bucket $b_1$, we have $\langle 2\rangle$ and $\langle 3\rangle$; and from the bucket $b_2$, we have $\langle 1\rangle$, $\langle 2\rangle$, and $\langle 3\rangle$.
By checking the intersections, we find that bucket $b_0$ intersects with bucket $b_2$ since they have $\langle 1\rangle$ in common.
Bucket $b_2$ also intersects with bucket $b_1$ since they have both $\langle 2\rangle$ and $\langle 3\rangle$ in common.
Understanding the meaning of intersections is crucial.
For example, the first intersection between buckets $b_0$ and $b_2$ means that there exist two input configurations (namely $\tuple{1}{1}$ and $\tuple{-4}{1}$) that only differ in the value of \x{} (from $\x=1$ in the first configuration $\tuple{1}{1}$ to $\x=-4$ in the second configuration $\tuple{-4}{1}$), where the first configuration leads to an output in bucket $b_0$ and the second configuration leads to an output in the other bucket $b_2$.
Therefore, we are able to assign a range to this variation by taking the minimum of bucket $b_0$ and the maximum of bucket $b_2$, resulting in the range $[-1,4]$ with a size of 5.
Similarly, we obtain the range $[1,4]$ with a size of 3 from the other intersection between buckets $b_1$ and $b_2$.
Finally, we return the maximum among all possible sizes, which is 5.
This result is an over-approximation of the impact value for the function $\rangename_\x(\landingprogram)$.

Similarly, from the result of the backward analysis, we repeat the steps to quantify the impact regarding the other input variable \y{}.
Therefore, the analysis result is an impact of 3 since the only buckets intersecting after projecting away \y{} are buckets $b_0$ and $b_1$, with abstract input configurations of $\langle 1\rangle$ for both, while bucket $b_2$ contains only the value $\langle -4\rangle$.

Regarding the impact definition \outcomesname, whenever we discover intersections among different buckets, we count the total number of different outcome values they carry.
Consequently, we find that the two intersections for both variables \x{} and \y{} hold four different outcome values in total.
In such case, our analysis would conclude that the two variables hold the same impact.
Indeed, the choice of the starting output buckets is critical.
For instance, with $b_0=\{0\}$, $b_1=\{1\}$, $b_2=\{2\}$, and $b_3=\{3\}$ one would recover maximum precision and obtain a useful analysis quantification.

In conclusion, the sound upper bound discovered by our analysis is always higher than the concrete one by construction of the theoretical framework.
The precision of our analysis is mostly affected by the choice of output buckets and the approximation induced by the backward analysis \sidenote{as outlined by the use cases shown in~\refsec{nfm24:experimental-results}.}

% \subsection{The \changesname{} impact definition}[The Changes Impact Definition]
\labsec{changes}

% \subsection{The \qlibraname{} impact definition}[The QLibra Impact Definition]
\labsec{qlibra}

% \input{src/chapters/extensional/sas21/evaluation-changes}
% \input{src/chapters/extensional/sas21/evaluation-qlibra}

\section{Neural Network Analysis}
\labsec{neural-network-analysis}

\subsection{Transition System for Neural Networks}

\subsection{Forward Reachability State Semantics for Neural Networks}

\subsection{Backward Reachability State Semantics for Neural Networks}

\section{The \changesname{} Impact Definition}[\changesname]
\labsec{changes-impact-definition}

\begin{definition}[Segments of Contiguous Regions]\labdef{segments}
\end{definition}

\begin{definition}[\changesname]\labdef{changes}
  Given an input variable $\definputvariable\in\inputvariables$, and an output descriptor $\outputdesc$,
  the quantity $\changes\in\dots$ is defined as:
  \begin{align*}
    \changes(\defsetoftraces) &\DefeQ \dots
  \end{align*}
\end{definition}

\denis{Here the algorithm for the abstract implementation of changes.}


\section{The \qlibraname{} Impact Definition}[\qlibraname]
\labsec{qlibra-impact-definition}

\begin{definition}[\qlibraname]\labdef{qlibra}
  Given an input variable $\definputvariable\in\inputvariables$, and an output descriptor $\outputdesc$,
  the quantity $\qlibra\in\dots$ is defined as:
  \begin{align*}
    \qlibra(\defsetoftraces) &\DefeQ \dots
  \end{align*}
\end{definition}


\section{Outcome Semantics}
\labsec{outcome-semantics}


\begin{definition}[Adjoints for the Outcome Semantics]
  \labdef{adjoints-outcome-semantics}
  \begin{align*}
    \outcomeabstraction \IN& \inputoutputtype \to \outcometype \\
    \outcomeabstraction(\defsetofsetofdependencies) \DefeQ& \setdef{
      \setdef{\inputoutputtuple{\defseq} \in\defsetofdependencies}{\retrieveoutput{\defseq} = \defstate}
    }{
      \defsetofdependencies\in\defsetofsetofdependencies \land
      \defstate \in \stateandbottom
    }\\
    \outcomeconcretization \IN& \outcometype \to \inputoutputtype \\
    \outcomeconcretization(\defsetofsetofdependencies) \DefeQ& \setdef{
      \dots
    }{
      \dots
    }
  \end{align*}
\end{definition}

\begin{theorem}\labthm{inputoutput-outcome-galois-connection}
  The input-output $\inputoutputsemanticsnoparam$ and outcome $\outcomesemanticsnoparam$ semantics form a \emph{Galois Connection}:
\begin{align*}
  \galoisbetweensemantics{inputoutput}{outcome}
\end{align*}
\end{theorem}

\begin{definition}[Outcome Semantics]\labdef{outcome-semantics}
  \begin{align*}
    \outcomesemanticsnoparam\DefeQ& \outcomeabstraction(\inputoutputsemanticsnoparam)
  \end{align*}
\end{definition}

\begin{theorem}\labthm{outcome-validation}
  \begin{align*}
    \inputoutputsemantics \subseteq \unused \IfF \outcomesemantics \subseteq \outcomeabstraction(\unused)
  \end{align*}
\end{theorem}


\section{Parallel Semantics}
\labsec{parallel-semantics}


\begin{definition}[Adjoints for the Parallel Semantics]
  \labdef{adjoints-parallel-semantics}
  \begin{align*}
    \parallelabstraction \IN& \outcometype \to \paralleltype \\
    \parallelabstraction(\defsetofsetofdependencies) \DefeQ& \setdef{
      \dots
    }{
      \dots
    }\\
    \parallelconcretization \IN& \paralleltype \to \outcometype \\
    \parallelconcretization(\defsetofsetofdependencies) \DefeQ& \setdef{
      \dots
    }{
      \dots
    }
  \end{align*}
\end{definition}

\begin{theorem}\labthm{outcome-parallel-galois-connection}
  The input-output $\outcomesemanticsnoparam$ and parallel $\parallelsemanticsnoparam$ semantics form a \emph{Galois Connection}:
\begin{align*}
  \galoisbetweensemantics{outcome}{parallel}
\end{align*}
\end{theorem}

\begin{definition}[Parallel Semantics]\labdef{parallel-semantics}
  \begin{align*}
    \parallelsemanticsnoparam\DefeQ& \parallelabstraction(\outcomesemanticsnoparam)
  \end{align*}
\end{definition}

\begin{theorem}\labthm{parallel-validation}
  \begin{align*}
    \outcomesemantics \subseteq \unused \IfF \parallelsemantics \subseteq \parallelabstraction(\unused)
  \end{align*}
\end{theorem}

\section{Abstract Implementation \texorpdfstring{$\abstractqlibra$}{Abstract QLibra}}[Abstract \texorpdfstring{$\abstractqlibra$}{QLibra}]
\labsec{abstract-qlibra}


\denis{Here the algorithm for the abstract implementation of changes.}

\section{Abstract Domains for Neural Network Verification}
\labsec{abstract-domains-for-neural-network-verification}

\subsection{Boxes}
\labsec{boxes}

\subsection{Symbolic Constant Propagation}
\labsec{symbolic-constant-propagation}

\subsection{DeepPoly}
\labsec{deeppoly}

\subsection{Neurify}
\labsec{neurify}


\section{Reduced Product}
\labsec{reduced-product}
