\pagelayout{wide} % No margins
\addpart{Quantitative Verification of Extensional Properties}
\labpart{extensional}
\pagelayout{margin} % Restore margins

\setchapterpreamble[u]{\margintoc}
%

\chapter{Quantitative Input Data Usage}
\labch{quantitative-input-data-usage}


In this chapter, we present a quantitative notion of input data usage, extending the definition introduced in the previous chapter.
We define three different quantifiers for the impact of the input variables on the program outcome, and we formalize a static analysis for verifying the resulting quantitative impact property of a program.
At the end, we present abstract versions of these quantifiers and show how they can be used to verify the corresponding quantitative property.
This chapter is based on work published at NASA Formal Methods Symposium (NFM) 2024 \cite{Mazzucato2024b}.
In the rest of this thesis, we will apply this quantitative framework in the context of neural networks and timing side-channel attacks.

\frenchdiv

\emph{Dans ce chapitre, nous présentons une notion quantitative de l'utilisation des données d'entrée, en étendant la définition introduite dans le chapitre précédent. Nous définissons trois mesures quantitatives différentes pour l'impact des variables d'entrée sur le résultat du programme, et nous formalisons une analyse statique pour vérifier la propriété d'impact quantitatif résultante d'un programme. À la fin, nous présentons des versions abstraites des mesures quantitatives et montrons comment elles peuvent être utilisées pour vérifier la propriété quantitative correspondante. Ce chapitre est basé sur un travail publié au Symposium sur les Méthodes Formelles de la NASA (NFM) 2024 \sidecite{Mazzucato2024b}. Dans le reste de cette thèse, nous appliquerons ce cadre quantitatif dans le contexte des réseaux neuronaux et des attaques par canal auxiliaire temporel.}

% \onlysectioncommands{\x,\y,\z,\lc,\exampleinput,\highlight,\inputa,\inputax,\inputay,\outputa,\inputb,\inputbx,\inputby,\outputb,\inputc,\inputcx,\inputcy,\outputc,\inputd,\inputdx,\inputdy,\outputd,\inpute,\inputex,\inputey,\outpute,\inputf,\inputfx,\inputfy,\outputf,\tracea,\traceax,\traceay,\traceb,\tracebx,\traceby,\tracec,\tracecx,\tracecy,\traced,\tracedx,\tracedy,\tracee,\traceex,\traceey,\tracef,\tracefx,\tracefy,\labelrotationangle, \outputvaluea,\outputvalueb,\outputvaluec,\outputvalued,\outputvaluee,\outputvaluef}


% %

\section{Landing Alarm System}
\labsec{landing-alarm-system}
\newcommand*{\x}{\texttt{angle}}
\newcommand*{\y}{\texttt{speed}}
\newcommand*{\z}{\texttt{risk}}
\newcommand*{\lc}{\texttt{landing\_coeff}}


\begin{marginlisting}
  \caption{Program for the landing-risk alarm system.}
  \labprog{landing-alarm-system}
  \vspace{0.5cm}
\begin{lstlisting}[language=customPython,escapechar=\%]
landing_coeff = abs(angle) + speed %\labline{compute-risk}%
if landing_coeff < 2 then %\labline{low-risk-cond}%
  risk = 0 %\labline{low-risk}%
else if landing_coeff > 5 then %\labline{high-risk-cond}%
  risk = 3 %\labline{high-risk}%
else %\labline{medium-risk-branch}%
  risk = floor(landing_coeff) - 2 %\labline{medium-risk}%
\end{lstlisting}
\end{marginlisting}


The goal of \refprog{landing-alarm-system}, referred to as \landingprogram, is to inform the pilot about the level of risk associated with the landing approach.
It takes two input variables, denoted as \x{} and \y, for the aircraft-airstrip alignment angle and the aircraft speed, respectively.
A value of 1 represents a good alignment while -4 a non-aligned angle, whereas 1, 2, 3 denote low, medium, and high speed.
A safer approach is indicated by lower speed.
The program \landingprogram{} computes first the landing risk coefficient, denoted as \lc, at \refline{compute-risk}.
This coefficient is obtained by taking the absolute value of the landing angle (accounting for both negative and positive approach angles) and adding it to the speed.
Afterwards, \refline{low-risk-cond} and~\refline{high-risk-cond} restrict \lc{} to range between values 2 and 5.
Values below 2 signify low danger, while values above 5 indicate high danger.
Respectively, we assign to the output variable \z{} the value of 0 for low danger and the value of 3 for high danger.
The medium degree of dangerousness is computed at \refline{medium-risk} and ranges between 1 and 2, which assigns to the output variable \z{} the largest integer less than, or equal, to $\lc-2$.
The output variable $\z{}$ is the danger level with possible values $\{0, 1, 2, 3\}$.

\begin{marginfigure}
\centering
\begin{tikzpicture}
  % Grid
  \draw[help lines, color=gray!30, dashed] (0,0) grid (2.9,3.9);
  % x-axis
  \draw[->,ultra thick] (0,0)--(3,0) node[right]{\x};
  % y-axis
  \draw[->,ultra thick] (0,0)--(0,4) node[above]{\y};
  % x-axis ticks
  \draw (0+1,0.1) -- (0+1,-0.1) node[below] {-4};
  \draw (1+1,0.1) -- (1+1,-0.1) node[below] {1};
  % y-axis ticks
  \foreach \y in {1,2,3}
  \draw (0.1,\y) -- (-0.1,\y) node[left] {\y};
  % Nodes
  \fill[color=seabornRed]   (0+1,0+1) circle[radius=2pt];
  \node[above right] at (0+1,0+1) {$3$};
  \fill[color=seabornRed]   (0+1,1+1) circle[radius=2pt];
  \node[above right] at (0+1,1+1) {$3$};
  \fill[color=seabornRed]   (0+1,2+1) circle[radius=2pt];
  \node[above right] at (0+1,2+1) {$3$};
  \fill[color=seabornGreen] (1+1,0+1) circle[radius=2pt];
  \node[above right] at (1+1,0+1) {$0$};
  \fill[color=seabornYellow] (1+1,1+1) circle[radius=2pt];
  \node[above right] at (1+1,1+1) {$1$};
  \fill[color=seabornOrange]    (1+1,2+1) circle[radius=2pt];
  \node[above right] at (1+1,2+1) {$2$};
\end{tikzpicture}
\caption{Input space composition of \refprog{landing-alarm-system}.}
\labfig{input-space-composition}
\end{marginfigure}

\reffig{input-space-composition} shows the input space composition of this system, where the label near each input represents the degree of risk assigned to the corresponding input configuration.
It is easy to note that a nonaligned angle of approach corresponds to a considerably higher level of risk, whereas the risk with a correct angle depends mostly on the aircraft speed.
Our goal is to develop a static analysis capable of quantifying the contribution of each input variable to the computation of the output variable $\z{}$.


\newcommand{\exampleinput}[1][\defaultprogramexampleletter]{\textsc{Input}_{#1}}

Understanding the effects of input variables on program executions allows us to reveal potential bugs or certify intended behavior.
Therefore, we present a framework to quantify the impact of input variables on the outcome of a program.
Intuitively, the definition of impact of a variable depends on various factors such as program structure, environment, developer expertise, and researcher intuition.
Different notions of impact leverage different traits of the input-output relations in computation.
As a result, it is not possible to formalize a definition that fits all requirements.

In general, all possible definitions of impact establish some relationship between variations of the input and the output values.
In this section, we introduce two impact notions.
The first impact definition focuses on \textit{the size of} extreme reachable values resulting from variations in the input variable.
The second exploits \textit{the number of} reachable outcomes.
Both definitions give us different insights on the program \landingprogram.
In particular, the first quantity tells us which variation in the values of input variables results in larger differences between output values, while the second indicates which variation reaches a greater number of output values.

\newcommand*{\highlight}[1]{\textcolor{seabornBlue}{#1}}
\newcommand*{\inputa}{\tuple{-4}{1}}
\newcommand*{\inputax}{\tuple{\highlight{-4}}{1}}
\newcommand*{\inputay}{\tuple{-4}{\highlight{1}}}
\newcommand*{\outputa}{\langle \outputvaluea\rangle} \newcommand*{\outputvaluea}{3}
\newcommand*{\inputb}{\tuple{-4}{2}}
\newcommand*{\inputbx}{\tuple{\highlight{-4}}{2}}
\newcommand*{\inputby}{\tuple{-4}{\highlight{2}}}
\newcommand*{\outputb}{\langle \outputvalueb\rangle} \newcommand*{\outputvalueb}{3}
\newcommand*{\inputc}{\tuple{-4}{3}}
\newcommand*{\inputcx}{\tuple{\highlight{-4}}{3}}
\newcommand*{\inputcy}{\tuple{-4}{\highlight{3}}}
\newcommand*{\outputc}{\langle \outputvaluec\rangle} \newcommand*{\outputvaluec}{3}
\newcommand*{\inputd}{\tuple{ 1}{1}}
\newcommand*{\inputdx}{\tuple{\highlight{ 1}}{1}}
\newcommand*{\inputdy}{\tuple{ 1}{\highlight{1}}}
\newcommand*{\outputd}{\langle \outputvalued\rangle} \newcommand*{\outputvalued}{0}
\newcommand*{\inpute}{\tuple{ 1}{2}}
\newcommand*{\inputex}{\tuple{\highlight{ 1}}{2}}
\newcommand*{\inputey}{\tuple{ 1}{\highlight{2}}}
\newcommand*{\outpute}{\langle \outputvaluee\rangle} \newcommand*{\outputvaluee}{1}
\newcommand*{\inputf}{\tuple{ 1}{3}}
\newcommand*{\inputfx}{\tuple{\highlight{ 1}}{3}}
\newcommand*{\inputfy}{\tuple{ 1}{\highlight{3}}}
\newcommand*{\outputf}{\langle \outputvaluef\rangle} \newcommand*{\outputvaluef}{2}
\newcommand*{\tracea}{\inputa\to\outputa}
\newcommand*{\traceax}{\inputax\to\outputa}
\newcommand*{\traceay}{\inputay\to\outputa}
\newcommand*{\traceb}{\inputb\to\outputb}
\newcommand*{\tracebx}{\inputbx\to\outputb}
\newcommand*{\traceby}{\inputby\to\outputb}
\newcommand*{\tracec}{\inputc\to\outputc}
\newcommand*{\tracecx}{\inputcx\to\outputc}
\newcommand*{\tracecy}{\inputcy\to\outputc}
\newcommand*{\traced}{\inputd\to\outputd}
\newcommand*{\tracedx}{\inputdx\to\outputd}
\newcommand*{\tracedy}{\inputdy\to\outputd}
\newcommand*{\tracee}{\inpute\to\outpute}
\newcommand*{\traceex}{\inputex\to\outpute}
\newcommand*{\traceey}{\inputey\to\outpute}
\newcommand*{\tracef}{\inputf\to\outputf}
\newcommand*{\tracefx}{\inputfx\to\outputf}
\newcommand*{\tracefy}{\inputfy\to\outputf}

\paragraph{First Impact Definition {\normalfont(\texorpdfstring{$\outcomesname$}{Outcomes})}.}
\labsec{overview:outcomes}
%
The first impact definition that we consider is
 $\outcomes(\defprogram)$, %(derived from $\outcomesentropy$),
where $\definputvariable$ is the input variable of interest and $\defprogram$ the program under analysis. Intuitively\sidenote{the formal definition is given in \refsec{quantitative-input-data-usage}}, $\outcomes$ returns the maximum number of outputs that are reachable from value variations of the input variable $\definputvariable$.
For the program
$\landingprogram$, the result is shown in column $\outcomesname(\landingprogram)$~of \reftab{overview}:
we obtain $\outcomesname_\x(\landingprogram)=2$ and $\outcomesname_\y(\landingprogram)=3$.

\newcommand*{\labelrotationangle}{30}
\begin{table*}[t]
  \centering
  \caption{Impact of for $\outcomesname(\landingprogram)$  and $\rangename(\landingprogram)$ definitions for both $\x$ and $\y$ variables. Computational features are \highlight{highlighted in blue}.}
  \labtab{overview}
  \begin{tabular}{c|c|c|c|c|c}
  \rotatebox[origin=c]{\labelrotationangle}{\textsc{Variable}}~ & ~\rotatebox[origin=c]{\labelrotationangle}{$\exampleinput[\landingprogram]$}~ & ~\textsc{Relevant Traces}~ & ~\rotatebox[origin=c]{\labelrotationangle}{\textsc{Outputs}}~ & ~\rotatebox[origin=c]{\labelrotationangle}{$\outcomesname$}~ & ~\rotatebox[origin=c]{\labelrotationangle}{$\rangename$}~ \\
  \toprule
  \multirow{6}{*}{\x}
   & $\inputax$ & $\traceax, \tracedx$ & $\{\outputvaluea,\outputvalued\}$ & \multirow{6}{*}{$2$} & \multirow{6}{*}{$3$} \\
  \cline{2-4}
   & $\inputbx$ & $\tracebx, \traceex$ & $\{\outputvalueb,\outputvaluee\}$ & & \\
  \cline{2-4}
   & $\inputcx$ & $\tracecx, \tracefx$ & $\{\outputvaluec,\outputvaluef\}$ & & \\
   \cline{2-4}
   & $\inputdx$ & $\tracedx, \traceax$ & $\{\outputvalued,\outputvaluea\}$ & & \\
   \cline{2-4}
   & $\inputex$ & $\traceex, \tracebx$ & $\{\outputvaluee,\outputvalueb\}$ & & \\
   \cline{2-4}
   & $\inputfx$ & $\tracefx, \tracecx$ & $\{\outputvaluef,\outputvaluec\}$ & & \\
  \midrule
  \multirow{12}{*}{\y}
   & \multirow{2}{*}{$\inputay$} & $\traceay, \traceby,$ & \multirow{2}{*}{$\{\outputvaluea\}$} & \multirow{12}{*}{$3$} & \multirow{12}{*}{$2$} \\
   & & $\tracecy$ & & & \\
  \cline{2-4}
   & \multirow{2}{*}{$\inputby$} & $\traceay, \traceby,$ & \multirow{2}{*}{$\{\outputvaluea\}$} & & \\
   & & $\tracecy$ & & & \\
  \cline{2-4}
   & \multirow{2}{*}{$\inputcy$} & $\traceay, \traceby,$ & \multirow{2}{*}{$\{\outputvaluea\}$} & & \\
   & & $\tracecy$ & & & \\
   \cline{2-4}
   & \multirow{2}{*}{$\inputdy$} & $\tracedy, \traceey,$ & \multirow{2}{*}{$\{\outputvalued,\outputvaluee, \outputvaluef\}$} & & \\
   & & $\tracefy$ & & & \\
   \cline{2-4}
   & \multirow{2}{*}{$\inputey$} & $\tracedy, \traceey,$ & \multirow{2}{*}{$\{\outputvalued,\outputvaluee, \outputvaluef\}$} & & \\
   & & $\tracefy$ & & & \\
   \cline{2-4}
   & \multirow{2}{*}{$\inputfy$} & $\tracedy, \traceey,$ & \multirow{2}{*}{$\{\outputvalued,\outputvaluee, \outputvaluef\}$} & & \\
   & & $\tracefy$ & & \\
   \bottomrule
  \end{tabular}
\end{table*}
The conclusion is that $\y$ has a greater influence than $\x$ on the output of the program.

\paragraph{Second Impact Definition {\normalfont(\texorpdfstring{$\rangename$}{Range})}.}
\labsec{overview:range}
%
The second impact definition we propose is $\rangename_\definputvariable$, which yields the maximum difference between the maximum and the minimum outputs that are reachable from value variations of the input variable $\definputvariable$.
The result for program $\landingprogram$ is shown in column $\rangename(\landingprogram)$~of \reftab{overview}:
the range of reachable outputs from variations of $\x$ is, at most, the interval $[0, 3]$, with a length of 3. Instead, the range of reachable outputs from variations of $\y$ is, at most, the interval $[0, 2]$, with a length of 2. Therefore, we obtain $\rangename_\x(\landingprogram)=3$ and $\rangename_\y(\landingprogram)=2$.
In other words, varying the angle of approach might drastically alter the landing risk, whereas the speed has less influence.
%
This is in contrast to the conclusion of \outcomesname{} where $\y$ has a greater impact than $\x$.
Although it may seem counterintuitive at first, the difference between the two impact instances is due to the different program traits they explore.
$\rangename$ quantifies over the variance in the extreme values of the set of output values, while $\outcomesname$ quantifies over the variance in the number of unique output values.
Consequently, changes in $\x$ yield a bigger variation in the degree of risk compared to $\y$, while changes in $\y$ reach far more risk levels compared to $\x$.
%
Note that, the impact definitions presented above are not computationally practical as they rely on a complete enumeration of all possible input configurations.
% Note that, enumerating all possible input configurations is not computationally practical.
Specifically, when dealing with more complex input space compositions, this approach is highly inefficient or even infeasible (as in the case of continuous input spaces).
As a consequence, our approach is based on an abstraction of input-output relations, which allows us to automatically infer a sound upper bound on the program's impact.

\paragraph{Static Analysis.}
\labsec{overview:static-analysis}

To quantify the impact of a program, one can rely solely on the input-output observations of the program.
Thus, our approach is based on an abstraction of input-output relations, which allows us to automatically infer a sound upper bound on the program's impact.

The analysis starts with a set of output abstractions called \textit{output buckets}.
A bucket is an abstract element representing a set of output states.
While this abstraction may limit the ability to precisely reason about the impact of output values within the same bucket, it permits automatic reasoning across different buckets.
Afterwards, an abstract interpretation-based static analyzer propagates each output bucket backward through the program under consideration.
The analyzer returns an abstract element for each output bucket, representing an over-approximation of the set of input configurations that lead to the output values inside the starting bucket.
This result contains also spurious input configurations that may not lead to a value inside the output bucket.
Based on the chosen impact definition \impactwrappername{} (e.g., \rangename{} or \outcomesname), we perform computations and comparisons on the abstract elements returned by the analysis to obtain an upper bound $\defbound'$. This upper bound is sound by construction of the theoretical framework, meaning that if $\defbound$ is the real (concrete) impact quantity obtained by \impactwrappername, then $\defbound\le\defbound'$.

Assuming \rangename{} as our impact of interest, given the set of abstract elements returned from the backward analysis (one for each bucket), we first project away the input variable $\definputvariable$ under consideration.
This means that our abstract elements now represent input configurations without the variable $\definputvariable$.
To be precise, we obtain \textit{partial input configurations} taking into account all possible variations of the input $\definputvariable$.
After the projection, we check for intersections among the abstract elements.
Any intersection may indicate two input configurations that only differ in the value of the variable $\definputvariable$, leading to two different output buckets.

Continuing our landing alarm system example, we start with three output buckets: $b_0=\{-1, 0\}$, $b_1=\{1, 2\}$, and $b_2=\{3, 4\}$ for instance.
Note that, only $\{0, 1, 2, 3\}$ are reachable \z{} values, but we may not have the exact post-condition of a program.
Therefore, for each of these three buckets, we run the backward analysis.
The backward analysis is parametric in the choice of the abstract domain, for instance, the disjunctive polyhedra domain based on the convex polyhedra domain~\sidecite{Cousot1978}.
Without showing the details of the analysis, starting with the bucket $b_0$, only the first branch of the if statement can be processed, \refline{low-risk}.
Therefore, \lc{} should be smaller than 2 at the end of \refline{compute-risk}.
Consequently, the sum of the absolute values of \x{} and \y{} should be smaller than 2.
Thus, our analysis discovers that only the input configuration $\tuple{1}{1}$ leads to the first bucket.
Similarly, we discover that from the bucket $b_1$, we reach the input configurations $\tuple{1}{2}$ and $\tuple{1}{3}$.
Lastly, from the bucket $b_2$, we reach the input configurations $\tuple{-4}{1}$, $\tuple{-4}{2}$, and $\tuple{-4}{3}$.

Regarding the case in which we quantify the impact of \x{}, we remove \x{} from each of the abstract elements returned by the analysis.
For example, from the bucket $b_0$, we have the abstract input configuration $\langle 1\rangle$\sidenote{The analysis discovers that only the input configuration $\tuple{1}{1}$ leads to the first bucket $b_0$. Thus, by removing \x{} from the input configuration, we obtain the tuple of a single element $\langle 1\rangle$, we use this notation for the rest of the section.}
; from the bucket $b_1$, we have $\langle 2\rangle$ and $\langle 3\rangle$; and from the bucket $b_2$, we have $\langle 1\rangle$, $\langle 2\rangle$, and $\langle 3\rangle$.
By checking the intersections, we find that bucket $b_0$ intersects with bucket $b_2$ since they have $\langle 1\rangle$ in common.
Bucket $b_2$ also intersects with bucket $b_1$ since they have both $\langle 2\rangle$ and $\langle 3\rangle$ in common.
Understanding the meaning of intersections is crucial.
For example, the first intersection between buckets $b_0$ and $b_2$ means that there exist two input configurations (namely $\tuple{1}{1}$ and $\tuple{-4}{1}$) that only differ in the value of \x{} (from $\x=1$ in the first configuration $\tuple{1}{1}$ to $\x=-4$ in the second configuration $\tuple{-4}{1}$), where the first configuration leads to an output in bucket $b_0$ and the second configuration leads to an output in the other bucket $b_2$.
Therefore, we are able to assign a range to this variation by taking the minimum of bucket $b_0$ and the maximum of bucket $b_2$, resulting in the range $[-1,4]$ with a size of 5.
Similarly, we obtain the range $[1,4]$ with a size of 3 from the other intersection between buckets $b_1$ and $b_2$.
Finally, we return the maximum among all possible sizes, which is 5.
This result is an over-approximation of the impact value for the function $\rangename_\x(\landingprogram)$.

Similarly, from the result of the backward analysis, we repeat the steps to quantify the impact regarding the other input variable \y{}.
Therefore, the analysis result is an impact of 3 since the only buckets intersecting after projecting away \y{} are buckets $b_0$ and $b_1$, with abstract input configurations of $\langle 1\rangle$ for both, while bucket $b_2$ contains only the value $\langle -4\rangle$.

Regarding the impact definition \outcomesname, whenever we discover intersections among different buckets, we count the total number of different outcome values they carry.
Consequently, we find that the two intersections for both variables \x{} and \y{} hold four different outcome values in total.
In such case, our analysis would conclude that the two variables hold the same impact.
Indeed, the choice of the starting output buckets is critical.
For instance, with $b_0=\{0\}$, $b_1=\{1\}$, $b_2=\{2\}$, and $b_3=\{3\}$ one would recover maximum precision and obtain a useful analysis quantification.

In conclusion, the sound upper bound discovered by our analysis is always higher than the concrete one by construction of the theoretical framework.
The precision of our analysis is mostly affected by the choice of output buckets and the approximation induced by the backward analysis \sidenote{as outlined by the use cases shown in~\refsec{nfm24:experimental-results}.}

% \section{Quantitative Input Data Usage}
\labsec{quantitative-input-data-usage}
\newcommand*{\x}{\texttt{angle}}
\newcommand*{\y}{\texttt{speed}}
\newcommand*{\z}{\texttt{risk}}
\newcommand*{\lc}{\texttt{landing\_coeff}}

In this section we introduce our quantitative framework with the formal definitions of \rangename{} and \outcomesname.


Our goal is to quantify the impact of a specific input variable on the computation of the program.
To this end, all the impact definition described by this work are extensional, \ie, they are based on the observation of input-output relations of program states.
Indeed, we base our definitions on the dependency semantics $\dependencysemanticsnoparam$, \refdef{dependency-semantics}.
We introduce the notion of impact, denoted by the function $\impactwrapper\in\setof\pairofstates\to\valuesposplus$, which maps program semantics to a non-negative domain of quantities, where $\definputvariable$ represents the input variable of interest in the program under analysis.

We implicitly assume the use of an \textit{output descriptor} $\outputdesc$ to determine the desired output of a program by observations on program states.

Specifically, $\reader\in \stateandbottom \to \valuesinf$ selects the output of interest from a given state and returns its corresponding value\sidenote{The option of returning $\pm\infty$ from the output descriptor is to deal with infinite traces, which do not have a final state ($\retrieveoutput\defseq = \statebottom$ for any $\defseq \in \infinitesequences$).}.
Additionally, $\filter\subseteq\valuesinf$ filters output states and selectively engages a subset of the potential outcomes.
Through this filtering mechanism, undesired outcomes are directly excluded, and a numerical value is ensured.

\inputprop{output-descriptor}
\marginprop{output-descriptor}

The above output characterization $\outputdesc$ is generic enough to cover plenty of use cases.
We leverage this output descriptor to provide the end user of the framework the flexibility to choose the interpretation and meaning of program outputs, without establishing it beforehand.

\begin{marginlisting}
\begin{lstlisting}[language=customPython]
landing_coeff =
  abs(angle) + speed
if landing_coeff < 2:
  risk = 0
else if landing_coeff > 5:
  risk = 3
else:
  risk =
    floor(landing_coeff) - 2
\end{lstlisting}
\end{marginlisting}

\begin{example}
  Consider the~\refprog{landing-alarm-system} for the landing alarm system with program states $\state=\setdef{\langle a, b, c, d \rangle}{a\in\{-4,1\}\land b\in\{1,2,3\}\land c\in\N \land d\in\{0,1,2,3\}}$, where $a$ is the value of $\x$, $b$ of $\y$, $c$ of $\lc$, and $d$ of $\z$.
  Here, we abuse the notation and use $\state$ as set of tuples instead of a map between variables and values, the two views are equivalent.
  The output descriptor is instantiated with
  \[
  \reader(x) \DefeQ \begin{cases}
    d & \text{if } x = \langle a, b, c, d \rangle \\
    +\infty & \text{otherwise}
  \end{cases}
  \]
  and $\filter=\{0,1,2,3\}$ filters $+\infty$ from the possible outputs.
  In other words, we are interested in the value of $\z$ for terminating traces.

  However, the end-user of the analysis may be interested in only a subset of the possible outcomes of the program.
  For instance, only about the risk levels in $\{0, 1, 2\}$, forgetting about the value $3$.
  It is crucial that our impact definitions remain sound to the user assumption on post-conditions, even when it is under-approximating the exact one.
  Thus, the filter specifies this information by $\filter=\{0, 1, 2\}$, which is a subset of all the possible values of the output variable \z.
\end{example}

\begin{example}
  For other contexts, rather than considering a single output variable one may be interested in a custom operation.
  % For example, in the context of neural network classifiers, the output of the program is the index of the output variable with highest value.
  For example, the output of a neural network classifier is the index of the output neuron holding the highest value.
  Hence, for a network with $n+1$ output neurons, we could instantiate
  \[
    \reader(x_0,\dots,x_{w+n-1},x_{w+n})=\argmax_{0\le j\le n} x_{w+j}
  \] where the function $\argmax_j X_j$ returns the \textit{argument} $j$ of the value holding the \textit{maximum} among the indexed family $X_j$.
  The filter $\filter$ could be the set of all indices $\{0,\dots,n\}$, hence permitting all possible outcomes from the reader $\reader$.
\end{example}



We can now define our property of interest, the \textit{$\defbound$-bounded impact property} $\bounded$.
By extension, $\bounded$ is the set of dependencies such that their impact, \wrt{} to the output descriptor $\outputdesc$ and the input variable $\definputvariable$, is below the threshold $\defbound\in\valuesposplus$.
%
\inputprop{bounded}


%
Following the definition of $\bounded$, our validation framework, \refeq{qualitative-soundness}, is instantiated as
%
\marginprop{collecting}
\begin{align}
  \labeq{bounded-soundness}
  \defprogram \satisfies \BOUNDED \IfF \collectingsemantics \subseteq \BOUNDED
\end{align}
%
We require $\impactwrapper$ to be monotonic, \ie, for any $X, Y\in \setof\finiteinfinitesequences$, it holds that $X \subseteq Y$ if and only if $\impactwrapper(X) \le \impactwrapper(Y)$.
Intuitively, this ensures that an impact applied to an over-approximation of the program semantics can only produce a higher quantity, enabling the definition of a sound terminating static analysis.
%
Next, we formalize the already introduced impact metrics \outcomesname{} and \rangename.

\subsection{The \outcomesname{} impact definition}[The Outcomes Impact Definition]
\labsec{outcomes}
%
Formally $\outcomes\in\setof\pairofstates\to\Nplus$ counts the number of different output values, allowed by the output descriptor, reachable by varying the input variable $\definputvariable\in\inputvariables$.
First, we define step-by-step the quantity $\outcomes$,
followed by the instantiation of this quantity within the context of example of \refprog{landing-alarm-system}.
We present the formal definition at the end.


Intuitively,
for any possible input configuration $\definput\in\reducedstate$, we collect all the traces that are starting from an input configuration that is a variation of $\definput$ on the input variable $\definputvariable$, \ie, $\setdef{
  \defseq\in \defsetoftraces
}{
  \retrieveinput{\defseq} \stateeq{\inputvariableswithouti} \definput
}$, where $\defsetoftraces\in\setof\finiteinfinitesequences$.
Then, we collect the output values of this set of traces by means of the output descriptor $\outputdesc$, \ie, $\setdef{
  \reader(\retrieveoutput\defseq)
}{
  \defseq \in \defsetoftraces \land
    \retrieveinput{\defseq} \stateeq{\inputvariableswithouti} \definput
}$. Specifically, this set contains all the output readings performed by $\reader$.
%
Afterwards, we extract the number of elements via the cardinality operator $\cardinality{\cdot}$.
Finally, we iterate through each input configuration $\definput$ and return the maximum value to ensure the greatest impact is preserved.

\begin{example}[Landing Alarm System]
  \label{ex:range}
  \newcommand*{\inputa}{\tuple{-4}{1}} \newcommand*{\outputa}{\langle \outputvaluea\rangle} \newcommand*{\outputvaluea}{3}
  \newcommand*{\inputb}{\tuple{-4}{2}} \newcommand*{\outputb}{\langle \outputvalueb\rangle} \newcommand*{\outputvalueb}{3}
  \newcommand*{\inputc}{\tuple{-4}{3}} \newcommand*{\outputc}{\langle \outputvaluec\rangle} \newcommand*{\outputvaluec}{3}
  \newcommand*{\inputd}{\tuple{ 1}{1}} \newcommand*{\outputd}{\langle \outputvalued\rangle} \newcommand*{\outputvalued}{0}
  \newcommand*{\inpute}{\tuple{ 1}{2}} \newcommand*{\outpute}{\langle \outputvaluee\rangle} \newcommand*{\outputvaluee}{1}
  \newcommand*{\inputf}{\tuple{ 1}{3}} \newcommand*{\outputf}{\langle \outputvaluef\rangle} \newcommand*{\outputvaluef}{2}
  \newcommand*{\tracea}{\inputa\to\outputa}
  \newcommand*{\traceb}{\inputb\to\outputb}
  \newcommand*{\tracec}{\inputc\to\outputc}
  \newcommand*{\traced}{\inputd\to\outputd}
  \newcommand*{\tracee}{\inpute\to\outpute}
  \newcommand*{\tracef}{\inputf\to\outputf}
  Let us revisit the example of the landing alarm system, with program states $\state=\setdef{\langle a, b, c, d \rangle}{a\in\{-4,1\}\land b\in\{1,2,3\}\land c\in\N \land d\in\Nle{3}}$.
  The input variables are $\inputvariables = \{\x,\y\}$, consequently the input configurations are
  $\reducedstate=\{\inputa, \inputb, \inputc, \inputd, \inpute, \inputf\}$.
%
  We begin by considering $\definputvariable=\x$ and $\inputa$ as the first input configuration to be explored.
  Hence, we collect all traces that are
  starting from an input configuration that is a variation of $\inputa$, \ie, $\setdef{
    \defseq \in \tracesemantics
  }{
    \retrieveinput{\defseq} \stateeq{\inputvariables\setminus\{\x\}} \inputa
  }$, where $\inputvariables\setminus\{\x\} = \{\y\}$ and consequently $\retrieveinput{\defseq} \stateeq{\{\y\}} \inputa$ holds whenever the initial state of $\defseq$ has $\y=1$. A possible trace of this set is $\langle 1, 1, 0, 0\rangle \to \langle 1, 1, 2, 0\rangle\to\langle 1, 1, 2, 0\rangle$ where, at the beginning, we randomly assign $\lc=0$ and $\z=0$, respectively the third and fourth component of the initial state.
%
  We collect the output values of this set of traces, $\setdef{
    \reader(\retrieveoutput\defseq)
  }{
    \defseq \in \tracesemantics \land
      \retrieveinput{\defseq}(\y) = 1
  }$.
  As a result, we obtain the set of output values $\{0, 3\}$.
  For instance, the output value $0$ is the result of the trace we exhibited previously, where the last state is $\langle 1, 1, 2, 0\rangle$ and thus the $\z$ variable of this trace is the last component with value $0$.
%
  Finally, the cardinality operator returns the value $2$, $\cardinality{\{0, 3\}} = 2$.
  By doing so for all possible input configurations in $\reducedstate$, we obtain $\outcomesname_{\x}(\tracesemantics)=3$.
  \reftab{range-x} and~\reftab{range-y} in overview (\refsec{overview}) illustrate the steps for $\outcomesname_{\x}(\tracesemantics)$ and $\outcomesname_{\y}(\tracesemantics)$ respectively.
\end{example}

\begin{definition}[\outcomesname]\labdef{outcomes}
  Given an input variable $\definputvariable\in\inputvariables$, and an output descriptor $\outputdesc$,
  the quantity $\outcomes\in\setof\finiteinfinitesequences\to\Rposplus$ is defined as
  %
  \begin{align}
    \labeq{outcomes}
    \outcomes(\defsetoftraces) &\DefeQ \sup_{\definput\in\reducedstate}
      \cardinality{\setdef{
        \reader(\retrieveoutput{\defseq})
      }{
        \defseq \in \defsetoftraces \land \retrieveinput{\defseq} \stateeq{\inputvariableswithouti} \definput
      }}
  \end{align}
  where $\sup(X)$ is the supremum operator, \ie, the smallest $q$ such that $q\ge x$ for all $x\in X$.
\end{definition}

It is easy to note that $\outcomes(\defsetoftraces)$ uses only input-output states (\cf, $\retrieveinput{\defseq}$ and $\retrieveoutput{\defseq}$) of traces $\defseq\in\defsetoftraces$.
Therefore, assuming $\defsetoftraces, \defsetoftraces'\in\setof\finiteinfinitesequences$ share the same set of input-output observations, $\outcomes(\defsetoftraces) = \outcomes(\defsetoftraces')$ holds.
This implies that, $\bounded$ is an extensional property when $\outcomes$ is used to instantiate $\impactwrapper$ in $\bounded$.
Furthermore, this impact definition is monotone in the amount of traces. That is, the more traces in input, the higher the impact as only more dependencies can satisfy the condition of \refeq{outcomes}, \cf~$\retrieveinput\defstate \stateeq{\inputvariableswithouti} \definputvariable$, and hence increase the number of outcomes.


\subsection{The \rangename{} impact definition}[The Range Impact Definition]
\labsec{range}

%
The quantity $\range\in\setof\pairofstates\to\Rposplus$ determines the
length of the range of output values from all the possible variations in the input variable $\definputvariable\in\inputvariables$.

\begin{example}[Landing Alarm System]
  \labexample{range}
  We revisit again the example of the landing alarm system.
  Assuming $\definputvariable=\x$, $\tuple{-4}{1}$ is the first input configuration to be explored, we collect all traces that are
  starting from an input configuration that is a variation of $\tuple{-4}{1}$.
  As before, we obtain $\setdef{
    \retrieveoutput\defseq(\z)
  }{
    \defseq \in \defsetoftraces \land
      \retrieveinput{\defseq}(\y) = 1
  }=\{0,3\}$.
%
  Here, we apply the length function, hence $\length(\{0,3\})=3$.
  By doing so for all possible input configurations $\reducedstate$, we obtain $\rangename_{\x}(\tracesemantics)=2$.
  \reftab{range-count} illustrates the steps for both $\rangename_{\x}(\tracesemantics)$ and $\rangename_{\y}(\tracesemantics)$.
\end{example}

  Formally,
\begin{definition}[\rangename]\labdef{range}
  Given an input variable $\definputvariable\in\inputvariables$, and an output descriptor $\outputdesc$,
  the quantity $\range\in\dependencytype\to\Rposplus$ is defined as
  %
  \begin{align*}
    \range(\defsetoftraces) &\DefeQ \sup_{\definput\in\reducedstate}
      \length(\setdef{
        \reader(\retrieveoutput{\defseq})
      }{
        \defseq \in \defsetoftraces \land \retrieveinput{\defseq} \stateeq{\inputvariableswithouti} \definput
      })
  \end{align*}
\end{definition}

This impact definition is monotone as well, in the amount of traces.
\begin{lemma}[\rangename{} is Monotonic]
  \lablemma{range-monotonic}
  For any $\defsetoftraces, \defsetoftraces'\in\dependencytype$, it holds that:
  \begin{align*}
    \defsetoftraces \subseteq \defsetoftraces' \ImplieS \range(\defsetoftraces) \le \range(\defsetoftraces')
  \end{align*}
\end{lemma}


% % \section{Filter Semantics}
\labsec{filter-semantics}
\newcommand*{\X}{X}
\newcommand*{\Y}{Y}
\newcommand*{\hiX}{\higher{X}}
\newcommand*{\hiY}{\higher{Y}}

From the dependency abstractions we derive the \textit{filter semantics}.
We exploit the output descriptor $\outputdesc$ to remove dependencies (tuple of input-output values) that are not relevant for our property, that is, not in $\filter$.

Formally, the pair of right-left adjoints $\tuple{\filterabstraction}{\filterconcretization}$ with $\filterabstraction, \filterconcretization\in\dependencytype\to\filtertype$ is defined as:
\begin{align}
  \labeq{filter-abstraction}
  \filterabstraction(\hiX)&\DefeQ\setdef{\setdef{\inputoutputtuple{\defseq}\in X}{\reader(\retrieveoutput{\defseq})\in\filter}}{X\in\hiX}\\
  \labeq{filter-concretization}
%   \filterconcretization(\hiY) &\DefeQ \setdef{\Y\setjoin\setdef{\inputoutputtuple{\defseq}}{\reader(\retrieveoutput{\defseq})\in Z\land \retrieveinput{\defseq}\in I}}{
%   \Y\in\hiY \land Z\subseteq\readerdomain\setminus\filter \land I\subseteq\state
% }
  \filterconcretization(\hiY) &\DefeQ \setdef{\Y\setjoin\setdef{\inputoutputtuple{\defseq}}{\tuple{\retrieveinput{\defseq}}{z}\in W \land \retrieveoutput{\defseq}\in \state \land \reader(\retrieveoutput{\defseq})=z}}{
  \Y\in\hiY \land W \subseteq \pair{\state}{(\readerdomain\setminus\filter)}
  }
\end{align}
where $
\filterabstraction$ abstracts away pair of states that lead to outputs that are not allowed by the output descriptor $\outputdesc$ maintaining the set-structure of $\hiX$.
% Similarly, double dot notation (written $\hihiX$) is a set of sets of sets of elements.
% The concretization $\filterconcretization(\hiY) \defeq \setdef{\X\in\setof\pairofstates}{
%   \exists \Y\in\hiY, Z\subseteq\readerdomain\setminus\filter, I\subseteq\state.~
%     \X=
%       \setdef{\inputoutputtuple{\defseq}}{\defseq\in \Y}\setjoin\setdef{\inputoutputtuple{\defseq}}{\reader(\retrieveoutput{\defseq})\in Z\land \retrieveinput{\defseq}\in I}
% }$ considers all the semantics $Y\in\setof\pairofstates$ such that it exists a semantics $\Y\in\hiY$ that shares the same input-output observations plus arbitrary observations made of discarded output values in $\readerdomain\setminus\filter$.
The concretization $\filterconcretization$ considers all the semantics $\Y\in\hiY$ extended with arbitrary observations made of discarded output values in $\readerdomain\setminus\filter$.
The abstract subset operator $\hiX \filtersubseteq \hiY$ checks subset inclusion among sets of traces allowed by the output descriptor $\outputdesc$, formally defined as:
\begin{align}
  \labeq{filter-subseteq}
  \hiX \filtersubseteq \hiY \DefifF
  \setdef{\X\in\hiX}{\foralldef{\inputoutputtuple{\defstate}\in\X}{\reader(\retrieveoutput{\defstate})\in\filter}}
  \subseteq
  \setdef{\Y\in\hiY}{\foralldef{\inputoutputtuple{\defstate}\in\Y}{\reader(\retrieveoutput{\defstate})\in\filter}}
\end{align}
%
Therefore, we define the following Galois connection.
%
\begin{theorem}
  \label{th:dependency-filter-galois}
  $\galoisbetweensemantics{dependency}{filter}$ is a Galois connection.
  \begin{proof}
    Given two set of sets of input-output observations $\hiX, \higher{Y}\in\setofsetof\pairofstates$ such that $\filterabstraction(\hiX)\filtersubseteq \higher{Y}$, we obtain that $\hiX\dependencysubseteq\filterconcretization(\higher{Y})$ since the concretization $\filterconcretization$ builds all the possible set of input-output observations enhanced with auxiliary observations made of output values previously discarded by $\filterabstraction(\hiX)$.
    On the other hand, we have $\hiX\dependencysubseteq\filterconcretization(\higher{Y})$.
    By abstracting we obtain $\filterabstraction(\hiX)$, which filters all semantics by removing traces that do not agree with $\outputdesc$.
    It holds that $\filterabstraction(\hiX)\filtersubseteq \higher{Y}$ since the $\filtersubseteq$ operator removes, in turn, traces that do not agree with $\outputdesc$.
  \end{proof}
\end{theorem}
% Note that, applying
% \begin{corollary}
%   Given $\hiX\in\filtertype$ such that $\foralldef{\X\in\hiX,\inputoutputtuple{\defseq}\in\X}{\reader(\retrieveoutput{\defseq})\in\filter}$, it holds that $\filterabstraction(\filterconcretization(\hiX))=\hiX$.
% \end{corollary}
%
We derive the filter semantics $\filtersemanticsnoparam\in\setofsetof\pairofstates$ as follows:
%
\begin{align}
  \labeq{filter}
  \filtersemanticsnoparam &\DefeQ \filterabstraction(\dependencysemanticsnoparam) = \filterabstraction(\{\setdef{
    \inputoutputtuple{\defseq}
  }{
    \defseq\in \tracesemanticsnoparam
  }\}) \\
    &\;=~ \{{\setdef{\inputoutputtuple{\defseq}\in \setdef{
      \inputoutputtuple{\defseq}
    }{
      \defseq\in \tracesemanticsnoparam
    }}{\reader(\retrieveoutput{\defseq})\in\filter}}\} \\
    &\;=~ \{{\setdef{\inputoutputtuple{\defseq}}{\defseq\in \tracesemanticsnoparam\land\reader(\retrieveoutput{\defseq})\in\filter}}\}
\end{align}
%
The next result shows that $\filtersemanticsnoparam$ allows sound and complete verification for proving that the impact of a program is bounded by $\defbound$.
%
\begin{theorem}
  \label{th:filter-soundness}
  $\collectingsemantics \subseteq \BOUNDED \IfF \filtersemantics \filtersubseteq \filterabstraction(\dependencyabstraction(\BOUNDED))$
  % \begin{proof}
  %   Let $\dependencysemantics \subseteq \dependencyabstraction(\bounded)$. From the Galois connection described in \refthm{dependency-filter-galois}, we have that $\filterabstraction(\dependencysemantics) \subseteq \filterabstraction(\dependencyabstraction(\bounded))$.
  %   From the definition of $\filtersemantics$, \refeq{filter}, we can conclude $\filtersemantics \subseteq \filterabstraction(\dependencyabstraction(\bounded))$. The other direction is symmetric.
  % \end{proof}
  \begin{proof}
    The implication $(\implies)$ follows from \refthm{dependency-soundness}, the monotonicity of $\filterabstraction$, \refthm{dependency-filter-galois}, and the definition of $\filtersemanticsnoparam$, \refeq{filter}, \ie,
    $
    \collectingsemanticsnoparam \subseteq \bounded \implies
      \dependencysemanticsnoparam \subseteq \dependencyabstraction(\bounded) \implies \filterabstraction(\dependencysemanticsnoparam) \filtersubseteq \filterabstraction(\dependencyabstraction(\bounded))\implies \filtersemanticsnoparam \filtersubseteq \filterabstraction(\dependencyabstraction(\bounded))
    $.
    Regarding the other implication $(\Leftarrow)$, by the definition of $\filtersemanticsnoparam$, $\dependencysemanticsnoparam$, and the property of Galois connections, we obtain
    \begin{align}
      \filtersemanticsnoparam
        \filtersubseteq &\;\, \filterabstraction(\dependencyabstraction(\bounded)) \\
      &~\implies \filterabstraction(\dependencysemanticsnoparam) \filtersubseteq \filterabstraction(\dependencyabstraction(\bounded)) &&(\text{by \refeq{filter}}) \\
      &~\implies \dependencysemanticsnoparam
        \subseteq \filterconcretization(\filterabstraction(\dependencyabstraction(\bounded))) &&(\text{by \refthm{dependency-filter-galois}}) \\
      &~\implies \setdef{\inputoutputtuple{\defseq}}{\defseq\in\tracesemanticsnoparam} \in \filterconcretization(\filterabstraction(\dependencyabstraction(\bounded))) &&(\text{by \refeq{dependency}})
    \end{align}
    We split two cases depending whether $\setdef{\inputoutputtuple{\defseq}}{\defseq\in\tracesemanticsnoparam}$ contains only traces allowed by $\outputdesc$:
    \begin{enumerate}[label=(\roman*)]
      \item $\foralldef{\defseq\in\tracesemanticsnoparam}{\reader(\retrieveoutput{\defseq})\in\filter}$ holds. In this case it is easy to note that $\setdef{\inputoutputtuple{\defseq}}{\defseq\in\tracesemanticsnoparam} \in \dependencyabstraction(\bounded)$ since $\filterconcretization$ and $\filterabstraction$ do not affect semantics that already satisfy the output descriptor. \label{first-case}
      \item $\existsdef{\defseq\in\tracesemanticsnoparam}{\reader(\retrieveoutput{\defseq})\not\in\filter}$ holds. Given $\defsetoftraces\subseteq\setdef{\inputoutputtuple{\defseq}}{\defseq\in\tracesemanticsnoparam}$ such that any trace $\defseq\in\defsetoftraces$ is allowed by $\reader(\retrieveoutput{\defseq})\in\filter$, from the previous case it is evident that $\defsetoftraces\in\dependencyabstraction(\bounded)$. By the definition of $\bounded$, \refeq{bounded}, it holds that any superset $\defsetoftraces'\supseteq\defsetoftraces$ belongs to $\bounded$ when $\defsetoftraces$ contains only traces allowed by the output descriptor and $\defsetoftraces\in\bounded$. Hence, it holds that $\setdef{\inputoutputtuple{\defseq}}{\defseq\in\tracesemanticsnoparam} \in \dependencyabstraction(\bounded)$. \label{second-case}
    \end{enumerate}
    In both cases~\ref{first-case} and~\ref{second-case} we obtain $\setdef{\inputoutputtuple{\defseq}}{\defseq\in\tracesemanticsnoparam} \in \dependencyabstraction(\bounded)$.
    The conclusion $\collectingsemanticsnoparam \subseteq \bounded$ trivially follows from the definition of $(\subseteq)$ and \refthm{dependency-soundness}.
  \end{proof}
\end{theorem}

This semantics still contains enough information to soundly and completely prove our any property $\bounded$, therefore potentially undecidable.
% Next step is to define sound-only semantics that are approximating the property of interest therefore leading to an impact bound which is higher than the actual yet undecidable one.

% \section{A Static Analysis for Quantitative Input Data Usage}
\labsec{static-analysis}

%
In this section, we introduce a sound computable static analysis to determine an upper bound on the impact of an input variable $\definputvariable$.
The soundness of the approach leverages two elements: $(1)$ an underlying abstract semantics $\backwardsemanticsnoparam$ to compute an over-approximation of the filter semantics $\filtersemanticsnoparam$; and $(2)$ a sound computable implementation of $\impactwrapper$, written $\impactinstance$, used in the property $\bounded$.

To quantify the usage of an input variable, we need to determine the input configurations leading to specific output values.
As our impact definitions $\outcomes$ and $\range$ measure over the different output values (i.e., $\reader(\retrieveoutput{\defstate})$) our underlying abstract semantics will be a \emph{backward} (co-)reachability semantics starting from \emph{disjoint} abstract post-conditions, over-approximating the (concrete) output values of the dependency semantics.
Specifically, we abstract the concrete output values with an indexed set $\buckets\in\vectorbuckets$ of $n$ disjoint \textit{output buckets}, where $\abstractdomainlattice$ is an abstract state domain with concretization function  $\abstractdomainconcretization\in\abstractdomain\to\setof\stateandbottom$. The choice of these output buckets is essential for obtaining a precise and meaningful analysis result.

For each output bucket $\bucket\in\abstractdomain$, our analysis computes an over-approximation of the dependency semantics restricted to the input configurations leading to $\abstractdomainconcretization(\bucket)$.
More formally, let $\reduce[\dependencysemanticsnoparam]{X} \defeq \setdef{\inputoutputtuple{\defstate}\in\dependencysemanticsnoparam}{\retrieveoutput{\defstate}\in X}$ be the reduction of the dependency semantics $\dependencysemanticsnoparam$ to the dependencies with final states in $X$.
%
Our static analysis is parametrized by an underlying backward abstract family\sidenote{A family of semantics is a set of program semantics parametrized by an initialization.}
of semantics $\backwardsemantics\in\backwardtype$ which computes the backward semantics $\backwardsemantics\bucket$ from a given output bucket $\bucket\in\abstractdomain$.
The concretization function $\backwardconcretization\in(\backwardtype)\to\abstractdomain\to\setof\pairofstates$ employs %the abstract concretization
$\abstractdomainconcretization$ to restore all possible input-output dependencies, \ie, $\backwardconcretization(\backwardsemantics)\bucket \defeq \setdef{\inputoutputtuple{\defstate}}{\retrieveinput{\defstate}\in\abstractdomainconcretization(\backwardsemantics\bucket)\land\retrieveoutput{\defstate}\in\abstractdomainconcretization(\bucket)}$.
We can thus define the soundness condition for the backward semantics with respect to the reduction of the dependency semantics.

\begin{definition}[Sound Over-Approximation for \texorpdfstring{$\backwardsemanticsnoparam$}{the backward semantics}]\label{def:sound-over-approximation}
  For all programs $\defprogram$, and output bucket $\bucket\in\abstractdomain$, the family of semantics $\backwardsemanticsnoparam$ is a \textup{sound over-approximation} of the dependency semantics $\dependencysemanticsnoparam$ reduced with  $\abstractdomainconcretization(\bucket)$, when it holds that:
  \[\reduceddependencysemantics \SubseteQ \backwardconcretization(\backwardsemantics)\bucket\]
\end{definition}

We define
$\multisemanticsnoparam\in\multitype$ as the backward semantics repeated on a set of output buckets $\buckets\in\vectorbuckets$, that is, $\multisemantics\buckets \defeq (\backwardsemantics\bucket)_{j\le n}$.
Again, the concretization function $\multiconcretization\in(\multitype)\to\vectorbuckets\to\setof\pairofstates$ employs the abstract concretization $\abstractdomainconcretization$ to restore all possible input-output dependencies over all the output buckets, \ie,
$\multiconcretization(\multisemantics)\buckets \defeq \bigsetjoin_{j\le n} \setdef{\inputoutputtuple{\defstate}}{\retrieveinput{\defstate}\in\abstractdomainconcretization((\multisemantics\buckets)_j)\land\retrieveoutput{\defstate}\in\abstractdomainconcretization(\bucket)}$.

\begin{lemma}[Sound Over-Approximation for \texorpdfstring{$\multisemanticsnoparam$}{the multi-bucket semantics}]\label{lm:sound-over-approximation-multi-bucket}
  For all programs $\defprogram$, output buckets $\buckets\in\vectorbuckets$, and a family of semantics $\backwardsemanticsnoparam$, the %multi-bucket family of
  semantics $\multisemanticsnoparam$ is a \textup{sound over-approximation} of the dependency semantics $\dependencysemanticsnoparam$ when reduced to $\bigsetjoin_{j\le n}\abstractdomainconcretization(\bucket)$:
  \[\reduce{\bigsetjoin_{j\le n}\abstractdomainconcretization(\bucket)} \SubseteQ \multiconcretization(\multisemantics)\buckets\]
\end{lemma}

Whenever the output buckets \textit{cover} the whole output space, $\multisemanticsnoparam$ is a sound over-approximation of $\dependencysemanticsnoparam$.
The concept of covering for output buckets ensures that no final states of the dependency semantics, \ie{} $\finalstatesdependency\defeq\setdef{\retrieveoutput{\defstate}}{\inputoutputtuple\defstate\in\dependencysemanticsnoparam}$, are missed from the analysis.

\begin{definition}[Covering]\label{def:covering}
  We say that the output buckets $\buckets\in\vectorbuckets$ \textit{cover} the whole output space whenever $\finalstatesdependency\subseteq\bigsetjoin_{j\le n}\abstractdomainconcretization(\bucket)$.
\end{definition}

% \newcommand{\resultofbucketj}{\hiX_{j}}
% \newcommand{\resultofbucketjk}{\hiX_{j,k}}
% \newcommand{\resultofbucket}{\hiX}

Next, we expect a sound implementation $\impactinstance\in\pair\vectorbuckets\vectorbuckets\to\valuesinf$ to return a bound on the impact which is always higher than the concrete counterpart $\impactwrapper$.

\begin{definition}[Sound Implementation]\label{def:sound-implementation}
  For all output buckets $\buckets$ and family of semantics $\backwardsemanticsnoparam$, $\impactinstance$ is a \textup{sound implementation} of $\impactwrappername$, whenever
  \[
    \impactwrapper(
      \multiconcretization(\multisemantics)\buckets
    ) \LE \impactinstance(\multisemantics\buckets, \buckets)
  \]
\end{definition}

% \newcommand{\resultofproject}{\higher{Y}}
% \newcommand{\resultofprojectj}{\higher{Y}_{j}}
% \newcommand{\resultofprojectjk}{\higher{Y}_{j,k}}


The next result shows that our static analysis is sound when employed to verify the property of interest $\bounded$ for the program $\defprogram$.
That is, if %the computation of
$\impactinstance$ returns the bound $\defbound'$, and $\defbound'\le\defbound$, then the program $\defprogram$ satisfies the property $\bounded$, \cf{} $\defprogram \satisfies \bounded$.


\begin{theorem}[Soundness] \label{th:soundness}
  Let $\bounded$ be the property of interest we want to verify for the program $\defprogram$ and the input variable $\definputvariable\in\inputvariables$.
  Whenever,
  \begin{enumerate}[label=(\roman*)]
    \item $\backwardsemanticsnoparam$ is sound with respect to $\dependencysemanticsnoparam$, \cf{} \refdef{sound-over-approximation}, and
    \item $\buckets$ covers the whole output space, \cf{} \refdef{covering}, and
    \item $\impactinstance$ is a sound implementation of $\impactwrapper$, \cf{} \refdef{sound-implementation},
  \end{enumerate}
  the following implication holds:
  \begin{align}
    \impactinstance(\multisemantics\buckets, \buckets) = \defbound' \LanD \defbound' \le \defbound \ImplieS \defprogram \satisfies \bounded
  \end{align}
\end{theorem}

Finally,
we define $\abstractrange$ and $\abstractoutcomes$
as possible implementations for $\range$ and $\outcomes$, respectively.
%
We assume the underlying abstract state domain $\abstractdomain$ is equipped with an
operator $\abstractdomainproject\in\abstractdomain\to\abstractdomain$
to project away the input variable $\definputvariable$.
For example, in the context of the interval domain, where each input variable is related to a possibly unbounded lower and upper bound, $\abstractdomainproject(\langle\definputvariable \mapsto [1, 3], j \mapsto [2, 4]\rangle) = \langle \definputvariable \mapsto [-\infty, \infty], j \mapsto [2, 4] \rangle$
removes the constraints related to $\definputvariable$.

The definition of $\abstractoutcomes$ first projects away the input variable $\definputvariable$ from all the given abstract values, then it collects all intersecting abstract values via the meet operator $\abstractdomainmeet$.
These intersections represent potential concrete input configurations where variations on the value of $\definputvariable$ lead to changes of program outcome, from a bucket to another.
We return the maximum number of abstract values that intersects after projections:
\begin{equation}\label{eq:abstract-outcomes}
  \abstractoutcomes(X^\natural, \buckets) \defeq \max~\setdef{\cardinalitynospaces{J}}{J \in \intersectallfunction((\abstractdomainproject(X^\natural_j))_{j\le\numberofbuckets})}
\end{equation}
Note the use of $\max$ instead of $\sup$ as in the concrete counterpart (\refeq{outcomes}) since the number of intersecting abstract values is bounded by $n$, i.e., the number of output buckets.
The function $\intersectallfunction$ takes as input an indexed set of abstract values and returns the set of indices of abstract values that intersect together, defined as follows:
\begin{equation*}
  \intersectallfunction(X^\natural\in\vectorbuckets) \defeq \setdef{J}{J \subseteq \N \land \forall j\le n, p\le n.~ j\in J \land p\in J \LanD X^\natural_j \abstractdomainmeet X^\natural_{p}}
\end{equation*}
Finding all the indices of intersecting abstract values is equivalent to find cliques in a graph, where each node represents an abstract value and an edge exists between two nodes if and only if the corresponding abstract values intersect.
Therefore, $\intersectallfunction$ can be efficiently implemented based on the graph algorithm by~\sidecite{Bron1973}.

Similarly, we define $\abstractrange$ as the maximum length of the range of the extreme values of the buckets represented by intersecting abstract values after projections.
In such case, we assume $\abstractdomain$ is equipped with an additional abstract operator $\abstractdomainlength\in\abstractdomain\to\valuesposplus$, which returns the length of the given abstract element, otherwise $+\infty$ if the abstract element is unbounded or represents multiple variables.
\begin{align}\label{eq:abstract-range}
  \abstractrange(X^\natural, \buckets) \DefeQ& \max~\seTDef{\abstractdomainlength(K)}{K \in I} \\
  \text{where}~
    I ~=~& \seTDef{\abstractdomainjoin\seTDef{\bucket}{j\in J}}{J \in \intersectallfunction((\abstractdomainproject(X^\natural_j))_{j\le\numberofbuckets})}
\end{align}

% \section{Experimental Results}
\labsec{experimental-results}
\newcommand*{\x}{\texttt{angle}}
\newcommand*{\y}{\texttt{speed}}
\newcommand*{\z}{\texttt{risk}}


The goal of this section is to highlight the potential of our static analysis for quantitative input data usage.
We implemented a proof-of-concept tool, called \impatto\sidenote{\url{https://github.com/denismazzucato/impatto}}, in Python 3 that employs the \interproc\sidenote{\url{https://github.com/jogiet/interproc}} abstract interpreter to perform the backward analysis.
Then, we exploited this tool to automatically derive a sound input data usage of three different scenarios.
As each impact result must be interpreted with respect to what the program computes, we analyze each scenario separately.

\subsection{Growth in a Time of Debt}
\labsec{rr}


Reinhart and Rogoff article ``Growth in a Time of Debt''~\sidecite{Reinhart2010} proposed a correlation between high levels of public debt and low economic growth,
and %. As a consequence, the article
was heavily cited to justify austerity measures around the world. %Notably,
One of the several errors discovered in the article is the incorrect usage of the input value relative to Norway's economic growth in 1964.
The data used in the article is publicly available but not the spreadsheet file. We reconstructed this simplified example based on
the technical critique by \sidetextcite{Herndon2014}, and an online discussion\sidenote{\url{https://economics.stackexchange.com/q/18553}}.
The~\refprog{rr}
computes the cross-country mean growth for the public debt-to-GDP $60-90\%$ category, key point to the article's conclusions.
The input data is the average growth rate for each country within this public dept-to-GDP category. The problem with this computation is that Norway has only one observation in such category, which alone could disrupt the mean computation among all the countries. Indeed, the year that Norway appears in the $60-90\%$ category achieved a growth rate of $10.2\%$, while the average growth rate for the other countries is $2.7\%$.
With such high rate, the mean growth rate raised to $3.4\%$, altering the article's conclusions.
We assume growth rate values between $-20\%$ and $20\%$ for all countries, consequentially, the output ranges are between these bounds as well. We instrumented the output buckets to cover the full output space in buckets of size $1$, \ie, $\setdef{t \le \texttt{avg} < t + 1}{-20 \le t \le 20}$.
%
\newcommand{\dg}{60}
\begin{table*}[t]
  \caption{Quantitative input usage for \refprog{rr} from the Reinhart and Rogoff's article.}
  \labtab{rr}
  \centering
  \begin{tabular}{c | ccccccccccc}
    \textsc{Impact} & \rotatebox{\dg}{\texttt{portugal1}} & \rotatebox{\dg}{\texttt{portugal2}} & \rotatebox{\dg}{\texttt{portugal3}} & \rotatebox{\dg}{\texttt{norway1}} & \rotatebox{\dg}{\texttt{uk1}} & \rotatebox{\dg}{\texttt{uk2}} & \rotatebox{\dg}{\texttt{uk3}} & \rotatebox{\dg}{\texttt{uk4}} & \rotatebox{\dg}{\texttt{us1}} & \rotatebox{\dg}{\texttt{us2}} & \rotatebox{\dg}{\texttt{us3}} \\
    \toprule
    \outcomesname{} & 5 & 5 & 5 & 10 & 2 & 2 & 2 & 2 & 3 & 3 & 3 \\
    \rangename{} & 5 & 5 & 5 & 10 & 2 & 2 & 2 & 2 & 3 & 3 & 3 \\
    \bottomrule
  \end{tabular}
\end{table*}
%
Results for both \outcomesname{} and \rangename{} are shown in \reftab{rr}.

\begin{lstlisting}[
  language=customPython,
  escapechar=\%,
  caption={Program computing the mean growth rate in the $60-90\%$ category.},
  label={lst:rr},
  % float,
  % floatplacement=H
]
 def mean_growth_rate_60_90(
     portugal1, portugal2, portugal3,
     norway1,
     uk1, uk2, uk3, uk4,
     us1, us2, us3):
   portugal_avg = (portugal1 + portugal2 + portugal3) / 3%\label{l:portugal-avg}%
   norway_avg = norway1%\label{l:norway-avg}%
   uk_avg = (uk1 + uk2 + uk3 + uk4) / 4%\label{l:uk-avg}%
   us_avg = (us1 + us2 + us3) / 3%\label{l:us-avg}%
   avg = (portugal_avg + norway_avg + uk_avg + us_avg) / 4%\label{l:final-avg}%
\end{lstlisting}
%\vspace{%-15pt}
%
%
The analysis discovers that the Norway's only observation for this category $\texttt{norway1}$ has the biggest impact on the output, as perturbations on its value are capable of reaching 10 different outcomes (\cf~column $\texttt{norway1}$), while the other countries only have 5, 2, and 3, respectively for Portugal, UK, and US.
The same applies to \rangename{} as the output buckets have size $1$ and all the input perturbations are only capable of reaching contiguous buckets. Hence, we obtain the same exact results.

Our analysis is able to discover the disproportionate impact of Norway's only observation in the mean computation, which would have prevented one of the several programming errors found in the article.
%Nevertheless,
From a review of~\refprog{rr}, it is clear that Norway's only observation has a greater contribution to the computation,
%of the average growth rate,
as it does not need to be averaged with other observations first.
However, such methodological error is less evident when dealing with a higher number of input observations ($1175$ observations in the original work) and the computation is hidden behind a spreadsheet.

% As noted in many reports, a possible solution would be to improve the weighting procedure or filter outliers.


\subsection{GPT-4 Turbo}
\labsec{gpt-4-turbo}

The second use case we present is drawn from Sam Altman's OpenAI keynote in September 2023\sidenote{\url{https://www.youtube.com/live/U9mJuUkhUzk?si=HOzuH3-gr_kTdhCt&t=2330}}, where he presented the GPT-4 Turbo.
This new version of the GPT-4 language model brings the ability to write and interpret code directly without the need of human interaction.
Hence, as showcased in the keynote, the user could prompt multiple information to the model, such as related to the organization of a holiday trip with friends in Paris, and the model automatically generates the code to compute the share of the total cost of the trip and run it in background.
In this environment, users are unable to directly view the code unless they access the backend console.
This limitation makes it challenging for them to evaluate whether the function has been implemented correctly or not, assuming users have the capability to do so.
%
From the keynote, we extracted the~\refprog{share-division} which computes the user's share of the total cost of a holiday trip to Paris, given the total cost of the Airbnb, the flight cost, and the number of friends going on the trip.
%
\begin{table}
  \caption{Quantitative input usage for \refprog{share-division} computing the share division among friends.}
  \labtab{gpt-4-turbo}
  \begin{tabular}{c | ccc}
    \textsc{Impact} & \rotatebox{45}{\texttt{airbnb\_total\_cost\_eur}} & \rotatebox{45}{\texttt{flight\_cost\_usd}} & \rotatebox{45}{\texttt{number\_of\_friends}} \\
    \toprule
    \outcomesname{} & 10 & 17 & 9 \\
    \rangename{} & 1099 & 1709 & 999 \\
    \bottomrule
  \end{tabular}
\end{table}
%

\begin{lstlisting}[
  language=customPython,
  escapechar=\%,
  label={lst:share-division},
  caption={Program computing share division for holiday planning among friends.},
  % float,
  % floatplacement=H
]
 def share_division(
     airbnb_total_cost_eur,
     flight_cost_usd,
     number_of_friends):
   share_airbnb = airbnb_total_cost_eur / number_of_friends
   usd_to_eur = 0.92
   flight_cost_eur = flight_cost_usd * usd_to_eur
   total_cost_eur = share_airbnb + flight_cost_eur
\end{lstlisting}
%
Regarding the input bounds, users are willing to spend between 500 and 2000 for the Airbnb, between 50 and 1000 for the flight, and travel with between 2 and 10 friends. As a result, they expect their share, variable $\texttt{total\_cost\_eur}$, to be between 90 and 1900.
To compute the impact of the input variables we choose the output buckets to cover the expected output space in buckets of size $100$, \ie, $\setdef{100t + 90 \le \texttt{total\_cost\_eur} < \min \{100(t + 1) + 90, 1900\}}{0 \le t \le 19}$.
The %analysis discovers similar
findings are similar for both the \outcomesname{} and \rangename{} analysis, see~\reftab{gpt-4-turbo}.
The input variable $\texttt{flight\_cost\_usd}$ has the biggest impact on the output, as perturbations on its value are capable of reaching 17 different output buckets (resp. a range of 1709 output values), while the other two, $\texttt{airbnb\_total\_cost\_eur}$ and $\texttt{number\_of\_friends}$, only reach 10 and 9 output buckets (resp. have ranges of size 1099 and 999), respectively.
%Since the output buckets reached by perturbations of input values are contiguous, the \rangename{} analysis shows similar findings: 1709, 1099, and 999, respectively for \texttt{flight\_cost\_usd}, \texttt{airbnb\_total\_cost\_eur}, and \texttt{number\_of\_friends},

These results confirm the user expectations about the proposed program from ChatGPT: the flight cost yields the biggest impact as it cannot be shared among friends.


\subsection{Termination Analysis (A)}
\labsec{termination-analysis-A}

\refprog{timing-analysis} is adapted from the termination category of the software verification competition \textsc{sv-comp}\sidenote{\url{https://sv-comp.sosy-lab.org/}}.
%the work of \sidecite{Gopan2006} .
%This use case comes from
Assuming both input positives, $\texttt{x},\texttt{y} \ge 0$, this program terminates in $\texttt{x}+1$ iterations if $\texttt{y} >50$, otherwise it terminates in $\texttt{x} - 2\texttt{y} + 103$ iterations.
We define $\texttt{counter}$ as the output variable, with output buckets defined as $\setdef{10k \le \texttt{counter} < 10(k+1)}{0 \le k < 50}$ and $\{\texttt{counter}\ge 500\}$. These output buckets represent cumulative ranges of iterations required for termination.
The analysis results are illustrated in~\reftab{termination-analysis}, they show that the input variable $\texttt{x}$ has the biggest impact.
Modifying the value of $\texttt{x}$ can result in the program terminating within any of the other 50 iteration ranges.
On the other hand, perturbations on $\texttt{y}$ can only result in the program terminating within 10 different iteration ranges.
%
\begin{table}
  \begin{tabular}{c | c@{\hskip 5pt}c}
    \textsc{Impact} & \rotatebox{0}{\texttt{x}} & \rotatebox{0}{\texttt{y}} \\
    \toprule
    \outcomesname{} & 50 & 10 \\
    \rangename{} & 499 & 99 \\
    \bottomrule
  \end{tabular}
  \caption{Quantitative input usage for \refprog{timing-analysis}.}
  \labtab{termination-analysis}
\end{table}
%
\begin{lstlisting}[
    language=customPython,
    escapechar=\%,
    label={lst:timing-analysis},
    caption={Timing analysis.},
    ]
 def example(x, y):
   counter = 0
   while x >= 0:
     if y <= 50:
       x += 1
     else
       x -= 1
     y += 1
     counter += 1
\end{lstlisting}
%
Such difference is motivated by the fact that $\texttt{y}$ is only used to determine the number of iterations in the case where $\texttt{y}$ is greater than 50, otherwise it is not used at all. Therefore, two values of $\texttt{y}$, \eg, $y_0$ and $y_1$, only result in two different ranges of iterations required to make the program terminate if either both of them are below $50$ or $y_0 < 50\land y_1 \ge 50$ or $y_0\ge50\land y_1 <50$, not in all the cases.

The given results can be interpreted as follows: the speed of termination of this loop is highly dependent on the value of $\texttt{x}$, while $\texttt{y}$ has a much smaller impact.
% This information could be used to attack such program just by looking at its timing behavior.
% For instance, given the number of iterations \texttt{counter}, we infer that the value of \texttt{x} is either $\texttt{counter} - 1$ or $\texttt{counter} + 2\texttt{y} - 103$. On the other hand, since \texttt{y} has less impact as discovered by our tool, we cannot infer much about its value.



\subsection{Termination Analysis (B)}
\labsec{app:termination-analysis-B}

This use case comes from the software verification competition SV-Comp\sidenote{\url{https://sv-comp.sosy-lab.org/}}, where the goal is to verify the termination of a program. \refprog{termination-a} and~\refprog{termination-b} have originally been proposed by \sidetextcite{Chen2012}, respectively these are Example (2.16) and Example (2.21) of such work.

\begin{lstlisting}[
  language=customPython,
  escapechar=\%,
  caption={Program Ex2.16 from software verification competition SV-Comp.},
  label={lst:termination-a}
]
def termination_a(x, y):
  while x > 0:
    x = y
    y = y - 1
  result = x + y
  return result
\end{lstlisting}

\refprog{termination-a} returns the value of \texttt{y} whenever $\texttt{x} = 0$, otherwise it returns $-1$.
We assume both input variables are positive up to $1000$, $0 \le \texttt{x} \le 1000$ and $0 \le \texttt{y} \le 1000$.
Regarding such a function, it is interesting to study its behaviors around $0$, thus the output bucket are $\{ \texttt{result < 0} \}, \{ \texttt{result} = 0 \}$, and $\{ \texttt{result > 0} \}$.
With the above parameters, the analysis \outcomesname{} returns 1 for both input variables.
Such result is not too interesting, but by looking at the internal stages of the analysis we notice that perturbations on the value of the variable \texttt{x} may be able to produce from an output negative value to zero or a positive one (and viceversa).
While perturbations on the value of the variable \texttt{y} are only able to produce from zero to positive (and viceversa).

As a second experiments, we consider the buckets from -1 to 19, $\setdef{ \texttt{result} = n}{-1 \le n \le 19}$, and we notice that the analysis \outcomesname{} returns 1 for the input variable \texttt{x} and 19 for \texttt{y}, meaning that the variable \texttt{y} is able to affect far more output values than \texttt{x}. However, combing the results of the previous experiment, only the variable \texttt{x} is able to affect the negative output values.

\begin{lstlisting}[
  language=customPython,
  escapechar=\%,
  caption={Program Ex2.21 from software verification competition SV-Comp.},
  label={lst:termination-b}
]
def termination_b(x, y):
  while x > 0:
    x = x + y
    y = -y - 1
  result = x + y
  return result
\end{lstlisting}

From the same work, we also consider \refprog{termination-b} which returns the value of \texttt{y} whenever $\texttt{x} = 0$, otherwise it returns $-1$.
Unfortunately, the backward analysis does not capture a precise loop invariant, thus both the analyses \outcomesname{} and \rangename are inconclusive in such case.
The key takeaway is that our analysis is highly dependent on the precision of the underlying backward analysis.

As a conclusion, even though SV-Comp proposes challenging benchmarks for termination, reachability, and safety analyses, they are not amenable for information flow analysis.
Most of the time, their examples involve loops with complex invariant, but as input-output relations, the variables involved are just zeroed out after the loop.
Drawing examples from their dataset is less appealing to our work.

\subsection{Linear Loops}
\labsec{linear-loops}

The last use case,~\refprog{linear-expression} computes the linear expression $(5x + 2y)$ via repeated additions.
Note that the invariant of the loop is indeed non-linear ($\texttt{result} = \texttt{i} * \texttt{x}$ and $\texttt{result} = \texttt{result}' + \texttt{i} * \texttt{y}$ respectively for the first and second loop, where $\texttt{result}'$ is the value of \texttt{result} before entering the second loop), but the loop is executed a fixed number of times, thus the analysis is able to compute the exact output buckets through loop unrolling.

\begin{lstlisting}[
  language=customPython,
  escapechar=\%,
  caption={Program computing the linear expression $(5x + 2y)$ via repeated additions.},
  label={lst:linear-expression}
]
def linear_expression(x, y):
  result = 0
  i = 0
  while i < 5:
    result = result + x
    i += 1
  i = 0
  while i < 2:
    result = result + y
    i += 1
\end{lstlisting}

For the analysis the input bounds are $0 \le \texttt{x} \le 1000$ and $0 \le \texttt{y} \le 1000$, while the output buckets are $\setdef{n * 100 \le \texttt{result} < (n + 1) * 100}{n \le 70}$.
Both analyses, \outcomesname{} and \rangename, show that \texttt{x} has an impact $\frac{5}{2}$ times bigger than \texttt{y} on the output.
Thus, the impact quantity provides insight about the termination speed.
Indeed, the loop for \texttt{x} is executed 5 times, while the one for \texttt{y} only 2.


\subsection{Landing Risk System}
\labsec{landing-risk}




\begin{figure*}[t]
  \centering
\begin{tikzpicture}
  % Grid
  \draw[help lines, color=gray!30, dashed] (-0.1,-0.1) grid (9.9,3.9);
  % x-axis
  \draw[->,ultra thick] (0,0)--(10,0) node[rotate=90,below]{\x};
  % y-axis
  \draw[->,ultra thick] (0,0)--(0,4) node[above]{\y};
  % x-axis ticks
  \foreach \x in {-4,-3,-2,-1,0,1,2,3,4}
      \draw (\x+5,0.1) -- (\x+5,-0.1) node[below] {\x};
  % y-axis ticks
  \foreach \y in {1,2,3}
      \draw (0.1,\y) -- (-0.1,\y) node[left] {\y};
  % Polyhedra
  \fill[color=seabornGreen, opacity=0.5] (5,3) -- (7,1) -- (3,1) -- cycle;
  \draw[color=seabornGreen, ultra thick] (5,3) -- (7,1) -- (3,1) -- cycle;
  % Polyhedra
  \fill[color=seabornYellow, opacity=0.5] (4,3) -- (5,3) -- (3,1) -- (2,1) -- cycle;
  \draw[color=seabornYellow, ultra thick] (4,3) -- (5,3) -- (3,1) -- (2,1) -- cycle;
  % % Polyhedra
  \fill[color=seabornYellow, opacity=0.5] (5,3) -- (6,3) -- (8,1) -- (7,1) -- cycle;
  \draw[color=seabornYellow, ultra thick] (5,3) -- (6,3) -- (8,1) -- (7,1) -- cycle;
  % Polyhedra
  \fill[color=seabornOrange, opacity=0.5] (3,3) -- (4,3) -- (2,1) -- (1,1) -- cycle;
  \draw[color=seabornOrange, ultra thick] (3,3) -- (4,3) -- (2,1) -- (1,1) -- cycle;
  % % Polyhedra
  \fill[color=seabornOrange, opacity=0.5] (6,3) -- (7,3) -- (9,1) -- (8,1) -- cycle;
  \draw[color=seabornOrange, ultra thick] (6,3) -- (7,3) -- (9,1) -- (8,1) -- cycle;
  % Polyhedra
  \fill[color=seabornRed, opacity=0.5] (1,3) -- (3,3) -- (1,1) -- cycle;
  \draw[color=seabornRed, ultra thick] (1,3) -- (3,3) -- (1,1) -- cycle;
  % Polyhedra
  \fill[color=seabornRed, opacity=0.5] (7,3) -- (9,3) -- (9,1) -- cycle;
  \draw[color=seabornRed, ultra thick] (7,3) -- (9,3) -- (9,1) -- cycle;
  % Nodes
  \fill[color=seabornRed] (0+1,0+1) circle[radius=2pt];
  \node[above left] at (0+1,0+1) {$3$};
  \fill[color=seabornRed] (0+1,1+1) circle[radius=2pt];
  \node[above left] at (0+1,1+1) {$3$};
  \fill[color=seabornRed]    (0+1,2+1) circle[radius=2pt];
  \node[above left] at (0+1,2+1) {$3$};
  \fill[color=seabornOrange] (1+1,0+1) circle[radius=2pt];
  \node[above left] at (1+1,0+1) {$2$};
  \fill[color=seabornRed]    (1+1,1+1) circle[radius=2pt];
  \node[above left] at (1+1,1+1) {$3$};
  \fill[color=seabornRed] (1+1,2+1) circle[radius=2pt];
  \node[above left] at (1+1,2+1) {$3$};
  \fill[color=seabornYellow]    (2+1,0+1) circle[radius=2pt];
  \node[above left] at (2+1,0+1) {$1$};
  \fill[color=seabornOrange] (2+1,1+1) circle[radius=2pt];
  \node[above left] at (2+1,1+1) {$2$};
  \fill[color=seabornRed] (2+1,2+1) circle[radius=2pt];
  \node[above left] at (2+1,2+1) {$3$};
  \fill[color=seabornGreen] (3+1,0+1) circle[radius=2pt];
  \node[above left] at (3+1,0+1) {$0$};
  \fill[color=seabornYellow] (3+1,1+1) circle[radius=2pt];
  \node[above left] at (3+1,1+1) {$1$};
  \fill[color=seabornOrange]    (3+1,2+1) circle[radius=2pt];
  \node[above left] at (3+1,2+1) {$2$};
  \fill[color=seabornGreen] (4+1,0+1) circle[radius=2pt];
  \node[above left] at (4+1,0+1) {$0$};
  \fill[color=seabornGreen]    (4+1,1+1) circle[radius=2pt];
  \node[above left] at (4+1,1+1) {$0$};
  \fill[color=seabornYellow]   (4+1,2+1) circle[radius=2pt];
  \node[above left] at (4+1,2+1) {$1$};
  \fill[color=seabornGreen]    (5+1,0+1) circle[radius=2pt];
  \node[above right] at (5+1,0+1) {$0$};
  \fill[color=seabornYellow]   (5+1,1+1) circle[radius=2pt];
  \node[above right] at (5+1,1+1) {$1$};
  \fill[color=seabornOrange]   (5+1,2+1) circle[radius=2pt];
  \node[above right] at (5+1,2+1) {$2$};
  \fill[color=seabornYellow]   (6+1,0+1) circle[radius=2pt];
  \node[above right] at (6+1,0+1) {$1$};
  \fill[color=seabornOrange]   (6+1,1+1) circle[radius=2pt];
  \node[above right] at (6+1,1+1) {$2$};
  \fill[color=seabornRed]   (6+1,2+1) circle[radius=2pt];
  \node[above right] at (6+1,2+1) {$3$};
  \fill[color=seabornOrange]   (7+1,0+1) circle[radius=2pt];
  \node[above right] at (7+1,0+1) {$2$};
  \fill[color=seabornRed]   (7+1,1+1) circle[radius=2pt];
  \node[above right] at (7+1,1+1) {$3$};
  \fill[color=seabornRed]   (7+1,2+1) circle[radius=2pt];
  \node[above right] at (7+1,2+1) {$3$};
  \fill[color=seabornRed]   (8+1,0+1) circle[radius=2pt];
  \node[above right] at (8+1,0+1) {$3$};
  \fill[color=seabornRed]   (8+1,1+1) circle[radius=2pt];
  \node[above right] at (8+1,1+1) {$3$};
  \fill[color=seabornRed]   (8+1,2+1) circle[radius=2pt];
  \node[above right] at (8+1,2+1) {$3$};
\end{tikzpicture}
\caption{Input space composition with continuous input values.}
\labfig{extended}
\end{figure*}

\begin{figure*}[t]
  \centering
  \begin{subfigure}{\textwidth}
  \begin{subfigure}[b]{0.24\textwidth}
    \begin{tikzpicture}[scale=0.8]
      % Grid
      \foreach \y in {0.5, 1.5, 2.5} {
        \draw[help lines, color=gray!30, dashed] (0,\y) -- (2.9,\y);
      }
      \foreach \x in {0.5, 1, 1.5, 2, 2.5} {
        \draw[help lines, color=gray!30, dashed] (\x, 0) -- (\x, 2.9);
      }
      % x-axis
      \draw[->,ultra thick] (0,0)--(3,0);
      % \draw[->,ultra thick] (0,0)--(3,0) node[rotate=90,below]{\x};
      % % y-axis
      \draw[->,ultra thick] (0,0)--(0,3) node[above]{\y};
      % % x-axis ticks
      \draw (0.5,0.1) -- (0.5,-0.1) node[below] {$-4$};
      % \draw (1,0.1) -- (1,-0.1) node[below] {$-2$};
      \draw (1,0.1) -- (1,-0.1);
      \draw (1.5,0.1) -- (1.5,-0.1) node[below] {$0$};
      % \draw (2,0.1) -- (2,-0.1) node[below] {$2$};
      \draw (2,0.1) -- (2,-0.1);
      \draw (2.5,0.1) -- (2.5,-0.1) node[below] {$4$};
      % % y-axis ticks
      \foreach \y in {1,2,3}
          \draw (0.1,\y-0.5) -- (-0.1,\y-0.5) node[left] {\y};
      % % Polyhedra
      \fill[color=seabornRed, opacity=0.5] (0.5,0.5) -- (2.5,0.5) -- (2.5,2.5) -- (0.5,2.5) -- cycle;
      \draw[color=seabornRed, ultra thick] (0.5,0.5) -- (2.5,0.5) -- (2.5,2.5) -- (0.5,2.5) -- cycle;
    \end{tikzpicture}
    % \caption{$\{\texttt{risk} = 3\}$}
  \end{subfigure}
  \hfill
  \begin{subfigure}[b]{0.23\textwidth}
    \begin{tikzpicture}[scale=0.8]
      % Grid
      \foreach \y in {0.5, 1.5, 2.5} {
        \draw[help lines, color=gray!30, dashed] (0,\y) -- (2.9,\y);
      }
      \foreach \x in {0.5, 1, 1.5, 2, 2.5} {
        \draw[help lines, color=gray!30, dashed] (\x, 0) -- (\x, 2.9);
      }
      % x-axis
      \draw[->,ultra thick] (0,0)--(3,0);
      % \draw[->,ultra thick] (0,0)--(3,0) node[rotate=90,below]{\x};
      % % y-axis
      % \draw[->,ultra thick] (0,0)--(0,3) node[above]{\y};
      % % x-axis ticks
      \draw (0.5,0.1) -- (0.5,-0.1) node[below] {$-4$};
      % \draw (1,0.1) -- (1,-0.1) node[below] {$-2$};
      \draw (1,0.1) -- (1,-0.1);
      \draw (1.5,0.1) -- (1.5,-0.1) node[below] {$0$};
      % \draw (2,0.1) -- (2,-0.1) node[below] {$2$};
      \draw (2,0.1) -- (2,-0.1);
      \draw (2.5,0.1) -- (2.5,-0.1) node[below] {$4$};
      % % y-axis ticks
      % \foreach \y in {1,2,3}
      %     \draw (0.1,\y-0.5) -- (-0.1,\y-0.5) node[left] {\y};
      % % Polyhedra
      \fill[color=seabornOrange, opacity=0.5] (0.5,0.5) -- (2.5,0.5) -- (2.25,2.5) -- (0.75,2.5) -- cycle;
      \draw[color=seabornOrange, ultra thick] (0.5,0.5) -- (2.5,0.5);
      \draw[color=seabornOrange, ultra thick] (0.75,2.5) -- (2.25,2.5);
      \draw[color=seabornOrange, ultra thick, dotted] (2.5,0.5) -- (2.25,2.5);
      \draw[color=seabornOrange, ultra thick, dotted] (0.75,2.5) -- (0.5,0.5);
    \end{tikzpicture}
    % \caption{$\{\texttt{risk} = 2\}$}
  \end{subfigure}
  % \hfill
  \begin{subfigure}[b]{0.23\textwidth}
    \begin{tikzpicture}[scale=0.8]
      % Grid
      \foreach \y in {0.5, 1.5, 2.5} {
        \draw[help lines, color=gray!30, dashed] (0,\y) -- (2.9,\y);
      }
      \foreach \x in {0.5, 1, 1.5, 2, 2.5} {
        \draw[help lines, color=gray!30, dashed] (\x, 0) -- (\x, 2.9);
      }
      % x-axis
      \draw[->,ultra thick] (0,0)--(3,0);
      % \draw[->,ultra thick] (0,0)--(3,0) node[rotate=90,below]{\x};
      % % y-axis
      % \draw[->,ultra thick] (0,0)--(0,3) node[above]{\y};
      % % x-axis ticks
      \draw (0.5,0.1) -- (0.5,-0.1) node[below] {$-4$};
      % \draw (1,0.1) -- (1,-0.1) node[below] {$-2$};
      \draw (1,0.1) -- (1,-0.1);
      \draw (1.5,0.1) -- (1.5,-0.1) node[below] {$0$};
      % \draw (2,0.1) -- (2,-0.1) node[below] {$2$};
      \draw (2,0.1) -- (2,-0.1);
      \draw (2.5,0.1) -- (2.5,-0.1) node[below] {$4$};
      % % y-axis ticks
      % \foreach \y in {1,2,3}
      %     \draw (0.1,\y-0.5) -- (-0.1,\y-0.5) node[left] {\y};
      % % Polyhedra
      \fill[color=seabornYellow, opacity=0.5] (1,0.5) -- (2,0.5) -- (1.75,2.5) -- (1.25,2.5) -- cycle;
      \draw[color=seabornYellow, ultra thick] (1,0.5) -- (2,0.5);
      \draw[color=seabornYellow, ultra thick] (1.75,2.5) -- (1.25,2.5);
      \draw[color=seabornYellow, ultra thick, dotted] (2,0.5) -- (1.75,2.5);
      \draw[color=seabornYellow, ultra thick, dotted] (1.25,2.5) -- (1,0.5);
    \end{tikzpicture}
    % \caption{$\{\texttt{risk} = 1\}$}
  \end{subfigure}
  % \hfill
  \begin{subfigure}[b]{0.24\textwidth}
    \begin{tikzpicture}[scale=0.8]
      % Grid
      \foreach \y in {0.5, 1.5, 2.5} {
        \draw[help lines, color=gray!30, dashed] (0,\y) -- (2.9,\y);
      }
      \foreach \x in {0.5, 1, 1.5, 2, 2.5} {
        \draw[help lines, color=gray!30, dashed] (\x, 0) -- (\x, 2.9);
      }
      % x-axis
      % \draw[->,ultra thick] (0,0)--(3,0);
      \draw[->,ultra thick] (0,0)--(3,0) node[rotate=90,below]{\x};
      % % y-axis
      % \draw[->,ultra thick] (0,0)--(0,3) node[above]{\y};
      % % x-axis ticks
      \draw (0.5,0.1) -- (0.5,-0.1) node[below] {$-4$};
      % \draw (1,0.1) -- (1,-0.1) node[below] {$-2$};
      \draw (1,0.1) -- (1,-0.1);
      \draw (1.5,0.1) -- (1.5,-0.1) node[below] {$0$};
      % \draw (2,0.1) -- (2,-0.1) node[below] {$2$};
      \draw (2,0.1) -- (2,-0.1);
      \draw (2.5,0.1) -- (2.5,-0.1) node[below] {$4$};
      % % y-axis ticks
      % \foreach \y in {1,2,3}
      %     \draw (0.1,\y-0.5) -- (-0.1,\y-0.5) node[left] {\y};
      % % Polyhedra
      \fill[color=seabornGreen, opacity=0.5] (1.25,0.5) -- (1.75,0.5) -- (1.5,2.5) -- cycle;
      \draw[color=seabornGreen, ultra thick] (1.25,0.5) -- (1.75,0.5);
      \draw[color=seabornGreen, ultra thick, dotted] (1.75,0.5) -- (1.5,2.5);
      \draw[color=seabornGreen, ultra thick, dotted] (1.5,2.5) -- (1.25,0.5);
    \end{tikzpicture}
    % \caption{$\{\texttt{risk} = 0\}$}
  \end{subfigure}
% \caption{Analysis result using the polyhedra domain.}
\label{fig:analysis-extended}
\end{subfigure}
\begin{subfigure}{\textwidth}
  \begin{subfigure}[b]{0.24\textwidth}
    \begin{tikzpicture}[scale=0.8]
      % Grid
      \foreach \y in {0.5, 1.5, 2.5} {
        \draw[help lines, color=gray!30, dashed] (0,\y) -- (2.9,\y);
      }
      \foreach \x in {0.5, 1, 1.5, 2, 2.5} {
        \draw[help lines, color=gray!30, dashed] (\x, 0) -- (\x, 2.9);
      }
      % x-axis
      \draw[->,ultra thick] (0,0)--(3,0);
      % \draw[->,ultra thick] (0,0)--(3,0) node[rotate=90,below]{\x};
      % % y-axis
      \draw[->,ultra thick] (0,0)--(0,3) node[above]{\y};
      \draw[dashed] (1.5,0)--(1.5,3);
      % % x-axis ticks
      \draw (0.5,0.1) -- (0.5,-0.1) node[below] {$-4$};
      % \draw (1,0.1) -- (1,-0.1) node[below] {$-2$};
      \draw (1,0.1) -- (1,-0.1);
      \draw (1.5,0.1) -- (1.5,-0.1) node[below] {$0$};
      % \draw (2,0.1) -- (2,-0.1) node[below] {$2$};
      \draw (2,0.1) -- (2,-0.1);
      \draw (2.5,0.1) -- (2.5,-0.1) node[below] {$4$};
      % % y-axis ticks
      \foreach \y in {1,2,3}
          \draw (0.1,\y-0.5) -- (-0.1,\y-0.5) node[left] {\y};
      % % Polyhedra
      \fill[color=seabornRed, opacity=0.5] (0.5,0.5) -- (1,2.5) -- (0.5,2.5) -- cycle;
      \fill[color=seabornRed, opacity=0.5] (2.5,0.5) -- (2.5,2.5) -- (2,2.5) -- cycle;
      \draw[color=seabornRed, ultra thick] (0.5,0.5) -- (1,2.5) -- (0.5,2.5) -- cycle;
      \draw[color=seabornRed, ultra thick] (2.5,0.5) -- (2.5,2.5) -- (2,2.5) -- cycle;
    \end{tikzpicture}
    % \caption{$\{\texttt{risk} = 3\}$}
  \end{subfigure}
  \hfill
  \begin{subfigure}[b]{0.23\textwidth}
    \begin{tikzpicture}[scale=0.8]
      % Grid
      \foreach \y in {0.5, 1.5, 2.5} {
        \draw[help lines, color=gray!30, dashed] (0,\y) -- (2.9,\y);
      }
      \foreach \x in {0.5, 1, 1.5, 2, 2.5} {
        \draw[help lines, color=gray!30, dashed] (\x, 0) -- (\x, 2.9);
      }
      % x-axis
      \draw[->,ultra thick] (0,0)--(3,0);
      \draw[dashed] (1.5,0)--(1.5,3);
      % \draw[->,ultra thick] (0,0)--(3,0) node[rotate=90,below]{\x};
      % % y-axis
      % \draw[->,ultra thick] (0,0)--(0,3) node[above]{\y};
      % % x-axis ticks
      \draw (0.5,0.1) -- (0.5,-0.1) node[below] {$-4$};
      % \draw (1,0.1) -- (1,-0.1) node[below] {$-2$};
      \draw (1,0.1) -- (1,-0.1);
      \draw (1.5,0.1) -- (1.5,-0.1) node[below] {$0$};
      % \draw (2,0.1) -- (2,-0.1) node[below] {$2$};
      \draw (2,0.1) -- (2,-0.1);
      \draw (2.5,0.1) -- (2.5,-0.1) node[below] {$4$};
      % % y-axis ticks
      % \foreach \y in {1,2,3}
      %     \draw (0.1,\y-0.5) -- (-0.1,\y-0.5) node[left] {\y};
      % % Polyhedra
      \fill[color=seabornOrange, opacity=0.5] (0.5,0.5) -- (0.75,0.5) -- (1,2.5) -- (0.75,2.5) -- cycle;
      \fill[color=seabornOrange, opacity=0.5] (2.25,0.5) -- (2.5,0.5) -- (2.25,2.5) -- (2,2.5) -- cycle;
      \draw[color=seabornOrange, ultra thick] (0.5,0.5) -- (0.75,0.5);
      \draw[color=seabornOrange, ultra thick] (2.25,0.5) -- (2.5,0.5);
      \draw[color=seabornOrange, ultra thick] (0.75,2.5) -- (1,2.5);
      \draw[color=seabornOrange, ultra thick] (2,2.5) -- (2.25,2.5);
      \draw[color=seabornOrange, ultra thick, dotted] (2.5,0.5) -- (2.25,2.5);
      \draw[color=seabornOrange, ultra thick] (2.25,0.5) -- (2,2.5);
      \draw[color=seabornOrange, ultra thick, dotted] (0.75,2.5) -- (0.5,0.5);
      \draw[color=seabornOrange, ultra thick] (1,2.5) -- (0.75,0.5);
    \end{tikzpicture}
    % \caption{$\{\texttt{risk} = 2\}$}
  \end{subfigure}
  % \hfill
  \begin{subfigure}[b]{0.23\textwidth}
    \begin{tikzpicture}[scale=0.8]
      % Grid
      \foreach \y in {0.5, 1.5, 2.5} {
        \draw[help lines, color=gray!30, dashed] (0,\y) -- (2.9,\y);
      }
      \foreach \x in {0.5, 1, 1.5, 2, 2.5} {
        \draw[help lines, color=gray!30, dashed] (\x, 0) -- (\x, 2.9);
      }
      % x-axis
      \draw[->,ultra thick] (0,0)--(3,0);
      \draw[dashed] (1.5, 0)--(1.5, 3);
      % \draw[->,ultra thick] (0,0)--(3,0) node[rotate=90,below]{\x};
      % % y-axis
      % \draw[->,ultra thick] (0,0)--(0,3) node[above]{\y};
      % % x-axis ticks
      \draw (0.5,0.1) -- (0.5,-0.1) node[below] {$-4$};
      % \draw (1,0.1) -- (1,-0.1) node[below] {$-2$};
      \draw (1,0.1) -- (1,-0.1);
      \draw (1.5,0.1) -- (1.5,-0.1) node[below] {$0$};
      % \draw (2,0.1) -- (2,-0.1) node[below] {$2$};
      \draw (2,0.1) -- (2,-0.1);
      \draw (2.5,0.1) -- (2.5,-0.1) node[below] {$4$};
      % % y-axis ticks
      % \foreach \y in {1,2,3}
      %     \draw (0.1,\y-0.5) -- (-0.1,\y-0.5) node[left] {\y};
      % % Polyhedra
      \fill[color=seabornYellow, opacity=0.5] (1,0.5) -- (1.25,0.5) -- (1.5,2.5) -- (1.75,0.5) -- (2,0.5) -- (1.75,2.5) -- (1.25,2.5) -- cycle;
      \draw[color=seabornYellow, ultra thick] (1,0.5) -- (1.25,0.5);
      \draw[color=seabornYellow, ultra thick] (1.75,0.5) -- (2,0.5);
      \draw[color=seabornYellow, ultra thick] (1.75,2.5) -- (1.25,2.5);
      \draw[color=seabornYellow, ultra thick, dotted] (2,0.5) -- (1.75,2.5);
      \draw[color=seabornYellow, ultra thick] (1.75,0.5) -- (1.5,2.5);
      \draw[color=seabornYellow, ultra thick, dotted] (1.25,2.5) -- (1,0.5);
      \draw[color=seabornYellow, ultra thick] (1.5,2.5) -- (1.25,0.5);
    \end{tikzpicture}
    % \caption{$\{\texttt{risk} = 1\}$}
  \end{subfigure}
  % \hfill
  \begin{subfigure}[b]{0.24\textwidth}
    \begin{tikzpicture}[scale=0.8]
      % Grid
      \foreach \y in {0.5, 1.5, 2.5} {
        \draw[help lines, color=gray!30, dashed] (0,\y) -- (2.9,\y);
      }
      \foreach \x in {0.5, 1, 1.5, 2, 2.5} {
        \draw[help lines, color=gray!30, dashed] (\x, 0) -- (\x, 2.9);
      }
      % x-axis
      % \draw[->,ultra thick] (0,0)--(3,0);
      \draw[->,ultra thick] (0,0)--(3,0) node[rotate=90,below]{\x};
      \draw[dashed] (1.5, 0)--(1.5, 3);
      % % y-axis
      % \draw[->,ultra thick] (0,0)--(0,3) node[above]{\y};
      % % x-axis ticks
      \draw (0.5,0.1) -- (0.5,-0.1) node[below] {$-4$};
      % \draw (1,0.1) -- (1,-0.1) node[below] {$-2$};
      \draw (1,0.1) -- (1,-0.1);
      \draw (1.5,0.1) -- (1.5,-0.1) node[below] {$0$};
      % \draw (2,0.1) -- (2,-0.1) node[below] {$2$};
      \draw (2,0.1) -- (2,-0.1);
      \draw (2.5,0.1) -- (2.5,-0.1) node[below] {$4$};
      % % y-axis ticks
      % \foreach \y in {1,2,3}
      %     \draw (0.1,\y-0.5) -- (-0.1,\y-0.5) node[left] {\y};
      % % Polyhedra
      \fill[color=seabornGreen, opacity=0.5] (1.25,0.5) -- (1.75,0.5) -- (1.5,2.5) -- cycle;
      \draw[color=seabornGreen, ultra thick] (1.25,0.5) -- (1.75,0.5);
      \draw[color=seabornGreen, ultra thick, dotted] (1.75,0.5) -- (1.5,2.5);
      \draw[color=seabornGreen, ultra thick, dotted] (1.5,2.5) -- (1.25,0.5);
      \draw[color=seabornGreen, ultra thick, dotted] (1.5,2.5) -- (1.5,0.5);
    \end{tikzpicture}
    % \caption{$\{\texttt{risk} = 0\}$}
  \end{subfigure}
% \caption{Analysis result after splitting the input space into two subspaces around $\texttt{angle}=0$.}
\end{subfigure}
\caption{Above, result of the analysis with convex polyhedra. Below, result after splitting the input space into two subspaces around $\texttt{angle}=0$.}
%\vspace{%-15pt}
\labfig{analysis}
\end{figure*}


We apply our quantitative analysis to~\refprog{landing-alarm-system} for the landing alarm system extended with the  continuous input space for the aircraft angle of approach, where $(-4 \le \texttt{angle} \le 4) \land (1 \le \texttt{speed} \le 3)$, see \reffig{extended}.
In this instance, the precision of the abstraction drastically drops as convex abstract domains are not able to capture the symmetric features of the input space around 0.
Indeed, the analysis result, first row of~\reffig{analysis}, is unable to reveal any difference in the input usage of input variables as all the abstract preconditions result of the backward analysis intersect together.
As a consequence, \outcomesname{} and \rangename{} are unable to provide any meaningful information, first row of \reftab{landing-risk}.

\begin{table}
  \caption{Quantitative input usage for~\refprog{landing-alarm-system}.}
  \labtab{landing-risk}
  \centering
  \begin{tabular}{cc|cc|cc}
    \multicolumn{2}{c|}{\multirow{2}{*}{~\textbf{Input Bounds}}} & \multicolumn{2}{c|}{\outcomesname} & \multicolumn{2}{c}{\rangename} \\ \cline{3-6}
    & & \texttt{angle} & \hspace{-5.5pt}\texttt{speed} & \texttt{angle} & \hspace{-5.5pt}\texttt{speed} \\ \hline\hline
    % $\texttt{angle} = -1 \lor \texttt{angle} = 4$ & \multirow{4}{*}{$1 \le \texttt{speed} \le 3$} & &
    % 1  &  2  & 3  & 2  \\ \cline{1-1} \cline{4-7}
    $-4 \le \texttt{angle} \le 4$ & \multirow{3}{*}{$~~\land 1 \le \texttt{speed} \le 3$} &
    3  &  3  & 3  & 3  \\ \cline{1-1} \cline{3-6}
    $-4 \le \texttt{angle} \le 0$ & &
    3  &  2  & 3  & 2  \\ \cline{1-1} \cline{3-6}
    $0 \le \texttt{angle} \le 4$ & &
    3 &  2 & 3 & 2 \\
  \end{tabular}
\end{table}

A possible approach to overcome the non-convexity of the input space is to split the input space into two subspaces (as a bounded set of disjunctive polyhedra), $-4 \le \texttt{angle} \le 0$ and $0 \le \texttt{angle} \le 4$, second and third row of \reftab{landing-risk}.
In the first subset $-4 \le \texttt{angle} \le 0$, we are able to perfectly captures the input regions that lead to each output bucket with our abstract analysis, second row of~\reffig{analysis}.
Therefore, we are able to recover the information that the input configurations from the bucket $\{\texttt{risk} =3\}$ do not intersect with the ones from the bucket $\{\texttt{risk} = 0\}$ after projecting away the axis \texttt{speed}.
As the end, our analysis notices that variations in the value of the input \texttt{angle} results in three possible output values, while variations in the \texttt{speed} input lead to two.
Similarly, regarding the range of values, variations in the \texttt{angle} input cover the entire spectrum of output values, whereas to the \texttt{speed} input only span a range of 2 since it exists no input value such that modifications in the \texttt{speed} value could obtain a range of output values bigger than 2.
The same reasoning applies to the other subspace with $0 \le \texttt{angle} \le 4$.



\section{The \texorpdfstring{$\defbound$}{k}-Bounded Impact Property}
\labsec{k-bounded-impact-property}

Quantitative input data usage should be understood as quantifying how much the input data influences the program outcomes.
To this end, we define the \emph{$\defbound$-bounded impact property}
$\genbounded$ as the set of program semantics such that the input variables
$\definputvariables\in\setof\inputvariables$ have an impact on the program outcome bounded by
$\defbound\in\valuesposplus$.
\marginnote{
  The set $\setof\inputvariables$ contains the input variables of a program.
}
\marginnote{
The set $\valuesposplus$ contains non-negative numerical values as well as with $+\infty$.
}
Depending on the comparison operator $\comparison\in\{\le,\ge\}$, the impact quantity is bounded by $\defbound$ either from above or below.
We employ the \emph{impact quantifier}:
\[\impactwrapper\in\tracetype\to\valuesposplus\]
to quantify the impact of the input variables $\definputvariables$ on the outcome of program computations.

\marginnote[*2]{
  As a comparison, here is the definition of the qualitative counterpart $\unused$:
  \begin{named}[\nrefdef{unused}]%
    \begin{eqnarray*}
  \unused\DefeQ
  \setdef{\tracesemantics}{\unusediowrapper(\tracesemantics)}
\end{eqnarray*}
%
  \end{named}%
}
\begin{definition}[$\defbound$-Bounded Impact Property]
  \labdef{bounded}
  Let $\definputvariables\in\setof\inputvariables$ be the set of input variables of interest, $\impactwrapper\in\tracetype\to\valuesposplus$ an impact quantifier, and $\defbound\in\valuesposplus$ the threshold.
  The \emph{$\defbound$-bounded impact property} $\genbounded$ is defined as follows:
  \begin{align*}
    \GENBOUNDED \DefeQ \setdef
    {\tracesemanticsnoparam \in \tracetype}
    {\impactwrapper(\tracesemanticsnoparam) \comparison \defbound}
  \end{align*}
  where the comparison operator $\comparison$ is either $\le$ or $\ge$.
\end{definition}


We allow the measured impact to be either less than or equal to ($\le$) or greater than or equal to ($\ge$) the threshold $\defbound$.
With such a definition, we allow abstractions of program semantics to return an always greater or, respectively, an always smaller impact than the actual one and thus provide a sound over-approximation of the impact property.

\begin{remark}
  A program $\defprogram$ satisfies the $\defbound$-bounded impact property $\genbounded$ if and only if the collecting semantics of $\defprogram$ implies $\genbounded$, formally:
  \begin{align*}
    \defprogram \satisfies \GENBOUNDED \IfF \collectingsemantics \subseteq \GENBOUNDED
  \end{align*}
\end{remark}

\begin{marginfigure}
  \caption{Illustration of the $\defbound$-bounded impact property $\bounded$ for different values of $\defbound$.}
  \labfig{le-bounded}
  \centering
  \resizebox{\textwidth}{!}{
    \begin{tikzpicture}
      % Parameters for the first ellipse
      \def\centerxone{0}
      \def\centeryone{0}
      \def\xradiusone{0.7}
      \def\yradiusone{1}
      \def\angleone{70}

      % Parameters for the second ellipse
      \def\centerxtwo{0.15}
      \def\centerytwo{0.25}
      \def\xradiustwo{1.25}
      \def\yradiustwo{1.5}
      \def\angletwo{68}

      % Parameters for the third ellipse
      \def\centerxthree{0.3}
      \def\centerythree{0.5}
      \def\xradiusthree{2}
      \def\yradiusthree{2}
      \def\anglethree{60}

      % Parameters for the fourth ellipse
      \def\centerxfour{0.4}
      \def\centeryfour{0.75}
      \def\xradiusfour{2.5}
      \def\yradiusfour{2.6}
      \def\anglefour{45}

      % Parameters for the fifth ellipse
      \def\centerxfive{0.5}
      \def\centeryfive{1}
      \def\xradiusfive{3}
      \def\yradiusfive{3.25}
      \def\anglefive{40}

      % Draw the first ellipse
      \draw (\centerxone,\centeryone) ellipse (\xradiusone cm and \yradiusone cm);
      \node[fill=white] at ({\centerxone + \xradiusone*cos(\angleone)}, {\centeryone + \yradiusone*sin(\angleone)}) {$\resize{\mathscr{B}}{\impactwrapper}[{\scaleto{\le}{5pt}0}]$};

      % Draw the second ellipse
      \draw (\centerxtwo,\centerytwo) ellipse (\xradiustwo cm and \yradiustwo cm);
      \node[fill=white] at ({\centerxtwo + \xradiustwo*cos(\angletwo)}, {\centerytwo + \yradiustwo*sin(\angletwo)}) {$\resize{\mathscr{B}}{\impactwrapper}[{\scaleto{\le}{5pt}1}]$};

      % Draw the third ellipse
      \draw (\centerxthree,\centerythree) ellipse (\xradiusthree cm and \yradiusthree cm);
      \node[fill=white] at ({\centerxthree + \xradiusthree*cos(\anglethree)}, {\centerythree + \yradiusthree*sin(\anglethree)}) {$\resize{\mathscr{B}}{\impactwrapper}[{\scaleto{\le}{5pt}2}]$};

      % Draw the fourth ellipse
      \draw (\centerxfour,\centeryfour) ellipse (\xradiusfour cm and \yradiusfour cm);
      \node[fill=white] at ({\centerxfour + \xradiusfour*cos(\anglefour)}, {\centeryfour + \yradiusfour*sin(\anglefour)}) {$\resize{\mathscr{B}}{\impactwrapper}[{\scaleto{\le}{5pt}3}]$};

      % Draw the fifth ellipse
      \node[fill=white] at ({\centerxfive + \xradiusfive*cos(\anglefive)}, {\centeryfive + \yradiusfive*sin(\anglefive)}) {$\dots$};

  \end{tikzpicture}
  }
  \end{marginfigure}

  \begin{marginfigure}
    \caption{Illustration of the $\defbound$-bounded impact property $\revbounded$ for different values of $\defbound$.}
    \labfig{ge-bounded}
    \centering
    \resizebox{\textwidth}{!}{
      \begin{tikzpicture}
        % Parameters for the first ellipse
        \def\centerxone{0}
        \def\centeryone{0}
        \def\xradiusone{0.5}
        \def\yradiusone{0.5}
        \def\angleone{110}

        % Parameters for the second ellipse
        \def\centerxtwo{0.15}
        \def\centerytwo{0.25}
        \def\xradiustwo{1.25}
        \def\yradiustwo{1.25}
        \def\angletwo{112}

        % Parameters for the third ellipse
        \def\centerxthree{0.3}
        \def\centerythree{0.5}
        \def\xradiusthree{2}
        \def\yradiusthree{1.75}
        \def\anglethree{120}

        % Parameters for the fourth ellipse
        \def\centerxfour{0.4}
        \def\centeryfour{0.75}
        \def\xradiusfour{2.5}
        \def\yradiusfour{2.6}
        \def\anglefour{135}


        % Draw the first ellipse
        \node[fill=white] at ({\centerxone + \xradiusone*cos(\angleone)}, {\centeryone + \yradiusone*sin(\angleone)}) {$\dots$};

        % Draw the second ellipse
        \draw (\centerxtwo,\centerytwo) ellipse (\xradiustwo cm and \yradiustwo cm);
        \node[fill=white] at ({\centerxtwo + \xradiustwo*cos(\angletwo)}, {\centerytwo + \yradiustwo*sin(\angletwo)}) {$\resize{\mathscr{B}}{\impactwrapper}[{\scaleto{\ge}{5pt}2}]$};

        % Draw the third ellipse
        \draw (\centerxthree,\centerythree) ellipse (\xradiusthree cm and \yradiusthree cm);
        \node[fill=white] at ({\centerxthree + \xradiusthree*cos(\anglethree)}, {\centerythree + \yradiusthree*sin(\anglethree)}) {$\resize{\mathscr{B}}{\impactwrapper}[{\scaleto{\ge}{5pt}1}]$};

        % Draw the fourth ellipse
        \draw (\centerxfour,\centeryfour) ellipse (\xradiusfour cm and \yradiusfour cm);
        \node[fill=white] at ({\centerxfour + \xradiusfour*cos(\anglefour)}, {\centeryfour + \yradiusfour*sin(\anglefour)}) {$\resize{\mathscr{B}}{\impactwrapper}[{\scaleto{\ge}{5pt}0}]$};


    \end{tikzpicture}
    }
    \end{marginfigure}

As a consequence of the $\defbound$-bounded impact property, \cf{} \refdef{bounded}, we notice that by enlarging the threshold $\defbound$, the set of program semantics that satisfy the property also increases.

\begin{lemma}[Monotonicity of the $\defbound$-Bounded Impact Property]
  \lablemma{bounded-monotonic}
  For all $\defbound, \defbound'\in\valuesposplus$ such that $\defbound \comparison \defbound'$, it holds that:
  \begin{align*}
    \GENBOUNDED \subseteq \mathscr{B}_{\impactwrapper}^{{\scaleto{\comparison}{5pt}\defbound'}}
  \end{align*}
\end{lemma}
\begin{proof}
  The proof is based on the fact that, for all $\defbound, \defbound'\in\valuesposplus$ such that $\defbound \comparison \defbound'$, it holds that $\impactwrapper(\defsetoftraces) \comparison \defbound \implies \impactwrapper(\defsetoftraces) \comparison \defbound'$ for any set of traces $\defsetoftraces\in\tracetype$.
  Consequently, we have that $\genbounded \subseteq \resize{\mathscr{B}}{\impactwrapper}[{\scaleto{\comparison}{5pt}\defbound'}]$.
\end{proof}





Note that, in both cases whether $\comparison$ is $\le$ or $\ge$, the $\defbound$-bounded impact property $\genbounded$ is always subset of $\resize{\mathscr{B}}{\impactwrapper}[{\scaleto{\comparison}{5pt}\defbound'}]$.
\reffig{le-bounded} compares the property $\bounded$ for different values of $\defbound$, the smallest being $\defbound = 0$ containing the program semantics whose input variables have no impact.
By increasing $\defbound$, the set of program semantics that satisfy the property also increases.
Regarding the other operator $\ge$, \reffig{ge-bounded} shows that $\resize{\mathscr{B}}{\impactwrapper}[{\scaleto{\ge}{5pt}0}]$ is the property containing all the program semantics, \ie, $\resize{\mathscr{B}}{\impactwrapper}[{\scaleto{\ge}{5pt}0}] = \tracetype$. By decreasing $\defbound$, the set of program semantics that satisfy the property also increases.
Therefore, the $\defbound$-bounded impact property $\bounded$ can be used to verify that the input variables $\definputvariables$ have an impact of \emph{at least} $\defbound$ on the program outcome. Whereas, the property $\revbounded$ can be used to verify that the impact is of \emph{at most} $\defbound$.


Next, we instantiate $\impactwrappername$ with the impact quantifiers of \outcomesname{}, \rangename{}, and \qusedname{}. Each of these quantifiers differently characterizes the impact of the input variables: \outcomesname{} counts the number of different outcomes, \rangename{} measures the range of outcome values, and \qusedname{} counts the number of input values that are \emph{not} used to produce the outcomes.
To this end, these quantifiers employ the concept of output observer, \cf{} \refdef*{output-observer}.
% We expect the output observer $\outputobs$ (\refdef*{output-observer}[*-2]) to return a quantifiable element, \ie{}, a value in $\valuesinf$.
% An output observer $\outputobs$ that returns a quantifiable element is called an \emph{output observer}.

% \begin{definition}[Output Descriptor]
%   \labdef{output-descriptor}
%   An \textup{output observer} is an output observer $\outputobs\in\stateandbottom\to\valuesinf$ that returns a quantifiable element.
% \end{definition}

% The above output observer $\outputobs$ is generic enough to cover plenty of use cases and still abstract enough to maintain the generality of the output abstraction introduced in the previous chapter.
% We leverage this output observer to provide the end user of the framework the flexibility to choose the meaning of program outputs, without establishing it beforehand.

\begin{example}\labexample{output-observer-landing}
  Consider the \refprog*{landing-alarm-system} for a landing alarm system with program states $\state=\setdef{\langle a, b, c, d \rangle}{a\in\{-4,1\}\land b\in\{1,2,3\}\land c\in\N \land d\in\{0,1,2,3\}}$, where $a$ is the value of $\text{angle}$, $b$ of $\texttt{speed}$, $c$ of $\texttt{landing\_coeff}$, and $d$ of $\texttt{risk}$.
  Here, we abuse the notation and write states in $\state$ as tuples of variable values, without considering program locations.
  The output observer is instantiated with
  \[
  \outputobs(\defstate) \DefeQ \begin{cases}
    \langle \top, \top, \top, d \rangle & \text{if } \defstate = \langle a, b, c, d \rangle \\
    \langle \top, \top, \top, \top \rangle & \text{otherwise}
  \end{cases}
  \]
  As a result, the output observer $\outputobs$ focuses on the value of the variable $\texttt{risk}$ when the last state of the trace is $\langle a, b, c, d \rangle$. Otherwise, in case of an infinite-length trace, the output observer maps to the state with all variables at $\top$.
  In other words, the output observer abstracts away the value of all variables but $\texttt{risk}$, the output variable.
  Note that, even though \refprog{landing-alarm-system} always terminates, we need to take into account that the input parameter of the output observer $\outputobs$ is either the final state or $\statebottom$.
  \reffig{output-observer-landing} depicts the traces generated by \refprog{landing-alarm-system} where the output values are abstracted by $\outputobs$.
\end{example}
\begin{marginfigure}[*-16]
  \centering
  \begin{tikzpicture}
    \node (linit) at (0,3) {$\locinit$};
    \node (lend) at (2,3) {$\loc{10}$};
    \draw[->] (linit) -- (lend);

    % Define the positions of the nodes
    \node (-41) at (0,2.5) {${\tuple{-4}{1}}$};
    \node (-42) at (0,2) {${\tuple{-4}{2}}$};
    \node (-43) at (0,1.5) {${\tuple{-4}{3}}$};
    \node (11) at (0,1) {${\tuple{1}{1}}$};
    \node (12) at (0,0.5) {${\tuple{1}{2}}$};
    \node (13) at (0,0) {${\tuple{1}{3}}$};

    \node (3) at (2,2) {${\langle 3 \rangle}$};
    \node (2) at (2,1.5) {${\langle 2 \rangle}$};
    \node (1) at (2,1) {${\langle 1 \rangle}$};
    \node (0) at (2,0.5) {${\langle 0 \rangle}$};

    % Draw the edges
    \draw[->] (-41) -- (3);
    \draw[->] (-42) -- (3);
    \draw[->] (-43) -- (3);

    \draw[->] (11) -- (0);
    \draw[->] (12) -- (1);
    \draw[->] (13) -- (2);

    \draw[decorate,decoration={brace,amplitude=8pt,mirror,raise=4pt},yshift=0pt,xshift=-0.1pt] (-0.5,2.7) -- (-0.5,-0.2) node [black,midway,xshift=-0.7cm,rotate=90] {$\outputsemanticsnoparam\semanticsof{\texttt{landing\_alarm\_system}}$};
\end{tikzpicture}
\caption{Graphical representation of the trace semantics of the \refprog{landing-alarm-system}.}
\labfig{output-observer-landing}
\end{marginfigure}

\begin{example}
  Originally introduced by \sidetextcite[*-4.5]{Giacobazzi2018}, the output observer $\outputobs$ should deal with any possible abstraction an end-user of the framework might want to use.
  For example, one could be interested in the parity of the output value.
  In this case, the output observer $\outputobs$ could be defined as:
  \[
  \outputobs(\defstate) \DefeQ \begin{cases}
    \langle \top, \top, \top, d \mod 2 \rangle & \text{if } \defstate = \langle a, b, c, d \rangle \\
    \langle \top, \top, \top, \top \rangle & \text{otherwise}
  \end{cases}
  \]
  \begin{marginfigure}[*-11]
    \centering
    \begin{tikzpicture}
      \node (linit) at (0,3) {$\locinit$};
      \node (lend) at (2,3) {$\loc{10}$};
      \draw[->] (linit) -- (lend);

      % Define the positions of the nodes
      \node (-41) at (0,2.5) {${\tuple{-4}{1}}$};
      \node (-42) at (0,2) {${\tuple{-4}{2}}$};
      \node (-43) at (0,1.5) {${\tuple{-4}{3}}$};
      \node (11) at (0,1) {${\tuple{1}{1}}$};
      \node (12) at (0,0.5) {${\tuple{1}{2}}$};
      \node (13) at (0,0) {${\tuple{1}{3}}$};

      \node (odd) at (2,1.5) {${\langle 1 \rangle}$};
      \node (even) at (2,1) {${\langle 0 \rangle}$};

      % Draw the edges
      \draw[->] (-41) -- (odd);
      \draw[->] (-42) -- (odd);
      \draw[->] (-43) -- (odd);

      \draw[->] (11) -- (even);
      \draw[->] (12) -- (odd);
      \draw[->] (13) -- (even);

      \draw[decorate,decoration={brace,amplitude=8pt,mirror,raise=4pt},yshift=0pt,xshift=-0.1pt] (-0.5,2.7) -- (-0.5,-0.2) node [black,midway,xshift=-0.7cm,rotate=90] {$\outputsemanticsnoparam\semanticsof{\texttt{landing\_alarm\_system}}$};
  \end{tikzpicture}
  \caption{Graphical representation of the trace semantics of the \refprog{landing-alarm-system} with the parity abstraction.}
  \labfig{parity-landing}
  \end{marginfigure}
  The parity of the value of $\texttt{risk}$ is mapped to the value $0$ if even and $1$ if odd.
  As a result, in our framework we would quantify the impact of the input variables on the parity of the output values.
  \reffig{parity-landing} shows the traces generated by \refprog{landing-alarm-system} where the output values are abstracted by their parity.
\end{example}



Next, we define the impact quantifiers of \outcomesname{}, \rangename{}, and \qusedname{}.

\subsection{The \outcomesname{} Impact Quantifier}[\outcomesname]
\labsec{outcomes}

Formally $\outcomes\in\tracetype\to\Nplus$ counts the number of different output values reachable by varying the input variables $\definputvariables\in\setof\inputvariables$.
% First, we define step-by-step the quantity $\outcomes$,
% followed by the formal definition.
% We instantiate of this quantity in the context of \refprog*{landing-alarm-system}[*-5] at the end, alongside with some properties of $\outcomesname$.


Intuitively, $\outcomes$ collects,
for any possible input configuration $\definput\in\reducedstate$, all the traces that are starting from an input configuration that is a variation of $\definput$ on the input variable $\definputvariables$, \ie, $\setdef{
  \defseq\in \defsetoftraces
}{
  \retrieveinput{\defseq} \stateeq{\inputvariableswithoutw} \definput
}$, where $\defsetoftraces\in\setof\finiteinfinitesequences$.
Then, it gathers the output values of this set of traces by means of the output observer $\outputobs$, \ie, $\setdef{
  \outputobs(\retrieveoutput\defseq)
}{
  \defseq \in \defsetoftraces \land
    \retrieveinput{\defseq} \stateeq{\inputvariableswithoutw} \definput
}$. Specifically, this set contains all the output abstractions performed by $\outputobs$.
%
Afterwards, it yields the number of these output values via the cardinality operator $\cardinality{\cdot}$.
Finally, through the supremum operator: $\sup$, it returns the maximum value to ensure that the greatest impact is preserved.
\marginnote[*-13]{
  The supremum operator $\sup$ yields the least upper bound of the given set of elements $X$:
  %  that is, the smallest $q$ such that $q$ is greater than or equal to all elements of the set.
  \begin{align*}
    \sup X \DefeQ x \text{ such that } \foralldef{y\in X}{x \ge y}
  \end{align*}
}

Formally, the $\outcomesname$ impact quantifier is defined as follows:

\begin{definition}[\outcomesname]\labdef{outcomes}
  Given an input variable $\definputvariables\in\setof\inputvariables$ and output observer $\outputobs\in\stateandbottom\to\stateandbottom$,
  the impact quantifier $\outcomes\in\tracetype\to\Nplus$ is defined as:
  \begin{align*}
    \outcomes(\defsetoftraces) &\DefeQ \sup_{\definput\in\state}
      \cardinality{\setdef{
        \outputobs(\retrieveoutput{\defseq})
      }{
        \defseq \in \defsetoftraces \land \retrieveinput{\defseq} \stateeq{\inputvariableswithoutw} \definput
      }}
  \end{align*}
\end{definition}


\begin{margintable}[*-9]
  \caption{Overview of the $\outcomesname$ impact quantifier for the variable $\texttt{angle}$ of \nrefprog{landing-alarm-system}.}
  \labtab{overview-outcomes-angle}
  \resizebox{\textwidth}{!}{
  \begin{tabular}{c|c|c}
    \textsc{Input} & \textsc{Traces} & \outcomesname \\
    \hline
    \hline
    $\tuple{\blue{-4}}{1}$ & \makecell{$\tuple{\blue{-4}}{1}\rightarrow\langle{3}\rangle$, \\ $\tuple{\blue{1}}{1}\rightarrow\langle{0}\rangle$} & 2 \\
    \hline
    $\tuple{\blue{-4}}{2}$ & \makecell{$\tuple{\blue{-4}}{2}\rightarrow\langle{3}\rangle$, \\ $\tuple{\blue{1}}{2}\rightarrow\langle{1}\rangle$} & 2 \\
    \hline
    $\tuple{\blue{-4}}{3}$ & \makecell{$\tuple{\blue{-4}}{3}\rightarrow\langle{3}\rangle$, \\ $\tuple{\blue{1}}{3}\rightarrow\langle{2}\rangle$} & 2 \\
    \hline
    $\tuple{\blue{1}}{1}$ & \makecell{$\tuple{\blue{1}}{1}\rightarrow\langle{0}\rangle$, \\ $\tuple{\blue{-4}}{1}\rightarrow\langle{3}\rangle$} & 2 \\
    \hline
    $\tuple{\blue{1}}{2}$ & \makecell{$\tuple{\blue{1}}{2}\rightarrow\langle{1}\rangle$, \\ $\tuple{\blue{-4}}{2}\rightarrow\langle{3}\rangle$} & 2 \\
    \hline
    $\tuple{\blue{1}}{3}$ & \makecell{$\tuple{\blue{1}}{3}\rightarrow\langle{2}\rangle$, \\ $\tuple{\blue{-4}}{3}\rightarrow\langle{3}\rangle$} & 2 \\
    \hline
    \hline
  \end{tabular}}
\end{margintable}
\begin{margintable}[*5]
  \caption{Overview of the $\outcomesname$ impact quantifier for the variable $\texttt{speed}$ of \nrefprog{landing-alarm-system}.}
  \labtab{overview-outcomes-speed}
  \resizebox{\textwidth}{!}{
  \begin{tabular}{c|c|c}
    \textsc{Input} & \textsc{Traces} & \outcomesname \\
    \hline
    \hline
    $\tuple{-4}{\blue{1}}$ & \makecell{$\tuple{-4}{\blue{1}}\rightarrow\langle{3}\rangle$, \\ $\tuple{-4}{\blue{2}}\rightarrow\langle{3}\rangle$ \\ $\tuple{-4}{\blue{3}}\rightarrow\langle{3}\rangle$} & 1 \\
    \hline
    $\tuple{-4}{\blue{2}}$ & \makecell{$\tuple{-4}{\blue{1}}\rightarrow\langle{3}\rangle$, \\ $\tuple{-4}{\blue{2}}\rightarrow\langle{3}\rangle$ \\ $\tuple{-4}{\blue{3}}\rightarrow\langle{3}\rangle$} & 1 \\
    \hline
    $\tuple{-4}{\blue{3}}$ & \makecell{$\tuple{-4}{\blue{1}}\rightarrow\langle{3}\rangle$, \\ $\tuple{-4}{\blue{2}}\rightarrow\langle{3}\rangle$ \\ $\tuple{-4}{\blue{3}}\rightarrow\langle{3}\rangle$} & 1 \\
    \hline
    $\tuple{1}{\blue{1}}$ & \makecell{$\tuple{1}{\blue{1}}\rightarrow\langle{0}\rangle$, \\ $\tuple{1}{\blue{2}}\rightarrow\langle{1}\rangle$ \\ $\tuple{1}{\blue{3}}\rightarrow\langle{2}\rangle$} & 3 \\
    \hline
    $\tuple{1}{\blue{1}}$ & \makecell{$\tuple{1}{\blue{1}}\rightarrow\langle{0}\rangle$, \\ $\tuple{1}{\blue{2}}\rightarrow\langle{1}\rangle$ \\ $\tuple{1}{\blue{3}}\rightarrow\langle{2}\rangle$} & 3 \\
    \hline
    $\tuple{1}{\blue{1}}$ & \makecell{$\tuple{1}{\blue{1}}\rightarrow\langle{0}\rangle$, \\ $\tuple{1}{\blue{2}}\rightarrow\langle{1}\rangle$ \\ $\tuple{1}{\blue{3}}\rightarrow\langle{2}\rangle$} & 3 \\
    \hline
    \hline
  \end{tabular}}
\end{margintable}
\begin{example}
  \labexample{outcomes}
  \newcommand*{\inputa}{\tuple{-4}{1}} \newcommand*{\outputa}{\langle \outputvaluea\rangle} \newcommand*{\outputvaluea}{3}
  \newcommand*{\inputb}{\tuple{-4}{2}} \newcommand*{\outputb}{\langle \outputvalueb\rangle} \newcommand*{\outputvalueb}{3}
  \newcommand*{\inputc}{\tuple{-4}{3}} \newcommand*{\outputc}{\langle \outputvaluec\rangle} \newcommand*{\outputvaluec}{3}
  \newcommand*{\inputd}{\tuple{ 1}{1}} \newcommand*{\outputd}{\langle \outputvalued\rangle} \newcommand*{\outputvalued}{0}
  \newcommand*{\inpute}{\tuple{ 1}{2}} \newcommand*{\outpute}{\langle \outputvaluee\rangle} \newcommand*{\outputvaluee}{1}
  \newcommand*{\inputf}{\tuple{ 1}{3}} \newcommand*{\outputf}{\langle \outputvaluef\rangle} \newcommand*{\outputvaluef}{2}
  \newcommand*{\tracea}{\inputa\to\outputa}
  \newcommand*{\traceb}{\inputb\to\outputb}
  \newcommand*{\tracec}{\inputc\to\outputc}
  \newcommand*{\traced}{\inputd\to\outputd}
  \newcommand*{\tracee}{\inpute\to\outpute}
  \newcommand*{\tracef}{\inputf\to\outputf}
  Let us revisit the example of the landing alarm system, \cf{} \refprog*{landing-alarm-system}[*-23], with program states $\state=\setdef{\langle a, b, c, d \rangle}{a\in\{-4,1\}\land b\in\{1,2,3\}\land c\in\N \land d\in\Nle{3}}$.
  For brevity, we refer to such program as program $\landingprogram$.
  The input variables are $\setof\inputvariables = \{\texttt{angle},\texttt{speed}\}$, consequently the input space is the set of program states reduced to the input variables, \ie{}
  $\reducedstate=\{\inputa, \inputb, \inputc, \inputd, \inpute, \inputf\}$.
%
  We begin by considering $\definputvariables=\texttt{angle}$ and $\inputa$ as the first input configuration to be explored.
  Hence, we collect all traces that are
  starting from an input configuration that is a variation of $\inputa$, \ie, $\setdef{
    \defseq \in \tracesemanticsnoparam\semanticsof{\landingprogram}
  }{
    \retrieveinput{\defseq} \stateeq{\inputvariables\setminus\{\texttt{angle}\}} \inputa
  }$, where $\inputvariables\setminus\{\texttt{angle}\} = \{\texttt{speed}\}$ and consequently $\retrieveinput{\defseq} \stateeq{\{\texttt{speed}\}} \inputa$ holds whenever the initial state of $\defseq$ has $\texttt{speed}=1$. A possible trace (with the intermediate states and the value of all the 4 variables) of this set is $\langle 1, 1, 0, 0\rangle \to \langle 1, 1, 2, 0\rangle\to\langle 1, 1, 2, 0\rangle$ where, at the beginning, we randomly assign $\texttt{landing\_risk}=0$ and $\texttt{risk}=0$, respectively the third and fourth component of the initial state.
%
  We collect the output values of this set of traces, $\setdef{
    \retrieveoutput\defseq(\texttt{risk})
  }{
    \defseq \in \tracesemanticsnoparam\semanticsof{\landingprogram} \land
      \retrieveinput{\defseq}(\texttt{speed}) = 1
  }$.
  As a result, we obtain the set of output values $\{0, 3\}$.
  For instance, the output value $0$ is the result of the trace we exhibited previously, where the last state is $\langle 1, 1, 2, 0\rangle$ and thus the $\texttt{risk}$ variable of this trace is the last component with value $0$.
%
  Finally, the cardinality operator returns the value $2$, $\cardinality{\{0, 3\}} = 2$.
  By doing so for all possible input configurations in $\reducedstate$, we obtain $\outcomesname_{\{\texttt{angle}\}}(\tracesemanticsnoparam\semanticsof{\landingprogram})=2$.
  \reftab{overview-outcomes-angle} illustrate the steps for $\outcomesname_{\{\texttt{angle}\}}(\tracesemanticsnoparam\semanticsof{\landingprogram})$ and \reftab{overview-outcomes-speed} for $\outcomesname_{\{\texttt{speed}\}}(\tracesemanticsnoparam\semanticsof{\landingprogram})$.
\end{example}

The quantifier presented above is monotone in the amount of traces in $\defsetoftraces$.

\begin{lemma}[\outcomesname{} is Monotonic]
  \lablemma{outcomes-monotonic}
For all set of traces $\defsetoftraces, \defsetoftraces'\in\tracetype$, it holds that:
  \begin{align*}
    \defsetoftraces \subseteq \defsetoftraces' \ImplieS \outcomes(\defsetoftraces) \le \outcomes(\defsetoftraces')
  \end{align*}
\end{lemma}
\begin{proof}
  The proof is based on the observation that, fixed an input state $\retrieveinput\defstate \in \state$, the set of output values $\setdef{
    \outputobs(\retrieveoutput{\defseq})
  }{
    \defseq \in \defsetoftraces \land \retrieveinput{\defseq} \stateeq{\inputvariableswithoutw} \definput
  }$ is a subset of $\setdef{
    \outputobs(\retrieveoutput{\defseq})
  }{
    \defseq \in \defsetoftraces' \land \retrieveinput{\defseq} \stateeq{\inputvariableswithoutw} \definput
  }$ since by hypothesis we have that $\defsetoftraces \subseteq \defsetoftraces'$.
  Hence, the cardinality of the first set is less than or equal to the cardinality of the second set.
  We conclude that $\outcomes(\defsetoftraces) \le \outcomes(\defsetoftraces')$.
\end{proof}

Let $\boundedoutcomes$ be the $\defbound$-bounded impact property when instantiated with the $\outcomesname$ impact quantifier.
The output semantics $\outputsemantics$ is sound and complete for validating $\boundedoutcomes$.

\siderefbox{def}{output-abstraction-semantics}
\siderefbox{def}{abstraction-dependency-semantics}
\siderefbox{def}{abstraction-output-abstraction-semantics}
\begin{lemma}[$\boundedoutcomes$ Validation]\lablemma{outcomes-validation}
  \begin{align*}
    \collectingsemantics \subseteq \BOUNDEDOUTCOMES \IfF \outputsemantics \subseteq \outputabstraction(\dependencyabstraction(\BOUNDEDOUTCOMES))
  \end{align*}
\end{lemma}
\begin{proof}[Proof (Sketch)]
  Let us consider the $\outputabstraction \circ \dependencyabstraction$ abstraction of the $\defbound$-bounded impact property $\boundedoutcomes$:
  \begin{eqnarray*}
    \lefteqn{\outputabstraction(\dependencyabstraction(\boundedoutcomes)) =} \\
    & \setdef{
      \defsetofdependencies \in \setof\pairofstates
    }{
      \sup_{\definput\in\state}
      \cardinality{\setdef{
        \defoutput
      }{
        \tuple{\retrieveinput{\defstate}}{\defoutput} \in \defsetofdependencies \land \retrieveinput{\defseq} \stateeq{\inputvariableswithoutw} \definput
      }} \comparison \defbound
    }
  \end{eqnarray*}
  Note that, all the intermediate states from the trace semantics are abstracted to input-output dependencies.
  The $\outcomesname$ impact quantifier does not consider the intermediate states, thus the abstraction to dependencies does not affect the validation of the property.
  Furthermore, even handling the output abstraction at the semantic level, by abstracting output states to abstract output states, does not affect the validation of the property as the $\outcomesname$ impact quantifier already abstracts the output values before counting them.
\end{proof}

For the \outcomesname{} quantifier, an impact quantity of zero implies that the input variables of interest have no impact on the program outcomes, \ie, the set of input variables is unused, provided that the given program semantics is deterministic.
On the other hand, non-deterministic statements in the program may mislead $\outcomesname$ to account for differences in the program output as a result of variations in the input variables.

\begin{example}\labexample{outcomes-nondeterminism}
  \marginnote{
    \begin{tikzpicture}[scale=0.8]
      \node (linit) at (-0.9,2) {$\locinit:$};
      \node (lend) at (-0.9,0) {$\loc{10}:$};
      % Define the positions of the nodes
      \node (TT) at (0,2) {${\tuple\true\true}$};
      \node (TF) at (1,2) {${\tuple\true\false}$};
      \node (FT) at (2,2) {${\tuple\false\true}$};
      \node (FF) at (3,2) {${\tuple\false\false}$};
      \node (outTT) at (0,0) {${\tuple\true\true}$};
      \node (outTF) at (1,0) {${\tuple\true\false}$};
      \node (outFT) at (2,0) {${\tuple\false\true}$};
      \node (outFF) at (3,0) {${\tuple\false\false}$};
      % Draw the edges
      \draw[->] (TT) -> (outTT);
      \draw[->] (TF) -> (outTT);
      \draw[->] (TT) -> (outTF);
      \draw[->] (TF) -> (outTF);
      \draw[->] (FT) -> (outFT);
      \draw[->] (FF) -> (outFT);
      \draw[->] (FT) -> (outFF);
      \draw[->] (FF) -> (outFF);
      \draw[decorate,decoration={brace,amplitude=8pt,mirror,raise=4pt},yshift=0pt,xshift=-0.1pt] (-1.1,2.2) -- (-1.1,-0.2) node [black,midway,xshift=-0.7cm,rotate=90] {$\tracesemanticsnoparam\semanticsof{\texttt{is\_passed}_\texttt{sci}}$};
  \end{tikzpicture}
  }
Consider the \refprog*{syntactic-vs-semantic-usage}[*-2] used in \refsec{input-data-usage} to illustrate the differences between syntactic and semantic input data usage. Notably: it contains a non-deterministic statement.
Assuming input values are $\N$, the $\outcomesname_{\texttt{x}}$ impact quantifier (for the variable $\texttt{x}$) is $+\infty$ as for any possible input value, the program can output any value in $\N$.
Due to the non-deterministic statement, the variable \texttt{x} has an infinite impact on the output values, even though the program does not use the input variable \texttt{x}, \cf{} \refexample{syntactic-vs-semantic-usage}.
\end{example}

Notably, the property $\boundedoutcomesle$ (with the operator $\le$) is subset-closed, \ie, if a program semantics satisfies the property, then all its subsets also satisfy the property.

\begin{lemma}[$\boundedoutcomesle$ is subset-closed]\lablemma{outcomes-subset-closed-le}
  For any two sets of traces $\defsetoftraces, \defsetoftraces'\in\tracetype$, it holds that:
  \begin{gather*}
    \defsetoftraces \subseteq \defsetoftraces' \land \defsetoftraces' \in \BOUNDEDOUTCOMESLE \ImplieS \defsetoftraces \in \BOUNDEDOUTCOMESLE
  \end{gather*}
\end{lemma}
\begin{proof}
  This proof follows directly from the monotonicity of the $\outcomes$ impact quantifier, \cf{} \reflemma{outcomes-monotonic}.
\end{proof}

Conversely, the property $\boundedoutcomesge$ (with the operator $\ge$) is superset-closed, \ie, if a program semantics satisfies $\boundedoutcomesge$, then all its supersets also satisfy $\boundedoutcomesge$.

\begin{lemma}[$\boundedoutcomesge$ is superset-closed]\lablemma{outcomes-superset-closed-ge}
  For any two sets of traces $\defsetoftraces, \defsetoftraces'\in\tracetype$, it holds that:
  \begin{gather*}
    \defsetoftraces \subseteq \defsetoftraces' \land \defsetoftraces \in \BOUNDEDOUTCOMESGE \ImplieS \defsetoftraces' \in \BOUNDEDOUTCOMESGE
  \end{gather*}
\end{lemma}
\begin{proof}
  Similarly to \reflemma{outcomes-subset-closed-le}, this proof follows directly from the monotonicity of the $\outcomes$ impact quantifier, \cf{} \reflemma{outcomes-monotonic}.
\end{proof}

\subsection{The \rangename{} Impact Quantifier}[\rangename]
\labsec{range}

The quantity $\range\in\tracetype\to\Rposplus$ determines the
size of the range of the output values for all the possible variations in the input variable $\definputvariables\in\setof\inputvariables$.
Let $\defoutputvariables$ be the set of output variables on which the impact is measured.
We are interested in the difference between the extreme values of the variable $\defoutputvariable$.
The difference of extreme values is computed by a distance measure $\distance$ computing the maximum distance between the set of given values. For instance, $\distance$ could be the Euclidean distance.

\begin{margintable}[*-10]
  \caption{Overview of the $\rangename$ impact quantifier for the variable $\texttt{angle}$ of \nrefprog{landing-alarm-system}.}
  \labtab{overview-range-angle}
  \resizebox{\textwidth}{!}{
  \begin{tabular}{c|c|c}
    \textsc{Input} & \textsc{Traces} & \rangename \\
    \hline
    \hline
    $\tuple{\blue{-4}}{1}$ & \makecell{$\tuple{\blue{-4}}{1}\rightarrow\langle{3}\rangle$, \\ $\tuple{\blue{1}}{1}\rightarrow\langle{0}\rangle$} & 3 \\
    \hline
    $\tuple{\blue{-4}}{2}$ & \makecell{$\tuple{\blue{-4}}{2}\rightarrow\langle{3}\rangle$, \\ $\tuple{\blue{1}}{2}\rightarrow\langle{1}\rangle$} & 2 \\
    \hline
    $\tuple{\blue{-4}}{3}$ & \makecell{$\tuple{\blue{-4}}{3}\rightarrow\langle{3}\rangle$, \\ $\tuple{\blue{1}}{3}\rightarrow\langle{2}\rangle$} & 1 \\
    \hline
    $\tuple{\blue{1}}{1}$ & \makecell{$\tuple{\blue{1}}{1}\rightarrow\langle{0}\rangle$, \\ $\tuple{\blue{-4}}{1}\rightarrow\langle{3}\rangle$} & 3 \\
    \hline
    $\tuple{\blue{1}}{2}$ & \makecell{$\tuple{\blue{1}}{2}\rightarrow\langle{1}\rangle$, \\ $\tuple{\blue{-4}}{2}\rightarrow\langle{3}\rangle$} & 2 \\
    \hline
    $\tuple{\blue{1}}{3}$ & \makecell{$\tuple{\blue{1}}{3}\rightarrow\langle{2}\rangle$, \\ $\tuple{\blue{-4}}{3}\rightarrow\langle{3}\rangle$} & 1 \\
    \hline
    \hline
  \end{tabular}}
\end{margintable}
\begin{margintable}[*5]
  \caption{Overview of the $\rangename$ impact quantifier for the variable $\texttt{speed}$ of \nrefprog{landing-alarm-system}.}
  \labtab{overview-range-speed}
  \resizebox{\textwidth}{!}{
  \begin{tabular}{c|c|c}
    \textsc{Input} & \textsc{Traces} & \rangename \\
    \hline
    \hline
    $\tuple{-4}{\blue{1}}$ & \makecell{$\tuple{-4}{\blue{1}}\rightarrow\langle{3}\rangle$, \\ $\tuple{-4}{\blue{2}}\rightarrow\langle{3}\rangle$ \\ $\tuple{-4}{\blue{3}}\rightarrow\langle{3}\rangle$} & 0 \\
    \hline
    $\tuple{-4}{\blue{2}}$ & \makecell{$\tuple{-4}{\blue{1}}\rightarrow\langle{3}\rangle$, \\ $\tuple{-4}{\blue{2}}\rightarrow\langle{3}\rangle$ \\ $\tuple{-4}{\blue{3}}\rightarrow\langle{3}\rangle$} & 0 \\
    \hline
    $\tuple{-4}{\blue{3}}$ & \makecell{$\tuple{-4}{\blue{1}}\rightarrow\langle{3}\rangle$, \\ $\tuple{-4}{\blue{2}}\rightarrow\langle{3}\rangle$ \\ $\tuple{-4}{\blue{3}}\rightarrow\langle{3}\rangle$} & 0 \\
    \hline
    $\tuple{1}{\blue{1}}$ & \makecell{$\tuple{1}{\blue{1}}\rightarrow\langle{0}\rangle$, \\ $\tuple{1}{\blue{2}}\rightarrow\langle{1}\rangle$ \\ $\tuple{1}{\blue{3}}\rightarrow\langle{2}\rangle$} & 2 \\
    \hline
    $\tuple{1}{\blue{1}}$ & \makecell{$\tuple{1}{\blue{1}}\rightarrow\langle{0}\rangle$, \\ $\tuple{1}{\blue{2}}\rightarrow\langle{1}\rangle$ \\ $\tuple{1}{\blue{3}}\rightarrow\langle{2}\rangle$} & 2 \\
    \hline
    $\tuple{1}{\blue{1}}$ & \makecell{$\tuple{1}{\blue{1}}\rightarrow\langle{0}\rangle$, \\ $\tuple{1}{\blue{2}}\rightarrow\langle{1}\rangle$ \\ $\tuple{1}{\blue{3}}\rightarrow\langle{2}\rangle$} & 2 \\
    \hline
    \hline
  \end{tabular}}
\end{margintable}

\begin{example}
  \labexample{range}
  We revisit again the example of the landing alarm system, program $\landingprogram$.
  Assuming $\definputvariables=\{\texttt{angle}\}$, $\tuple{-4}{1}$ is the first input configuration to be explored, we collect all traces that are
  starting from an input configuration that is a variation of $\tuple{-4}{1}$.
  As before, we obtain $\setdef{
    \retrieveoutput\defseq(\texttt{risk})
  }{
    \defseq \in \defsetoftraces \land
      \retrieveinput{\defseq}(\texttt{speed}) = 1
  }=\{0,3\}$ as the set of output values for the output variable \texttt{risk}.
%
  Here, we are interested in the difference between the extreme values of the outcomes. Hence, we apply the size function: $\distance(\{0,3\})=3$.
  By doing so for all possible input configurations $\reducedstate$, we obtain $\rangename_{\{\texttt{angle}\}}(\tracesemanticsnoparam\semanticsof{\landingprogram})=3$.
  Respectively, \reftab{overview-range-angle} illustrates the steps for $\rangename_{\{\texttt{angle}\}}(\tracesemanticsnoparam\semanticsof{\landingprogram})$ and \reftab{overview-range-speed} for $\rangename_{\{\texttt{speed}\}}(\tracesemanticsnoparam\semanticsof{\landingprogram})$.
\end{example}

  Formally, the $\rangename$ impact quantifier is defined as follows:

\begin{definition}[\rangename]\labdef{range}
  Given the set of input variables of interest $\definputvariables\in\setof\inputvariables$ and the output variables $\defoutputvariables$,
  the impact quantifier $\range\in\tracetype\to\Rposplus$ is defined as
  %
  \begin{align*}
    \range(\defsetoftraces) &\DefeQ \sup_{\definput\in\reducedstate}
      \distance(\setdef{
        \outputobs(\retrieveoutput{\defseq})(\defoutputvariables)
      }{
        \defseq \in \defsetoftraces \land \retrieveinput{\defseq} \stateeq{\inputvariableswithoutw} \definput
      })
  \end{align*}
\end{definition}

The above quantifier employs the auxiliary function $\distance$. For instance, such distance could be instantiated as the Euclidean distance: $\distance(X) \defeq \max_{x, x' \in X} \sqrt{\sum_k (x - x')^2}$ if $X$ is non-empty, otherwise $\distance(X) \defeq 0$. From here, we consider the Euclidean distance as the default distance measure.

We note that also $\rangename$ is monotonic in the amount of traces in $\defsetoftraces$.

\begin{lemma}[\rangename{} is Monotonic]
  \lablemma{range-monotonic}
For all set of traces $\defsetoftraces, \defsetoftraces'\in\tracetype$, it holds that:
  \begin{align*}
    \defsetoftraces \subseteq \defsetoftraces' \ImplieS \range(\defsetoftraces) \le \range(\defsetoftraces')
  \end{align*}
\end{lemma}
\begin{proof}
  Similarly to \reflemma{outcomes-monotonic}, the proof is based on the observation that the set of output values $\setdef{
    \outputobs(\retrieveoutput{\defseq})(\defoutputvariables)
  }{
    \defseq \in \defsetoftraces \land \retrieveinput{\defseq} \stateeq{\inputvariableswithoutw} \definput
  }$ is a subset of the set of output values $\setdef{
    \outputobs(\retrieveoutput{\defseq})(\defoutputvariables)
  }{
    \defseq \in \defsetoftraces' \land \retrieveinput{\defseq} \stateeq{\inputvariableswithoutw} \definput
  }$ since by hypothesis we have that $\defsetoftraces \subseteq \defsetoftraces'$.
  Hence, the size of the first set is less than or equal to the cardinality of the second set.
  We conclude that $\range(\defsetoftraces) \le \range(\defsetoftraces')$.
\end{proof}


Let $\boundedrange$ be the $\defbound$-bounded impact property when instantiated with the $\rangename$ impact quantifier.
The output semantics $\outputsemantics$ is sound and complete for validating $\boundedrange$.
\siderefbox{def}{output-abstraction-semantics}
\siderefbox{def}{abstraction-dependency-semantics}
\siderefbox{def}{abstraction-output-abstraction-semantics}
\begin{lemma}[$\boundedrange$ Validation]\lablemma{range-validation}
  \begin{align*}
    \collectingsemantics \subseteq \BOUNDEDRANGE \IfF \outputsemantics \subseteq \outputabstraction(\dependencyabstraction(\BOUNDEDRANGE))
  \end{align*}
\end{lemma}
\begin{proof}[Proof (Sketch)]
  Let us consider the (double) abstraction of the $\defbound$-bounded impact property $\boundedrange$:
  \begin{align*}
    & \outputabstraction(\dependencyabstraction(\boundedrange)) = \\
    &\quad
    \setdef{
      \defsetofdependencies \in \setof\pairofstates
    }{ \\ & \qquad
      \sup_{\definput\in\state}
      \distance({\setdef{
        \retrieveoutput{\defstate}(\defoutputvariables)
      }{
        \inputoutputtuple{\defstate} \in \defsetofdependencies \land \retrieveinput{\defseq} \stateeq{\inputvariableswithoutw} \definput
      }}) \comparison \defbound
    }
  \end{align*}
  As noticed before, the $\rangename$ impact quantifier does not consider the intermediate states and abstracts the output states before applying the size function.
  Hence, neither the abstraction to dependencies nor the abstraction to output values affect the validation of the property.
\end{proof}

\reflemma{range-monotonic} and \reflemma{range-validation} show that the $\rangename$ impact quantifier can be used to certify that a program has impact of \emph{at most} $\defbound$. In the present of deterministic systems, an impact quantity of zero implies that the input variable is unused.
Otherwise, similarly to $\rangename$, the input variable may have an impact on the program computation even if not used\sidenote{See \refexample{outcomes-nondeterminism} for a similar discussion of the \outcomesname{} impact quantifier when applied to programs containing nondeterministic statements.}.

Notably, the property $\boundedrangele$ (with the operator $\le$) is subset-closed, \ie, if a program semantics satisfies the property, then all its subsets also satisfy the property.

\begin{lemma}[$\boundedrangele$ is subset-closed]\lablemma{range-subset-closed-le}
  For any two sets of traces $\defsetoftraces, \defsetoftraces'\in\tracetype$, it holds that:
  \begin{gather*}
    \defsetoftraces \subseteq \defsetoftraces' \land \defsetoftraces' \in \BOUNDEDOUTCOMESLE \ImplieS \defsetoftraces \in \BOUNDEDOUTCOMESLE
  \end{gather*}
\end{lemma}
\begin{proof}
  This proof follows directly from the monotonicity of the $\range$ impact quantifier, \cf{} \reflemma{range-monotonic}.
\end{proof}

Conversely, the property $\boundedrangege$ (with the operator $\ge$) is superset-closed, \ie, if a program semantics satisfies $\boundedrangege$, then all its supersets also satisfy $\boundedrangege$.

\begin{lemma}[$\boundedrangege$ is superset-closed]\lablemma{range-superset-closed-ge}
  For any two sets of traces $\defsetoftraces, \defsetoftraces'\in\tracetype$, it holds that:
  \begin{gather*}
    \defsetoftraces \subseteq \defsetoftraces' \land \defsetoftraces \in \BOUNDEDOUTCOMESGE \ImplieS \defsetoftraces' \in \BOUNDEDOUTCOMESGE
  \end{gather*}
\end{lemma}
\begin{proof}
  Similarly to \reflemma{range-subset-closed-le}, this proof follows directly from the monotonicity of the $\range$ impact quantifier, \cf{} \reflemma{range-monotonic}.
\end{proof}

\subsection{The \qusedname{} Impact Quantifier}[\qusedname]
\labsec{qused}

The quantifier $\qused\in\tracetype\to\Nplus$ counts the number of input values for the variables in $\definputvariables\in\setof\inputvariables$ of $\definputvariables$ that are \emph{not} used to produce the outcomes.
Unlike \outcomesname{} and \rangename{}, even in the presence of non-deterministic statements, the impact of an input variable is zero \wrt{} \qusedname{} if and only if the input variable is not used to produce any outcome.

Given a set of traces $\defsetoftraces$, for each possible outcome of a trace $\deftrace\in\defsetoftraces$, \qusedname{} collects the traces $\deftrace'\in\defsetoftraces$ that share the same output abstraction, \ie, $\outputobs(\retrieveoutput{\deftrace'}) = \outputobs(\retrieveoutput\defstate)$. Then, it gathers the set of input values of the variables in $\definputvariables$ that correspond to these traces, called $I_\deftrace(\defsetoftraces)$.
The number of missing input values is obtained by subtracting
the set of input values from traces that share the same output abstraction, \cf{} $I_\deftrace(\defsetoftraces)$,
from
the set of input values that can produce the outcome $\retrieveoutput\defstate$ by variations of the variables in $\definputvariables$, called $Q_\definputvariables(I_\deftrace(\defsetoftraces))$.
Finally, $\qused$ yields the maximum number of input values that are not used to produce the outcomes.


Formally, the $\qusedname$ impact quantifier is defined as follows:

\begin{definition}[\qusedname]\labdef{qused}
  Given an input variable $\definputvariables\in\setof\inputvariables$,
  the impact quantifier $\qused\in\tracetype\to\Nplus$ is defined as
  %
  % \begin{eqnarray*}
  %   \qused(\defsetoftraces) \DefeQ
  %   \sup_{\deftrace \in \defsetoftraces}
  %     \spacer\left|
  %       \begin{array}{l}
  %         \setdef{\retrieveinput\defstate(\definputvariables)}{\retrieveinput\defstate \in \state} \setminus \\
  %         \quad \setdef{
  %           \retrieveinput{\defseq'}(\definputvariables)
  %           }{
  %             \deftrace' \in \defsetoftraces \LanD
  %                 \outputobs(\retrieveoutput{\deftrace}) = \outputobs(\retrieveoutput{\deftrace'})
  %           }
  %       \end{array}
  %     \right|
  % \end{eqnarray*}
  \begin{align*}
    \qused(\defsetoftraces) &\DefeQ \sup_{\deftrace \in \defsetoftraces}
    \cardinality{
        Q_\definputvariables(I_\deftrace(\defsetoftraces)) \setminus I_\deftrace(\defsetoftraces)
      } \\
      I_{\deftrace}(\defsetoftraces) &\spacearound= \setdef{
        \retrieveinput{\defseq'}
      }{
        \defseq' \in \defsetoftraces \land
        \outputobs(\retrieveoutput{\deftrace}) = \outputobs(\retrieveoutput{\defseq'})
      } \\
      Q_\definputvariables(\defsetofstates) &\spacearound= \setdef{
        \defstate' \in \state
      }{
        \defstate \in \defsetofstates \land
          \defstate' \stateeq{\inputvariableswithoutw} \defstate
      }
  \end{align*}
  where $I_{\deftrace}(\defsetoftraces)$ is the set of input states of traces in $\defsetoftraces$ that share the same output abstraction as $\deftrace$ and $Q_\definputvariables(\defsetofstates)$ is the set of states that are variations from states in $\defsetofstates$ only in the variables $\definputvariables$.
\end{definition}


\begin{example}
  To demonstrate the $\qusedname$ impact quantifier, we employ four simple programs.
  The first program we consider, \refprog{id}, returns the value of the only input variable $\texttt{x}$ without any modification.
  The second program, \refprog{random}, returns a random value.
  The last two, \refprog{mix1} and \refprog{mix2}, combine together non-deterministic and deterministic statements.
  For simplicity, variables are assumed to range in $\{0, 1, 2\}$.

  \begin{marginlisting}
    \caption{The identity program.}
    \labprog{id}
    \vspace{15pt}
  \begin{lstlisting}[style=mystyle,language=customPython]
id(x):
  return x;
 \end{lstlisting}
  \end{marginlisting}
  The trace semantics of \refprog{id} is:
  \begin{align*}
    \tracesemanticsnoparam\semanticsof{\texttt{id}}
    =
    \left\{
      \begin{array}{l}
        \tracetuple{\locinit}{\langle 0\rangle}\rightarrow\tracetuple{\loc{3}}{\langle 0\rangle},\\
        \tracetuple{\locinit}{\langle 1\rangle}\rightarrow\tracetuple{\loc{3}}{\langle 1\rangle},\\
        \tracetuple{\locinit}{\langle 2\rangle}\rightarrow\tracetuple{\loc{3}}{\langle 2\rangle}
      \end{array}
    \right\}
  \end{align*}
  \begin{marginfigure}[*-1.5]
    \begin{tikzpicture}[scale=0.8]
      \node (linit) at (-0.9,2) {$\locinit:$};
      \node (lend) at (-0.9,0) {$\loc{3}:$};

      % Define the positions of the nodes
      \node (0) at (0,2) {${\langle 0 \rangle}$};
      \node (1) at (1,2) {${\langle 1 \rangle}$};
      \node (2) at (2,2) {${\langle 2 \rangle}$};

      \node (out0) at (0,0) {${\langle 0 \rangle}$};
      \node (out1) at (1,0) {${\langle 1 \rangle}$};
      \node (out2) at (2,0) {${\langle 2 \rangle}$};

      % Draw the edges
      \draw[->] (0) -> (out0);
      \draw[->] (1) -> (out1);
      \draw[->] (2) -> (out2);

      \draw[decorate,decoration={brace,amplitude=8pt,mirror,raise=4pt},yshift=0pt,xshift=-0.1pt] (-1.1,2.2) -- (-1.1,-0.2) node [black,midway,xshift=-0.7cm,rotate=90] {$\tracesemanticsnoparam\semanticsof{\texttt{id}}$};
  \end{tikzpicture}
  \caption{Graphical representation of the trace semantics of \refprog{id}.}
  \labfig{id-traces}
  \end{marginfigure}
  As explained above, for any $\deftrace\in \tracesemanticsnoparam\semanticsof{\texttt{id}}$ we collect the set of input states that belong to traces $\deftrace'\in\tracesemanticsnoparam\semanticsof{\texttt{id}}$ that share the same output abstraction as $\deftrace$, \cf{} $I_{\deftrace}(\tracesemanticsnoparam\semanticsof{\texttt{id}})$.
  In this case, the set of input states that can produce the outcome $0$ is $\{\tracetuple{\locinit}{\langle 0\rangle}\}$, for the outcome $1$ is $\{\tracetuple{\locinit}{\langle 1\rangle}\}$, and for the outcome $2$ is $\{\tracetuple{\locinit}{\langle 2\rangle}\}$, all singletons.
  Hence, the set of states that are variations, \cf{} $Q_{\{\texttt{x}\}}(I_{\deftrace}(\tracesemanticsnoparam\semanticsof{\texttt{id}}))$, is the complement of each singleton.
  Hence, for the three set of input states, we obtain the values $2, 2, 2$, respectively.
  Thus, $\qusedname_{\{\texttt{x}\}}(\tracesemanticsnoparam\semanticsof{\texttt{id}})=2$, meaning that the input variable $\texttt{x}$ is used.

\begin{marginlisting}[*-3]
  \caption{The random program.}
  \labprog{random}
  \vspace{15pt}
\begin{lstlisting}[style=mystyle,language=customPython]
random(x):
  return rand();
  \end{lstlisting}
\end{marginlisting}
  The trace semantics of \refprog{random} is:
  \begin{align*}
    \tracesemanticsnoparam\semanticsof{\texttt{random}}
    =
    \left\{
      \begin{array}{l}
        \tracetuple{\locinit}{\langle 0\rangle}\rightarrow\tracetuple{\loc{3}}{\langle \anyvalue\rangle}, \\
        \tracetuple{\locinit}{\langle 1\rangle}\rightarrow\tracetuple{\loc{3}}{\langle \anyvalue\rangle}, \\
        \tracetuple{\locinit}{\langle 2\rangle}\rightarrow\tracetuple{\loc{3}}{\langle \anyvalue\rangle}
      \end{array}
    \right\}
  \end{align*}

  \begin{marginfigure}[*-5]
    \begin{tikzpicture}[scale=0.8]
      \node (linit) at (-0.9,2) {$\locinit:$};
      \node (lend) at (-0.9,0) {$\loc{3}:$};

      % Define the positions of the nodes
      \node (0) at (0,2) {${\langle 0 \rangle}$};
      \node (1) at (1,2) {${\langle 1 \rangle}$};
      \node (2) at (2,2) {${\langle 2 \rangle}$};

      \node (out0) at (0,0) {${\langle 0 \rangle}$};
      \node (out1) at (1,0) {${\langle 1 \rangle}$};
      \node (out2) at (2,0) {${\langle 2 \rangle}$};

      % Draw the edges
      \draw[->] (0) -> (out0);
      \draw[->] (0) -> (out1);
      \draw[->] (0) -> (out2);
      \draw[->] (1) -> (out0);
      \draw[->] (1) -> (out1);
      \draw[->] (1) -> (out2);
      \draw[->] (2) -> (out0);
      \draw[->] (2) -> (out1);
      \draw[->] (2) -> (out2);

      \draw[decorate,decoration={brace,amplitude=8pt,mirror,raise=4pt},yshift=0pt,xshift=-0.1pt] (-1.1,2.2) -- (-1.1,-0.2) node [black,midway,xshift=-0.7cm,rotate=90] {$\tracesemanticsnoparam\semanticsof{\texttt{random}}$};
  \end{tikzpicture}
  \caption{Graphical representation of the trace semantics of \refprog{random}.}
  \labfig{random-traces}
  \end{marginfigure}
  In this case, any outcome can be produced by any input value, thus the set of input states is $\{\tracetuple{\locinit}{\langle 0\rangle}, \tracetuple{\locinit}{\langle 1\rangle}, \tracetuple{\locinit}{\langle 2\rangle}\}$ for all outcomes.
  Since all possible permutations of the variable \texttt{x} are already in each set of input states from outcomes, we obtain the value $0$.
  Thus, $\qusedname_{\{\texttt{x}\}}(\tracesemanticsnoparam\semanticsof{\texttt{random}})=0$, meaning that the input variable $\texttt{x}$ is not used.

  \begin{marginlisting}[*-4]
    \caption{First example combining random and constant value.}
    \labprog{mix1}
    \vspace{25pt}
  \begin{lstlisting}[style=mystyle,language=customPython]
mix1(x):
  if x > 0:
    return rand();
  else:
    return 0;
  \end{lstlisting}
\end{marginlisting}
  The trace semantics of \refprog{mix1} is:
  \begin{align*}
    \tracesemanticsnoparam\semanticsof{\texttt{mix1}}
    =
    \left\{
      \begin{array}{l}
        \tracetuple{\locinit}{\langle 0\rangle}\rightarrow\tracetuple{\loc{6}}{\langle 0\rangle}, \\
        \tracetuple{\locinit}{\langle 1\rangle}\rightarrow\tracetuple{\loc{6}}{\langle \anyvalue\rangle}, \\
        \tracetuple{\locinit}{\langle 2\rangle}\rightarrow\tracetuple{\loc{6}}{\langle \anyvalue\rangle}
      \end{array}
      \right\}
  \end{align*}
  \begin{marginfigure}[*-2]
    \begin{tikzpicture}[scale=0.8]
      \node (linit) at (-0.9,2) {$\locinit:$};
      \node (lend) at (-0.9,0) {$\loc{6}:$};

      % Define the positions of the nodes
      \node (0) at (0,2) {${\langle 0 \rangle}$};
      \node (1) at (1,2) {${\langle 1 \rangle}$};
      \node (2) at (2,2) {${\langle 2 \rangle}$};

      \node (out0) at (0,0) {${\langle 0 \rangle}$};
      \node (out1) at (1,0) {${\langle 1 \rangle}$};
      \node (out2) at (2,0) {${\langle 2 \rangle}$};

      % Draw the edges
      \draw[->] (0) -> (out0);
      \draw[->] (1) -> (out0);
      \draw[->] (1) -> (out1);
      \draw[->] (1) -> (out2);
      \draw[->] (2) -> (out0);
      \draw[->] (2) -> (out1);
      \draw[->] (2) -> (out2);

      \draw[decorate,decoration={brace,amplitude=8pt,mirror,raise=4pt},yshift=0pt,xshift=-0.1pt] (-1.1,2.2) -- (-1.1,-0.2) node [black,midway,xshift=-0.7cm,rotate=90] {$\tracesemanticsnoparam\semanticsof{\texttt{mix1}}$};
  \end{tikzpicture}
  \caption{Graphical representation of the trace semantics of \refprog{mix1}.}
  \labfig{mix1-traces}
  \end{marginfigure}
  In this case, the set of input states that can produce the outcome $0$ is $\{\tracetuple{\locinit}{\langle 0\rangle}, \tracetuple{\locinit}{\langle 1\rangle}, \tracetuple{\locinit}{\langle 2\rangle}\}$, for the outcome $1$ is $\{\tracetuple{\locinit}{\langle 1\rangle}, \tracetuple{\locinit}{\langle 2\rangle}\}$, and for the outcome $2$ is $\{\tracetuple{\locinit}{\langle 1\rangle}, \tracetuple{\locinit}{\langle 2\rangle}\}$.
  Respectively, the input state $\tracetuple{\locinit}{\langle 0 \rangle}$ is missing from both sets of input states that can produce the outcomes $1$ and $2$.
  Thus, we yield the highest value of missing states, obtaining $\qusedname_{\{\texttt{x}\}}(\tracesemanticsnoparam\semanticsof{\texttt{mix1}})=1$.


  \begin{marginlisting}
    \caption{Second example combining random and constant value.}
    \labprog{mix2}
    \vspace{25pt}
  \begin{lstlisting}[style=mystyle,language=customPython]
mix2(x, y):
  if x and y:
    return True
  elif not x and y:
    return rand()
  else:
    return False
 \end{lstlisting}
  \end{marginlisting}
  For this last program we introduce another variable \texttt{y}, we simplify even further the variable to boolean values $\{\true, \false\}$.
  The trace semantics of \refprog{mix2} is:
  \begin{align*}
    \tracesemanticsnoparam\semanticsof{\texttt{mix2}}
    =
    \left\{
      \begin{array}{l}
        \tracetuple{\locinit}{\langle \true, \true\rangle}\rightarrow\tracetuple{\loc{8}}{\langle \true\rangle}, \\
        \tracetuple{\locinit}{\langle \true, \false\rangle}\rightarrow\tracetuple{\loc{8}}{\langle \false\rangle}, \\
        \tracetuple{\locinit}{\langle \false, \true\rangle}\rightarrow\tracetuple{\loc{8}}{\langle \true\rangle}, \\
        \tracetuple{\locinit}{\langle \false, \true\rangle}\rightarrow\tracetuple{\loc{8}}{\langle \false\rangle}, \\
        \tracetuple{\locinit}{\langle \false, \false\rangle}\rightarrow\tracetuple{\loc{8}}{\langle \false\rangle}
      \end{array}
      \right\}
  \end{align*}
  \begin{marginfigure}
    \begin{tikzpicture}[scale=0.8]
      \node (linit) at (-0.9,2) {$\locinit:$};
      \node (lend) at (-0.9,0) {$\loc{8}:$};

      % Define the positions of the nodes
      \node (TT) at (0,2) {${\tuple\true\true}$};
      \node (TF) at (1,2) {${\tuple\true\false}$};
      \node (FT) at (2,2) {${\tuple\false\true}$};
      \node (FF) at (3,2) {${\tuple\false\false}$};

      \node (T) at (0.5,0) {${\langle \true \rangle}$};
      \node (F) at (2.5,0) {${\langle \false \rangle}$};

      % Draw the edges
      \draw[->] (TT) -> (T);
      \draw[->] (TF) -> (F);
      \draw[->] (FT) -> (T);
      \draw[->] (FT) -> (F);
      \draw[->] (FF) -> (F);

      \draw[decorate,decoration={brace,amplitude=8pt,mirror,raise=4pt},yshift=0pt,xshift=-0.1pt] (-1.1,2.2) -- (-1.1,-0.2) node [black,midway,xshift=-0.7cm,rotate=90] {$\tracesemanticsnoparam\semanticsof{\texttt{mix2}}$};
  \end{tikzpicture}
  \caption{Graphical representation of the trace semantics of \refprog{mix2}.}
  \labfig{mix2-traces}
  \end{marginfigure}
  In this case, the set of input states that can produce the outcome $\true$ is $\{\tracetuple{\locinit}{\langle \true, \true\rangle}, \tracetuple{\locinit}{\langle \false, \true\rangle}\}$, for the other outcome $\false$ the set of input states is $\{\tracetuple{\locinit}{\langle \true, \false\rangle}, \tracetuple{\locinit}{\langle \false, \true\rangle}, \tracetuple{\locinit}{\langle \false, \false\rangle}\}$.
  In case $\definputvariables=\{\texttt{x}\}$, the state $\tracetuple{\locinit}{\langle \true, \true\rangle}$ is missing from the set of input states that can produce the outcome $\false$, obtained by permuting the variable \texttt{x} on the input state $\tracetuple{\locinit}{\langle \false, \true\rangle}$.
  Instead, the set of states $\{\tracetuple{\locinit}{\langle \true, \true\rangle}, \tracetuple{\locinit}{\langle \false, \true\rangle}\}$ already contains all the possible permutations of the variable \texttt{x}.
  Thus, we yield the highest value of missing states, obtaining $\qusedname_{\{\texttt{x}\}}(\tracesemanticsnoparam\semanticsof{\texttt{mix2}})=1$.
  On the other hand, if $\definputvariables=\{\texttt{y}\}$, the states $\tracetuple{\locinit}{\langle \false, \false\rangle}$ and $\tracetuple{\locinit}{\langle \true, \false\rangle}$ are missing from the set of input states that can produce the outcome $\true$. No state is missing from the set of input states that can produce the outcome $\false$.
  Thus, we yield the highest value of missing states, obtaining $\qusedname_{\{\texttt{y}\}}(\tracesemanticsnoparam\semanticsof{\texttt{mix2}})=2$.

  From these examples we notice that the $\qusedname$ impact quantifier is able to capture that a variable is not used in a program, even in the presence of non-deterministic statements, \cf{} \refprog{random}.
  Moreover, $\qusedname$ also discriminates between different amount of impact from \refprog{id}, \refprog{mix1}, and \refprog{mix2}.
  Indeed, \refprog{id} is, intuitively, the program where the input variable has the highest impact possible as the output is exactly its value, while \refprog{random} the lowest as the output is random.
\end{example}

Next, we show the validation of the $\defbound$-bounded impact property when instantiated with the $\qusedname$ impact quantifier.
% Note the use of the $\ge$ operator\sidenote{Previously, for the \outcomesname{} and \rangename{} impact quantifiers, the $\defbound$-bounded impact property employed the $\le$ comparison operator.} in our property $\boundedqused$. This design choice is due to the fact that the $\qusedname$ impact quantifier is neither monotonic nor anti-monotonic in the amount of traces in $\defsetoftraces$. Thus, the same way the syntactic dependency analysis consider an under-approximation of dependencies among variables to return a sound result, the static analysis for $\qusedname$ considers an under-approximation of the ``missing'' input values. In detail, it considers an under-approximation of the set of input values $Q_\definputvariables(I_\deftrace(\defsetoftraces)) \setminus I_\deftrace(\defsetoftraces)$ for the given set of traces $\defsetoftraces$. This under-approximation may lead to a lower value of $\qusedname$ than the actual value, hence the use of the $\ge$ operator as the concrete bound is always higher.

\siderefbox{def}{output-abstraction-semantics}
\siderefbox{def}{abstraction-dependency-semantics}
\siderefbox{def}{abstraction-output-abstraction-semantics}
\begin{lemma}[$\boundedqused$ Validation]\lablemma{qused-validation}
  \begin{align*}
    \collectingsemantics \subseteq \BOUNDEDQUSED \IfF \outputsemantics \subseteq \outputabstraction(\dependencyabstraction(\BOUNDEDQUSED))
  \end{align*}
\end{lemma}
\begin{proof}[Proof (Sketch)]
  Let us consider the $\outputabstraction \circ \dependencyabstraction$ abstraction of the $\defbound$-bounded impact property $\boundedqused$:
  \marginnote{
    \begin{align*}
      &I_{\retrieveoutput{\defstate}}(\defsetofdependencies) \spacearound= \\
      &\quad \seTDef{
        \retrieveinput{\defstate'}
      }{
        \inputoutputtuple{\defstate'} \in \defsetofdependencies \land
        \outputobs(\retrieveoutput{\defstate}) = \outputobs(\retrieveoutput{\defstate'})
      } \\
      &Q_\definputvariables(\defsetofstates) \spacearound= \\
      &\quad\setdef{
        \defstate' \in \state
      }{
        \defstate \in \defsetofstates \land
          \defstate' \stateeq{\inputvariableswithoutw} \defstate
      }
    \end{align*}
  }
  \begin{eqnarray*}
    \lefteqn{\outputabstraction(\dependencyabstraction(\boundedqused)) =} \\
    &
    \setdef{
      \defsetofdependencies \in \setof\pairofstates
    }{
      \sup_{\inputoutputtuple{\defstate} \in \defsetofdependencies}
    \cardinality*{
        Q_\definputvariables(I_{\retrieveoutput{\defstate}}(\defsetofdependencies)) \setminus I_{\retrieveoutput{\defstate}}(\defsetofdependencies)
      } \comparison \defbound
    }
  \end{eqnarray*}
  As noticed before, the $\qusedname$ impact quantifier does not consider the intermediate states and abstracts the output states before any computation.
  Hence, neither the abstraction to dependencies nor the abstraction to output values affect the validation of the property.
\end{proof}

The next example shows that $\qusedname$ is neither monotonic nor anti-monotonic in the amount of traces, meaning that, by adding or removing traces from a given set $\defsetoftraces\in\tracetype$ the quantity $\qused(\defsetoftraces)$ may increase or decrease.
Notably, the major consequence is that $\boundedqused$ is neither subset- nor superset-closed. Thus, not all the subsets, respectively supersets, of a potential over-approximation, respectively under-approximation, of the trace semantics satisfy the $\boundedqused$ property.
% To address this issue, from the result of the abstract analysis we compute an under-approximation of the input values that are definitely missing from the set of input values that can produce the outcomes. This under-approximation provides a lower bound on the $\qusedname$ impact quantifier, hence the use of the $\ge$ operator in the $\defbound$-bounded impact property.

\begin{example}\labexample{qused-not-monotonic}
In this example, we show that the $\qusedname$ impact quantifier is neither monotonic nor anti-monotonic in the amount of traces.
Specifically, we consider two input variables with boolean values $\{\true, \false\}$, the \refprog{id2} returns the second component of the input without any modification.
\begin{marginlisting}
  \caption{Identity function on the second component.}
  \labprog{id2}
  \vspace{25pt}
\begin{lstlisting}[style=mystyle,language=customPython]
id2(x, y):
  return y
 \end{lstlisting}
\end{marginlisting}
The trace semantics of \refprog{id2} is:
\begin{align*}
  \tracesemanticsnoparam\semanticsof{\texttt{id2}}
  =
  \left\{
    \begin{array}{l}
      \tracetuple{\locinit}{\langle \true, \true\rangle}\rightarrow\tracetuple{\loc{3}}{\langle \true\rangle}, \\
      \tracetuple{\locinit}{\langle \true, \false\rangle}\rightarrow\tracetuple{\loc{3}}{\langle \false\rangle}, \\
      \tracetuple{\locinit}{\langle \false, \true\rangle}\rightarrow\tracetuple{\loc{3}}{\langle \true\rangle}, \\
      \tracetuple{\locinit}{\langle \false, \false\rangle}\rightarrow\tracetuple{\loc{3}}{\langle \false\rangle}
    \end{array}
    \right\}
\end{align*}
\begin{marginfigure}
  \begin{tikzpicture}[scale=0.8]
    \node (linit) at (-0.9,2) {$\locinit:$};
    \node (lend) at (-0.9,0) {$\loc{3}:$};

    % Define the positions of the nodes
    \node (TT) at (0,2) {${\tuple\true\true}$};
    \node (TF) at (1,2) {${\tuple\true\false}$};
    \node (FT) at (2,2) {${\tuple\false\true}$};
    \node (FF) at (3,2) {${\tuple\false\false}$};

    \node (T) at (0.5,0) {${\langle \true \rangle}$};
    \node (F) at (2.5,0) {${\langle \false \rangle}$};

    % Draw the edges
    \draw[->] (TT) -> (T);
    \draw[->] (TF) -> (F);
    \draw[->] (FT) -> (T);
    \draw[->] (FF) -> (F);

    \draw[decorate,decoration={brace,amplitude=8pt,mirror,raise=4pt},yshift=0pt,xshift=-0.1pt] (-1.1,2.2) -- (-1.1,-0.2) node [black,midway,xshift=-0.7cm,rotate=90] {$\tracesemanticsnoparam\semanticsof{\texttt{id2}}$};
\end{tikzpicture}
\caption{Graphical representation of the trace semantics of \refprog{id2}.}
\labfig{id2-traces}
\end{marginfigure}
For the set of traces $\tracesemanticsnoparam\semanticsof{\texttt{id2}}$, we have that $\qusedname_{\{\texttt{x} \}}(\tracesemanticsnoparam\semanticsof{\texttt{id2}})=0$ as no input value is missing from both the sets of input values that can produce the outcomes $\true$ and $\false$. In fact, the variable \texttt{x} is \emph{not used} in the program.
Let us consider the bigger set of traces $\defsetoftraces \supseteq \tracesemanticsnoparam\semanticsof{\texttt{id2}}$:
\begin{align*}
  \defsetoftraces
  =
  \left\{
    \begin{array}{l}
      \tracetuple{\locinit}{\langle \true, \true\rangle}\rightarrow\tracetuple{\loc{3}}{\langle \true\rangle}, \\
      \tracetuple{\locinit}{\langle \true, \false\rangle}\rightarrow\tracetuple{\loc{3}}{\langle \false\rangle}, \\
      \tracetuple{\locinit}{\langle \false, \true\rangle}\rightarrow\tracetuple{\loc{3}}{\langle \true\rangle}, \\
      \tracetuple{\locinit}{\langle \false, \true\rangle}\rightarrow\tracetuple{\loc{3}}{\langle \false\rangle}, \\
      \tracetuple{\locinit}{\langle \false, \false\rangle}\rightarrow\tracetuple{\loc{3}}{\langle \false\rangle}
    \end{array}
    \right\}
\end{align*}
\begin{marginfigure}
  \begin{tikzpicture}[scale=0.8]
    % \node (linit) at (-0.9,2) {$\locinit:$};
    % \node (lend) at (-0.9,0) {$\loc{3}:$};

    % Define the positions of the nodes
    \node (TT) at (0,2) {${\tuple\true\true}$};
    \node (TF) at (1,2) {${\tuple\true\false}$};
    \node (FT) at (2,2) {${\tuple\false\true}$};
    \node (FF) at (3,2) {${\tuple\false\false}$};

    \node (T) at (0.5,0) {${\langle \true \rangle}$};
    \node (F) at (2.5,0) {${\langle \false \rangle}$};

    % Draw the edges
    \draw[->] (TT) -> (T);
    \draw[->] (TF) -> (F);
    \draw[->] (FT) -> (T);
    \draw[->] (FT) -> (F);
    \draw[->] (FF) -> (F);

    \draw[decorate,decoration={brace,amplitude=8pt,mirror,raise=4pt},yshift=0pt,xshift=-0.1pt] (-0.5,2.2) -- (-0.5,-0.2) node [black,midway,xshift=-0.7cm] {$\defsetoftraces$};
\end{tikzpicture}
\caption{Graphical representation of $\defsetoftraces$.}
\labfig{defsetoftraces-traces}
\end{marginfigure}
This set of traces could be obtained by the \refprog{mix2}.
In this case, $\qusedname_{\{\texttt{x} \}}(\defsetoftraces)=1$ as the input value $\tracetuple{\locinit}{\langle \true, \true\rangle}$ is missing from the set of input values that can produce the outcome $\false$, \cf, the set $\{ \tracetuple{\locinit}{\langle \true, \false\rangle}, \tracetuple{\locinit}{\langle \false, \true\rangle}, \tracetuple{\locinit}{\langle \false, \false\rangle} \}$.
Note that, in this case, the variable \texttt{x} is \emph{used} in the program.
Lastly, let us increase again the set of traces to $\defsetoftraces' \supseteq \defsetoftraces$:
\begin{align*}
  \defsetoftraces'
  =
  \left\{
    \begin{array}{l}
      \tracetuple{\locinit}{\langle \true, \true\rangle}\rightarrow\tracetuple{\loc{3}}{\langle \true\rangle}, \\
      \tracetuple{\locinit}{\langle \true, \true\rangle}\rightarrow\tracetuple{\loc{3}}{\langle \false\rangle}, \\
      \tracetuple{\locinit}{\langle \true, \false\rangle}\rightarrow\tracetuple{\loc{3}}{\langle \false\rangle}, \\
      \tracetuple{\locinit}{\langle \false, \true\rangle}\rightarrow\tracetuple{\loc{3}}{\langle \true\rangle}, \\
      \tracetuple{\locinit}{\langle \false, \true\rangle}\rightarrow\tracetuple{\loc{3}}{\langle \false\rangle}, \\
      \tracetuple{\locinit}{\langle \false, \false\rangle}\rightarrow\tracetuple{\loc{3}}{\langle \false\rangle}
    \end{array}
    \right\}
\end{align*}
\begin{marginfigure}[*-6]
  \begin{tikzpicture}[scale=0.8]
    \node (linit) at (-0.9,2) {$\locinit:$};
    \node (lend) at (-0.9,0) {$\loc{6}:$};

    % Define the positions of the nodes
    \node (TT) at (0,2) {${\tuple\true\true}$};
    \node (TF) at (1,2) {${\tuple\true\false}$};
    \node (FT) at (2,2) {${\tuple\false\true}$};
    \node (FF) at (3,2) {${\tuple\false\false}$};

    \node (T) at (0.5,0) {${\langle \true \rangle}$};
    \node (F) at (2.5,0) {${\langle \false \rangle}$};

    % Draw the edges
    \draw[->] (TT) -> (T);
    \draw[->] (TT) -> (F);
    \draw[->] (TF) -> (F);
    \draw[->] (FT) -> (T);
    \draw[->] (FT) -> (F);
    \draw[->] (FF) -> (F);

    \draw[decorate,decoration={brace,amplitude=8pt,mirror,raise=4pt},yshift=0pt,xshift=-0.1pt] (-1.1,2.2) -- (-1.1,-0.2) node [black,midway,xshift=-0.7cm,rotate=90] {$\tracesemanticsnoparam\semanticsof{\texttt{false\_or\_rand}}$};
\end{tikzpicture}
\caption{Graphical representation of the trace semantics of \refprog{false-or-rand}.}
\labfig{false-or-rand-traces}
\end{marginfigure}
In this case, $\qusedname_{\{\texttt{x} \}}(\defsetoftraces')=0$ as the input $\tracetuple{\locinit}{\langle \true, \true\rangle}$ is not missing anymore in the set of input values that can produce the outcomes $\false$.
For instance, a program that may generate such a set of traces is the \refprog{false-or-rand}.
\begin{marginlisting}
  \caption{False or random program.}
  \labprog{false-or-rand}
  \vspace{15pt}
\begin{lstlisting}[style=mystyle,language=customPython]
false_or_rand(x, y):
  if not y:
    return False
  else:
    return rand()
 \end{lstlisting}
\end{marginlisting}
This example shows that \[\qusedname_{\{\texttt{x} \}}(\tracesemanticsnoparam\semanticsof{\texttt{id2}}) \LE \qusedname_{\{\texttt{x} \}}(\defsetoftraces) \GE \qusedname_{\{\texttt{x} \}}(\defsetoftraces')\] where $\tracesemanticsnoparam\semanticsof{\texttt{id2}} \subseteq \defsetoftraces \subseteq \defsetoftraces'$.
Thus, the $\qusedname$ impact quantifier is neither monotonic nor anti-monotonic in the amount of traces, as the quantifier $\qusedname$ may increase or decrease by adding or removing traces from a given set.
\end{example}





Unlike for the previous impact quantifiers, it holds that, whenever $\qused(\tracesemantics)$ is 0, the program $\defprogram$ does not use the input variable $\definputvariables$, even in the presence of non-determinism.
Thus, we can establish the following equivalence:

\siderefbox{def}{abstract-unused}
\siderefbox{def}{qused}
\begin{lemma}[Unused Equivalence]\lablemma{qualitative-quantitative-equivalence}
  For all program $\defprogram$, it holds that:
  \begin{gather*}
    \unusediowrapper(\tracesemantics) \IfF \qused(\tracesemantics) = 0
  \end{gather*}
\end{lemma}
\begin{proof}
  To show ($\Rightarrow$) we assume that $\unusediowrapper(\tracesemantics)$.
  As a consequence, we have that any outcome either is not reachable or is reachable from every input value of the variables $\definputvariables$.
  Thus, for any $\deftrace\in\defsetoftraces$, the set of input values that can produce that outcome is the set of all input values, \ie{}, $I_{\deftrace}(\defsetoftraces) = \values$.
  Since we remove from $Q_{\definputvariables}(I_{\deftrace}(\defsetoftraces))$ the set of input values, we obtain a value of 0 for all traces, hence $\qused(\tracesemantics) = 0$.

  To show ($\Leftarrow$) we assume that $\qused(\tracesemanticsnoparam) = 0$.
  By definition, for all $\retrieveoutput\defstate$, the set of input values that can produce the outcome $\retrieveoutput\defstate$ is the set of all input values, otherwise $\qused(\tracesemanticsnoparam)$ would not be 0.
  Thus, for all traces $\deftrace\in\tracesemantics$ and input values $\defvalue\in\values$ such that $\retrieveinput{\deftrace}(\definputvariables) \neq \defvalue$, we have that it exists a trace $\deftrace'$ such that $\outputobs(\retrieveoutput{\deftrace'}) = \outputobs(\retrieveoutput{\deftrace})$ and $\retrieveinput{\deftrace'}(\definputvariables) = \defvalue$, otherwise it would not be possible that the set of input values that can produce any outcome is the set of all input values. Hence, $\unusediowrapper(\tracesemantics)$.
\end{proof}


\begin{marginfigure}
  \caption{Graphical comparison between the unused $\unused$ and the $\defbound$-bounded impact property $\boundedqused$.}
  \labfig{unused-bounded}
  % shows on top a bar going from $\defbound=0$ (left) to $\defbound=1$, $\defbound=2$, and finally $\defbound=\dots$ (right); this is a dashed bar with a bigger dot for numbers, label ar above.
  % Below, we have a dotted vertical line going from each dot to the bottom of the figure.
  % Here we have four layers, each one with two curly braces, splitting in two the layer.
  \centering
  \begin{tikzpicture}[scale=0.9]
    \def\first{0};
    \def\fourth{3.9};
    \def\second{\fourth/3*1};
    \def\third{\fourth/3*2};

    \def\slack{0.3};
    \def\beforefirst{\first-\slack};
    \def\beforesecond{\second-\slack};
    \def\beforethird{\third-\slack};
    \def\beforefourth{\fourth-\slack};
    \def\afterfirst{\first+\slack};
    \def\aftersecond{\second+\slack};
    \def\afterthird{\third+\slack};
    \def\afterfourth{\fourth+2*\slack};

    \def\layerheight{1.1};
    \def\firstlevel{-0.3};
    \def\secondlevel{\firstlevel-\layerheight};
    \def\thirdlevel{\secondlevel-\layerheight};
    \def\fourthlevel{\thirdlevel-\layerheight};

    \draw[dashed] (\first,0) -- (\fourth + 0.5,0);
    \foreach \x in {\first, \second, \third, \fourth}
    \draw (\x cm,3pt) -- (\x cm,-3pt);
    \draw (\first,0) node[above=4pt] {$\defbound=0$};
    \draw (\second,0) node[above=4pt] {$\defbound=1$};
    \draw (\third,0) node[above=4pt] {$\defbound=2$};
    \draw (\fourth,0) node[above=4pt] {$\defbound=\cdots$};
    \draw[dotted] (\first,0) -- (\first,\fourthlevel);
    \draw[dotted] (\second,0) -- (\second,\fourthlevel);
    \draw[dotted] (\third,0) -- (\third,\fourthlevel);
    \draw[dotted] (\fourth,0) -- (\fourth,\fourthlevel);

    % horizontal curly brackets, two each level
    \draw [decorate,decoration={brace,amplitude=4pt},xshift=0pt,yshift=0pt] (\afterfirst,\firstlevel) -- (\beforefirst,\firstlevel) node [black,midway,yshift=-0.4cm,xshift=0.7cm] {$\boundedqusedle[0] = \unused$};
    \draw [decorate,decoration={brace,amplitude=4pt},xshift=0pt,yshift=0pt] (\afterfourth,\firstlevel) -- (\beforesecond,\firstlevel) node [black,midway,yshift=-0.4cm, xshift=0.5cm] {$\boundedqusedge[1] = \neg\unused$};
    \draw [decorate,decoration={brace,amplitude=4pt},xshift=0pt,yshift=0pt] (\aftersecond,\secondlevel) -- (\beforefirst,\secondlevel) node [black,midway,yshift=-0.4cm] {$\boundedqusedle[1]$};
    \draw [decorate,decoration={brace,amplitude=4pt},xshift=0pt,yshift=0pt] (\afterfourth,\secondlevel) -- (\beforethird,\secondlevel) node [black,midway,yshift=-0.4cm] {$\boundedqusedge[2]$};
    \draw [decorate,decoration={brace,amplitude=4pt},xshift=0pt,yshift=0pt] (\afterthird,\thirdlevel) -- (\beforefirst,\thirdlevel) node [black,midway,yshift=-0.4cm] {$\boundedqusedle[2]$};
    \draw [decorate,decoration={brace,amplitude=4pt},xshift=0pt,yshift=0pt] (\afterfourth,\thirdlevel) -- (\beforefourth,\thirdlevel) node [black,midway,yshift=-0.4cm] {$\boundedqusedge[3]$};
    \draw [decorate,decoration={brace,amplitude=4pt},xshift=0pt,yshift=0pt] (\afterfourth,\fourthlevel) -- (\beforefirst,\fourthlevel) node [black,midway,yshift=-0.4cm] {$\boundedqusedle[+\infty] = \boundedqusedge[0]$};
  \end{tikzpicture}
\end{marginfigure}

As a consequence of \reflemma{qualitative-quantitative-equivalence}, we obtain an equivalence between the (un)used property (\cf{} \refdef*{unused}) and the $\defbound$-bounded impact property (\cf{} \refdef*{bounded}) instantiated with the $\qusedname$ quantifier:
\begin{align*}
  \unused \spacearound{=} \BOUNDEDQUSEDLE[0] && \text{if } \comparison = \le\\
  \neg\unused \spacearound{=} \BOUNDEDQUSEDGE[1] && \text{if } \comparison = \ge
\end{align*}
meaning that if a program semantics does not utilize the variables $\definputvariables$, then it has an impact of \emph{at most} $0$ (and vice versa).
Conversely, if a program semantics utilizes the variables $\definputvariables$, then it has an impact of \emph{at least} $1$ (and vice versa).
\reffig{unused-bounded} depicts an overview of the relationship between the unused property $\unused$ and the various $\defbound$-bounded impact properties $\boundedqused$ instantiated with different thresholds and comparison operators.


In the rest of this chapter, we present the static analysis to verify the $\defbound$-bounded impact property of a program and show the abstract version of the quantifiers introduced in this chapter.

\section{Abstract Quantitative Input Data Usage}
\labsec{abstract-quantitative-input-data-usage}

In this section, we introduce a sound and computable static analysis to verify the $\defbound$-bounded impact property of a program.
The soundness of the approach leverages two elements: (1) a backward abstract semantics, and (2) sound and computable implementations of the quantifiers \outcomesname{}, \rangename{}, and \qusedname{} (written as $\abstractoutcomesname$, $\abstractrangename$, and $\abstractqusedname$ respectively).


To quantify the usage of the input variables $\definputvariables$, we mostly need to determine the input configurations leading to specific output values.
As our impact quantifiers measure over the reachable output values (more precisely, over their abstraction by $\outputobs$) our underlying abstract semantics will be a \emph{backward} (co-)reachability semantics starting from \emph{disjoint} abstract post-conditions, over-approximating the (concrete) output values of the dependency semantics.
Specifically, we abstract the concrete output values with an indexed set $\buckets\in\vectorbuckets$ of $n$ disjoint \textit{output buckets}, where $\abstractdomainlattice$ is an abstract state domain with concretization function  $\abstractdomainconcretization\in\abstractdomain\to\setof\stateandbottom$. The choice of these output buckets is a parameter of the analysis and a \emph{good} choice is essential for obtaining a precise and meaningful analysis result.

For each output bucket $\bucket\in\abstractdomain$, where $1 \le j \le n$, our analysis computes an over-approximation of the dependency semantics restricted to the input configurations leading to $\abstractdomainconcretization(\bucket)$.
% More formally, the reduction of the output-abstraction semantics $\outputsemanticsnoparam$ to the dependencies with abstraction of a final state in $X$ is defined as:
% \[
%   \reduce[\outputsemanticsnoparam]{X} \DefeQ \setdef{
%     \setdef{
%       \tuple{\retrieveinput\defstate}{\outputobs(\retrieveoutput{\defstate})}\in\defsetofdependencies
%     }{
%       \retrieveoutput{\defstate} \in X
%       }
%   }{
%     \defsetofdependencies\in \outputsemanticsnoparam
%   }
% \]
%
Our static analysis is parametrized by an underlying backward abstract family\sidenote[][*-2]{A family of semantics is a set of program semantics parametrized by an initialization.}
of semantics $\backwardsemanticsnoparam\in\backwardtype$ which computes the backward semantics $\backwardsemanticsnoparam(\bucket)$ from a given output bucket $\bucket\in\abstractdomain$.
For instance, we could use the backward semantics of \refdef*{backward-semantics} to compute $\backwardsemanticsnoparam(\bucket)$.
The concretization function $\backwardconcretization\in(\backwardtype)\to\abstractdomain\to\outputtype$ employs the concretization of the abstract domain, \cf{} $\abstractdomainconcretization$, to restore all possible dependencies from input states to output values, formally:
\[
  \backwardconcretization(\backwardsemanticsnoparam)\bucket \DefeQ
  \setof{\setdef{
    \inputoutputtuple{\defstate}
  }{
    \retrieveinput{\defstate}\in\abstractdomainconcretization(\backwardsemanticsnoparam(\bucket))\land\retrieveoutput{\defstate}\in\abstractdomainconcretization(\bucket)
    }
  }
  \]
In words, $\backwardconcretization$ collects all the input-output dependencies from the application of the backward semantics to each output bucket $\bucket$.
Since the abstract semantics may discover more dependencies than the concrete ones, the concretization returns all combination of dependencies.
In such a way, we ensure that the (restriction of the) output-abstraction semantics is a subset of the abstract one.
%
We can thus define the soundness condition for the backward semantics with respect to the reduction of the output-abstraction semantics.

\siderefbox{def}{output-reduction}

\begin{definition}[Sound Over-Approximation for \texorpdfstring{$\backwardsemanticsnoparam$}{the backward semantics}]\labdef{sound-over-approximation}
  For all programs $\defprogram$, and output bucket $\bucket\in\abstractdomain$, the family of semantics $\backwardsemanticsnoparam$ is a \textup{sound over-approximation} of the output-abstraction semantics $\outputsemanticsnoparam$ reduced with  $\abstractdomainconcretization(\bucket)$, when it holds that:
  \[\reducedoutputsemantics \SubseteQ \backwardconcretization(\backwardsemantics)\bucket\]
\end{definition}

We define
$\multisemanticsnoparam\in\multitype$ as the backward semantics repeated on a set of output buckets $\buckets\in\vectorbuckets$ as follows:

\begin{definition}[Multi-Bucket Semantics \texorpdfstring{$\multisemanticsnoparam$}{}]\labdef{multi-semantics}
  We define
$\multisemanticsnoparam\in\multitype$ as the backward semantics repeated on a set of output buckets $\buckets\in\vectorbuckets$, that is:
\begin{align*}
\multisemantics\buckets \DefeQ (\backwardsemantics\bucket)_{j\le n}
\end{align*}
\end{definition}


The concretization function $\multiconcretization$ maps the multi-bucket semantics $\multisemanticsnoparam$ from a given set of output buckets $\buckets$ to a set of sets of dependencies in $\outputtype$.
Specifically, for each output bucket $\bucket\in\buckets$, $\multiconcretization$ concretizes the output from $\abstractdomainconcretization(\bucket)$ and the input states from the application of the backward semantics to $\bucket$, \ie, $\abstractdomainconcretization((\multisemantics\buckets)_j)$.
Hence, all the possible input-output dependencies are collected for all buckets.
Lastly, to account for the fact that the abstract semantics may discover more dependencies than the concrete ones, the concretization returns all combinations of dependencies.
\begin{definition}[Multi-Bucket Concretization \texorpdfstring{$\multiconcretization$}{}]\labdef{multi-concretization}
  We define the concretization function $\multiconcretization\in(\multitype)\to\vectorbuckets\to\outputtype$ as:
\begin{gather*}
  \multiconcretization(\multisemantics)\buckets \DefeQ
  \bigsetjoin_{j\le n} \backwardconcretization(\backwardsemantics)\bucket
\end{gather*}
\end{definition}

Next, we show the soundness of the multi-bucket semantics with respect to the output-abstraction semantics.

\begin{lemma}[Sound Over-Approximation for \texorpdfstring{$\multisemanticsnoparam$}{the Multi-Bucket Semantics}]\lablemma{sound-over-approximation-multi-bucket}
  For all programs $\defprogram$, output buckets $\buckets\in\vectorbuckets$, and a family of semantics $\backwardsemanticsnoparam$, the %multi-bucket family of
  semantics $\multisemanticsnoparam$ is a \textup{sound over-approximation} of the output-abstraction semantics $\outputsemanticsnoparam$ when reduced to $\bigsetjoin_{j\le n}\abstractdomainconcretization(\bucket)$:
  \[\reduce{\bigsetjoin\limits_{j\le n}\abstractdomainconcretization(\bucket)} \SubseteQ \multiconcretization(\multisemantics)\buckets\]
\end{lemma}
\begin{proof}
  From \refdef{multi-semantics} we have that $\multiconcretization(\multisemanticsnoparam)\buckets = \bigsetjoin_{j\le n}\backwardconcretization(\backwardsemanticsnoparam(\bucket))$.
  Then, from \refdef{sound-over-approximation}, we obtain that $\foralldef{j \le n}{\reducedoutputsemanticsnoparam \subseteq \backwardconcretization(\backwardsemanticsnoparam(\bucket))}$.
  Thus, by monotonicity of the union operator over set inclusion, it holds that $\bigsetjoin_{j\le n}\reducedoutputsemanticsnoparam \subseteq \bigsetjoin_{j\le n}\backwardconcretization(\backwardsemanticsnoparam(\bucket))$. We conclude by:
  \begin{align*}
    \bigsetjoin_{j\le n}\reducedoutputsemanticsnoparam &= \bigsetjoin_{j\le n} \setdef{\inputoutputtuple{\defstate}\in\outputsemanticsnoparam}{\retrieveoutput{\defstate}\in\abstractdomainconcretization(\bucket)}
      && \text{by $\reducenoparam{X}$} \\
    &= \setdef{\inputoutputtuple{\defstate}\in\outputsemanticsnoparam}{\retrieveoutput{\defstate}\in\bigsetjoin_{j \le n}\abstractdomainconcretization(\bucket)}
      && \text{by set definition} \\
    &= \reducenoparam{\bigsetjoin\limits_{j \le n}\abstractdomainconcretization(\bucket)}
      && \text{by $\reducenoparam{X}$}
  \end{align*}
  \end{proof}

We introduce the concept of \emph{covering} for output buckets to ensure that no potential final state is missed from the analysis.

\begin{definition}[Covering]\label{def:covering}
  Let $\outputobs\in\stateandbottom\to\stateandbottom$ be the output observer, we say that the output buckets $\buckets\in\vectorbuckets$ \textit{cover} the subset of potential outcomes whenever it holds that:
  \[
    \setdef{\outputobs(\retrieveoutput\defstate)}{\retrieveoutput{\defstate} \in \stateandbottom} \SubseteQ \setdef{\outputobs(\retrieveoutput\defstate)}{\retrieveoutput\defstate\in\bigsetjoin_{j\le n}\abstractdomainconcretization(\bucket)}
  \]
\end{definition}


Whenever the output buckets \textit{cover} the subset of outcomes abstracted by $\outputobs$, $\multisemanticsnoparam$ is a sound over-approximation of $\outputsemanticsnoparam$.

\begin{lemma}\lablemma{sound-approximation-covering}
  Let $\buckets\in\vectorbuckets$ be the set of output buckets such that it covers the subset of potential outcomes of a program $\defprogram$.Then, it holds that the semantics $\multisemanticsnoparam$ is a \textup{sound over-approximation} of the output-abstraction semantics $\outputsemanticsnoparam$:
  \[
    \outputsemantics \SubseteQ \multiconcretization(\multisemantics)\buckets
  \]
\end{lemma}
\begin{proof}
  Follows directly from \reflemma{sound-over-approximation-multi-bucket} and \refdef{covering}. The covering of the output buckets ensures that no potential final state is missed and thus let the reduction of the output-abstraction semantics unnecessary, as it holds that $\outputsemantics = \reduce{\bigsetjoin_{j\le n}\abstractdomainconcretization(\bucket)}$.
\end{proof}



% Additionally, the output buckets needs to be compatible with the concrete outcomes of the program, meaning that two output states cannot map to the same output abstraction if they belong to different output buckets.

% \begin{definition}[Compatibility]\labdef{compatibility}
%   Given the output buckets $\buckets\in\vectorbuckets$ and the output observer $\outputobs\in\stateandbottom\to\valuesinf$, we say that $\buckets$ is \textup{compatible} with $\outputobs$, whenever it holds:
%   \[ \foralldef{\defstate_j \in \abstractdomainconcretization(\bucket), \defstate_p \in \abstractdomainconcretization(\bucket[p])}{\outputobs(\defstate_j) \neq \outputobs(\defstate_p)} \ImplieS \bucket \neq \bucket[p] \]
% \end{definition}

So far, we presented the semantics $\multisemanticsnoparam$ as an abstraction of the output-abstraction semantics $\outputsemanticsnoparam$ via \reflemma{sound-approximation-covering}.
Next, this section shows the concept of sound implementation of the impact quantifiers.
We expect a sound implementation $\impactinstance\in\pair\vectorbuckets\vectorbuckets\to\valuesinf$ to return a bound on the impact which is always higher (or always lower depending on the ordering operator $\comparison$ of $\genbounded$) than the concrete counterpart $\impactwrapper$.

\begin{definition}[Sound Implementation]\labdef{sound-implementation}
  For all output buckets $\buckets$ and family of semantics $\multisemanticsnoparam$,
  whenever,
  \begin{enumerate}[label=(\roman*)]
    \item $\backwardsemanticsnoparam$ is sound with respect to $\outputsemanticsnoparam$, \cf{} \refdef*{sound-over-approximation}[*-12], and
    \item $\buckets$ covers the subset of potential outcomes, \cf{} \refdef*{covering}[*-8],
  \end{enumerate}
  we say that $\impactinstance$ is a \textup{sound implementation} of $\impactwrapper$ if and only if:
  \begin{eqnarray*}
  \impactwrapper(
    \tracesemantics
    ) \ComparisoN \impactinstance(\multisemantics\buckets, \buckets)
  \end{eqnarray*}
\end{definition}

\refdef{sound-implementation} ensures that the soundness of the static analysis is preserved when computing the quantity $\defbound$.
We define the concretization of the bound measured by the abstract impact $\impactinstance(\multisemantics\buckets, \buckets)$ via $\boundconcretization \in \valuesinf \to \collectingtype$.
The idea behind the concretization $\boundconcretization$ is to filter the set of sets of traces concretized by $\multiconcretization$ that yield an impact lower than the bound computed in the abstract, \cf{} $\impactinstance(\multisemantics\buckets, \buckets)$.
Before defining the concretization, we show an example where not all subsets of $\multiconcretization$ satisfy the property $\genbounded$.
% \marginnote[*7]{The operator $\singletonelimination \in \setof A \to \setof A$ removes the set around a singleton set, \ie, $\singletonelimination(\{a\}) = a$.}

\marginnote{
  \begin{tikzpicture}[scale=1]
    \node (linit) at (-0.9,2) {$\locinit:$};
    \node (lend) at (-0.9,0) {$\loc{3}:$};
    % Define the positions of the nodes
    \node (TT) at (0,2) {${\tuple\true\true}$};
    \node (TF) at (1,2) {${\tuple\true\false}$};
    \node (FT) at (2,2) {${\tuple\false\true}$};
    \node (FF) at (3,2) {${\tuple\false\false}$};
    \node (T) at (0.5,0) {${\langle \true \rangle}$};
    \node (F) at (2.5,0) {${\langle \false \rangle}$};
    % Draw the edges
    \draw[->] (TT) -> (T);
    \draw[->] (TF) -> (F);
    \draw[->] (FT) -> (T);
    \draw[->] (FF) -> (F);
    \draw[decorate,decoration={brace,amplitude=8pt,mirror,raise=4pt},yshift=0pt,xshift=-0.1pt] (-1.1,2.2) -- (-1.1,-0.2) node [black,midway,xshift=-0.7cm,rotate=90] {$\tracesemanticsnoparam\semanticsof{\texttt{id2}}$};
\end{tikzpicture}
}
\begin{example}
  We assume we are interested in the validation of the property $\boundedqusedge$, where the impact quantifier is $\qusedname$ and the comparison operator for quantity bounds is $\ge$.
  We consider the trace semantics of the program \refprog{id2}, \ie, $\tracesemanticsnoparam\semanticsof{\texttt{id2}}$, depicted on the side.
  Consider an abstract semantics $\multisemanticsnoparam\semanticsof{\texttt{id2}}$ that concretize to the following sets of dependencies:
  \begin{align*}
    \multiconcretization(\multisemanticsnoparam) =
    \setof*{\left\{
    \begin{array}{l}
      \tuple{\tracetuple{\locinit}{\langle \true, \true\rangle}}{\tracetuple{\loc{3}}{\langle \true\rangle}}, \\
      \tuple{\tracetuple{\locinit}{\langle \true, \false\rangle}}{\tracetuple{\loc{3}}{\langle \false\rangle}}, \\
      \tuple{\tracetuple{\locinit}{\langle \false, \true\rangle}}{\tracetuple{\loc{3}}{\langle \anyvalue\rangle}}, \\
      \tuple{\tracetuple{\locinit}{\langle \false, \false\rangle}}{\tracetuple{\loc{3}}{\langle \false\rangle}}
    \end{array}
    \right\}}
  \end{align*}
  As noted in \refexample{qused-not-monotonic}, the $\qusedname_{\{\texttt{x}\}}$ applied to the biggest set of the concretized dependencies above is 1.
  We assume the abstract impact returns the bound 1.
  In this case, we filter from $\multiconcretization(\multisemantics)$ the set of dependencies that yield an impact lower than the bound 1 as the validation of the property $\boundedqused$ employs the $\ge$ operator.
  For instance, the following set of dependencies is removed:
  \begin{align*}
    \left\{
    \begin{array}{l}
      \tuple{\tracetuple{\locinit}{\langle \true, \true\rangle}}{\tracetuple{\loc{3}}{\langle \true\rangle}}, \\
      \tuple{\tracetuple{\locinit}{\langle \true, \false\rangle}}{\tracetuple{\loc{3}}{\langle \false\rangle}}, \\
      \tuple{\tracetuple{\locinit}{\langle \false, \true\rangle}}{\tracetuple{\loc{3}}{\langle \true\rangle}}, \\
      \tuple{\tracetuple{\locinit}{\langle \false, \false\rangle}}{\tracetuple{\loc{3}}{\langle \false\rangle}}
    \end{array}
    \right\}
  \end{align*}
  as $\qusedname_{\{\texttt{x}\}}$ is 0 in such case (\cf{} \refexample{qused-not-monotonic}).
\end{example}

\begin{definition}[Concretization of the Measured Quantity]\labdef{concretization-quantitative}
  The concretization of a measured quantity $\defbound$ is $\boundconcretization\in\valuesinf\to\collectingtype$, defined as:
  \begin{align*}
    &\boundconcretization(\impactinstance(\multisemantics\buckets, \buckets)) \DefeQ \\
    & \quad
    \setdef{\defsetoftraces \in \dependencyconcretization \circ \outputconcretization \circ \multiconcretization(\multisemantics)\buckets}{
      \impactwrapper(\defsetoftraces) \comparison \impactinstance(\multisemantics\buckets, \buckets)
    }
  \end{align*}
\end{definition}

% Note that, it holds an equivalence with the $\defbound$-bounded impact property:
% \[
%   \boundconcretization(\defbound) = \genbounded
% \]

% Additionally, we note that the concretization $\boundconcretization$ is monotonically increasing, respectively decreasing, depending on comparison operator $\comparison$.

% \begin{lemma}[Monotonicity of $\boundconcretization$]\lablemma{monotonicity-bound-concretization}
%   Depending on the comparison operator $\comparison$, for all quantities $\defbound',\defbound\in\valuesposplus$ measured by the abstract impact $\impactinstance$ it holds that:
%   \begin{enumerate}[label=(\roman*)]
%     \item $\defbound' \le \defbound \ImplieS \leconcretization(\defbound') \subseteq \leconcretization(\defbound)$
%     \item $\defbound' \ge \defbound \ImplieS \geconcretization(\defbound') \supseteq \geconcretization(\defbound)$
%   \end{enumerate}
% \end{lemma}
% \begin{proof}
%   The proof follows directly from the definition of the concretization $\boundconcretization$. By increasing the bound $\defbound'$, the set of traces in $\leconcretization(\defbound')$ can only increase as we weaken the condition. On the other hand, $\geconcretization(\defbound')$ can only decrease as we strengthen the condition.
% \end{proof}

The concretization $\boundconcretization$ maintains the set of traces concretized, in turns, by $\multiconcretization$, $\outputconcretization$ (\cf{} \refdef*{concretization-dependency-semantics}), and $\dependencyconcretization$ (\cf{} \refdef*{concretization-output-abstraction-semantics}) that yield an impact lower than the bound computed in the abstract, \cf, $\impactinstance(\multisemantics\buckets, \buckets)$.

\begin{remark}
  Based on the impact quantifier $\impactwrappername$, the $\defbound$-bounded impact property $\genbounded$ can be subset-closed, meaning that whenever a set of traces satisfies the property, all its subsets also satisfy the property.
In this case, it would hold that
\begin{align*}
  \multiconcretization(\multisemantics)\buckets
  \SubseteQ \boundconcretization(\impactinstance(\multisemantics\buckets, \buckets))
\end{align*}
However, this is not always the case.
Indeed, some subsets of traces may not satisfy the property $\genbounded$.
\end{remark}


The concretization $\boundconcretization$ is used to prove that the abstract semantics $\multisemanticsnoparam$ is sound for proving the property $\genbounded$.


\siderefbox{def}{concretization-quantitative}
\siderefbox{def}{bounded}
\begin{lemma}\lablemma{soundness-boundconcretization}
  A program $\defprogram$ has an impact of at most $\defbound$ (respectively, at least $\defbound$ depending on the comparison operator $\comparison$) only if the set of dependencies concretized from the quantity $\impactwrapper(\tracesemantics)$
  are a subset of the semantics in $\genbounded$.
  \begin{align*}
    \boundconcretization(
      \impactwrapper(\tracesemantics)
    ) \SubseteQ \GENBOUNDED  \ImplieS \defprogram \satisfies \GENBOUNDED
  \end{align*}
\end{lemma}
\begin{proof}
  By hypothesis, we assume that $\dependencyconcretization \circ \outputconcretization \circ \boundconcretization(
    \defbound'
  ) \subseteq \genbounded$ where $\defbound' = \impactwrapper(\tracesemantics)$.
  We have that $\outputsemantics \subseteq \boundconcretization(\defbound')$ by definition of $\boundconcretization$, \cf{} \refdef{concretization-quantitative}.
  Then, by monotonicity of $\outputconcretization$ and $\dependencyconcretization$, \cf{} \refdef{right-left-adjoints-for-the-output-abstraction-semantics} and \refdef{right-left-adjoints-for-the-dependency-semantics} respectively, we have that $\dependencyconcretization(\outputconcretization(\outputsemantics)) \subseteq \dependencyconcretization(\boundconcretization(\defbound'))$.
  Thus, since $\dependencyconcretization \circ \outputconcretization \circ \boundconcretization(
    \defbound'
  ) \subseteq \genbounded$, we have that $\dependencyconcretization(\outputconcretization(\outputsemantics)) \subseteq \genbounded$.
  Depending on the instantiation of the impact quantifier, we conclude the proof by \reflemma{outcomes-validation} for the $\outcomesname$ impact quantifier, \reflemma{range-validation} for the $\rangename$ impact quantifier, and \reflemma{qused-validation} for the $\qusedname$ impact quantifier.
\end{proof}

Finally, the next result shows that our static analysis is sound when employed to verify the property of interest $\genbounded$ for the program $\defprogram$.
That is, if %the computation of
$\impactinstance$ returns the bound $\defbound'$, and $\defbound'\comparison\defbound$, then the program $\defprogram$ satisfies the property $\genbounded$, \cf{} $\defprogram \satisfies \genbounded$.

\begin{theorem}[Soundness] \labthm{soundness}
  Let $\genbounded$ be the property of interest we want to verify for the program $\defprogram$ and the input variable $\definputvariables\in\setof\inputvariables$.
  Whenever,
  \begin{enumerate}[label=(\roman*)]
    \item \label{p:first} $\backwardsemanticsnoparam$ is sound with respect to $\outputsemanticsnoparam$, \cf{} \refdef*{sound-over-approximation}[*-5], and
    \item \label{p:second} $\buckets$ covers the subset of potential outcomes, \cf{} \refdef*{covering}[*-1.5], and
    \item \label{p:third} $\impactinstance$ is a sound implementation of $\impactwrapper$, \cf{} \refdef*{sound-implementation}[*3],
\end{enumerate}
  the following implication holds:
  \begin{align*}
    \impactinstance(\multisemantics\buckets, \buckets) = \defbound' \land \defbound' \comparison \defbound \ImplieS \defprogram \satisfies \GENBOUNDED
  \end{align*}
\end{theorem}
\begin{proof}
  From \ref{p:first}, \ref{p:second}, and \ref{p:third}, we have that:
  \begin{align*}
    \impactwrapper(\tracesemantics) \ComparisoN \impactinstance(\multisemantics\buckets, \buckets) \ComparisoN \defbound'
  \end{align*}
  From the hypothesis $\defbound' \comparison \defbound$, we obtain $\impactwrapper(\tracesemantics)\comparison \defbound$.
  By \refdef{concretization-quantitative}, we have that $\boundconcretization(\impactwrapper(\tracesemantics)) \subseteq \genbounded$.
  We conclude by \reflemma{soundness-boundconcretization}.
\end{proof}


Next,
we define $\abstractrangename$, $\abstractoutcomesname$, and $\abstractqusedname$
as possible implementations for $\rangename$, $\outcomesname$, and $\qusedname$, respectively.

%
We assume the underlying abstract state domain $\abstractdomain$ is equipped with an
operator $\abstractdomainproject\in\abstractdomain\to\abstractdomain$
to project away the input variables $\definputvariables$, that is, the reduction in dimensionality of the abstract element by projecting its image to the reduced variable space.
\begin{example}
In the context of the interval domain, where each input variable is related to a possibly unbounded lower and upper bound, $\abstractdomainprojecti(\langle\definputvariable \mapsto [1, 3], j \mapsto [2, 4]\rangle) = \langle \definputvariable \mapsto [-\infty, +\infty], j \mapsto [2, 4] \rangle$
  removes the constraints related to the variables in $\{\definputvariable\}$.
  Often, whenever a variable maps to $\abstractdomaintop$, or equivalently $[-\infty, +\infty]$, we hide it from the abstract element, \eg, instead of $\langle \definputvariable \mapsto [-\infty, +\infty], j \mapsto [2, 4] \rangle$ we often write $\langle j \mapsto [2, 4] \rangle$.
\end{example}
%
% We define the soundness condition on the project operator to ensure that $\abstractdomainproject(\defstate^\natural)$ represents all the concrete states result of perturbations on the variable $\definputvariables$ from a state represented by an abstract value $\defstate^\natural$.
We define the condition a sound projection operator $\abstractdomainproject$ must satisfy to ensure that no concrete state is missed by the projection.

The projection operator is sound whenever the projection of the variables $\definputvariables$ from the abstract value $\defstate^\natural$ contains all the concrete states that can be obtained by perturbing the variables $\definputvariables$ from a concrete state represented by $\defstate^\natural$.
In fact, we could define the \emph{concrete} project operator $\project\in\setof\state\to\setof\state$ as it follows:
\[
  \project(\defsetofstates) \DefeQ \setdef{\defstate'}{\defstate\in\defsetofstates \land \defstate \stateeq{\definputvariables} \defstate'}
\]

It follows the classic definition of soundness.
\begin{definition}[Sound Projection Operator]\labdef{soundness-project}
  For all abstract states $\defstate^\natural$, the projection of the abstract state $\abstractdomainproject(\defstate^\natural)$ is sound whenever it holds that:
  \[
    \project(\abstractdomainconcretization(\defstate^\natural)) \subseteq \abstractdomainproject(\defstate^\natural)
    \]
\end{definition}

\subsection{Abstract Implementation \texorpdfstring{$\abstractoutcomes$}{Abstract Outcomes}}[Abstract \texorpdfstring{$\abstractoutcomes$}{Outcomes}]
\labsec{abstract-outcomes}


The definition of $\abstractoutcomes$ first projects away the input variables $\definputvariables$ from all the given abstract values, then it collects all possible abstract values resulting from the meet operation over any two resulting abstract domain elements.
These intersections represent concrete input configurations where variations on the values of $\definputvariables$ \emph{may} lead to changes of program outcome, from a bucket to another.
We return the maximum number of abstract values that intersect after projections:
\begin{definition}[\texorpdfstring{$\abstractoutcomes$}{Abstract Outcomes}]\labdef{abstract-outcomes}
  We define $\abstractoutcomes\in\pair\vectorbuckets\vectorbuckets\to\valuesinf$ as:
  \begin{equation*}
  \abstractoutcomes(\presfrombuckets, \buckets) \DefeQ \max~\setdef{\cardinalitynospaces{J}}{J \in \intersectallfunction((\abstractdomainproject(\prefrombucket))_{j\le\numberofbuckets})}
  \end{equation*}
  where $\presfrombuckets\in\vectorbuckets$ is the result of the abstract semantics $\multisemanticsnoparam$.
\end{definition}
Note the use of $\max$ instead of $\sup$ as in the concrete counterpart (\refdef*{outcomes}[*-5]) since the number of intersecting abstract values is finite and bounded by $n$, the number of output buckets.
The function $\intersectallfunction$ takes as input an indexed set of abstract values and returns the set of indices of abstract values that intersect together, defined as follows:

\begin{definition}[\texorpdfstring{$\intersectallfunction$}{Intersect Function}]\labdef{intersect-all-function}
  Given a vector of $\numberofbuckets$ abstract elements $\presfrombuckets\in\vectorbuckets$, we define $\intersectallfunction\in\vectorbuckets\to\setof\N$ as:
  \begin{eqnarray*}
    \lefteqn{\intersectallfunction(\presfrombuckets) \DefeQ} \\
    & \setdef{J \subseteq \N}{\forall j\le n, p\le n.~ j\in J \land p\in J \LanD \prefrombucket \abstractdomainmeet \prefrombucket[p]}
  \end{eqnarray*}
\end{definition}


Finding all the indices of intersecting abstract values is equivalent to find cliques in a graph, where each node represents an abstract value and an edge exists between two nodes if and only if the corresponding abstract values intersect.
Therefore, $\intersectallfunction$ can be efficiently implemented based on the graph algorithm by~\sidetextcite{Bron1973}.
All together, the abstract implementation $\abstractoutcomes$ returns the value of the maximum among all the sizes of the maximum cliques for each projection.
%

\begin{example}\labexample{abstract-outcomes}
  Consider \refprog*{landing-alarm-system}[*-3] with program states defined as $\state=\setdef{\langle a, b, c, d \rangle}{a\in\{-4,1\}\land b\in\{1,2,3\}\land c\in\N \land d\in\Nle{3}}$. The output buckets represent the four levels of landing risk: low, medium-low, medium-high, and high. Using the interval domain: $\buckets \in \vectorbuckets[4]$ such that $\bucket[\textup{low}] = \langle \texttt{risk} \mapsto 0 \rangle$, $\bucket[\textup{mid\-/low}] = \langle \texttt{risk} \mapsto 1 \rangle$, $\bucket[\textup{mid\-/high}] = \langle \texttt{risk} \mapsto 2 \rangle$, and $\bucket[\textup{high}] = \langle \texttt{risk} \mapsto 3 \rangle$.
  \nrefdef{covering} is satisfied as the output buckets cover all the possible outcomes of the program for the variable \texttt{risk}.
  We assume the backward analysis to return the following intervals:
  \begin{align*}
    \backwardsemanticsnoparam(\bucket[\textup{low}]) \spacearound{&=} \prefrombucket[\textup{low}] \spacearound{=} \langle \texttt{angle} \mapsto 1, \texttt{speed} \mapsto [1, 3] \rangle \\
    \backwardsemanticsnoparam(\bucket[\textup{mid\-/low}]) \spacearound{&=} \prefrombucket[\textup{mid\-/low}] \spacearound{=} \langle \texttt{angle} \mapsto 1, \texttt{speed} \mapsto [2, 3] \rangle \\
    \backwardsemanticsnoparam(\bucket[\textup{mid\-/high}]) \spacearound{&=} \prefrombucket[\textup{mid\-/high}] \spacearound{=} \langle \texttt{angle} \mapsto 1, \texttt{speed} \mapsto [2, 3] \rangle \\
    \backwardsemanticsnoparam(\bucket[\textup{high}]) \spacearound{&=} \prefrombucket[\textup{high}] \spacearound{=} \langle \texttt{angle} \mapsto -4, \texttt{speed} \mapsto [1, 3] \rangle \\
  \end{align*}
  Indeed, the backward analysis over-approximates the concrete semantics as, for example, the input configuration $\texttt{angle} = 1$ and $\texttt{speed} = 3$ does not lead to a low risk of approach.
  We are interested in the input variable \texttt{angle}, thus the projections are:
  \begin{align*}
    \abstractdomainproject[{\{\texttt{angle}\}}](\prefrombucket[\textup{low}]) \spacearound{&=} \langle \texttt{speed} \mapsto [1, 3] \rangle \\
    \abstractdomainproject[{\{\texttt{angle}\}}](\prefrombucket[\textup{mid\-/low}]) \spacearound{&=} \langle \texttt{speed} \mapsto [2, 3] \rangle \\
    \abstractdomainproject[{\{\texttt{angle}\}}](\prefrombucket[\textup{mid\-/high}]) \spacearound{&=} \langle \texttt{speed} \mapsto [2, 3] \rangle \\
    \abstractdomainproject[{\{\texttt{angle}\}}](\prefrombucket[\textup{high}]) \spacearound{&=} \langle \texttt{speed} \mapsto [1, 3] \rangle \\
  \end{align*}
  It is easy to notice that the abstract elements resulting from the projection intersect between each other. Therefore, we obtain that the set of indices of buckets that intersect is:
  \begin{eqnarray*}
  \lefteqn{\intersectallfunction((\abstractdomainproject[{\{\texttt{angle}\}}](\prefrombucket[\textup{low}]), \dots, \abstractdomainproject[{\{\texttt{angle}\}}](\prefrombucket[\textup{high}]))) ={}} \\
& \setof{\{\textup{low}, \textup{mid\-/low}, \textup{mid\-/high}, \textup{high}\}}
\end{eqnarray*}
  The maximum size is $4$ at $\cardinalitynospaces{\{\textup{low}, \textup{mid\-/low}, \textup{mid\-/high}, \textup{high}\}}$.
  Therefore, $\abstractoutcomesname_{\{\texttt{angle}\}}(\presfrombuckets, \buckets) = 4$, meaning that variations of the input variable \texttt{angle} may lead to changes of program outcome from four different buckets.
  On the other hand, projecting away the variable \texttt{speed} from the abstract intervals results in:
  \begin{align*}
    \abstractdomainproject[{\{\texttt{speed}\}}](\prefrombucket[\textup{low}]) \spacearound{&=} \langle \texttt{angle} \mapsto 1 \rangle \\
    \abstractdomainproject[{\{\texttt{speed}\}}](\prefrombucket[\textup{mid\-/low}]) \spacearound{&=} \langle \texttt{angle} \mapsto 1 \rangle \\
    \abstractdomainproject[{\{\texttt{speed}\}}](\prefrombucket[\textup{mid\-/high}]) \spacearound{&=} \langle \texttt{angle} \mapsto 1 \rangle \\
    \abstractdomainproject[{\{\texttt{speed}\}}](\prefrombucket[\textup{high}]) \spacearound{&=} \langle \texttt{angle} \mapsto -4 \rangle \\
  \end{align*}
  In this case, only a few abstract elements intersect, specifically:
  \begin{eqnarray*}
    \lefteqn{\intersectallfunction((\abstractdomainproject[{\{\texttt{speed}\}}](\prefrombucket[\textup{low}]), \dots, \abstractdomainproject[{\{\texttt{speed}\}}](\prefrombucket[\textup{high}]))) ={}} \\
  & \left\{
    \begin{array}{l}
    \{\textup{low}\}, \{\textup{mid\-/low}\}, \{\textup{mid\-/high}\}, \{\textup{high}\}, \\
    \{\textup{low}, \textup{mid\-/low}\}, \{\textup{low}, \textup{mid\-/high}\},
    \{\textup{mid\-/low}, \textup{mid\-/high}\}\\
    \{\textup{low}, \textup{mid\-/low}, \textup{mid\-/high}\}
    \end{array}
    \right\}
  \end{eqnarray*}
  The maximum clique is of size $3$, \cf{} \cardinalitynospaces{\{\textup{low}, \textup{mid\-/low}, \textup{mid\-/high}\}}, hence $\abstractoutcomesname_{\{\texttt{speed}\}}(\presfrombuckets, \buckets) = 3$. This means that variations of the input variable \texttt{speed} may lead to changes of program outcome in three different buckets.
\end{example}

In order to prove that the abstract impact $\abstractoutcomes$ is a sound over-approximation of the concrete impact $\outcomes$, we require the output buckets $\buckets$ to be \textit{compatible} with the output observer $\outputobs$.
That is, any two output states $\retrieveoutput\defstate, \retrieveoutput\defstate'$ that produce different output readings, \ie, $\outputobs(\retrieveoutput\defstate) \neq \outputobs(\retrieveoutput\defstate')$, belong to different buckets, \ie, $\retrieveoutput\defstate \in \abstractdomainconcretization(\bucket) $, $ \retrieveoutput\defstate' \in \abstractdomainconcretization(\bucket[p]) $, and $ \bucket \neq \bucket[p]$.
Intuitively, compatibility ensures that the counting intersecting buckets in the abstract does not miss any concrete outcome.
%
\begin{definition}[Compatibility]\labdef{compatibility}
  Given the output buckets $\buckets\in\vectorbuckets$ and the output observer $\outputobs\in\stateandbottom\to\stateandbottom$, we say that $\buckets$ is \textup{compatible} with $\outputobs$, whenever it holds:
  \[ \foralldef{\defstate_j \in \abstractdomainconcretization(\bucket), \defstate_p \in \abstractdomainconcretization(\bucket[p])}{\outputobs(\defstate_j) \neq \outputobs(\defstate_p)} \ImplieS \bucket \neq \bucket[p] \]
\end{definition}
%
Note that, $\outcomes$ is bounded by the number of buckets when the conditions of covering and compatibility hold for the output buckets.
\begin{lemma}[$\outcomes$ Upper Bound]\lablemma{outcomes-upper-bound}
  When the $\numberofbuckets$ output buckets $\buckets\in\vectorbuckets$ are compatible, \cf{} \refdef{compatibility}, and cover the subset of potential outcomes, \cf{} \refdef*{covering}, it holds that
  $\outcomes(\tracesemanticsnoparam) \le n$.
\end{lemma}
\begin{proof}
  We notice that $\outcomes(\tracesemanticsnoparam) \le \cardinalitynospaces{\setdef{\outputobs(\retrieveoutput\defstate)}{\retrieveoutput\defstate\in\stateandbottom}}$ as the set of outputs for the trace semantics is always bigger than any set of outputs.
  It is easy to note that the cardinality of $\setdef{\outputobs(\retrieveoutput\defstate)}{\retrieveoutput\defstate\in\stateandbottom}$ is bounded from above by $n$ since any two output states $\retrieveoutput\defstate, \retrieveoutput\defstate'$ that produce different outputs, \ie, $\outputobs(\retrieveoutput\defstate) \neq \outputobs(\retrieveoutput\defstate')$, belong to different buckets, \ie, $\retrieveoutput\defstate \in \abstractdomainconcretization(\bucket) \land \retrieveoutput\defstate' \in \abstractdomainconcretization(\bucket[p]) \land \bucket \neq \bucket[p]$ (by compatibility, \cf{} \refdef{compatibility}).
  Where the existence of the two buckets is guaranteed by covering (\cf{} \refdef{covering}).
  Therefore, there are at most $n$ different outputs.
\end{proof}

The next result shows that the abstract impact $\abstractoutcomes$ is a sound over-approximation of the concrete impact $\outcomes$.

\refdef@{outcomes}[*-4]
\begin{lemma}[$\abstractoutcomes$ is a Sound Implementation of $\outcomes$]\lablemma{abstractoutcomes-is-sound}
  Let $\definputvariables\in\setof\inputvariables$ be the input variable of interest, $\abstractdomain$ the abstract domain, $\backwardsemanticsnoparam$ the family of semantics, and $\buckets\in\vectorbuckets$ the starting output buckets.
  Whenever the following conditions hold:
  \begin{enumerate}[label=(\roman*)]
    % \item \label{proof:a} $\backwardsemanticsnoparam$ is sound with respect to $\outputsemanticsnoparam$, \cf{} \refdef*{sound-over-approximation},
    \item \label{proof:b2} $\buckets$ covers the subset of potential outcomes, \cf{} \refdef*{covering}[*-3.5],
    \item \label{proof:b1} $\buckets$ is compatible with $\outputobs$, \cf{} \refdef*{compatibility}[*1], and
    \item \label{proof:d} $\abstractdomainproject[]$ is sound, \cf{} \refdef*{soundness-project}[*5];
  \end{enumerate}
  then, $\abstractoutcomes$ is a sound implementation of $\outcomes$.
\end{lemma}
\begin{proof}[Proof (Sketch)]
  We need to prove that
  \begin{gather*}
    \outcomes(\tracesemanticsnoparam) = \sup_{\definput\in\reducedstate}
    \cardinality{\setdef{\outputobs(\retrieveoutput{\deftrace})}{\deftrace\in\tracesemanticsnoparam\land\retrieveinput{\deftrace} \stateeq{\inputvariableswithoutw} \definput}} \\
    \LE \\
    \abstractoutcomes(\multiconcretization(\multisemanticsnoparam)\buckets, \buckets) = \\
    \max~\setdef{\cardinalitynospaces{J}}{J \in \intersectallfunction((\abstractdomainproject(\backwardconcretization(\backwardsemanticsnoparam)\bucket))_{j\le\numberofbuckets})}
  \end{gather*}
  Let $\overline{\definput}\in\reducedstate$ be such that:
  \begin{gather*}
    \sup_{\definput\in\reducedstate}
    \cardinality{\setdef{\outputobs(\retrieveoutput{\deftrace})}{\deftrace\in\tracesemanticsnoparam\land\retrieveinput{\deftrace} \stateeq{\inputvariableswithoutw} \definput}}
    \spacearound{=} \\
    \cardinality{\setdef{\outputobs(\retrieveoutput{\deftrace})}{\deftrace\in\tracesemanticsnoparam\land\retrieveinput{\deftrace} \stateeq{\inputvariableswithoutw} \overline{\definput}}}
  \end{gather*}
  Let $\overline{J} \in \intersectallfunction((\abstractdomainproject(\backwardconcretization(\backwardsemanticsnoparam)\bucket))_{j\le\numberofbuckets})$ be such that:
  \begin{gather*}
    \max~\setdef{\cardinalitynospaces{J}}{J \in \intersectallfunction((\abstractdomainproject(\backwardconcretization(\backwardsemanticsnoparam)\bucket))_{j\le\numberofbuckets})}
    \spacearound{=}
    \cardinalitynospaces{\overline{J}}
  \end{gather*}
  We need to show that $\cardinality{\setdef{\outputobs(\retrieveoutput{\deftrace})}{\deftrace\in\tracesemanticsnoparam\land\retrieveinput{\deftrace} \stateeq{\inputvariableswithoutw} \overline{\definput}}} \le \cardinalitynospaces{\overline{J}}$.
  By contradiction, we assume that $\cardinality{\setdef{\outputobs(\retrieveoutput{\deftrace})}{\deftrace\in\tracesemanticsnoparam\land\retrieveinput{\deftrace} \stateeq{\inputvariableswithoutw} \overline{\definput}}} > \cardinalitynospaces{\overline{J}}$, meaning that the number of distinct output abstraction from states in $\tracesemanticsnoparam$ from perturbation of the input $\retrieveinput\defstate$ is higher than the number of indices in the abstract. Hence, either:
  \begin{enumerate}[label=(\alph*)]
    \item \label{dsa1} there exist output abstractions that are not matched by any output bucket,
    \item \label{dsa2} some output abstractions are matched by the same output bucket, or
    \item \label{dsa3} there are output abstractions in $\tracesemanticsnoparam$ from perturbation of the input $\retrieveinput\defstate$ that are not considered in the abstract.
  \end{enumerate}
  By \ref{proof:b1}, the first case \ref{dsa1} is not possible as covering, \cf{} \refdef{covering}, ensures that all the output abstractions are matched by the output buckets.
  By \ref{proof:b2}, the second case \ref{dsa2} is not possible as compatibility, \cf{} \refdef{compatibility}, ensures that any two output abstractions that produce different outputs belong to different output buckets.
  The third case \ref{dsa3} is not possible by \ref{proof:d} as the soundness of the projection operator ensures that all the concrete states result of perturbations on the variable $\definputvariables$ from a state represented by an abstract value $\defstate^\natural$ are considered in the abstract.
  Therefore, the assumption is false, and we conclude that $\abstractoutcomes$ is a sound implementation of $\outcomes$.
\end{proof}

\begin{example}
  We show that the result of the abstract implementation $\abstractoutcomes$ in \refexample{abstract-outcomes} is a sound over-approximation of the concrete implementation $\outcomes$.
  \marginnote{
    \[
  \outputobs(\defstate) \DefeQ \left\{\begin{array}{l}
    \langle \top, \top, \top, d \rangle \\
    \qquad \text{if } \defstate = \langle a, b, c, d \rangle \\
    \langle \top, \top, \top, \top \rangle \\
    \qquad \text{otherwise}
  \end{array}\right.
  \]
  }
  We assume the output observer $\outputobs$ is defined as in \refexample{output-observer-landing}, where the fourth component of the state represents the value of the output variable \texttt{risk}.
  Indeed, the output buckets of \refexample{abstract-outcomes} are compatible with the output observer $\outputobs$ as any two output states are mapped to different output buckets.
  \marginnote{
    \begin{align*}
      \bucket[\textup{low}] \spacearound{&=} \langle \texttt{risk} \mapsto 0 \rangle \\
      \bucket[\textup{mid\-/low}] \spacearound{&=} \langle \texttt{risk} \mapsto 1 \rangle \\
      \bucket[\textup{mid\-/high}] \spacearound{&=} \langle \texttt{risk} \mapsto 2 \rangle \\
      \bucket[\textup{high}] \spacearound{&=} \langle \texttt{risk} \mapsto 3 \rangle
    \end{align*}}
  As expected by \refthm*{soundness}[*6], the bound computed by the abstract implementation $\abstractoutcomes$ is always greater or equal to the concrete one:
  \begin{align*}
    & \outcomesname_{\{\texttt{angle}\}}(\tracesemanticsnoparam\semanticsof{\landingprogram}) = 2 \\
    &\qquad\LE \abstractoutcomesname_{\{\texttt{angle}\}}(\presfrombuckets, \buckets) = 4
  \end{align*}
  \begin{align*}
    & \outcomesname_{\{\texttt{speed}\}}(\tracesemanticsnoparam\semanticsof{\landingprogram}) = 3 \\
    &\qquad\LE \abstractoutcomesname_{\{\texttt{speed}\}}(\presfrombuckets, \buckets) = 3 \\
  \end{align*}
  where $\landingprogram$ is the program of the landing alarm system, \cf{} \refprog{landing-alarm-system}.
\end{example}


\subsection{Abstract Implementation \texorpdfstring{$\abstractrange$}{Abstract Range}}[Abstract \texorpdfstring{$\abstractrange$}{Range}]
\labsec{abstract-range}

We define $\abstractrange$ as the maximum distance of the range of the extreme values of the buckets represented by intersecting abstract values after projections.
In such case, we assume $\abstractdomain$ is equipped with an additional abstract operator $\abstractdomaindistance\in\abstractdomain\to\valuesposplus$, which returns the size of the given abstract element, otherwise $+\infty$ if the abstract element is unbounded or represents multiple variables.

\begin{example}
  In the context of the interval domain, where each input variable is related to a possibly unbounded lower and upper bound, $\abstractdomaindistance(\langle \defvariable \mapsto [2, 4]\rangle) = 2$.
  On the other hand, whenever the input abstract element is unbounded, the size is $+\infty$, \eg, $\abstractdomaindistance(\langle \defvariable \mapsto [0, +\infty]\rangle) = +\infty$.
  % The function $\abstractdomaindistance$ expects only a single variable to be constrained in the abstract domain, or in other words, only one variable is allowed to be not $\top$.
\end{example}

The abstract range $\abstractrange$ first projects away the input variables $\definputvariables$ from all the given abstract values.
Then, it collects all the possible intersections among the projected abstract values.
These intersections represent concrete input configurations where variations on the values of $\definputvariables$ \emph{may} lead to changes of program outcome, from a bucket to another.
All the possible combination of intersections are joined together to find the maximum range of the extreme values of the buckets.

\begin{definition}[\texorpdfstring{$\abstractrange$}{Abstract Range}]\labdef{abstract-range}
  Let $\defoutputvariables$ be the output variables of interest.
We define $\abstractrange\in\pair\vectorbuckets\vectorbuckets\to\valuesposplus$ as:
\begin{align*}
\abstractrange(\presfrombuckets, \buckets) \DefeQ& \max~\seTDef{\abstractdomaindistance(\abstractdomainproject[\defoutputvariables](K))}{K \in I} \\
\text{where}~
I ~=~& \seTDef{\abstractdomainjoin\seTDef{\bucket}{j\in J}}{J \in \intersectallfunction((\abstractdomainproject(\prefrombucket))_{j\le\numberofbuckets})}
\end{align*}
where $\presfrombuckets\in\vectorbuckets$ is the result of the abstract semantics $\multisemanticsnoparam$.
\end{definition}

\begin{example}\labexample{abstract-range}
  Similarly to \refexample{abstract-outcomes}, we show the computation of $\abstractrange$ for the program \refprog*{landing-alarm-system}.
  The computation of $\abstractrange$ works as for $\abstractoutcomes$ until the computation of the intersections.
  Hence, we obtain the same projections for the variable \texttt{angle}:
  \begin{align*}
    &\intersectallfunction((\abstractdomainproject[{\{\texttt{angle}\}}](\prefrombucket[\textup{low}]), \dots, \abstractdomainproject[{\{\texttt{angle}\}}](\prefrombucket[\textup{high}]))) ={} \\
  &\quad \setof{\{\textup{low}, \textup{mid\-/low}, \textup{mid\-/high}, \textup{high}\}}
  \end{align*}
  Next, $\abstractrangename_{\{\texttt{angle}\}}$ computes the size of the output buckets joined together for any combination of bucket indices.
  The maximum is achieved whenever both the low and high buckets, respectively buckets $\bucket[\textup{low}]$ and $\bucket[\textup{high}]$, are joined together, as they yield the minimum and maximum value for the variable \texttt{risk}. The size of bucket $\bucket[\textup{low}]$ joined with $\bucket[\textup{high}]$ is $3$, \ie, $\abstractdomaindistance(\bucket[\textup{low}] \abstractdomainjoin \bucket[\textup{high}]) = \abstractdomaindistance(\langle \texttt{risk} \mapsto [0, 3] \rangle) = 3$.
  Therefore, $\abstractrangename_{\{\texttt{angle}\}}(\presfrombuckets, \buckets) = 3$.
  On the other hand, projecting away the variable \texttt{speed} from the abstract intervals results in:
  \begin{eqnarray*}
    \lefteqn{\intersectallfunction((\abstractdomainproject[{\{\texttt{speed}\}}](\prefrombucket[\textup{low}]), \dots, \abstractdomainproject[{\{\texttt{speed}\}}](\prefrombucket[\textup{high}]))) ={}} \\
  & \left\{
    \begin{array}{l}
    \{\textup{low}\}, \{\textup{mid\-/low}\}, \{\textup{mid\-/high}\}, \{\textup{high}\}, \\
    \{\textup{low}, \textup{mid\-/low}\}, \{\textup{low}, \textup{mid\-/high}\},
    \{\textup{mid\-/low}, \textup{mid\-/high}\}\\
    \{\textup{low}, \textup{mid\-/low}, \textup{mid\-/high}\}
    \end{array}
    \right\}
  \end{eqnarray*}
  In this case, we obtain the following sizes for the combination of buckets:
  \begin{align*}
    \abstractdomaindistance(\bucket[\textup{low}]) \spacearound{&=} \abstractdomaindistance(\langle \texttt{risk} \mapsto 0 \rangle) \spacearound{=} 0 \\
    \abstractdomaindistance(\bucket[\textup{mid\-/low}]) \spacearound{&=} \abstractdomaindistance(\langle \texttt{risk} \mapsto 1 \rangle) \spacearound{=} 0 \\
    \abstractdomaindistance(\bucket[\textup{mid\-/high}]) \spacearound{&=} \abstractdomaindistance(\langle \texttt{risk} \mapsto 2 \rangle) \spacearound{=} 0 \\
    \abstractdomaindistance(\bucket[\textup{high}]) \spacearound{&=} \abstractdomaindistance(\langle \texttt{risk} \mapsto 3 \rangle) \spacearound{=} 0 \\
    %
    \abstractdomaindistance(\bucket[\textup{low}] \abstractdomainjoin \bucket[\textup{mid\-/low}]) \spacearound{&=} \abstractdomaindistance(\langle \texttt{risk} \mapsto [0, 1] \rangle) \\ \spacearound{&=} 1 \\
    \abstractdomaindistance(\bucket[\textup{low}] \abstractdomainjoin \bucket[\textup{mid\-/high}]) \spacearound{&=} \abstractdomaindistance(\langle \texttt{risk} \mapsto [0, 2] \rangle) \\ \spacearound{&=} 2 \\
    \abstractdomaindistance(\bucket[\textup{mid\-/low}] \abstractdomainjoin \bucket[\textup{mid\-/high}]) \spacearound{&=} \abstractdomaindistance(\langle \texttt{risk} \mapsto [1, 2] \rangle) \\ \spacearound{&=} 1 \\
    \abstractdomaindistance(\bucket[\textup{low}] \abstractdomainjoin \bucket[\textup{mid\-/low}] \abstractdomainjoin \bucket[\textup{mid\-/high}])
     \spacearound{&=} \abstractdomaindistance(\langle \texttt{risk} \mapsto [0, 2] \rangle) \\ \spacearound{&=} 2
  \end{align*}
  Therefore, $\abstractrangename_{\{\texttt{speed}\}}(\presfrombuckets, \buckets) = 2$.
\end{example}

To prove that $\abstractrange$ is a sound implementation of $\range$, we require the classic soundness condition on the abstract operator $\abstractdomaindistance$, ensuring that the abstract distance is always greater than the concrete one.

\begin{definition}[Soundness of \texorpdfstring{$\abstractdomaindistance$}{Size}]\labdef{soundness-size}
  Given an abstract value $\defstate^\natural\in\abstractdomain$ and the output variables of interest $\defoutputvariables$, it holds that:
  \[\distance(\setdef{\outputobs(\defstate)(\defoutputvariables)}{\defstate\in\abstractdomainconcretization(\defstate^\natural)}) \LE \abstractdomaindistance(\defstate^\natural)\]
\end{definition}

The next result shows that the abstract impact $\abstractrange$ is a sound over-approximation of the concrete impact $\range$, \cf{} \refdef*{range}[*-16.2].

\begin{lemma}[$\abstractrange$ is a Sound Implementation of $\range$]\lablemma{abstractrange-is-sound}
  Let $\definputvariables\in\setof\inputvariables$ be the input variable of interest, $\abstractdomain$ the abstract domain, $\backwardsemanticsnoparam$ the family of semantics, and $\buckets\in\vectorbuckets$ the starting output buckets.
  Whenever the following conditions hold:
  \begin{enumerate}[label=(\roman*)]
    \item \label{proof:ab} $\buckets$ covers the subset of potential outcomes, \cf{} \refdef*{covering}[*-16.7],
    \item \label{proof:ac} $\buckets$ is compatible with $\outputobs$, \cf{} \refdef*{compatibility}[*-12.3],
    \item \label{proof:ad} $\abstractdomainproject[]$ is sound, \cf{} \refdef*{soundness-project}[*-8.5], and
    \item \label{proof:ae} $\abstractdomaindistance$ is sound, \cf{} \refdef*{soundness-size}[*-4];
  \end{enumerate}
  then, $\abstractrange$ is a sound implementation of $\range$.
\end{lemma}
\begin{proof}[Proof (Sketch)]
  We need to prove that
  \begin{gather*}
    \range(\tracesemanticsnoparam) = \sup_{\definput\in\reducedstate}
    \distance(\setdef{\outputobs(\retrieveoutput{\deftrace})(\defoutputvariable)}{\deftrace\in\tracesemanticsnoparam\land\retrieveinput{\deftrace} \stateeq{\inputvariableswithoutw} \definput})
    \LE \\
    \abstractrange(\multiconcretization(\multisemanticsnoparam)\buckets, \buckets) = \max~\setdef{\abstractdomaindistance(K)}{K \in I}
  \end{gather*}
  where $I = \setdef{\abstractdomainjoin\seTDef{\bucket}{j\in J}}{J \in \intersectallfunction((\abstractdomainproject(\backwardconcretization(\backwardsemanticsnoparam)\bucket))_{j\le\numberofbuckets})}$.
  Let $\overline{\definput}\in\reducedstate$ be such that:
  \begin{gather*}
    \sup_{\definput\in\reducedstate}
    \distance(\setdef{\outputobs(\retrieveoutput{\deftrace})(\defoutputvariables)}{\deftrace\in\tracesemanticsnoparam\land\retrieveinput{\deftrace} \stateeq{\inputvariableswithoutw} \definput})
    \spacearound{=} \\
    \distance(\setdef{\outputobs(\retrieveoutput{\deftrace})(\defoutputvariables)}{\deftrace\in\tracesemanticsnoparam\land\retrieveinput{\deftrace} \stateeq{\inputvariableswithoutw} \overline{\definput}})
  \end{gather*}
  Let $\overline{K} \in I$ be such that:
  \begin{gather*}
    \max~\setdef{\abstractdomaindistance(K)}{K \in I}
    \spacearound{=}
    \abstractdomaindistance(\overline{K})
  \end{gather*}
  where $\overline{K} = \abstractdomainjoin\seTDef{\bucket}{j\in J}$ and $J \in \intersectallfunction((\abstractdomainproject(\backwardconcretization(\backwardsemanticsnoparam)\bucket))_{j\le\numberofbuckets})$.
  We need to show that $\distance(\setdef{\outputobs(\retrieveoutput{\deftrace})(\defoutputvariables)}{\deftrace\in\tracesemanticsnoparam\land\retrieveinput{\deftrace} \stateeq{\inputvariableswithoutw} \overline{\definput}}) \le \abstractdomaindistance(\overline{K})$.
  By contradiction, we assume that $\distance(\setdef{\outputobs(\retrieveoutput{\deftrace})(\defoutputvariables)}{\deftrace\in\tracesemanticsnoparam\land\retrieveinput{\deftrace} \stateeq{\inputvariableswithoutw} \overline{\definput}}) > \abstractdomaindistance(\overline{K})$, meaning that the size of output abstraction over the variable $\defoutputvariables$ from states in $\tracesemanticsnoparam$ is higher than the size of $\overline{K}$ in the abstract. Hence, either:
  \begin{enumerate}[label=(\alph*)]
    \item \label{au1} there exist output abstractions that are not matched by any output bucket,
    \item \label{au2} some output abstractions are matched by the same output bucket,
    \item \label{au3} there are output abstractions in $\tracesemanticsnoparam$ from perturbation of the input $\retrieveinput\defstate$ that are not considered in the abstract, or
    \item \label{au4} the size of the abstract element is not greater than the size of the output abstraction.
  \end{enumerate}
  Similarly to the proof of \reflemma{abstractoutcomes-is-sound}, we can show that the first three cases are not possible by \ref{proof:ab}, \ref{proof:ac}, and \ref{proof:ad}.
  The fourth case \ref{au4} is not possible by \ref{proof:ae} as the soundness of the size operator ensures that the size of the abstract element is always greater than the size of the output abstraction.
  Therefore, the assumption is false, and we conclude that $\abstractrange$ is a sound implementation of $\range$.
\end{proof}

\begin{example}
  Let \texttt{risk} be the output variable of interest.
  The quantities computed by the abstract implementation $\abstractrange$ in \refexample{abstract-range} are sound over-approximations of the concrete implementation $\range$.
  \begin{align*}
    & \rangename_{\{\texttt{angle}\}}(\tracesemanticsnoparam\semanticsof{\textup{landing-alarm-system}}) = 3 \\
    &\qquad\LE \abstractrangename_{\{\texttt{angle}\}}(\multiconcretization(\multisemanticsnoparam)\buckets, \buckets) = 3
  \end{align*}
  \begin{align*}
    & \rangename_{\{\texttt{speed}\}}(\tracesemanticsnoparam\semanticsof{\textup{landing-alarm-system}}) = 2 \\
    &\qquad\LE \abstractrangename_{\{\texttt{speed}\}}(\presfrombuckets, \buckets) = 2
  \end{align*}
  as expected by \refthm*{soundness}[*-4].
\end{example}

\subsection{Abstract Implementation \texorpdfstring{$\abstractqused$}{Abstract QUsed}}[Abstract \texorpdfstring{$\abstractqused$}{QUsed}]
\labsec{abstract-qused}

We define $\abstractqused$ as the number of input values that are missing from perturbation to the input variables $\definputvariables$ to reach the same outcomes.

Specifically, $\abstractqused$ first computes the input values of the variable in $\definputvariables$ that are missing coming from the same output bucket.
These values represent missing partial input configurations that the concrete impact quantifier $\qused$ may count.
To compute the number of missing (total) input configurations, we multiply the number of missing input values for the number of possible values of the other input variables.
This number of possible values of the other input variables is computed by projecting away all the variables but one from the abstract element and then applying the abstract counting operator $\abstractdomaincount$ from the whole input space $\abstractdomaintop\in\abstractdomain$.
The result is a measure of the influence of the input variables on the output of the program, \ie, the number of input configurations that may lead to different outcomes.
\begin{definition}[\texorpdfstring{$\abstractqused$}{Abstract QUsed}]\labdef{abstract-qused}
  We define $\abstractqused\in\pair\vectorbuckets\vectorbuckets\to\valuesposplus$ as:
  \begin{align*}
    \abstractqused(\presfrombuckets, \buckets) \DefeQ& \max_{j \le \numberofbuckets}\spacer
    \abstractdomaincount(\abstractdomainprojectothers(\abstractdomaintop \setminus \prefrombucket)) \cdot \prod_{\defvariable\in \definputvariables} l_{\defvariable}
  \end{align*}
  where $\presfrombuckets\in\vectorbuckets$ is the result of the abstract semantics $\multisemanticsnoparam$ and $l$ is the total number of input values a variable takes:
  \begin{align*}
    l_{\defvariable} \DefeQ& \abstractdomaincount(\abstractdomainproject[\inputvariables \setminus \{\defvariable\}](\abstractdomaintop))
  \end{align*}
\end{definition}

\begin{example}\labexample{abstract-qused}
  We show the computation of $\abstractqused$ for the program \refprog*{landing-alarm-system}.
  For the sake of the example, we consider as abstract domain the set of points for each variable, otherwise with the use of interval abstract domain, the result would be too imprecise.
  The backward analysis returns the following set of points:
  \begin{align*}
    \backwardsemanticsnoparam(\bucket[\textup{low}]) \spacearound{&=} \prefrombucket[\textup{low}] \spacearound{=} \langle \texttt{angle} \mapsto \{1\}, \texttt{speed} \mapsto \{1, 2, 3\} \rangle \\
    \backwardsemanticsnoparam(\bucket[\textup{mid\-/low}]) \spacearound{&=} \prefrombucket[\textup{mid\-/low}] \spacearound{=} \langle \texttt{angle} \mapsto \{1\}, \texttt{speed} \mapsto \{2, 3\} \rangle \\
    \backwardsemanticsnoparam(\bucket[\textup{mid\-/high}]) \spacearound{&=} \prefrombucket[\textup{mid\-/high}] \spacearound{=} \langle \texttt{angle} \mapsto \{1\}, \texttt{speed} \mapsto \{2, 3\} \rangle \\
    \backwardsemanticsnoparam(\bucket[\textup{high}]) \spacearound{&=} \prefrombucket[\textup{high}] \spacearound{=} \langle \texttt{angle} \mapsto \{-4\}, \texttt{speed} \mapsto \{1, 2, 3\} \rangle
  \end{align*}
  We are interested in the impact of the input variable \texttt{angle}, hence we project away the other variables from the complement of the given abstract intervals, obtaining:
  \begin{align*}
    &\abstractdomainproject[{\{\texttt{speed}\}}](\abstractdomaintop \setminus \prefrombucket[\textup{low}]) \\
    &\quad\spacearound{=} \abstractdomainproject[{\{\texttt{speed}\}}](\langle \texttt{angle} \mapsto \{-4, 1\}, \texttt{speed} \mapsto \{1, 2, 3\} \rangle \setminus \prefrombucket[\textup{low}]) \\
    &\quad\spacearound{=} \abstractdomainproject[{\{\texttt{speed}\}}](\langle \texttt{angle} \mapsto \{-4\}, \texttt{speed} \mapsto \emptyset\rangle) \spacearound{=}\abstractdomainbottom \\
    &\abstractdomainproject[{\{\texttt{speed}\}}](\abstractdomaintop \setminus \prefrombucket[\textup{mid\-/low}]) \\
    &\quad\spacearound{=} \abstractdomainproject[{\{\texttt{speed}\}}](\langle \texttt{angle} \mapsto \{-4, 1\}, \texttt{speed} \mapsto \{1, 2, 3\} \rangle \setminus \prefrombucket[\textup{mid\-/low}]) \\
    &\quad\spacearound{=} \abstractdomainproject[{\{\texttt{speed}\}}](\langle \texttt{angle} \mapsto \{-4\}, \texttt{speed} \mapsto \{1\}\rangle) \\
    &\quad\spacearound{=} \langle \texttt{angle} \mapsto \{-4\} \rangle \\
    &\abstractdomainproject[{\{\texttt{speed}\}}](\abstractdomaintop \setminus \prefrombucket[\textup{mid\-/high}]) \\
    &\quad\spacearound{=} \abstractdomainproject[{\{\texttt{speed}\}}](\langle \texttt{angle} \mapsto \{-4, 1\}, \texttt{speed} \mapsto \{1, 2, 3\} \rangle \setminus \prefrombucket[\textup{mid\-/high}]) \\
    &\quad\spacearound{=} \abstractdomainproject[{\{\texttt{speed}\}}](\langle \texttt{angle} \mapsto \{-4\}, \texttt{speed} \mapsto \{1\}\rangle) \\
    &\quad\spacearound{=} \langle \texttt{angle} \mapsto \{-4\} \rangle \\
    &\abstractdomainproject[{\{\texttt{speed}\}}](\abstractdomaintop \setminus \prefrombucket[\textup{high}]) \\
    &\quad\spacearound{=} \abstractdomainproject[{\{\texttt{speed}\}}](\langle \texttt{angle} \mapsto \{-4, 1\}, \texttt{speed} \mapsto \{1, 2, 3\} \rangle \setminus \prefrombucket[\textup{high}]) \spacearound{=} \abstractdomainbottom
  \end{align*}
  Then, $\abstractqusedname_{\{\texttt{angle}\}}$ counts the number of input configurations:
  \begin{align*}
    \abstractdomaincount(\abstractdomainbottom) \spacearound{&=} 0 \\
    \abstractdomaincount(\langle \texttt{angle} \mapsto \{-4\} \rangle) \spacearound{&=} 1
  \end{align*}
  This value has to be multiplied by the number of possible values for the other input variable \texttt{speed}, which is $3$.
  The total number of missing input configurations is $3$ for the variable \texttt{angle}, hence $\abstractqusedname_{\{\texttt{angle}\}}(\presfrombuckets, \buckets) = 3$.
  On the other hand, when interested in the variable \texttt{speed} we project away the variable \texttt{angle} from the abstract intervals, obtaining:
  \begin{align*}
    &\abstractdomainproject[{\{\texttt{angle}\}}](\abstractdomaintop \setminus \prefrombucket[\textup{low}]) \\
    &\quad\spacearound{=} \abstractdomainproject[{\{\texttt{angle}\}}](\langle \texttt{angle} \mapsto \{-4, 1\}, \texttt{speed} \mapsto \{1, 2, 3\} \rangle \setminus \prefrombucket[\textup{low}]) \\
    &\quad\spacearound{=} \abstractdomainproject[{\{\texttt{angle}\}}](\langle \texttt{angle} \mapsto \{-4\}, \texttt{speed} \mapsto \emptyset\rangle) \spacearound{=}\abstractdomainbottom \\
    &\abstractdomainproject[{\{\texttt{angle}\}}](\abstractdomaintop \setminus \prefrombucket[\textup{mid\-/low}]) \\
    &\quad\spacearound{=} \abstractdomainproject[{\{\texttt{angle}\}}](\langle \texttt{angle} \mapsto \{-4, 1\}, \texttt{speed} \mapsto \{1, 2, 3\} \rangle \setminus \prefrombucket[\textup{mid\-/low}]) \\
    &\quad\spacearound{=} \abstractdomainproject[{\{\texttt{angle}\}}](\langle \texttt{angle} \mapsto \{-4\}, \texttt{speed} \mapsto \{1\}\rangle) \\
    &\quad\spacearound{=} \langle \texttt{speed} \mapsto \{1\} \rangle \\
    &\abstractdomainproject[{\{\texttt{angle}\}}](\abstractdomaintop \setminus \prefrombucket[\textup{mid\-/high}]) \\
    &\quad\spacearound{=} \abstractdomainproject[{\{\texttt{angle}\}}](\langle \texttt{angle} \mapsto \{-4, 1\}, \texttt{speed} \mapsto \{1, 2, 3\} \rangle \setminus \prefrombucket[\textup{mid\-/high}]) \\
    &\quad\spacearound{=} \abstractdomainproject[{\{\texttt{angle}\}}](\langle \texttt{angle} \mapsto \{-4\}, \texttt{speed} \mapsto \{1\}\rangle) \\
    &\quad\spacearound{=} \langle \texttt{speed} \mapsto \{1\} \rangle \\
    &\abstractdomainproject[{\{\texttt{angle}\}}](\abstractdomaintop \setminus \prefrombucket[\textup{high}]) \\
    &\quad\spacearound{=} \abstractdomainproject[{\{\texttt{angle}\}}](\langle \texttt{angle} \mapsto \{-4, 1\}, \texttt{speed} \mapsto \{1, 2, 3\} \rangle \setminus \prefrombucket[\textup{high}]) \spacearound{=} \abstractdomainbottom
  \end{align*}
  Then, $\abstractqusedname_{\{\texttt{speed}\}}$ counts the number of input configurations:
  \begin{align*}
    \abstractdomaincount(\abstractdomainbottom) \spacearound{&=} 0 \\
    \abstractdomaincount(\langle \texttt{speed} \mapsto \{1\} \rangle) \spacearound{&=} 1 \\
  \end{align*}
  As before, we multiply such value by the number of possible values for the other input variable \texttt{angle}, \cf{} 2.
  The total number of missing input configurations is then $2$ for the variable \texttt{speed}, hence $\abstractqusedname_{\{\texttt{speed}\}}(\presfrombuckets, \buckets) = 2$.
\end{example}

To prove that $\abstractqused$ is a sound implementation of $\qused$, we require the sound soundness condition on the abstract operator $\abstractdomaincount$, ensuring that the abstract count is always lower than the concrete one.
This requirement comes from the comparison operator $\ge$ in the property $\boundedqused$.
Note that, the concrete counterpart of the abstract $\abstractdomaincount$ is the cardinality operator $\cardinalitynospaces{\cdot}$.

\begin{definition}[Soundness of \texorpdfstring{$\abstractdomaincount$}{Count}]\labdef{soundness-count}
  Given an abstract value $\defstate^\natural\in\abstractdomain$ and the set of input variables of interest $\definputvariables\in\setof\inputvariables$, it holds that:
  \[\cardinality{\setdef{\retrieveinput{\defstate}(\definputvariables)}{\defstate\in\abstractdomainconcretization(\defstate^\natural)}} \GE \abstractdomaincount(\defstate^\natural) \]
\end{definition}

The next result shows that the abstract impact $\abstractqused$ is a sound over-approximation of the concrete impact $\qused$, \cf{} \refdef*{qused}[*-5].

\begin{lemma}[$\abstractqused$ is a Sound Implementation of $\qused$]\lablemma{abstractqused-is-sound}
  Let $\definputvariables\in\setof\inputvariables$ be the input variable of interest, $\abstractdomain$ the abstract domain, $\backwardsemanticsnoparam$ the family of semantics, and $\buckets\in\vectorbuckets$ the starting output buckets.
  Whenever the following conditions hold:
  \begin{enumerate}[label=(\roman*)]
    \item \label{pp1} $\buckets$ covers the subset of potential outcomes, \cf{} \refdef*{covering}[*-2.5],
    \item \label{pp2} $\buckets$ is compatible with $\outputobs$, \cf{} \refdef*{compatibility}[*2],
    \item \label{pp3} $\abstractdomainproject[]$ is sound, \cf{} \refdef*{soundness-project}[*6], and
    \item \label{pp4} $\abstractdomaincount$ is sound, \cf{} \refdef*{soundness-count}[*10.8];
  \end{enumerate}
  then, $\abstractqused$ is a sound implementation of $\qused$.
\end{lemma}
\begin{proof}[Proof (Sketch)]
  We need to prove that
    \begin{gather*}
      \qused(\tracesemanticsnoparam) \DefeQ
      \sup_{\deftrace \in \defsetoftraces}
    \cardinality{
        Q_\definputvariables(I_\deftrace(\defsetoftraces)) \setminus I_\deftrace(\defsetoftraces)
      } \\
    \GE \\
    \abstractqused(\multiconcretization(\multisemanticsnoparam)\buckets, \buckets) = \max_{j \le \numberofbuckets}\spacer
    \abstractdomaincount(\abstractdomainprojectothers(\abstractdomaintop \setminus \prefrombucket)) \cdot \prod_{\defvariable\in \definputvariables} l_{\defvariable}
  \end{gather*}
  Let $\overline{\defseq}\in\tracesemanticsnoparam$ be such that:
  \begin{gather*}
    \sup_{\deftrace \in \defsetoftraces}
    \cardinality{
        Q_\definputvariables(I_\deftrace(\defsetoftraces)) \setminus I_\deftrace(\defsetoftraces)
      }
    \spacearound{=} \\
    \cardinality{
        Q_\definputvariables(I_{\overline{\deftrace}}(\defsetoftraces)) \setminus I_{\overline{\deftrace}}(\defsetoftraces)
      }
  \end{gather*}
  Let $\overline{j} \le \numberofbuckets$ be such that:
  \begin{gather*}
    \max_{j \le \numberofbuckets}\spacer
    \abstractdomaincount(\abstractdomainprojectothers(\abstractdomaintop \setminus \prefrombucket)) \cdot \prod_{\defvariable\in \definputvariables} l_{\defvariable}
    \spacearound{=}
    \abstractdomaincount(\abstractdomainprojectothers(\abstractdomaintop \setminus \prefrombucket[\overline{j}])) \cdot \prod_{\defvariable\in \definputvariables} l_{\defvariable}
  \end{gather*}
  By contradiction, we assume that $\cardinality{
    Q_\definputvariables(I_{\overline{\deftrace}}(\defsetoftraces)) \setminus I_{\overline{\deftrace}}(\defsetoftraces)
  } < \abstractdomaincount(\abstractdomainprojectothers(\abstractdomaintop \setminus \prefrombucket[\overline{j}])) \cdot \prod_{\defvariable\in \definputvariables} l_{\defvariable}$, meaning that the number of missing input configurations is lower than the number of missing input values for the variable $\definputvariables$ coming from the same output bucket.
  Hence, either:
  \begin{enumerate}[label=(\alph*)]
    \item \label{qq1} there exist output abstractions that are not matched by any output bucket,
    \item \label{qq2} some output abstractions are matched by the same output bucket,
    \item \label{qq3} there are output abstractions in $\tracesemanticsnoparam$ from perturbation of the input $\retrieveinput\defstate$ that are not considered in the abstract, or
    \item \label{qq4} counting the number of input configurations in the abstract is not greater than the number of input configurations in the concrete.
  \end{enumerate}
  Again, we can show that the first three cases are not possible by hypothesis \ref{pp1}, \ref{pp2}, and \ref{pp3}. By hypothesis \ref{pp4}, the soundness of the count operator ensures that the number of input configurations in the abstract is always greater than the number of input configurations in the concrete. Therefore, the assumption is false, and we conclude that $\abstractqused$ is a sound implementation of $\qused$.
\end{proof}


\begin{example}
  The quantities computed by the abstract implementation $\abstractqused$ in \refexample{abstract-qused} are sound over-approximations of the concrete implementation $\qused$:
  \begin{align*}
    & \qusedname_{\{\texttt{angle}\}}(\tracesemanticsnoparam\semanticsof{\landingprogram}) = 3 \\
    &\qquad\GE \abstractqusedname_{\{\texttt{angle}\}}(\presfrombuckets, \buckets) = 3
  \end{align*}
  \begin{align*}
    & \qusedname_{\{\texttt{speed}\}}(\tracesemanticsnoparam\semanticsof{\landingprogram}) = 2 \\
    &\qquad\GE \abstractqusedname_{\{\texttt{speed}\}}(\presfrombuckets, \buckets) = 2
  \end{align*}
  as expected by \refthm*{soundness}, where $\landingprogram$ is the program of the landing alarm system, \cf{} \refprog{landing-alarm-system}.
  In this case, the bound computed in the abstract is the same as the concrete one, as the abstract domain is precise enough to capture the exact number of missing input configurations.
\end{example}


\section{Summary}
\labsec{quantitative-input-data-usage-summary}

In this chapter, we introduced the quantifiers $\outcomes$, $\range$, and $\qused$ to measure the impact of input variables on the output of a program.
We developed a theoretical framework to verify the $\defbound$-bounded impact property of a program and showed the abstract version of the quantifiers.
In the next chapter, we present the evaluation of the quantifiers on a set of use cases.
Later, we will show how we handle the quantification of the impact of input variables in the context of neural network models.


\frenchdiv

\emph{Dans ce chapitre, nous avons introduit les mesures quantitatives $\outcomes$, $\range$ et $\qused$ pour quantifier l'impact des variables d'entrée sur la sortie d'un programme. Nous avons développé un cadre théorique pour vérifier la propriété d'impact bornée par $\defbound$ d'un programme et montré la version abstraite des mesures quantitatives. Dans le prochain chapitre, nous présenterons l'évaluation des mesures quantitatives sur un ensemble de scénarios. Plus tard, nous montrerons comment nous gérons la quantification de l'impact des variables d'entrée dans le contexte des modèles de réseaux neuronaux.}
