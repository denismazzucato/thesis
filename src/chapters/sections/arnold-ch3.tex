\setchapterpreamble[u]{\margintoc}


\chapter{Input Data Usage}
\labch{input-data-usage}
\index{Input Data Usage}

\marginemptybox{5cm}

In this chapter, we introduce the notion of input data usage as proposed by \sidecite{Urban2018}.
In particular, we include output abstractions in the definition of an unused input variable, obtaining a more general definition.
Through this chapter, we study relations among abstract non-interference and the unused predicate and their use in the context of non-deterministic programs.
Then, we present a \emph{sound and complete} hierarchy of semantics that contains only, and exactly, the information needed to reason about the usage of input variables.
Finally, we show a \emph{sound} abstraction of the unused property collecting syntactic dependencies among variables.
Later in the next chapters, we will define a quantitative measure of usage of input variables, extending the qualitative notion presented in this chapter.


\frenchdiv

\emph{Dans ce chapitre, nous introduisons la notion d'utilisation des données d'entrée telle que proposée par \textcite{Urban2018}. En particulier, nous incluons des abstractions de sortie dans la définition d'une variable d'entrée non utilisée, obtenant ainsi une définition plus générale. Tout au long de ce chapitre, nous étudions les relations entre la non-interférence abstraite et le prédicat de non-utilisation, ainsi que leur utilisation dans le contexte des programmes non déterministes. Ensuite, nous présentons une hiérarchie de sémantiques \emph{correcte et complète} qui contient uniquement, et exactement, les informations nécessaires pour raisonner sur l'utilisation des variables d'entrée. Enfin, nous montrons une abstraction \emph{correcte} de la propriété de non-utilisation en collectant les dépendances syntaxiques entre les variables. Dans les prochains chapitres, nous définirons une mesure quantitative de l'utilisation des variables d'entrée, en étendant la notion qualitative présentée dans ce chapitre.}





\section{Summary}
\labsec{input-data-usage-summary}

In this chapter, we introduced the notion of input data usage and defined an abstract version inspired by the definition of abstract non-interference.
We presented the hierarchy of semantics that allows reasoning about the usage of input variables of a program.
We showed a computable semantics collecting syntactic dependencies among variables.
In the next chapter, we will define a quantitative counterpart of input data usage, able to measure the impact of variations in the input data on the outcome of a program.
Notably, we will exploit the output abstraction to obtain numerical values from the output states.

\frenchdiv

\emph{Dans ce chapitre, nous avons introduit la notion d'utilisation des données d'entrée et défini une version abstraite inspirée de la définition de la non-interférence abstraite. Nous avons présenté la hiérarchie des sémantiques permettant de raisonner sur l'utilisation des variables d'entrée d'un programme. Nous avons montré une sémantique calculable qui collecte les dépendances syntaxiques entre les variables. Dans le prochain chapitre, nous définirons une contrepartie quantitative de l'utilisation des données d'entrée, capable de mesurer l'impact des variations des données d'entrée sur le résultat d'un programme. Notamment, nous exploiterons l'abstraction de sortie pour obtenir des valeurs numériques à partir des états de sortie.}
