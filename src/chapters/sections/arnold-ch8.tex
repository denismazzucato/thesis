\pagelayout{wide} % No margins
\addpart{Quantitative Verification of Intensional Properties}
\labpart{intensional}
\pagelayout{margin} % Restore margins

\setchapterpreamble[u]{\margintoc}

\chapter{Quantitative Static Timing Analysis}
\labch{quantitative-static-timing-analysis}


\marginemptybox{6.8cm}

This chapter presents a static analysis for quantifying the impact of input variables on the number of iterations of a program.
The analysis employs a semantic global loop bound analysis to derive an over-approximation of the impact quantity.
First, we introduce the maximal trace semantics augmented with a global loop counter to model the number of iterations of all the loops of a program.
Then, we show an abstract global loop bound analysis, followed by the impact quantification, which is employed to verify the corresponding $\defbound$-impact property.
This chapter is based on the work presented at the 31st Static Analysis Symposium (SAS 2024)~\sidecite{Mazzucato2024c}.
\refch{sas24-eval} will discuss some implementation details and optimizations, as well as the evaluation of the analysis presented on this chapter.


\frenchdiv

\emph{Ce chapitre présente une analyse statique pour quantifier l'impact des variables d'entrée sur le nombre d'itérations d'un programme. L'analyse utilise une analyse sémantique des bornes globales des boucles pour dériver une sur-approximation de la quantité d'impact. Tout d'abord, nous introduisons la sémantique des traces maximales augmentée d'un compteur global de boucles pour modéliser le nombre d'itérations de toutes les boucles d'un programme. Ensuite, nous présentons une analyse abstraite des bornes globales des boucles, suivie par la quantification de l'impact, qui est utilisée pour vérifier la propriété d'impact borné par $\defbound$ correspondante. Ce chapitre est basé sur les travaux présentés lors du 31e Symposium sur l'analyse statique (SAS 2024)~\cite{Mazzucato2024c}. \refch{sas24-eval} discutera de certains détails d'implémentation et optimisations, ainsi que de l'évaluation de l'analyse présentée dans ce chapitre.}







\section{Summary}

This chapter presented a static analysis for quantifying the impact of input variables on the number of iterations of a program.
The following chapter presents the experimental evaluation of our approach and concludes the main body of this thesis.


\frenchdiv

\emph{Ce chapitre a présenté une analyse statique pour quantifier l'impact des variables d'entrée sur le nombre d'itérations d'un programme. Le chapitre suivant présente l'évaluation expérimentale de notre approche et conclut le corps principal de cette thèse.}
