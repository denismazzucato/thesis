\pagelayout{wide} % No margins
\addpart{Conclusion}
\labpart{conclusion}
\pagelayout{margin} % Restore margins

% \setchapterpreamble[u]{\margintoc}
% \chapter{Related Work}
% \labch{related-work}

\chapter{Conclusion and Future Directions}
\labch{conclusion}


The aim of this thesis is to develop static analyses based on abstract interpretation to soundly quantify the impact of input data.
We proposed a formal quantitative framework to reason about the impact of input data, parametrized in the notion of impact quantifier to satisfy different needs.
The static analyses we developed are \emph{automatic} and \emph{sound}, meaning that they are guaranteed to provide an upper bound on the impact of input data automatically without any user intervention.
We focused on two main applications: the first one is to quantify the impact of input features on the classification of neural networks, and the second one is to quantify the impact of input data on the execution time of programs.
Our experiments show evidence that our analyses are effective in practice.

\paragraph{Quantitaitve Input Usage.} Our first contribution is the design of a novel quantitative framework to reason about the impact of input data according to a given impact quantifier of interest. We presented three different impact quantifiers: \outcomesname{}, \rangename{}, and \qusedname{}. Each of these quantifiers differently characterizes the impact of the input variables: \outcomesname{} counts the number of different outcomes, \rangename{} measures the range of outcome values, and \qusedname{} counts the number of input values that are \emph{not} used to produce the outcomes.
We defined a static analysis to automatically compute an upper bound on the impact of input data according to a sound implementation of the impact quantifier of interest. The results are guaranteed to be sound, meaning that the computed quantities are formally guaranteed to be an upper bound on the actual impact of input data.
A prototype implementation of our framework, called \impatto,\sidenote{\impattourl} initially demonstrated the feasibility of our approach on a set of demonstrative programs.

\paragraph{Impact of Input Features.} Our second contribution is the application of our quantitative framework in the context of neural networks.
As we noticed that the \outcomesname{}, \rangename{}, and \qusedname{} quantifiers are unable to capture the impact of input features on the classification of neural networks, we introduced two new impact quantifiers: \changesname{} and \qlibraname{}.
These quantifiers are specifically designed to tackle the non-linear behavior of the input space of neural networks.
For the evaluation of our proposed method, we extended two tools: \impatto{} and \libra{}.\sidenote{\libraurl}
We demonstrated that our two quantifiers are capable of capturing the impact of input features on the classification of neural networks.

\paragraph{Impact on Execution Time.} Our third contribution is the application of our quantitative framework to quantify the impact of input data on the number of iterations of loops in programs. This information can be used to estimate the impact of input variables on the execution time of programs.
We implemented in a tool called \timesec\sidenote{\timesecurl} a static analysis to automatically compute an upper bound on the impact of input data on the number of iterations of loops in programs.
Notably, we certified that the \bignum{} library\sidenote{\bignumurl} is free of timing side-channel vulnerabilities.
To the best of our knowledge, our approach provides strictly stronger guarantees than previous empirical results.


\frenchdiv

We conclude with some perspectives on future research directions we would like to study.

\paragraph{Abstract Domains.}

throughout this thesis, we used a multitude of abstract domains to engine our static analyses. For instance, in \refch{quantitative-input-data-usage}, the backward analysis was parametrized by convex numerical abstract domain, such as interval domain~\sidecite{Cousot1978}, octagon domain~\sidecite{Min_e2006a}, or the polyhedra domain~\cite{Cousot1978}. In \refch{quantitative-fairness}, the forward pre-analysis employed the \symbolic{} \cite{Wang2018b}, \deeppoly{} \cite{Singh2019}, and \neurify{} \cite{Wang2018a} abstract domains for neural network analysis. In the future, the development of new relational abstract domains to discover specific non-linear variable relations could drastically improve the analysis precision.
In fact, it is often improbable to find tight quantification bounds when the abstract domain is not expressive enough to capture the program behavior precisely enough during the abstract analysis.

The development of new abstract domains is a challenging task.
However, as each impact quantifier is designed to answer a specific question, the development of new abstract domains could be guided by the programs on which the quantifiers are applied.
By restricting to the programs of interest, the abstract domain could be tailored to capture the specific behavior of that set of programs.
The quantification of the impact would automatically benefit from the increased precision of the abstract domain.

\paragraph{Non-Termination and Non-Determinism.}

An interesting direction for future work is to extend our approach to address non-termination and non-determinism.
A possible solution to integrate non-termination in the quantitative framework would be to run a termination analysis \sidecite{Chatterjee2021,Gonnord2015,Urban2015b}, alongside the backward analysis, to add -- as an output value -- the potential non-termination state.
With such information, we could refine our quantitative analysis to discover termination-aware impact quantities.
Indeed, by knowing that some executions do not terminate after a certain loop could improve the precision of the quantitative bound as we avoid considering all the successive iterations as potential executions.

To address non-determinism we could consider the sequence of all possible non-deterministic choices as a parameter of the semantics~\sidecite{Cousot2012a,Parolini2024}.
On an orthogonal direction, the development of quantiative properties that natively handle non-determinism at the property reasoning level could raise the need of ad-hoc semantics. For instance, the unused property (\cf{} \refdef*{unused-predicate}) and the $\defbound$-bounded impact property \wrt{} the \qusedname{} quantifier (\cf{} \refdef*{qused}) handle natively the non-determinism and do not require any specific semantics.

\paragraph{Data Science Code.}

A broader comparison of our quantitative input data usage with close related properties, such as quantitative data leakage, could resort in interesting discoveries.
Exploring this direction in the future would involve addressing further verification challenges posed by data science code \sidecite{Negrini2023,Subotic2022b,Drobnjakovic2024}. Especially, by the dynamic nature of code notebooks, which makes them susceptible to programming errors that are less common in other development environments. Notebooks allow users to execute code out of order or modify variables on the fly, leading to errors and inconsistencies in the analysis process. Such issues can have significant consequences, particularly in fields where data-driven decisions play a crucial role.

Furthermore, data science notebooks heavily rely on external libraries like NumPy or pandas for dataset analysis and manipulation. Quantifying the impact of programs utilizing these libraries is a major challenge due to their complex and low-level implementations. Although the source code is usually available, analyzing it is a demanding task and often impractical. Therefore, employing custom abstractions of these libraries can facilitate verification by abstracting unnecessary details and focusing on the high-level logic of the libraries.

\paragraph{Neural Network Verification.}

It remains for future work to implement support for other activation functions than \relu s. It would also be straightforward to adapt \libra{} to support other fairness notions such as individual fairness \sidecite{Dwork2012}.
%
Moreover, we plan to design and equip \libra{} with a smarter reduced product between domains, able to also exchange symbolic bounds along with the concrete bounds.
%
Finally, we intend to extend our approach to other machine learning models, such as support vector machines \sidecite{Cristianini2010} or decision tree ensembles \sidecite{Breiman2001,Friedman2001}.


\paragraph{Quantitative Properties.}

Another promising direction is the exploration of new impact definitions for cyber-physical systems~\sidecite{Kwiatkowska2016}.
It could also be interesting to exploit an impact definition to analyze the impact of abstract domains in static program analyzers, \eg, by using pre-metrics as defined in~\sidecite{Campion2022,Campion2023}.

\paragraph{Post-\textsc{Spectre} Era.}

As discussed in \refch{quantitative-static-timing-analysis}, certifying that some sensitive input variables do not impact the execution time can prevent timing side-channel attacks.
This implication may not be true anymore in the post-\textsc{Spectre} attack era \sidecite{Kocher2018,Lipp2020}.
Microarchitectural features, as out-of-order and speculative execution, make constant-time programs still vulnerable to timing side-channel attacks \sidecite{Cauligi2020}.
By incorporating microarchitectural features in the analysis of timing side-channel attacks, as done by \sidecite{Guarnieri2020,Bard2024}, our approach could be extended to consider the impact of input variables also in the presence of out-of-order and speculative execution.
