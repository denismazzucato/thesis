%----------------------------------------------------------------------------------------
%	PREFACE
%----------------------------------------------------------------------------------------

\chapter*{Abstract}
\addcontentsline{toc}{chapter}{Abstract} % Add the preface to the table of contents as a chapter


The aim of this thesis is to develop mathematically sound and practically efficient methods for improving the reliability of software systems.
Our approach is based on abstract interpretation, a formal framework to systematically design approximate program behaviors.
We focus on the development of a quantitative framework for measuring the impact of input variables on the program behavior.
Advances in the understanding of input data usage in software systems produce a more correct, performant, and secure software, particularly in safety-critical systems where these factors are non-negotiable.


To achieve this goal, we propose a novel quantitative input usage framework to discriminate between input variables based on their influence on the program behavior.
This framework is flexible, parametrized in the notion of impact to suit various needs, thereby providing a method for both certification of intended behaviors and identification of potential flaws.
By employing abstract interpretation, our results are guaranteed to be sound, meaning that the quantified bounds on the impact of input variables are always greater (or always lower depending on the underlying over- or under-approximation) than the actual impact.
The major challenge, however, is to ensure that the quantification is precise enough to be useful in practice, while still being computationally efficient.


The quantitative framework is applied to verify both extensional and intensional properties of software. Extensional properties are based on the input-output behavior of programs, while intensional properties concern how the computations are performed. Specifically, the extensional property measures how variations in input data influence program output, whereas the intensional property assesses the impact of input data on the number of iterations within the program's loops during execution.
We quantify the impact of input variables by analyzing input-output dependencies in the program under evaluation.

To demonstrate the practical applicability of this framework, we implemented it into three distinct tools: \libra, \impatto, and \timesec, each designed for a specific type of quantitative analysis. \libra{} focuses on fairness of neural networks, \impatto{} is tailored for general-purpose software reliability analysis, and \timesec{} specializes in assessing side-channel vulnerabilities. The effectiveness of these tools was validated through extensive experimental evaluations across diverse use cases, from the quantification of biased space in neural networks to the detection of timing-based side-channel attacks in cryptographic libraries.
\libra's application in neural network analysis has provided insights into how input features influence model behavior, contributing to the development of more reliable machine learning systems.
\timesec{} has been shown to effectively quantify the impact of input variables in the Amazon Web Services \bignum{} library, enabling the identification of critical variables that may affect system stability or security.

Overall, the quantitative framework proposed in this thesis advances the understanding of input data usage in software systems, providing theoretical framework based on abstract interpretation and practical tools for quantifying the impact of input data. The experimental results highlight the utility of our quantitative framework, demonstrating its potential.


\chapter*{Résumé}
% \addcontentsline{toc}{chapter}{Abstract} % Add the preface to the table of contents as a chapter

\emph{
The aim of this thesis is to develop mathematically sound and practically efficient methods for improving the reliability of software systems.
Our approach is based on abstract interpretation, a formal framework to systematically design approximate program behaviors.
We focus on the development of a quantitative framework for measuring the impact of input variables on the program behavior.
Advances in the understanding of input data usage in software systems produce a more correct, performant, and secure software, particularly in safety-critical systems where these factors are non-negotiable.}

\emph{
To achieve this goal, we propose a novel quantitative input usage framework to discriminate between input variables based on their influence on the program behavior.
This framework is flexible, parametrized in the notion of impact to suit various needs, thereby providing a method for both certification of intended behaviors and identification of potential flaws.
By employing abstract interpretation, our results are guaranteed to be sound, meaning that the quantified bounds on the impact of input variables are always greater (or always lower depending on the underlying over- or under-approximation) than the actual impact.
The major challenge, however, is to ensure that the quantification is precise enough to be useful in practice, while still being computationally efficient.}

\emph{
The quantitative framework is applied to verify both extensional and intensional properties of software. Extensional properties are based on the input-output behavior of programs, while intensional properties concern how the computations are performed. Specifically, the extensional property measures how variations in input data influence program output, whereas the intensional property assesses the impact of input data on the number of iterations within the program's loops during execution.
We quantify the impact of input variables by analyzing input-output dependencies in the program under evaluation.}

\emph{
To demonstrate the practical applicability of this framework, we implemented it into three distinct tools: \libra, \impatto, and \timesec, each designed for a specific type of quantitative analysis. \libra{} focuses on fairness of neural networks, \impatto{} is tailored for general-purpose software reliability analysis, and \timesec{} specializes in assessing side-channel vulnerabilities. The effectiveness of these tools was validated through extensive experimental evaluations across diverse use cases, from the quantification of biased space in neural networks to the detection of timing-based side-channel attacks in cryptographic libraries.
\libra's application in neural network analysis has provided insights into how input features influence model behavior, contributing to the development of more reliable machine learning systems.
\timesec{} has been shown to effectively quantify the impact of input variables in the Amazon Web Services \bignum{} library, enabling the identification of critical variables that may affect system stability or security.}

\emph{
Overall, the quantitative framework proposed in this thesis advances the understanding of input data usage in software systems, providing theoretical framework based on abstract interpretation and practical tools for quantifying the impact of input data. The experimental results highlight the utility of our quantitative framework, demonstrating its potential.}

\index{Preface}
