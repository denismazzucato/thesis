%----------------------------------------------------------------------------------------
%	PREFACE
%----------------------------------------------------------------------------------------

\chapter*{Abstract}
\addcontentsline{toc}{chapter}{Abstract} % Add the preface to the table of contents as a chapter

The aim of this thesis is to develop mathematically sound and practically efficient methods for improving the reliability of software systems.
This work is grounded in the theory of Abstract Interpretation, a formal framework for approximating program behaviors.
In particular, this thesis focuses on quantifying the impact of input variables on program execution, a critical aspect for ensuring correctness, performance, and security of software systems.

To achieve this, we present a novel quantitative input usage framework to discriminate between input variables based on their impact on the program. This framework allows the identification of variables that disproportionately affect the system, and can be used to certify intended behavior or reveal potential flaws.
The notion of impact is parametric to the framework, providing flexibility to adapt to different contexts and requirements.

In particular, we explore the application of this quantitative framework for verifying both intensional and extensional properties. Extensional properties refer to the input-output behavior of a program, while intensional properties also encompass the internal states.

The results presented in this thesis have been implemented into three tools: Libra, Impatto, and TimeSec.
Experimental results show the quantification of impact in a variety of scenarios, including the evaluation of fairness for neural networks and the detection of side-channel vulnerabilities.

\chapter*{Résumé}
% \addcontentsline{toc}{chapter}{Abstract} % Add the preface to the table of contents as a chapter

\index{preface}
