%----------------------------------------------------------------------------------------
%	PREFACE
%----------------------------------------------------------------------------------------

\chapter*{Abstract}
\addcontentsline{toc}{chapter}{Abstract} % Add the preface to the table of contents as a chapter


The aim of this thesis is to develop mathematically sound and practically efficient methods for improving the reliability of software systems through the formal framework of abstract interpretation, a formal framework designed to approximate program behaviors. This research centers on developing a quantitative framework to measure the impact of input variables on program behavior--a step forward for ensuring software correctness, performance, and security, particularly in safety critical systems where these factors are non-negotiable.


To achieve this goal, we propose a novel quantitative input usage framework to discriminate between input variables based on their influence on program behavior.
This framework is flexible, parametrized in the notion of impact to suit various needs, thereby providing a method for both certification of intended behaviors and identification of potential flaws.
By employing abstract interpretation, our results are guaranteed to be sound, meaning that the quantified bounds on the impact of input variables are always greater (or always lower depending on the underlying over- or under-approximation) than the actual impact.
The major challenge, however, is to ensure that the quantification is precise enough to be useful in practice, while still being computationally efficient.


The quantitative framework is applied to verify both extensional and intensional properties of software. Extensional properties are concerned with the input-output behavior of programs, while intensional properties also involve the examination of intermediate states. Specifically, we quantify the impact of input variables by analyzing input-output dependencies in the program under evaluation. The extensional property measures how variations in input data influence program output, whereas the intensional property assesses the impact of input data on the number of iterations within the program’s loops during execution.

To demonstrate the practical applicability of this framework, we implemented it into three distinct tools: \libra, \impatto, and \timesec, each designed for a specific type of quantitative analysis. \libra{} focuses on fairness of neural networks, \impatto{} is tailored for general-purpose software reliability analysis, and \timesec{} specializes in assessing side-channel vulnerabilities. The effectiveness of these tools was validated through extensive experimental evaluations across diverse use cases, including the quantification of biased space in neural networks and the detection of timing-based side-channel attacks in cryptographic libraries.

The experimental results highlight the utility of our quantitative framework, demonstrating its potential. For instance, \timesec{} has been shown to effectively quantify the impact of input variables in various real-world programs, enabling the identification of critical variables that may affect system stability or security. Similarly, \libra's application in neural network analysis has provided insights into how input features influence model behavior, contributing to the development of more reliable machine learning systems.

Overall, this thesis contributes to the field of static analysis by introducing the quantitative framework on top of abstract interpretation. The proposed framework can thus automatically benefit from years of recent advancement and future developments in the field as more precise abstract domains can be easily plugged in to improve the precision of the quantification. This not only enhances the reliability of software systems but also provides a scalable and adaptable approach to address the challenges of input data impact quantification.


\chapter*{Résumé}
% \addcontentsline{toc}{chapter}{Abstract} % Add the preface to the table of contents as a chapter

\index{preface}
