%----------------------------------------------------------------------------------------
%	PREFACE
%----------------------------------------------------------------------------------------
\chapter*{Abstract}
\addcontentsline{toc}{chapter}{Abstract} % Add the preface to the table of contents as a chapter

\index{Preface}

The aim of this thesis is to develop mathematically sound and practically efficient methods for improving the reliability of software systems.
Our approach is based on abstract interpretation, a formal framework to systematically design approximate program behaviors.
We focus on the development of a quantitative framework for measuring the impact of input variables on the program behavior.
Advances in the understanding of input data usage in software systems produce a more correct, performant, and secure software, particularly in safety-critical systems where these factors are non-negotiable.


To achieve this goal, we propose a novel quantitative input usage framework to discriminate between input variables based on their influence on the program behavior.
This framework is flexible, parametrized in the notion of impact to suit various needs, thereby providing a method for both certification of intended behaviors and identification of potential flaws.
By employing abstract interpretation, our results are guaranteed to be sound, meaning that the quantified bounds on the impact of input variables are always greater (or always lower depending on the underlying over- or under-approximation) than the actual impact.
The major challenge, however, is to ensure that the quantification is precise enough to be useful in practice, while still being computationally efficient.


The quantitative framework is applied to verify both extensional and intensional properties of software. Extensional properties are based on the input-output behavior of programs, while intensional properties concern how the computations are performed. Specifically, the extensional property measures how variations in input data influence program output, whereas the intensional property assesses the impact of input data on the number of iterations within the program's loops during execution.
We quantify the impact of input variables by analyzing input-output dependencies in the program under evaluation.

To demonstrate the practical applicability of this framework, we implemented it into three distinct tools: \libra, \impatto, and \timesec, each designed for a specific type of quantitative analysis. \libra{} focuses on fairness of neural networks, \impatto{} is tailored for general-purpose software reliability analysis, and \timesec{} specializes in assessing side-channel vulnerabilities. The effectiveness of these tools was validated through extensive experimental evaluations across diverse use cases, from the quantification of biased space in neural networks to the detection of timing-based side-channel attacks in cryptographic libraries.
\libra's application in neural network analysis has provided insights into how input features influence model behavior, contributing to the development of more reliable machine learning systems.
\timesec{} has been shown to effectively quantify the impact of input variables in the Amazon Web Services \bignum{} library, enabling the identification of critical variables that may affect system stability or security.

Overall, the quantitative framework proposed in this thesis advances the understanding of input data usage in software systems, providing theoretical framework based on abstract interpretation and practical tools for quantifying the impact of input data. The experimental results highlight the utility of our quantitative framework, demonstrating its potential.


\chapter*{Résumé}
% \addcontentsline{toc}{chapter}{Abstract} % Add the preface to the table of contents as a chapter

{\em
L'objectif de cette thèse est de développer des méthodes à la fois mathématiquement solides et pratiquement efficaces pour améliorer la fiabilité des systèmes logiciels. Notre approche repose sur l'interprétation abstraite, un cadre formel permettant de concevoir systématiquement des comportements de programme approximatifs. Nous nous concentrons sur le développement d'un cadre quantitatif pour mesurer l'impact des variables d'entrée sur le comportement du programme. Les avancées dans la compréhension de l'utilisation des données d'entrée dans les systèmes logiciels permettent de produire des logiciels plus corrects, performants et sécurisés, en particulier dans les systèmes critiques pour la sécurité où ces facteurs sont non négociables.

Pour atteindre cet objectif, nous proposons un nouveau cadre d'utilisation quantitative des entrées permettant de discriminer les variables d'entrée en fonction de leur influence sur le comportement du programme. Ce cadre est flexible, paramétré par la notion d'impact pour répondre à divers besoins, fournissant ainsi une méthode à la fois pour la certification des comportements attendus et pour l'identification des défauts potentiels. En utilisant l'interprétation abstraite, nos résultats sont garantis d'être corrects, ce qui signifie que les limites quantifiées de l'impact des variables d'entrée sont toujours supérieures (ou toujours inférieures en fonction de l'approximation sous-jacente, supérieure ou inférieure) à l'impact réel. Le principal défi est toutefois de garantir que la quantification soit suffisamment précise pour être utile en pratique, tout en restant efficace sur le plan computationnel.

Le cadre quantitatif est appliqué pour vérifier à la fois les propriétés extensionnelles et intentionnelles des logiciels. Les propriétés extensionnelles se basent sur le comportement entrée-sortie des programmes, tandis que les propriétés intentionnelles concernent la manière dont les calculs sont effectués. Plus précisément, la propriété extensionnelle mesure comment les variations des données d'entrée influencent la sortie du programme, tandis que la propriété intentionnelle évalue l'impact des données d'entrée sur le nombre d'itérations au sein des boucles du programme lors de l'exécution. Nous quantifions l'impact des variables d'entrée en analysant les dépendances entrée-sortie dans le programme évalué.

Pour démontrer l'applicabilité pratique de ce cadre, nous l'avons implémenté dans trois outils distincts : \libra, \impatto, et \timesec, chacun étant conçu pour un type spécifique d'analyse quantitative. \libra{} se concentre sur l'équité des réseaux neuronaux, \impatto{} est adapté à l'analyse de la fiabilité des logiciels à usage général, et \timesec{} est spécialisé dans l'évaluation des vulnérabilités aux canaux auxiliaires. L'efficacité de ces outils a été validée par des évaluations expérimentales approfondies dans divers cas d'utilisation, allant de la quantification de l'espace biaisé dans les réseaux neuronaux à la détection des attaques par canal auxiliaire temporel dans les bibliothèques cryptographiques. L'application de \libra{} dans l'analyse des réseaux neuronaux a fourni des informations sur la manière dont les caractéristiques d'entrée influencent le comportement du modèle, contribuant ainsi au développement de systèmes d'apprentissage automatique plus fiables. \timesec{} a démontré son efficacité en quantifiant l'impact des variables d'entrée dans la bibliothèque \bignum{} d'Amazon Web Services, permettant l'identification de variables critiques susceptibles d'affecter la stabilité ou la sécurité du système.

Dans l'ensemble, le cadre quantitatif proposé dans cette thèse fait progresser la compréhension de l'utilisation des données d'entrée dans les systèmes logiciels, en fournissant un cadre théorique basé sur l'interprétation abstraite et des outils pratiques pour quantifier l'impact des données d'entrée. Les résultats expérimentaux mettent en évidence l'utilité de notre cadre quantitatif, démontrant son potentiel.
}
