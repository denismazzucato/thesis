%----------------------------------------------------------------------------------------
%	PREFACE
%----------------------------------------------------------------------------------------

\chapter*{Abstract}
\addcontentsline{toc}{chapter}{Abstract} % Add the preface to the table of contents as a chapter


The aim of this thesis is to develop mathematically sound and practically efficient methods for improving the reliability of software systems through the formal framework of abstract interpretation, a formal framework designed to approximate program behaviors. This research centers on developing a quantitative framework to measure the impact of input variables on program behavior--a step forward for ensuring software correctness, performance, and security, particularly in safety critical systems where these factors are non-negotiable.


To achieve this goal, we propose a novel quantitative input usage framework to discriminate between input variables based on their influence on program behavior.
This framework is flexible, parametrized in the notion of impact to suit various needs, thereby providing a method for both certification of intended behaviors and identification of potential flaws.
By employing abstract interpretation, our results are guaranteed to be sound, meaning that the quantified bounds on the impact of input variables are always greater (or always lower depending on the underlying over- or under-approximation) than the actual impact.
The major challenge, however, is to ensure that the quantification is precise enough to be useful in practice, while still being computationally efficient.


The quantitative framework is applied to verify both extensional and intensional properties of software. Extensional properties are concerned with the input-output behavior of programs, while intensional properties also involve the examination of intermediate states. Specifically, we quantify the impact of input variables by analyzing input-output dependencies in the program under evaluation. The extensional property measures how variations in input data influence program output, whereas the intensional property assesses the impact of input data on the number of iterations within the program’s loops during execution.

To demonstrate the practical applicability of this framework, we implemented it into three distinct tools: \libra, \impatto, and \timesec, each designed for a specific type of quantitative analysis. \libra{} focuses on fairness of neural networks, \impatto{} is tailored for general-purpose software reliability analysis, and \timesec{} specializes in assessing side-channel vulnerabilities. The effectiveness of these tools was validated through extensive experimental evaluations across diverse use cases, including the quantification of biased space in neural networks and the detection of timing-based side-channel attacks in cryptographic libraries.

The experimental results highlight the utility of our quantitative framework, demonstrating its potential. For instance, \timesec{} has been shown to effectively quantify the impact of input variables in various real-world programs, enabling the identification of critical variables that may affect system stability or security. Similarly, \libra's application in neural network analysis has provided insights into how input features influence model behavior, contributing to the development of more reliable machine learning systems.

Overall, this thesis contributes to the field of static analysis by introducing the quantitative framework on top of abstract interpretation. The proposed framework can thus automatically benefit from years of recent advancement and future developments in the field as more precise abstract domains can be easily plugged in to improve the precision of the quantification. This not only enhances the reliability of software systems but also provides a scalable and adaptable approach to address the challenges of input data impact quantification.


\chapter*{Résumé}
% \addcontentsline{toc}{chapter}{Abstract} % Add the preface to the table of contents as a chapter

\emph{L'objectif de cette thèse est de développer des méthodes mathématiquement rigoureuses et pratiquement efficaces pour améliorer la fiabilité des systèmes logiciels en utilisant le cadre formel de l'interprétation abstraite, un cadre conçu pour approximer les comportements des programmes. Cette recherche se concentre sur le développement d'un cadre quantitatif permettant de mesurer l'impact des variables d'entrée sur le comportement des programmes—une avancée essentielle pour garantir la correction, la performance et la sécurité des logiciels, en particulier dans les systèmes critiques où ces facteurs sont non négociables.}

\emph{Pour atteindre cet objectif, nous proposons un cadre novateur d'utilisation quantitative des entrées afin de discriminer les variables d'entrée en fonction de leur influence sur le comportement des programmes. Ce cadre est flexible, paramétré par la notion d'impact pour s'adapter à divers besoins, offrant ainsi une méthode à la fois pour la certification des comportements attendus et pour l'identification des failles potentielles. En employant l'interprétation abstraite, nos résultats sont garantis d'être sûrs, ce qui signifie que les bornes quantifiées sur l'impact des variables d'entrée sont toujours supérieures (ou inférieures selon l'approximation sous-jacente) à l'impact réel. Le principal défi est toutefois de s'assurer que la quantification est suffisamment précise pour être utile en pratique, tout en restant efficace sur le plan computationnel.}

\emph{Le cadre quantitatif est appliqué pour vérifier à la fois les propriétés extentionnelles et intensionnelles des logiciels. Les propriétés extentionnelles concernent le comportement entrée-sortie des programmes, tandis que les propriétés intensionnelles impliquent également l'examen des états intermédiaires. Plus précisément, nous quantifions l'impact des variables d'entrée en analysant les dépendances entrée-sortie dans le programme évalué. La propriété extentionnelle mesure l'influence des variations des données d'entrée sur la sortie du programme, tandis que la propriété intensionnelle évalue l'impact des données d'entrée sur le nombre d'itérations des boucles du programme pendant son exécution.}

\emph{Pour démontrer l'applicabilité pratique de ce cadre, nous l'avons implémenté dans trois outils distincts : \libra{}, \impatto{} et \timesec{}, chacun conçu pour un type spécifique d'analyse quantitative. \libra{} se concentre sur l'équité des réseaux de neurones, \impatto{} est conçu pour l'analyse de la fiabilité des logiciels à usage général, et \timesec{} est spécialisé dans l'évaluation des vulnérabilités aux canaux auxiliaires. L'efficacité de ces outils a été validée par des évaluations expérimentales approfondies sur divers cas d'utilisation, notamment la quantification des biais dans les réseaux de neurones et la détection des attaques par canal auxiliaire fondées sur le temps dans les bibliothèques cryptographiques.}

\emph{Les résultats expérimentaux soulignent l'utilité de notre cadre quantitatif, démontrant son potentiel. Par exemple, \timesec{} s'est avéré efficace pour quantifier l'impact des variables d'entrée dans divers programmes réels, permettant l'identification des variables critiques susceptibles d'affecter la stabilité ou la sécurité du système. De même, l'application de \libra{} dans l'analyse des réseaux de neurones a fourni des insights sur la manière dont les caractéristiques d'entrée influencent le comportement du modèle, contribuant au développement de systèmes d'apprentissage automatique plus fiables.}

\emph{Dans l'ensemble, cette thèse apporte une contribution significative au domaine de l'analyse statique en introduisant un cadre quantitatif sur la base de l'interprétation abstraite. Le cadre proposé peut ainsi automatiquement bénéficier des avancées récentes et des développements futurs dans le domaine, car des domaines abstraits plus précis peuvent être facilement intégrés pour améliorer la précision de la quantification. Cela améliore non seulement la fiabilité des systèmes logiciels, mais fournit également une approche évolutive et adaptable pour relever les défis de la quantification de l'impact des données d'entrée.}

\index{preface}
